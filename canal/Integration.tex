\chapter{Integration in der komplexen Analysis}
Erinnerung $f: [a,b] \rightarrow \C$ stetig.
\[ \int_a^b f(t) \dd t = \lim_{\epsilon \rightarrow 0} \sum_{i = 0}^{N-1} f( \tau_i ) \Delta t_i \quad t_i = t_{i+1} - t_i \]
Zu jedem $\epsilon$ wird eine Zerlegung des Intervalls $[a,b]$ gewählt
\[ a = t_0 \leq t_1 < \dotsb < t_N = b \]
mit $\abs{\Delta t_i} <\epsilon$, und Auswertungspunkte $\tau_i$ mit $t_i \leq \tau_i \leq t_{i+1}$ \\
Mit dieser Definition gilt:
\[ \int_a^b f(t) \dd t = \int_a^b \Re( f(t) ) \dd t + \imath \int_a^b \Im( f(t) ) \dd t \]
Sei jetzt $f: \C \supset \Omega \rightarrow \C$ stetig, $\gamma$ eine Kurve in $\Omega$ mit einer Orientierung (=Durchlaufsinn) \\
\begin{def*}
	\[ \int_\gamma f(z) \dd z = \lim_{\epsilon \rightarrow 0} \sum_{i = 0}^{N-1} f(\zeta_i) \Delta z_i , \quad \delta z_i = z_{i+1} - z_i \]
	für $\gamma$ eine Kurve von $p$ nach $q$, wählen wir zu jedem $\epsilon$ eine Zerlegung also eine Folge $p = z_0 , z_1 , \dotsc , z_N = q$ von Punkten auf $\gamma$ mit $\abs{z_{i+1} - z_i} < \epsilon$ für alle $i$, sowie Auswertungspunkte $\zeta_i$ auf $\gamma$ zwischen $z_i$ und $z_{i+1}$
\end{def*}
Sei $t \mapsto z(t)$ eine Parameterdarstellung von $\gamma, a \leq t \leq b, z(a) = p , z(b) = q$. Gegeben eine Zerlegung $a = t_0 < \dotsb < t_N = b$ von $[a,b]$ erhalten wir eine Zerlegung $p = z(t_0) , \dotsc , z(t_N) = q$ von $\gamma$. Auswertungspunkte $\tau_i$ zwischen $t_i , t_{i+1}$ $\rightsquigarrow$ Auswertungspunkte $\zeta_i = z(\tau_i)$ zwischen $z(t_i) , z(t_N)$
\[ \begin{split}
	\int_\gamma f(z) \dd z	&= \lim_{\epsilon \rightarrow 0} \sum_{i = 0}^{N-1} f(z(\tau_i)) \underbrace{\frac{z(t_{i+1} - z(t_i)}{t_{i+1}}_{=\dot{z}(\tau_i)} - t_i} \Delta t_i \quad \Delta t_i = t_{i+1} - t_i \\
						&\overset{\text{MWS}}{=} \int_a^b f(z(t)) \dot{z}(t) \dd t
\end{split} \]
( $t \mapsto z(t)$ stetig differenzierbar ) \\
Resultat: Sei $\gamma$ eine Kurve von $p$ nach $q$ in $\Omega$ mit stetig differenzierbaren Parameterdarstellung $t \mapsto z(t) , a \leq t \leq b , z(a) = p , z(b) = q$. Dann gilt
\[ \int_\gamma f(z) \dd z = \int_a^b f(z(t)) \dot{z}(t) \dd t \]
( Merkhilfe: $\dd z = \frac{\dd z}{\dd t} \dd t$ ) \\
\begin{bsp}
	$\gamma$ Strecke von $0$ nach $a \in \C$ von $f(z) = z^n$. \\
	Parameterdarstellung von $\gamma$: $t \mapsto \underbrace{ta}_{z(t)}$ $0 \leq t \leq 1$
	\[ \begin{split}
		\int_\gamma z^n \dd z	&= \int_0^1 z(t)^n \dot{z}(t) \dd t \\
							&= \int_0^1 (ta)^n a \dd t \\
							&= a^{n+1} \int_0^1 t^n \dd t \\
							&= \frac{a^{n+1}}{n+1}
	\end{split} \]
\end{bsp}
\begin{bsp}
	$\gamma$ ein Kreis mit Radius $r$ Mittelpunkt $0$ von $r$ nach $r$ in Gegenuhrzeigersinn. Parameterdarstellung:
	\begin{gather*}
		t \mapsto z(t) = re^{\imath t} \quad 0 \leq t \leq 2\pi \\
		\dot{z}(t) = \imath re^{\imath t}
		\begin{split}
			\int_\gamma z^n \dd z	&= \int_0^{2\pi} (re^{\imath t})^n \imath re^{\imath t} \dd t \\
								&= r^{n+1} \imath \int_0^{2\pi} e^{\imath t(n+1)} \dd t \\
								&= \begin{cases}
									0		&n \neq -1	\\
									2\pi\imath	&n = -1	
								\end{cases}
		\end{split}
	\end{gather*}
\end{bsp}
\begin{bsp}
	\[ \begin{split}
		\int_\gamma \overline{z}^n \dd z	&= \int_0^{2\pi} (re^{-\imath t})^n \imath re^{-\imath t} \dd t \\
								&= r^{n+1} \imath \int_0^{2\pi} e^{\imath t(1-n)} \dd t \\
								&= \begin{cases}
									0			& n \neq 1	\\
									2\pi\imath r^2	&n = 1	
								\end{cases}
	\end{split} \]
	(hängt nicht trivial von $r$ ab)
\end{bsp}
\begin{bsp}
		$\gamma$ = Intervall $[a,b]$ von $a$ nach $b$, $f$ stetig $\Omega \rightarrow \C$ \\
		Parameterdarstellung $t \mapsto z(t) = t$ $a \leq t \leq b$, $\dot{z}(t) = 1$
		\[ \int_\gamma f(z) \dd z = \int_a^b f(t) \cdot 1 \dd t \]
\end{bsp}
Das Riemmanintegral der reellen Analysis ist also der Spezialfall wo $\gamma$ eine Strecke auf der reellen Achse ist.

\section{Eigenschaften des Linienintegrals \texorpdfstring{$\int_\gamma$}{}}
\begin{enumerate}
	\item \[ \int_\gamma (\lambda f(z) + \mu g(z) ) \dd z = \lambda \int_\gamma f(z) \dd z + \mu \int_\gamma g(z) \dd z ; \lambda ,\mu \in \C \]
	\item \[ int_\gamma f(z) \dd z = \int_{\gamma_1} f(z) \dd z + \int_{\gamma_2} f(z) \dd z \]
		$\gamma$ die Vereinigung von $\gamma_1$ und $\gamma_2$ ist, wobei der Endpunkt von $\gamma_1$ = Anfangspunkt von $\gamma_2$
\end{enumerate}
Eine Kurve $t \mapsto z(t)$ heisst \textbf{geschlossen} falls $z(b) = z(a)$. Aus 2 folgt: Ist $\gamma$ geschlossen, so ist $\int_\gamma f(z) \dd z$ unabhängig von der Wahl des Anfangpunktes. \\
\begin{bew}
	\[ \begin{split}
		\int_\gamma f(z) \dd z	&\underset{2}{=} \int_{\gamma_1} f(z) \dd z + \int_{\gamma_2} f(z) \dd z \\
							&= \int_{\gamma_2} f(z) \dd z + \int_{\gamma_1} f(z) \dd z \\
							&= \int_{\tilde{\gamma}} f(z) \dd z
	\end{split} \]
\end{bew}
\begin{enumerate}[start=3]
	\item \[ \int_{-\gamma} f(z) \dd z = - \int_\gamma f(z) \dd z \]
		wobei $-\gamma$ die Kurve $\gamma$ mit entgegengesetztem Durchlaufsinn bezeichnet. \\
		\begin{bsp*}
			\[ \int_b^a f(t) \dd t = -\int_a^b f(t) \dd t \]
		\end{bsp*}
		\begin{bew}
			Wenn $t \mapsto z(t)$ eine Parameterdarstellung von $\gamma$ ist mit $a \leq t \leq b$ dann $t \mapsto z(-t)$ $-b \leq t \leq -a$ ist eine Parameterdarstellung  von $-\gamma$.
			\begin{gather*}
				\frac{\dd}{\dd t} (z(-t)) = -\dot{z}(-t) \\
				\begin{split}
					\int_{-\gamma} f(z) \dd z	&= \int_{-b}^{-a} f(z(-t))(-\dot{z}(t)) \dd z \\
										&= \int_b^a f(z(s)) \dot{z}(s) \dd s \\
										&= -\int_a^b f(z(t)) \dot{z}(t) \dd t \\
										&= -\int_\gamma f(z) \dd z
				\end{split}
			\end{gather*}
		\end{bew}
	\item \[ \abs{\int_\gamma f(z) \dd z} \leq \max_{z \in \gamma} \abs{f(z)} \cdot \text{ Länge von } \gamma \]
		Dreiecksungleichung $\abs{a+b} \leq \abs{a} + \abs{b}$ , $\abs{a_1 + \dotsb + a_N} \leq \abs{a_1} + \dotsb + \abs{a_N}$
		\[ \abs{\sum_{i = 0}^{N-1} f(\zeta_i) \Delta z_i} \leq \sum_{i = 0}^{N-1} \underbrace{\abs{f(\zeta_i)}}_{\leq \max_{z \in \gamma} \abs{f(z)}} \abs{\Delta z_i} \leq \max_{z \in \gamma} \abs{f(z)} \sum_{i = 0}^{N-1} \underbrace{\abs{\Delta z_i}}_{\text{Länge des Polygonzugs } z_0 , \dotsc , z_N} \]
		Im Grenzwert $\epsilon \rightarrow 0$ erhalten wir die Behauptung
\end{enumerate}
\begin{def*}[note = analytische Stammfunktion , index = analytische Stammfunktion , indexformat = {2!1~ 1!~2}]
	Sei $f: \Omega \rightarrow \C$ analytisch. Eine \textbf{analyische Stammfunktion} von $f$ ist eine analytische Funktion $F: \Omega \rightarrow \C$ mit $F'(z) = f(z) \forall z \in \Omega$
\end{def*}
\begin{satz*}
	Wenn $f: \Omega \rightarrow \C$ analytisch eine Stammfunktion $F$ hat und $\gamma$ eine Kurve in $\Omega$ von $p$ nach $q$ ist, dann gilt
	\[ \int_\gamma f(z) \dd z = F(q) - F(p) \]
	(Insbesondere hängt $\int_\gamma f(z) \dd z$ nur von der Endpunkten $p$ und $q$ von $\gamma$ und nicht vom Verlauf von $gamma$)
\end{satz*}
\begin{korr*}
	In diesem Fall gilt
	\[ \int_\gamma f(z) \dd z = 0 \]
	für alle geschlossene Kurven $\gamma$.
\end{korr*}
\begin{bsp}
	\begin{gather*}
		f(z) = z^n , n \in \Z \setminus \{ -1 \} \\
		F(z) = \frac{z^{n+1}}{n+1} \\
		\Omega = \C \: (\text{oder $\C^*$ falls $n < 0$)} \\
		\int_\gamma z^n \dd z = \frac{q^{n+1}}{n+1} - \frac{p^{n+1}}{n+1} \text{ für alle Kurven $\gamma$ von $p$ nach $q$}
	\end{gather*}
\end{bsp}
\begin{bsp}
	\[ f(z) = \frac{1}{z} \quad \Omega = \C^* \]
	$f$ hat keine Stammfunktion. Hätte $f$ eine Stammfunktion, so wäre
	\[ \int_{\abs{z}=1} \frac{1}{z} \dd z = 0 \]
	aber es gilt
	\[ \int_{\abs{z}=1} \frac{1}{z} \dd z = 2\pi\imath \]
\end{bsp}
\begin{bsp}
	\[ f(z) = \frac{1}{z} \text{ auf } \Omega = \C \setminus \R^{\leq 0} \]
	Der Hauptwert $\Log$ des Logarithmus ist eine Stammfunktion ($\Log'(z) = \frac{1}{z}$) \\
	(Der Kreis $\abs{z}=1$ liegt nicht in $\Omega$)
\end{bsp}

\subsection{Welche Funktionen haben Stammfunktionen?}
\begin{def*}[note = sternförmig , index = sternförmig]
	Ein Gebiet $\Omega \subset \C$ heisst \textbf{sternförmig} bezüglich $a \in \Omega$ falls mit $z \in \Omega$ auch die Strecke $[a,z]$ von $a$ nach $z$ ebenfalls in $\Omega$ liegt.
\end{def*}
\begin{bsp}
	$\Omega = \C$ für jeden $a$
\end{bsp}
\begin{bsp}
	$\Omega =$ Kreisscheibe, alle $a$
\end{bsp}
\begin{bsp}
	BILD
\end{bsp}\todo{Copy Bild from Simon}
\begin{bsp}
	$\Omega = \C \setminus \R^{<0}$, $a$ z.B. $1$
\end{bsp}
\begin{satz*}
	Sei $\Omega$ sternförmig bezüglich $a$. Jede analytische Funktion auf $\Omega$ besitzt eine Stammfunktion. Sie ist eindeutig bis auf eine additive Konstante.
	\begin{bew}[note = Beweis und Formel]
		Sei $f: \Omega \rightarrow \C$ analytisch
		\[ F(z) = \int_{[a,z]} f(\zeta) \dd \zeta \]
		\begin{enumerate}[label = (\alph*)]
			\item $F$ ist analytisch \\
				Parameterdarstelung von $[a,z] : t \mapsto a + t(z-a) , 0 \leq t \leq 1$
				\[ F(z) = \int_0^1 f(a + t(z-a)) (z-a) \dd t \]
				Der Integrand ist (als Funktion von $z$) für jedes $t$ analytisch. \enquote{Grenzwert unter dem Integral} $\implies$ komplexe Ableitung existiert.
			\item \begin{gather*}
				\begin{split}
					F'(z)
						&= \lim_{h \rightarrow 0} \frac{F(z+h) - F(z)}{h} \qquad h \text{ parallel zu } z-a \\
						&= \lim_{h \rightarrow 0} \frac{1}{h} \left( \int_{[a,z+h]} f(\zeta) \dd \zeta - \int_{[a,z]} f(\zeta) \dd \zeta \right) \\
						&= \lim_{h \rightarrow 0} \frac{1}{h} \int_{[z,z+h]} f(\zeta) \dd \zeta \\
						&= \lim_{h \rightarrow 0} \cancel{\frac{1}{h}} \int_0^1 f(z + th) \cdot \cancel{h} \dd t \\
						&= f(z)
				\end{split} \\
				t \mapsto z + th , 0 \leq t \leq 1
			\end{gather*}
		\end{enumerate}
	\end{bew}
	\begin{bew}[note = Eindeutigkeit]
		Sind $F_1 , F_2$ zwei Stammfunktionen, dann ist $(F_1 - F_2)' = 0 \iff F_1 = F_2 + c$
	\end{bew}
\end{satz*}
\todo{Too long}
\begin{bsp*}
	$f(z) = \frac{1}{z}$ auf $\Omega = \C \setminus \R^{<0}$ \\
	$F(z) = \int_{[1,z]} \frac{1}{\zeta} \dd \zeta$ ist eine Stammfunktion von $\frac{1}{z}$ ($F'(z) = \frac{1}{z}$) $F(1) = 0 = \Log 1$
	\begin{folge}[head = Es folgt , note = (Alternative Definition von $\Log$)]
		\[ \begin{split}
			\int_{[1,z]} \frac{1}{\zeta} \dd \zeta
				&= \Log z \quad z \in \Omega \\
				&= \int_0^1 \frac{1}{1 + t(z-1)} (z-1) \dd t
		\end{split} \]
	\end{folge}
\end{bsp*}
Wir haben gesehen, dass:
\begin{enumerate}[label = \arabic*)]
	\item Wenn $f$ eine analytische Stammfunktion besitzt, dann $\int_\gamma f(z) \dd z = 0$ für alle geschlossene Kurven in $\Omega$
	\item Ist $f: \Omega \rightarrow \C$ analytisch auf einem sternförmigen Gebeit $\Omega$, dann hat $f$ eine analytische Stammfunktion $F(z) = \int_{[a,z]} f(\zeta) \dd \zeta$
\end{enumerate}
$\implies$
\begin{satz*}[note = Satz von Cauchy für sternförmige Gebiete , index = Satz von Cauchy für sternförmige Gebiete , indexformat = {1!~23456 3!12~456}]
	$\Omega$ sternförmig , $f: \Omega \rightarrow \C$ analytisch \\
	$\implies \int_\gamma f(z) \dd z = 0$ für alle geschlossenen $\gamma$ in $\Omega$
\end{satz*}
\begin{bem}[note = zu Satz von Cauchy]
	Gilt allgemein für einfach zusammenhängende Gebiete $\Omega$ (alle sternförmigen Gebiete sind einfach zusammenhängend)
\end{bem}
\begin{bem}[note = zu Satz von Cauchy]
	Gilt nicht für allgemeine Gebeite z.B. $\Omega = \C \setminus \{ 0 \}$ nicht einfach zusammenhängend
	\[ f(z) = \frac{1}{z} \implies \int_{\abs{z}=1} \frac{1}{z} \dd z = 2\pi\imath \neq 0 \]
\end{bem}
\subsection{Anwendung}
\[ \Omega = \C , f(z) = e^{-\frac{z^2}{2}} \]
Fouriertransformierte von $f$:
\[ \hat{f}(\omega) = \int_{-\infty}^{\infty} e^{\frac{z^2}{2} - \imath\omega z} \dd z \]
Wir wissen
\[ \hat{f}(0) = \int_{-\infty}^{\infty} e^{\frac{z^2}{2}} \dd z = \sqrt{2\pi} \]
Quadratische Ergänzung
\begin{gather*}
	\begin{split}
		\hat{f}(\omega) =
			&= \int_{-\infty}^{\infty} e^{-\frac{1}{2} (z + \imath \omega)^2 - \frac{\omega^2}{2}} \dd z \\
			&= e^{-\frac{\omega}{2}} \int_{-\infty}^{\infty} e^{\frac{1}{2} (z + \imath\omega)^2} \dd z \\
			&= e^{-\frac{\omega}{2}} \lim_{L \rightarrow \infty} \int_{-L}^L e^{-\frac{1}{2} (z + \imath\omega)^2} \dd z
	\end{split} \\
	\gamma = \gamma_1 + \gamma_4 + (-\gamma_2) + (-\gamma_3) \\
	0 = \int_\gamma e^{-\frac{1}{2} (z + \imath\omega)^2} \dd z = I_1 - I_2 - I_3 + I_4 \\
	I_j = \int_{\gamma_j} e^{-\frac{1}{2} (z + \imath\omega)^2} \dd z \quad j = 1 , 2 , 3 , 4 \\
	I_1 = \int_{-L}^L e^{-\frac{1}{2} (z + \imath\omega)^2} \dd z \\
	I_2 = \int_{-L}^L e^{-\frac{1}{2} (t - \imath\omega + \imath\omega)^2} \dd t = \int_{-L}^L e^{-\frac{1}{2} t^2} \dd t \\
	\gamma_2 : z(t) = -\imath\omega + t , -L \leq t \leq L
	\intertext{Zu zeigen: $I_3 , I_4 \rightarrow 0$ für $L \rightarrow \infty$}
	I_4 = \int_0^1 e^{-(L - \imath\omega t)^2} \cdot (-\imath\omega) \dd t = \int_0^1 \underbrace{e^{-\frac{L^2}{2} + \imath\omega L + \frac{\omega^2 t^2}{2}} (-\imath\omega)}_{\abs{\cdot} \leq e^{-\frac{L^2}{2}} \cdot \abs{e^{\imath\omega Lt}} \cdot e^{\frac{\omega^2}{2}} \cdot \abs{\omega}} \dd t \\
	\gamma_4 : z(t) = L - \imath\omega t , 0 \leq t \leq 1 \\
	\abs{I_4} \leq e^{-\frac{L^2}{2} + \frac{\omega^2}{2}} \abs{\omega} \overset{L \rightarrow \infty}{\rightarrow} 0
	\intertext{Analog für $I_3$}
	\text{Es folgt:} \lim_{L \rightarrow \infty} I_1 = \lim_{L \rightarrow \infty} I_2 \\
	\int_{-\infty}^\infty e^{\frac{1}{2} (z + \imath\omega)^2} \dd z = \int_{-\infty}^{\infty} e^{-\frac{1}{2} z^2} \dd z = \sqrt{2\pi} \\
	\hat{f}(\omega) = e^{-\frac{\omega^2}{2}} \lim_{L \rightarrow \infty} I_{1/2} = e^{-\frac{\omega^2}{2}} \sqrt{2\pi}
\end{gather*}
\begin{folge}[note = Folgerung von Cauchy , index = Folgerung von Cauchy , indexformat = {3!12~}]
	Jede auf einem einfach zusammenhängenden Gebiet analytische Funktion besitzt eine Stammfunktion, nämmlich
	\[ F(z) = \int_\gamma f(\zeta) \dd \zeta \]
	wobei $\gamma$ eine beliebige Kurve von einem festen Punkt $a$ nach $z$ ist.
	\begin{bew}[note = für sternförmige Gebiete]
		\begin{gather*}
			F(z) = \int_\gamma f(\zeta) \dd \zeta = \int_\gamma f(\zeta) \dd \zeta + \int_{\tilde{\gamma}} f(\zeta) \dd \zeta \\
			\tilde{\gamma} = [a,z] - \gamma \text{ geschlossen} , \int_{\tilde{\gamma}} f(\zeta) \dd \zeta = 0
		\end{gather*}
	\end{bew}
\end{folge}
\begin{def*}[note = Umlaufzahl , index = Umlaufzahl]
	Sei $\gamma$ eine geschlossene Kurve, $a \in \C$ , $a \notin \gamma$.
	\begin{beh}
		\[ \frac{1}{2\pi\imath} \int_\gamma \frac{1}{z-a} \dd z \eqqcolon n(\gamma,a) \]
	\end{beh}
	ist eine ganze Zahl. Sie heisst \textbf{Umlaufzahl} von $\gamma$ bezüglich $a$.
	\begin{bew}
		Zur Vereinfachung setzen wir $a = 0$
		\[ n(\gamma,0) = \frac{1}{2\pi\imath} \int_\gamma \frac{1}{z} \dd z \]
		$\frac{1}{z}$ hat eine Stammfunktion in jedem Sektor $S$ mit Öffnungswinkel $< 2\pi$ \\
		Ist $S \subset \C \setminus \R^{\leq 0}$ kann man $\Log$ nehmen, sonst ei anderer stetigen Zweig $\Log_S$ der Logarithmus, z.B. für $S = \left\{ \frac{3\pi}{4} < \arg z < \frac{5\pi}{4} \right\}$:
		\begin{gather*}
			\Log_S (re^{\imath\phi}) = \log r + \imath\phi , \text{ wobei } \frac{3\pi}{4} < \phi < \frac{5\pi}{4} \\
			\int_\gamma \frac{1}{z} \dd z = \int_{\gamma_1} \frac{1}{z} \dd z + \dotsb + \int_{\gamma_n} \frac{1}{z} \dd z
		\end{gather*}
		wobei jeder Teilstück $\gamma_j$ von $z_j$ nach $z_{j+1}$ in einem solchen Sektor liegt.
		\begin{gather*}
			\int_{\gamma_j} \frac{1}{z} \dd z = \log_{S_j} z_{j+1} - \log_{S_j} z_j = \log_{S_j} \frac{r_{j+1}}{r_j} + \imath ( \underbrace{\phi_{j+1} - \phi_j}_{\Delta\phi_j ; \abs{\Delta\phi_j} < 2\pi} ) \\
			\int_\gamma \frac{1}{z} \dd z = \cancel{\log r_2} - \cancel{\log r_1} + \imath\Delta\phi_1 + \cancel{\log r_3} - \cancel{\log r_2} + \imath\Delta\phi_2 + \dotsb + \cancel{\log r_1} - \cancel{\log r_n} + \imath\Delta\phi_n = \imath\Delta\phi = \text{ Vielfaches von $2\pi$ (da $\gamma$ geschlossen)} \\
			\implies \int_\gamma \frac{1}{z} \dd z = \imath\Delta\phi = \imath n(\gamma,0) \cdot 2\pi
		\end{gather*}
	\end{bew}
	Rechenmethode: \\
	Wähle Halbgerade von $a$ die $\gamma$ in endlich vielen Punkten trifft. \\
	$n(\gamma,a) = $ Anzahl Kreuzungen im positiven Sinn $-$ Anzahl Kreuzungen im negativen Sinn
\end{def*}
\todo{Too long}
\begin{lem*}
	Der Satz von Cauchy gilt auch, wenn man annimmt, das $f$ auf $\Omega \setminus \{ a \}$ analytisch ($a \in \Omega$ fest) und stetig auf $\Omega$ ist. (Beweis siehe Blatter)
\end{lem*}
\begin{satz*}[note = Cauchy-Integralformel , index = Cauchy Integralformel , indexformat = {1!~-2}]
	Sei $f: \Omega \rightarrow \C$ auf einem einfach zusammenhängenden Gebiet $\Omega$. Dann gilt für jede geschlossene Kurve $\gamma$ und $a \in \Omega , a \notin \gamma$
	\[ \frac{1}{2\pi\imath} \int_\gamma \frac{f(z)}{z-a} \dd z = n(\gamma,a) f(a) \]
\end{satz*}
\begin{bsp*}
	\[ \int_{\abs{z}=1} \frac{e^z}{z} \dd z = 2\pi\imath \cdot 1 \cdot e^0 = 2\pi\imath \]
\end{bsp*}
\begin{bew}
	\[ \frac{1}{2\pi\imath} \int_\gamma \frac{f(z)}{z-a} \dd z = \frac{1}{2\pi\imath} \int_\gamma \frac{f(z) - f(a)}{z-a} \dd z + \frac{1}{2\pi\imath} \underbrace{\int_\gamma \frac{f(a)}{z-a} \dd z}_{f(a) \cdot 2\pi\imath n(\gamma,a)} \]
	Die Funktion
	\[ g(z) = \begin{cases}
		\frac{f(z) - f(a)}{z-a} &z \neq a \\
		f'(a) &z =a
	\end{cases} \]
	ist stetig auf $\Omega$ und analytisch auf $\Omega \setminus \{ a \}$ \\
	\[ \text{Lemma } \implies \int_\gamma g(z) \dd z = \int_\gamma \frac{f(z) - f(a)}{z-a} \dd z = 0 \quad \blacksquare \]
\end{bew}
\begin{satz*}[note = Satz von Cauchy , index = Satz von Cauchy , indexformat = {3!12~ 1!~23}]
	$f: \Omega \rightarrow \C$ analytisch
	\[ \int_\gamma f(z) \dd z \]
	für alle geschlossene Kurven $\gamma$
\end{satz*}
\begin{satz*}[note = Cauchy-Integralformel , index = Cauchy Integralformel , indexformat = {1!~-2}]
	$f: \Omega \rightarrow \C$ analytisch
	\[ \frac{1}{2\pi\imath} \int_\gamma \frac{f(z)}{z-a} \dd z = n(\gamma,a) f(a) \]
	für alle geschlossene $\gamma$ in $\Omega$ die nicht durch $a$ gehen
\end{satz*}
\begin{bsp*}
	für $\gamma$ ein Kreis mit $a$ in seinem Inneren
	\[ \frac{1}{2\pi\imath} \int_K \frac{f(z)}{z-a} \dd z = f(a) \]
\end{bsp*}
Nützlich: Cauchy Formel als Darstellungssatz:
\[ f(z) = \frac{1}{2\pi\imath} \int_K \frac{f(\zeta)}{\zeta-z} \dd \zeta \]
$f(z)$ ist auf eine Kreisscheibe eindeutig bestimmt durch die Werte von $f$ auf dem Rand.
\begin{gather*}
	\frac{\dd}{\dd z} \frac{1}{\zeta - z} = \frac{1}{(\zeta - z)^2}
	f'(z) = \frac{1}{2\pi\imath} \int_K \frac{f(\zeta)}{(\zeta - z)^2} \dd \zeta \\
	f''(z) = \frac{2}{2\pi\imath} \int_K \frac{f(\zeta)}{(\zeta - z)^3} \dd \zeta \\
	f'''(z) = \frac{2 \cdot 3}{2\pi\imath} \int_K \frac{f(\zeta)}{(\zeta - z)^4} \dd \zeta
\end{gather*}
Induktion: \\
Analytische Funktionen sind unendlich oft differenzierbar. Alle höhere Ableitungen sind analytisch. Es gilt für die $n$-te Ableitung:
\[ f^{(n)}(z) = \frac{n!}{2\pi\imath} \int_K \frac{f(\zeta)}{(\zeta - z)^{n+1})} \dd \zeta \]

\subsubsection{Anwendung: Mittelwert-Eigenschaft}
$K$ = Kreis, $\abs{z-a} = r$
\begin{gather*}
	t \mapsto z(t) = a + re^{\imath t} \\
	0 \leq t \leq 2\pi \\
	\begin{split}
		f(a) = \frac{1}{2\pi\imath} \int_{\abs{z-a} = r} \frac{f(z)}{z-a} \dd z
			&= \frac{1}{2\pi\imath} \int_0^{2\pi} \frac{f(a + re^{\imath t})}{re^{\imath t}} \underbrace{\imath re^{\imath t}}_{\dot{z}(t)} \dd t \\
			&= \frac{1}{2\pi} \int_0^{2\pi} f(a + re^{\imath t}) \dd t
	\end{split}
\end{gather*}

\subsubsection{Riemannscher Hebbarkeitssatz}
\begin{lem*}[note = Technisches Lemma (s.Autographie)]
	Die Formel
	\[ F(z) = \int_{[a,z]} f(\zeta) \dd \zeta \]
	für die Stammfunktion von $f$ gilt auch wenn $f$ analytisch in $\Omega \setminus \{a\}$ ist und stetig auf $\Omega$
\end{lem*}
\begin{satz*}[note = Riemannscher Hebbarkeitssatz , index = Riemannscher Hebbarkeits satz , indexformat = {12.3 3!12-~}]
	Sei $f$ analytisch auf $\Omega \setminus \{a\}$, beschränkt ($\abs{f(z)} < M$) uaf einer Umgebung von $a$. Dann hat $f$ eine analytische Fortsetzung auf $\Omega$
\end{satz*}
\begin{bsp*}
	$f(z) = \frac{z}{e^z - 1}$ nicht definiert in $z = 0$ \\
	Betrachte $\frac{e^z - 1}{z}$
	\begin{gather*}
		\frac{e^z - 1}{z} = \frac{1 + z + \frac{z^2}{2!} + \dotsb - 1}{z} = 1 + \frac{z}{2!} + \frac{z^2}{3!} + \dotsb \\
		\lim_{z \rightarrow 0} \frac{e^z - 1}{z} = 1 \\
		\implies \lim_{z \rightarrow 0} \frac{z}{e^z - 1} = 1
	\end{gather*}
\end{bsp*}
\begin{folge}
	Es folgt aus dem Satz:
	\[ \tilde{f}(z) = \begin{cases}
		f(z) &z \neq 0 \\
		1 ^z = 0
	\end{cases} \]
	ist eine analytische Fortsetzung von $f$ in einer Umgebung von $0$
\end{folge}
Beweis: s.Blatter

Taylorreihe: \\
Sei $f$ analytisch auf $\Omega$. Dann gilt für alle $a \in \Omega$
\[ f(z) = f(a) + f'(a)(z-a) + \frac{f''(a)}{2!}(z-a)^2 + \dotsb \]
Diese Reihe konvergiert für alle $z$ in der grössten in $\Omega$ enthaltener offener Kreischeibe um $a$.
\begin{bew}[note = {Bewies für $a = 0$}]
	\begin{gather*}
		f(z) = \frac{1}{2\pi\imath} \int_K \frac{f(\zeta)}{\zeta - z} \dd z \quad K = \text{ Kreis um $0$ mit Radius } r \\
		\abs{z} < \abs{\zeta} = r \\
		\frac{1}{\zeta - z} = \frac{1}{\zeta} \frac{1}{1-\frac{z}{\zeta}} = \frac{1}{\zeta} + \frac{z}{\zeta^2} + \frac{z^2}{\zeta^3} + \dotsb \quad \abs{\frac{z}{\zeta}} < 1 \\
		f(z) = \frac{1}{2\pi\imath} \sum_{n=0}^\infty \frac{f(\zeta)}{\zeta^{n+1}} \dd \zeta z^n = \sum_{n=0}^\infty \frac{f^{(n)}(0)}{n!} z^n
	\end{gather*}
\end{bew}
$f: \Omega \rightarrow \C , \Omega \subset \C$ eine zusammenhängende analytische Funktion.
\begin{gather*}
	\int_\gamma f(z) \dd z = 0 \\
	\int_\gamma \frac{f(z)}{z-a} \dd z = 2\pi\imath f(a) \\
	n! \int_\gamma \frac{f(z)}{(z-a)^{n+1}} \dd z = 2\pi\imath f^{(n)}(a)
\end{gather*}
\begin{satz*}[note = Satz von Liouville , index = Satz von Liouville , indexformat = {3!12~ 1!~23}]
	$f: \C \rightarrow \C$ analytisch \\
	$f$ sei beschränkt, d.h. $\abs{f(z)} \leq M , z \in \C$ \\
	$\implies f$ ist konstant
	\begin{gather*}
		\begin{split}
			f'(a)
				&= \frac{1}{2\pi\imath} \int_{\gamma_r} \frac{f(z)}{(z-a)^2} \dd z \rightarrow \abs{f'(a)} \leq \frac{1}{2\pi\imath} 2\pi r M \frac{1}{r^2} \\
				&= \frac{M}{r} \rightarrow 0 \: ( r \rightarrow \infty )
		\end{split} \\
		f'(a) = 0
	\end{gather*}
\end{satz*}
$f: \C \rightarrow \C$ analytisch
\[ \abs{f(z)} \leq M ( 1 + \abs{z}^n ) \]
$f$ ist ein Polynom

\[ \abs{f^{(n)}(a)} \leq \frac{M}{r^n} n! \]
$M$ Maximum auf einer Kreisscheibe mit Radius $r$

$\gamma_1 , \dotsc , \gamma_n$ geschlossene Kurven \\
$\gamma = \gamma_1 + \dotsb + \gamma_n$ Zyklus \\
\[ \int_\gamma f(z) \dd z = \sum_{j=1}^n \int_{\gamma_j} f(z) \dd z \]

\begin{def*}[note = nullhomologer Zyklus , index = nullhomologer Zyklus , indexformat = {2!1 }]
	\[ \forall z \notin \Omega : n(\gamma , z) = 0 \]
\end{def*}

$f: \Omega \rightarrow \C$ analytisch, $\Omega$ beliebiges Gebiet, $\gamma$ nullhomologer Zyklus \\
\[ \int_\gamma f(z) \dd z = 0 \]
$G \subset \Omega$ Gebiet mit $\partial G$
\[ \int_{\partial G} \frac{f(z)}{z-a} \dd z =2\pi\imath n( \partial G , a ) f(a) \quad a \in G \]

\begin{gather*}
	f: \Omega \rightarrow \C \\
	\int_{\partial G} \frac{f(z)}{z-w} \dd z = 2\pi\imath f(w) n( \partial G , w ) \\
	0 = \int_{\partial D_b} \frac{f(z)}{z-w} \dd z - \int_{\partial D_a} \frac{f(z)}{z-w} \dd z
\end{gather*}

\subsection{Laurentreihen und Residuum}
\begin{def*}[note = Laurent-Reihe , index = Laurent Reihe , indexformat = {1!~-2 2!1-~}]
	Eine Potenzreihe der Form
	\[ \sum_{k = -\infty}^{\infty} c_k z^k = \sum_{k=1}^{\infty} c_{-k} \frac{1}{z_k} + \sum_{k=0}^{\infty} c_k z^k \]
	heisst \textbf{Laurent-Reihe}
\end{def*}
\begin{def*}[note = Randzyklus , index = Rand zyklus , indexformat = {1.2 2!~.1}]
	\textbf{Randzyklus} $\partial G$ eines Gebeits $G$
	\[ \begin{cases}
		n(\partial G,z) = 1 &\text{für } z \in G \\
		n(\partial G,z) = 0 &\text{für } z \notin G \cup \partial G
	\end{cases} \]
	(\enquote{Gebiet liegt in Umlaufrichtung links})
\end{def*}
\begin{gather*}
	f: \Omega \rightarrow \C \text{ analytisch} \\
	G = \{ z \in \C : a < \abs{z - z_0} < b \} \\
	w \in G \\
	f(w) = \frac{1}{2\pi\imath} \int_{\partial G} \frac{f(z)}{z-w} \dd z = \frac{1}{2\pi\imath} \int_{\partial D_b} \frac{f(z)}{z-w} \dd z - \frac{1}{2\pi\imath} \int_{\partial D_a} \frac{f(z)}{z-w} \dd z \\
	g_b(w) \coloneqq \int_{\partial D_b} \frac{f(z)}{z-w} \dd z , \abs{w-z_0} < b = \abs{z-z_0} \text{ (dort ist $g_b$ analytisch)} \\
	\frac{1}{z-w} = \frac{1}{z - z_0 - (w - z_0)} = \frac{1}{z-z_0} \cdot \frac{1}{1 - \frac{w-z_0}{z-z_0}} = \frac{1}{z-z_0} \sum_{j=0}^{\infty} \left( \frac{w-z_0}{z-z_0} \right)^j \\
	\begin{split}
		g_b(w)
			&=  \int_{\partial D_b} \frac{f(z)}{z-z_0} \cdot \sum_{j=0}^{\infty} \left( \frac{w-z_0}{z-z_0} \right)^j \\
			&= \sum_{j=0}^{\infty} (w-z_0)^j \int_{\partial D_b} \underbrace{\frac{f(z)}{(z-z_0)^{j+1}}}_{c_j} \dd z \\
			&= \sum_{j=0}^{\infty} c_j \cdot (w-z_0)^j
	\end{split} \\
	g_a(w) \coloneqq \int_{\partial D_a} \frac{f(z)}{z-w} \dd z \quad \abs{w-z_0} > a \text{ (dort ist $g_a$ analytisch)} \\
	\frac{1}{z-w} = \frac{1}{w-z_0} \cdot \frac{1}{\underbrace{\frac{z-z_0}{w-z_0}}_{\abs{\cdot} < 1} - 1} = -\frac{1}{w-z_0} \sum_{j=0}^{\infty} \left( \frac{z-z_0}{w-z_0} \right)^j \\
	\begin{split}
		g_a(w)
			&= -\sum_{j=0}^{\infty} (w-z_0)^{-j-1} \int_{\partial D_a} \frac{f(z)}{(z-z_0)^{-j}} \dd z \\
			&= -\sum_{j = -\infty}^{-1} (w-z_0)^j \underbrace{\int_{\partial D_a} \frac{f(z)}{(z-z_0)^{j+1}} \dd z}_{c_j} \\
			&= -\sum_{j = -\infty}^{-1} (w-z_0)^j \cdot c_j
	\end{split} \\
	\begin{split}
		f(w)
			&= \frac{1}{2\pi\imath} \left( \sum_{j=0}^{\infty} (w-z_0)^j c_j + \sum_{j = -\infty}^{-1} (w-z_0)^j c_j \right) \\
			&= \sum_{j = -\infty}^{\infty} d_j (w-z_0)^j \\
			&\implies \text{ Laurent-Reihe von $f$ bezüglich Entwickelungspunkt $z_0$}
	\end{split} \\
	d_j = \frac{1}{2\pi\imath} \int_{\partial D_r} \frac{f(z)}{(z-w)^{j+1}} \dd z
\end{gather*}
\begin{bsp*}
	\begin{gather*}
		f(z) = \frac{1}{z^2 - 1} \implies \text{ Singularitäten bei } \pm 1 \\
		\begin{split}
			\frac{1}{z^2 - 1}
				&= \frac{1}{(z-1)(z+1)} \\
				&= \frac{1}{z-1} \cdot \frac{1}{2 + (z-1)} \\
				&= \frac{1}{z-1} \frac{1}{1 - \frac{1-z}{2}} \cdot \frac{1}{2} \\
				&= \frac{1}{2(z-1)} \sum_{j=0}^{\infty} \left( \frac{-z + 1}{2} \right)^j \\
				&= \frac{1}{2(z-1)} \sum_{j=0}^{\infty} (z-1)^j \left( -\frac{1}{2} \right)^j \\
				&= -\sum_{j=-1}^{\infty} (z-j)^j \left( -\frac{1}{2} \right)^{j+2}
		\end{split} \\
		0 < \abs{z-1} < 2 \\
		\begin{split}
			\frac{1}{z^2 - 1}
				&= \frac{1}{z-1} \cdot \frac{1}{2 + (z-1)} \\
				&= \frac{1}{(z-1)^2} \frac{1}{1 + \frac{2}{z-1}} \\
				&= \frac{1}{(z-1)^2} \cdot \sum_{j=0}^{\infty} \left( \frac{-2}{z-1} \right)^j \\
				&= \sum_{j = -\infty}^{-2} (-2)^{j+2} (z-1)^j \\
				&= \frac{1}{(z-1)^2} - \frac{2}{(z-1)^3} + \dotsb
		\end{split}
	\end{gather*}
\end{bsp*}
\begin{bsp*}
	\[ \begin{split}
		f(z)
			&= \frac{z+1}{z^3 - z^2} \\
			&= \frac{1}{z^2} \cdot \frac{z+1}{z-1} \\
			&= \frac{1}{z^2} \cdot \frac{z}{z-1} + \frac{1}{z^2} \cdot \frac{1}{z-1} \\
			&= -\frac{1}{z} \frac{1}{1-z} - \frac{1}{z^2} \frac{1}{1-z} \\
			&= -\frac{1}{z} \sum_{j=0}^{\infty} z^j - \frac{1}{z^2} \sum_{j=0}^{\infty} z^j \\
			&= -\frac{1}{z^2} - \frac{2}{z} - 2 \sum_{j=0}^{\infty} z^j
	\end{split} \]
\end{bsp*}

\subsection{Isolierte Singularitäten}
\begin{enumerate}
	\item Hauptteil nicht vorhanden: $f(z) = \sum_{j=0}^{\infty} c_j (z-z_0)^j$ \\
		$\implies z_0$ ist hebbare Singularität
		$f(z_0) \coloneqq \lim_{z \rightarrow z_0} f(z)$ (Grenzwert existiert)
	\item Hauptteil besteht aus endlich vielen Termen
		\begin{gather*}
			h(z) = \sum_{j=1}^N c_{-j} (z-z_0)^{-j} \\
			\implies \lim_{z \rightarrow z_0} f(z) = \infty \\
			g(z) \coloneqq \begin{cases}
				\frac{1}{f(z)} &z \neq z_0 \\
				0 &z = 0
			\end{cases} \text{ $g$ analytisch in Umgebung von $z_0$, dort $g(z) \neq 0$ für $\forall z \neq z_0$}
			\intertext{$g$ hat NS endlicher Ordnung}
			g(z) = (z-z_0)^N , g_0(z) , g_1(z) \neq 0 \\
			\begin{split}
				f(z)
					&= \frac{1}{(z-z_0)^N} \cdot \frac{1}{\underbrace{g_1(z)}_{\text{$f_1(z)$ analytisch in Umgebung von $z_0$}}} \\
					&= \frac{1}{(z-z_0)^N} \cdot \sum_{j=0}^{\infty} d_j (z-z_0)^j \\
					&= d_0 (z-z_0)^{-N} + \dotsb + d_{N-1} (z-z_0)^{-1} + \dotsb
			\end{split} \\
		\end{gather*}
		$\implies f$ hat Pol $N$-ter Ordnung in $z_0$
	\item \[ \text{Hauptteil } = \sum_{j=1}^{\infty} d_j (z-z_0)^{-j} \]
		\begin{bsp*}
			\[ e^{\frac{1}{z}} = \sum_{j=0}^{\infty} \frac{1}{j!} \left( \frac{1}{z} \right)^j \]
		\end{bsp*}
		\enquote{$f$ hat eine wesentliche Singularität in $z_0$}
\end{enumerate}
\begin{satz*}[note = Satz von Picard , index = Satz von Picard , indexformat = {3!12~ 1!~23}]
	Falls $f$ eine wesentliche Singluratität in $z_0$ hat, so nimmt $f$ in jeder Umgebung $\dot{U}$ von $z_0$ jeden Wert $w \in \C$ unendlich oft an, bis auf höchstens eine Ausnahme.
\end{satz*}
$\implies e^{\frac{1}{z}} = w$ hat $\infty$ viele Lösungen für $\forall w \neq 0$, auch falls $\abs{z} < \epsilon$\\
Eine analytische Funktion auf Ringgebiet $R = \{ z \in \C | r < \abs{z-a} < R \}, 0 \leq r < R \leq \infty$ kann in eine Laurentreihe
\[ f(z) = \sum_{n=-\infty}^{\infty} c_n (z-a)^n \]
Die Laurentkoeffizienten $c_n$ sind eindeutig bestimmt.
\[ c_n = \frac{1}{2\pi\imath} \int_{\abs{z-a} = s} f(z) (z-a)^{-n-1} \dd z \text{ für jedes } s , r < s < R \]
\begin{bem}
	Dieselbe Funktion kann verschiedene Laurent-Reihen in verschiedenen Ringgebieten haben.
\end{bem}
\begin{bsp*}
	\begin{gather*}
		\frac{1}{1-z} = \begin{cases}
			\sum_{n=0}^{\infty} z^n &0 < \abs{z} < 1 \\
			-\sum_{n=-\infty}^{-1} &1 < \abs{z} < \infty
		\end{cases} \\
		\frac{1}{1-z} = \frac{\frac{1}{z}}{\frac{1}{z} - 1} = -\frac{1}{z} \frac{1}{1 - \frac{1}{z}} = -\sum_{n=1}^{\infty} \frac{1}{z^n} \quad \left( \abs{\frac{1}{z}} < 1 \right)
	\end{gather*}
\end{bsp*}

\subsubsection{Speziallfall: Laurent-Reihe um einen isolierten Singularität}
$f$ analytisch auf $\Omega \setminus \{a\}, a \in \Omega$ \\
In einer Umgebung von $a$ hat $f$ eine Laurent-Entwickelung
\[ f(z) = \sum_{n=-\infty}^{\infty} c_n (z-a)^n \quad 0 < \abs{z-a} < R \]
in den Punktierten Kreisscheibe um $a$ mit Radius $R$.
\[ c_n = \frac{1}{2\pi\imath} \int_{\abs{z-a} = \epsilon} f(z) (z-a)^{-n-1} \dd z \quad 0 < \epsilon < R \]
Der Koeffizient $c_{-1} = \frac{1}{2\pi\imath} \int_{\abs{z-a} = \epsilon} f(z) \dd z$ heisst Residuum von $f$ an der Stelle $a$
\[ \res_{z=a} f(z) = \res (f|a) = \frac{1}{2\pi\imath} \int_{\abs{z-a} = \epsilon} f(z) \dd z \]
\begin{bsp*}
	\[ f(z) = \frac{e^z - 1}{z^3} \]
	hat eine Isolierte Singularität $z=0$
	\begin{gather*}
		\res_{z=0} f(z) = \frac{1}{2\pi\imath} \int_{\abs{z-a} = \epsilon} \frac{e^z - 1}{z^3} \dd z
		\intertext{Zum Ausrechnen}
		\frac{e^z - 1}{z^3} = \frac{z + \frac{z^2}{2!} + \frac{z^3}{3!} + \dotsb }{z^3} = \frac{1}{z^2} + \frac{1}{2z} + \dotsb \\
		\res_{z=0} f(z) = \text{ Koeffizient von } \frac{1}{z} = \frac{1}{2}
	\end{gather*}
\end{bsp*}
\begin{satz*}[note = Residuumsatz , index = Residuum satz , indexformat = {2!1-~ 1!~-2 1.2}]
	Sei $f$ analytisch auf einem Gebiet $\Omega$ ausser an isolierten Singularitäten. Sei $G \subset \Omega$ mit Randzyklus $\partial G$ in $\Omega$ der nicht durch die Singularitäten geht. Seien $a_1 , \dotsc , a_n$ die isolierten Singularitäten in $G$. Dann gilt
	\[ \frac{1}{2\pi\imath} \int_{\partial G} f(z) \dd z = \res_{z=a_1} f(z) + \dotsb + \res_{z=a_n} f(z) \]
	\begin{bew}
		$G' = G \setminus$ kleine Kreisscheiben um $a_1 , \dotsc , a_n$. $f$ ist analytisch auf $G'$. Cauchy:
		\[ \begin{split}
			\int_{\partial G'} f(z) \dd z
				&= 0 \\
				&= \int_{\partial G} f(z) \dd z - \int_{\abs{z-a_1} = \epsilon} f(z) \dd z \\
				&- \dotsb - \int_{\abs{z-a_n} = \epsilon} f(z) \dd z
		\end{split} \]
	\end{bew}
	\begin{bew}[note = 2. Erklärung]
		Die Integrale $\int_\gamma f(z) \dd z$ änder sich nicht wenn man $\gamma$ deformiert (im Definitionsbereich):
		\[ \int_\gamma f(z) \dd z = \int_{\gamma'} f(z) \dd z + \underbrace{\int_{\gamma''} f(z) \dd z}_{=0 \text{ Cachy}} \]
		Also gilt
		\begin{gather*}
			\int_{\partial G} f(z) \dd z = \int_{\gamma'} f(z) \dd z \\
			\begin{split}
				\int_{\text{Strecken}}
					&= 0 \\
					&= \sum_{j=1}^n \int_{\abs{z-a_j} = \epsilon} f(z) \dd z \\
					&= 2\pi\imath \sum _{j=1}^n \res_{z=a_j} f(z)
			\end{split}
		\end{gather*}
	\end{bew}
\end{satz*}
\begin{bsp*}
	\begin{gather*}
		I = \int_0^{2\pi} \frac{1}{2 + \cos \phi} \dd \phi \\
		z = e^{\imath\phi} \\
		\cos \phi = \frac{e^{\imath\phi} + e^{-\imath\phi}}{2} = \frac{z - z^{-1}}{2} \\
		\dd z = \imath e^{\imath\phi} \dd \phi= \imath z \dd \phi \\
		I = \frac{1}{\imath} \int_{\abs{z}=1} \frac{1}{2 + \frac{z + z^{-1}}{2}} \frac{1}{z} \dd z = \frac{1}{\imath} \int_{\abs{z}=1} \frac{2}{z^2 + 4z + 1} \dd z
	\end{gather*}
\end{bsp*}
$a \in \C$ heisst isolierte Singularität einer analytischen Funktion $f$ wenn $f$ in $a$ nicht definiert ist aber es gibt eine punktierte Kreisscheibe $0 < \abs{z-a} < r$ um $a$ wo $f$ definiert ist.
\begin{bsp*}
	\[ f = \frac{1}{\sin \frac{1}{z}} \]
	nicht definiert für $z = 0 , \frac{1}{\pi n} , n \in \Z$
\end{bsp*}
Um isolierten Singularitäten hat $f$ eine Laurent-Entwickelung
\[ f(z) = \dotsb +  \frac{c_{-2}}{(z-a)^2} + \frac{c_{-1}}{z-a} + c_0 + c_1 (z-a) + c_2 (z-a)^2 + \dotsb \quad r > \abs{z-a} > 0 \]
\begin{enumerate}[label = (\alph*)]
	\item $\dotsb = c_{-2} = c_{-1} = 0$ $f(z) = c_0 + c_1 (z-a) + c_2 (z-a)^2 + \dotsb$ Hebbare Singularität
		\begin{bsp*}
			\[ \frac{\sin z}{z} \]
		\end{bsp*}
	\item $\dotsb + c_{-(N+2)} = c_{-(N+1)} = 0$, $c_{-N} \neq 0$ ($N > 0$)
		\[ f(z) = \underbrace{\frac{c_{-N}}{(z-a)^N} + \frac{c_{-N+1}}{(z-a)^{N-1}} + \dotsb + \frac{c_{-1}}{z-a}}_{\text{Hauptteil}} + c_0 + c_1 (z-a) + \dotsb \]
		$a$ eine Polstelle der Ordnung $N$. $f$ hat einen Pol der Ordnung $N$ an der Stelle $a$
		\begin{def*}[note = einfacher Pol , index = einfacher Pol , indexformat = {2!1~}]
			Ein Pol der Ordnung 1 heisst \textbf{einfacher Pol}
			\[ f(z) = \frac{c_{-1}}{z-a} + c_0 + \dotsb \]
		\end{def*}
	\item Wesentliche Singularitäten = isolierte Singularität der keine Pole und nicht hebbar sind.
\end{enumerate}
\begin{enumerate}[label = (\alph*)]
	\item $\lim_{z \rightarrow a} f(z)$ existiert
	\item $\lim_{z \rightarrow a} \abs{f(z)} = \infty$
	\item $\lim_{z \rightarrow a} \abs{f(z)}$ existiert nicht auch als uneigentlicher ($\infty$) Limes
\end{enumerate}
Wie rechnet man das Residuum, $c_{-1}$ an einer einfachen Polstelle? \\
$f$ hat einen einfachen Pol an der Stelle $a$ genau dann wenn der Grenzwert $\lim_{z \rightarrow a} (z-a) f(z)$ existiert; dieser Grenzwert ist das Residuum $c_{-1}$
\begin{enumerate}[label = \arabic*)]
	\item $\res_{z=a} f(z) = \lim_{z \rightarrow a} (z-a) f(z)$ ($a$ einfache Polstelle)
	\item Sei $f(z) = \frac{p(z)}{q(z)}$, $p,q$ analytisch in einer Umgebung von $a \in \C$. Sei $a$ eine einfache Nullstelle von $q$, d.h. $q(z) = (z-a) h(z)$ wobei $h(z)$ analytisch und $h(a) \neq 0$ (oder $q(z) = a_1 (z-a) + a_2 (z-a)^2 + a^3 (z-a)^3 + \dotsb , a_1 \neq 0$)
		\begin{gather*}
			\begin{split}
				\lim_{z \rightarrow a} (z-a) \frac{p(z)}{q(z)}
					&= \lim_{z \rightarrow a} \frac{(z-a) p(z)}{a_1 (z-a) + a_2 (z-a)^2 + \dotsb} \\
					&= \lim_{z \rightarrow a} \frac{p(z)}{\underbrace{a_1 + a_2 (z-a) + \dotsb}_{=h(z)}} \\
					&= \frac{p(a)}{a_1} \\
					&= \frac{p(a)}{q'(a)}
			\end{split} \\
			\res_{z=a} \frac{p(z)}{q(z)} = \frac{p(a)}{q'(a)}
		\end{gather*}
		$p,q$ analytisch, $q$ hat eine einfache Nullstelle an der Stelle $a$
\end{enumerate}
\begin{bsp*}
	\begin{gather*}
		\begin{split}
			\int_0^{2\pi} \frac{1}{\cos \phi + 2}
				&= \frac{1}{\imath} \int_{\abs{z}=1} \frac{2}{z^2 + 4z + 1} \dd z \\
				&\overset{\text{\scriptsize{Residuensatz}}}{=} \frac{1}{\imath} 2\pi\imath \sum_{\substack{\abs{a}<1 \\ \text{Singularitäten}}} \res_{z=a} \frac{2}{z^4 + 4z + 1}
		\end{split} \\
		\qquad \text{Singularitäten: } z^2 + 4z + 1 = 0 \\
		\qquad a_{1,2} = -2 \pm \sqrt{3} \\
		= \frac{1}{\imath} 2\pi\imath \res_{z = -2 + \sqrt{3}} \frac{2}{z^4 + 4z + 1}
		\intertext{Verwende 2) mit $p = 2 , q = z^4 + 4z + 1$}
		= \frac{1}{\imath} 2\pi\imath \left. \frac{2}{2z + 4} \right|_{z = -2 + \sqrt{3}} = \frac{2\pi}{\sqrt{3}} \\
		\int_0^{2\pi} = \frac{2 + \cos \phi} \dd \phi = \frac{2\pi}{\sqrt{3}}
	\end{gather*}
\end{bsp*}
Allgemeiner: Wenn $R$ eine rationale Funktion in zwei Variablen ist,
\[ \begin{split}
	\int_0^{2\pi} R(\cos \phi , \sin \phi)
		&= \begin{vmatrix*}[l] z = e^{\imath\phi} \\ \dd z = \imath e^{\imath \phi} \dd \phi \\ \dd \phi = \frac{1}{\imath} \frac{\dd z}{z} \end{vmatrix*} \\
		&= \frac{1}{\imath} \int_{\abs{z}=1} R\left( \frac{1}{2} (z + z^{-1}) , \frac{1}{2\imath} (z - z^{-1} \right) \\
		&= 2\pi \sum_{\substack{\text{Singularität } a \\ \text{mit } \abs{a} < 1}} R\left( \frac{1}{2} (z + z^{-1}) , \frac{1}{2\imath} (z - z^{-1} \right) \frac{1}{2}
\end{split} \]
\begin{bsp*}
	\[ \int_{-\infty}^{\infty} \frac{1}{1 + x^2} \dd x = \lim_{R \rightarrow \infty} \int_{-R}^R \frac{1}{1 + x^2} \dd x \]
	Sei $\gamma$ der Weg, der aus $[-R,R]$ und dem Halbkreis $\gamma' : \abs{z} = R , \Im z > 0$
	\begin{gather*}
		\abs{\int_{\gamma'} \frac{1}{1 + z^2} \dd z} \leq \underbrace{\max_{\abs{z}=R} \abs{\frac{1}{1 + z^2}}}_{\leq \frac{c}{R^2}} \pi R \underset{R \rightarrow \infty} \rightarrow \infty \quad \text{für $R$ gross} \\
		\begin{split}
			\lim_{R \rightarrow \infty} \int_{-R}^R \frac{1}{1 + x^2} \dd x
				&= \lim_{R \rightarrow \infty} \int_\gamma \frac{1}{1 + z^2} \dd z \\
				&= 2\pi\imath\res_{z=\imath} \lim_{z \rightarrow \imath} (z-\imath) \frac{1}{1+z^2} \\
				&= 2\pi\imath \frac{1}{2\imath}
		\end{split} \\
		\int_{-\infty}^{\infty} \frac{1}{1 + x^2} \dd x = \pi \\
		1 + z^2 = (z+\imath)(z-\imath)
	\end{gather*}
	\begin{bem}
		Wir hätten den Integrationsweg auch in der unteren Halbebene schliessen können, mit demselben Resultat (Übung)
	\end{bem}
\end{bsp*}
\begin{bsp*}
	Fouriertransformierte von $\frac{1}{1 + x^2}$
	\begin{gather*}
		\int_{-\infty}^{\infty} \frac{1}{1 + x^2} e^{-\imath\omega x} \dd x \quad \omega \in \R \\
		\int_{\gamma} \frac{1}{1 + x^2} e^{-\imath\omega z} \dd z \\
		e^{-\imath\omega z} \overset{\substack{y > 0 \\ z = x + \imath y}}{=} e^{-\imath\omega x} e^{\omega y} \text{ beschränkt falls } \omega < 0 \\
		\begin{split}
			\int_{-\infty}^{\infty} \frac{1}{1 + x^2} e^{-\imath\omega x} \dd x
				&\underset{\omega < 0}{=} \lim_{R \rightarrow \infty} \int_{\gamma} \frac{1}{1 + z^2} e^{-\imath\omega z} \dd z \\
				&= 2\pi\imath \res_{z=\imath} \frac{1}{1 + z^2} e^{-\imath\omega z} \dd z \\
				&= 2\pi\imath \lim_{z \rightarrow \imath} (z-\imath) \frac{1}{1 + z^2} e^{-\imath\omega z} \\
				&= \pi e^{\omega} \quad (\omega \leq 0)
		\end{split}
	\end{gather*}
	Für $\omega > 0$: Entweder verwende die untere Halbebene oder \textbf{bemerke, dass}
	\begin{gather*}
		\int_{-\infty}^{\infty} \frac{1}{1 + x^2} e^{-\imath\omega x} \dd x = \int_{-\infty}^{\infty} \frac{1}{1 + x^2} e^{\imath\omega x} \dd x = \hat{f}(\omega) \\
		\omega > 0 : \hat{f}(\omega) = \hat{f}(-\omega) = \pi e^{-\abs{\omega}} \\
		\text{Schlussresultat: } \hat{f}(\omega) \int_{-\infty}^{\infty} \frac{1}{1+x^2} e^{-\imath\omega x} \dd x = \pi e^{-\abs{\omega}}
	\end{gather*}
\end{bsp*}
\begin{bsp*}
	\[ \int_0^{\infty} \frac{t^{\lambda}}{1 + t^2} \dd t \quad -1 < \lambda < 1 \implies \lambda - 2 < -1 \]
	Idee: Betrachte den Integrationsweg: BILD \\
	$I_{\gamma} = I_1 + I_2 + I_3 + I_4$ \\
	Idee: $I_1 , I_3 \rightarrow 0$ für $r \rightarrow 0 , R \rightarrow \infty , \delta \rightarrow 0$ \\
	$I_2 , I_4$ Polynomial zu $I$
	\begin{gather*}
		\left( I_j = \int_{\gamma_j} \frac{e^{\lambda \Log z}}{1 + z^2} \dd z \right) \\
		I_{\gamma} = \int_{\gamma} \frac{e^{\lambda \Log z}}{1 + z^2} \dd z
		\intertext{$I_2$: Parameterdarstellung von $-\gamma_2$:}
		t \mapsto t ^{\imath(\pi - \delta)} \quad r < t < R \\
		\begin{split}
			I_2
				&= - \int_r^R \frac{e^{\lambda(\Log t + \imath(\pi - \delta))}}{1 + t^2 e^{2\imath(\pi - \delta)}} \dd t \\
				&= -e^{\imath\lambda(\pi - \delta)} \int_r^R \frac{t^2}{1 + t^2 e^{-2\imath\delta}} \dd t \\
				&\underset{\substack{\delta \rightarrow 0 \\ r \rightarrow 0 \\ R \rightarrow \infty}}{\rightarrow} -e^{\imath\lambda\pi} \int_0^{\infty} \frac{t^2}{1 + t^2} \dd t
		\end{split} \\
		I_4: t \mapsto te^{-\imath(\pi-\delta)} \\
		I_4 \underset{\substack{\delta \rightarrow 0 \\ r \rightarrow 0 \\ R \rightarrow \infty}}{\rightarrow} e^{-\imath\lambda\pi} \int_0^{\infty} \frac{t^2}{1 + t^2} \dd t \\
		\abs{I_1} \leq C R^{\lambda - 2} 2\pi R = 2\pi C R^{\lambda - 1} \underset{R \rightarrow \infty}{\rightarrow} 0 \quad (\lambda < 1) \\
		\abs{I_3} \leq \mathcal{C} r^{\lambda} 2\pi r = 2\pi \mathcal{C} r^{\lambda + 1} \underset{r \rightarrow 0}{\rightarrow} 0 \quad (\lambda > -1) \\
		\begin{split}
			I_{\gamma}
				&= I_1 + I_2 + I_3 + I_4 \\
				&\underset{\substack{\delta \rightarrow 0 \\ r \rightarrow 0 \\ R \rightarrow \infty}}{\rightarrow} 0 + -e^{\imath\lambda\pi} I + 0 + e^{-\imath\lambda\pi} I = (e^{-\imath\lambda\pi} - e^{+\imath\lambda\pi} ) I
		\end{split} \\
		I_{\gamma} = 2\pi\imath \left( \res_{z=\imath} \frac{e^{\lambda \Log z}}{1 + z^2} + \res_{z=-\imath} \frac{e^{\lambda \log z}}{1 + z^2} \right) \\
		2\pi\imath \frac{e^{\lambda \Log z}}{2\imath} + \frac{e^{\lambda \Log z}}{-2\imath} \\
		\int_0^{\infty} \frac{t^{l\lambda}}{1 + t^2} \dd t = \pi \frac{\sin\left( \frac{\pi\lambda}{2} \right)}{\sin(\pi\lambda)} = \frac{\pi}{2} \frac{1}{\cos\left(\frac{\pi\lambda}{2}\right)}
	\end{gather*}
\end{bsp*}
\todo{Too long}
\todo{Bild}

\subsection{Residuen an Polen höheren Ordnung}
$f(z)$ hat einen Pol der Ordnung $\leq n$ and der Stelle $a$, $f(z) = \frac{c_{-n}}{(z-a)^n} + \frac{c_{-n+1}}{(z-a)^{n-1}} + \dotsb + \frac{c_{-1}}{z-a} + c_0 + c_1 (z-a) + \dotsb ; 0 < \abs{z-a} < r$ $\iff$ $g(z) = (z-a)^n f(z)$ hat eine hebbare Singularität an der Stelle $a$ d.h. $g$ hat eine Taylorreihe um $a$:
\begin{gather*}
	g(z) = c_{-n} + c_{-n+1} (z-a) + \dotsb + c_{-1} (z-a)^{n-1} + \dotsb \\
	\res_{z=a} f(z) = \text{ Koeffizient von } (z-a)^{n-1} \text{ in der Taylorreihe} = \frac{g^{(n-1)}(a)}{(n-1)!}
\end{gather*}
\begin{bsp*}
	\begin{gather*}
		\begin{split}
			\int_{-\infty}^{\infty} \frac{e^{\imath t}}{(1+t^2)^2} \dd t
				&= \lim_{R \rightarrow \infty} \int_{\gamma_R} \frac{e^{\imath z}}{(z^2+1)^2} \dd z \\
				&= \lim_{R \rightarrow \infty} 2\pi\imath \res_{z=\imath} \frac{e^{\imath z}}{(z^2+1)^2}
		\end{split} \\
		\qquad z^2 + 1 = \underbrace{(z-\imath)}_{\rightarrow 0 \Leftarrow z \rightarrow \imath}(z+\imath) \\
		\qquad \frac{e^{\imath z}}{(z^2+1)^2} = \frac{1}{(z-\imath)^2} \underbrace{\left( \frac{e^{\imath z}}{(z+1)^2} \right)}_{g(z)} \\
		= \frac{1}{(z-1)^2} \left( g(\imath) + \underbrace{g'(\imath)(z-\imath)}_{=\res_{z=\imath} \frac{e^{\imath z}}{(z^2+1)^2}} + \frac{g''(\imath)}{2!}(z-a)^2 + \dotsb \right) \\
		\begin{split}
			g'(\imath)
				= \left. \left( \frac{\imath e^{\imath z}}{(z+\imath)^2} - 2 \frac{e^{\imath z}}{(z+\imath)^3} \right) \right|_{z=\imath} \\
				&= \frac{\imath}{2\imath} e^{-1} - \frac{2}{(2\imath)^3} e^{-1} \\
				&= \frac{1}{4e}(-\imath-\imath) = \frac{-\imath}{2e}
		\end{split} \\
		\int_{-\infty}^{\infty} \frac{e^{\imath t}}{(1+t^2)^2} \dd t = 2\pi\imath g'(\imath) = \frac{\pi}{e}
	\end{gather*}
\end{bsp*}
\begin{bsp*}
	Fourierreihe von $f(t) = \frac{1}{a + \sin t}$ ($2\pi$-periodisch) ; $\abs{a} > 1$
	\begin{gather*}
		f(t) = \sum_{n=-\infty}^{\infty} c_n e^{\imath t} \\
		c_n = \frac{1}{2\pi} \int_0^{2\pi} \frac{1}{a + \sin t} e^{-\imath nt} \dd t \\
		\begin{split}
			\begin{vmatrix*}[l] z = e^{\imath t} \\ \dd z = \imath z \dd t \end{vmatrix*}
				&= \frac{1}{2\pi\imath} \int_{\abs{z}=1} \frac{1}{a+ \frac{2\imath}(z-z^{-1})} z^{-n} \frac{1}{z} \dd z \\
				&= \frac{1}{2\pi\imath} \int_{\abs{z}=1} \frac{2\imath}{z^2 + 2\imath a - 1} z^{-n} \dd z
		\end{split} \\
		z^2 + 2\imath a - 1 = (z-z_+)(z-z_-) \\
		z_\pm = \imath \underbrace{( -1 \pm \sqrt{a^2 + 1} )}_{\in \R}
	\end{gather*}
	Sei z.B. $a > 0$. \\
	Pole in der Einheitskreisscheibe $z = z_+$ (und falls $n > 0$ $z = 0$). \\
	Berechne $c_n$ für $n \leq 0$ (kein Pl bei $0$), verwende $c_n = \overline{c_{-n}}$ für $n > 0$ (Siehe *)
	\begin{gather*}
		n \leq 0 \\
		c_n = \res_{z=z_+} \frac{2\imath}{(z-z_+)(z-z_-)} z^{-n} = \frac{2\imath z_+^{-n}}{z_+ - z_-} \quad (n \leq 0) \\
		= \frac{1}{2\pi} \int_0^{2\pi} \frac{e^{\imath nt}}{a + \sin t} \dd t = \begin{cases}
			\frac{\imath^{-m}}{\sqrt{a^2 - 1}} (\sqrt{a^2 - 1} - a)^{\abs{n}} & a > 1 \\
			\frac{-\imath^{-m}}{\sqrt{a^2 - 1}} (-\sqrt{a^2 - 1} - a)^{\abs{n}} & a < 1
		\end{cases}
	\end{gather*}
\end{bsp*}
