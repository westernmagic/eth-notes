\chapter{Laplace Transformationen}
$f(t)$ sei eine für $t \geq 0$ definierte Funktion.
\begin{def*}[ note = Laplace-Transformierte , index = Laplace Transformierte , indexformat = {1!~-2 2!1-~}]
	Die \textbf{Laplace-Transformierte} von $f$ ist
	\[ F(s) = \int_0^{\infty} f(t) e^{-st} \dd t \]
	$s \in \C$, so dass das Integral existiert.
\end{def*}
\begin{bsp*}
	\begin{gather*}
		f(t) = t^n \quad n = 0 , 1 , 2 , \dotsc \\
		F(s) = \int_0^{\infty} t^n e^{-st} \dd t \\
		\text{definiert für } \Re(s) > 0 \: ( e^{-st} = e^{-\Re(s) t} \underbrace{e^{-\imath\Im(s) t}}_{\abs{\cdot} = 1})
		\intertext{Für $n = 0$}
		\int_0^{\infty} e^{-st} \dd t = \left[ -\frac{1}{s} e^{-st} \right]_0^{\infty} = \frac{1}{s}
		\intertext{$n > 0$}
		\begin{split}
			\int_0^{\infty} \underbrace{t^n}_{\downarrow} \underbrace{e^{-st}}_{\uparrow} \dd t
				&= \cancel{\left[ t^n \frac{1}{-s} e^{-st} \right]_0^{\infty}} - \int_0^{\infty} nt^{n-1} \frac{1}{-s} e^{-st} \dd t \\
				&= \frac{n}{s} \int_0^{\infty} t^{n-1} e^{-st} \dd t
		\end{split}
		\intertext{Iteriere}
		\int_0^{\infty} t^n e^{-st} \dd t =  \frac{n}{s} \cdot \frac{n-1}{s} \cdot \frac{n-2}{s} \cdots \frac{1}{s} \int_0^{\infty} e^{-st} \dd t = \frac{n!}{s^{n+1}}
	\end{gather*}
\end{bsp*}
\begin{bsp*}
	\begin{gather*}
		f(t) = e^{at} \quad a \in \C \\
		F(s) = \int_0^{\infty} e^{at-st} \dd t = \int_0^{\infty} e^{(a-s)t} \dd t = \frac{1}{s-a} , \Re(s) > \Re(a)
	\end{gather*}
\end{bsp*}
Sei $f(t)$ eine vis auf Sprungstellen stetige Funktion, mit
\[ \abs{f(t)} \leq C e^{at} \quad C , a \in \R \]
Dann ist $F(s)$ für alle $s \in \C$ mit $\Re(s) > a$. $F(s)$  ist analytisch in dieser Halbebene.
\begin{bsp*}[ note = Heaviside Funktion , index = Heaviside Funktion]
	\begin{gather*}
		H(t) = \begin{cases}
			1 & t \geq 0 \\
			0 & t < 0
		\end{cases}
		\intertext{oder allgemeiner}
		H(t-a) = \begin{cases}
			1 & t \geq a \\
			0 & t < a
		\end{cases}
		\intertext{Für $a \geq 0$}
		f(t) = H(t-a) \\
		\begin{split}
			F(s)
				&= \int_0^{\infty} H(t-s) e^{-st} \dd t \\
				&= \int_a^{\infty} e^{-st} \dd t \\
				&= \left[ \frac{e^{-st}}{-s} \right]_0^{\infty} \\
				&= \frac{e^{-as}}{s}
		\end{split}
	\end{gather*}
\end{bsp*}

\section{Notationen}
\[ F = \L[f] \]
Die Laplacetransformation $\L$ bildet $f$ auf $F$ ab.
\[ f(t) \laplace F(s) \]
z.B $t^n \laplace \frac{n!}{s^{n+1}}$

\section{Inverse Laplace-Transformation}
\begin{def*}[ note = inverse Laplace-Transformierte , index = inverse Laplace Transformierte , indexformat = {2!~-3!1~ 3!2-~!1}]
	$f(t)$ heisst \textbf{inverser Laplace-Transformierte} von $F(s)$ falls $F = \L[f]$. Wir schreiben $f = \L^{-1}[F]$
\end{def*}
%\begin{satz*}[ note = Satz von Lerch , \index = Satz von Lerch , indexformat = {3!12~ 1!~23}]
%	Die Laplace-Transformation ist im wesentlichen injektiv: gilt $\L[f] = \L[g]$ für Funktionen $f,g$ stetig bis auf Sprungstellen $\abs{f(t)} \leq C_1 e^{a_1 t} , \abs{g(t)} \leq C_2 e^{a_2 t}$ dann gilt $f = g$ ausser möglicherweise an den Sprungstellen. Also ist $\L^{-1}[F]$ (bis auf Sprungstellen) wohldefiniert (wenn es existiert)
%\end{satz*}
\todo{Fix}
\begin{bem}
	Es gibt eine Formel, in den Anwendungen meistens nutzlos. Besser: Tabellen konsultation
\end{bem}
\begin{bsp*}
	\[ \L^{-1}\left[ \frac{1}{s^2} \right] = t \]
\end{bsp*}

\section{Eigenschaften}
Aus der Linearität des Integrals folgt
\begin{enumerate}[ label = (\arabic*) ]
	\item \[ \L[ \lambda f + \mu g ] = \lambda \L[f] + \mu \L[g]  \quad \lambda , \mu \in C \]
		\begin{bsp*}
			\[ \frac{2}{s} + \frac{1}{s^2} \Laplace 2 + t \]
		\end{bsp*}
	\item Verschibungssatz
		\begin{gather*}
			f \laplace F \quad g \laplace G \\
			f(t) = e^{-at} g(t) \implies F(s) = G(s + a)
		\end{gather*}
		\begin{bew}
			\[ F(s) = \int_0^{\infty} e^{-at} g(t) e^{-st} \dd t = \int_0^{\infty} g(t) e^{-(s+a)t} \dd t \]
		\end{bew}
	\item
		\begin{gather*}
			\L\left[ \frac{\dd f}{\dd t} \right] = ? \\
			\text{Sei } g(t) = \frac{\dd f(t)}{\dd t} , f \text{ differenzierbar } , \abs{f} \leq C e^{at} \\
			\begin{split}
				G(s)
					&= \int_0^{\infty} \underbrace{\frac{\dd f}{\dd t}}_{\uparrow} \underbrace{e^{-st}}_{\downarrow} \\
					&= \left[ f(t) e^{-st} \right]_0^{\infty} - \int_0^{\infty} f(t) e^{-st} (-s) \dd t \\
					= -f(0) + s F(s)
			\end{split} \\
			\L\left[ \frac{\dd f}{\dd t} \right] = s F(s) - f(0)
			\intertext{Iteriere}
			\L\left[ \frac{\dd^2 f}{\dd t^2} \right] = s \L\left[ \frac{\dd f}{\dd t} \right] - \frac{\dd f}{\dd t}(s) = s^2 F(s) - s f(0) - f'(0)
			\intertext{Mit Induktion}
			F = \L[f] \\
			\L\left[ \frac{\dd^n f}{\dd t^n} \right] = s^n F(s) - \sum_{j=0}^{n-1} s^{n-1-j} \frac{\dd^j f}{\dd t^j}(s)
		\end{gather*}
\end{enumerate}

\begin{bsp*}[ note = Erste Anwendung: Anfangswertproblem ]
	\begin{gather*}
		\left\{ \begin{matrix*}[l]
			y'(t) = a y(t) \\
			y(0) = 1
		\end{matrix*} \right. \\
		\text{Sei } Y = \L[y] \\
		\overbrace{s Y(s)}^{\L[y']} - y(0) = \overbrace{a Y(s)}^{\L[ay]} \\
		s Y(s) - 1 = a Y(s) \\ 
		(s-a) Y(s) = 1 \\
		Y(s) = \frac{1}{s-a}
		\intertext{Tabelle:}
		y(t) = e^{at}
	\end{gather*}
\end{bsp*}

\section{Weitere Beispiele von LT}
\begin{bsp}
	\begin{gather*}
		f(t) = \sin(at) \quad a \in \R \\
		\begin{split}
			F(s)
				&= \int_0^{\infty} \sin(at) e^{-st} \dd t \quad (\Re s > 0) \\
				&= \int_0^{\infty} \frac{1}{2\imath} (e^{\imath a t} - e^{-\imath a t} ) e^{-st} \dd t \\
				&= \frac{1}{2\imath} \left( \frac{1}{s - \imath a} - \frac{1}{s + \imath a} \right)  \\
				&= \frac{1}{2\imath} \frac{2\imath a}{s^2 + a^2} \\
				&= \frac{a}{s^2 + a^2}
		\end{split}
	\end{gather*}
\end{bsp}
\begin{bsp}
	\begin{gather*}
		f(t) = \cos(at) = \frac{1}{a} \frac{\dd}{\dd t} \sin(at) \quad a \neq 0 \\
		F(s) = \frac{1}{a} \left( s \cdot \frac{a}{s^2 + a^2} - \underbrace{0}_{\sin(a0)} \right) = \frac{s}{s^2 + a^2}
	\end{gather*}
\end{bsp}

\begin{enumerate}[ resume  , label = (\arabic*) ]
	\item Skalartransformation \\
		Sei $a > 0$.
		\begin{gather*}
			g(t) = f(at) \\
			G(t) = \int_0^{\infty} f(at) e^{-st} \dd t = \int_0^{\infty} f(\tilde{t}) e^{\frac{-s}{a}\tilde{t}} \frac{1}{a} \dd \tilde{t} \\
			\tilde{t} = at , t = \frac{\tilde{t}}{a} , \dd \tilde{t} = a \dd t \\
			=\frac{1}{a} F\left( \frac{s}{a} \right)
		\end{gather*}
	\item Falltungsgesetz
		\begin{gather*}
			f , g : [ 0 , \infty ) \rightarrow \C \\
			\text{Faltung: } (f * g)(t) = \int_0^t f(t-t') g(t') \dd t'
		\end{gather*}
		\begin{satz*}[note = Flatungssatz , index = Faltungs satz , indexformat = {1!~-2 2!1-~}]
			\[ \L[f*g](s) = F(s) G(s) \]
			\begin{bew}
				\[ \begin{split}
					\L[f*g](s)
						&= \int_0^{\infty} \int_0^t f(t-t') g(t') \dd t' e^{-st} \dd t \\
						&= \int_0^{\infty} \int_0^t f(t-t') g(t') e^{-st} \dd t' \dd t \\
						&= \int_0^{\infty} \left( \underbrace{\int_{t'}^{\infty} f(t-t') g(t') e^{-st}}_{0 \leq u = t-t' < \infty} \dd t \right) \dd t' \\
						&= \int_0^{\infty} \left( \int_0^{\infty} f(u) g(t') e^{-s(u+t')} \dd u \right) \dd t' \\
						&= \int_0^{\infty} \underbrace{\left( \int_0^{\infty} f(u) e^{-su} \dd u \right)}_{F(s)} g(t') e^{-st'} \dd t' \\
						&= F(s) G(s)
				\end{split} \]
			\end{bew}
		\end{satz*}
	\item Zweite Ableitungsregel
		\[ f(t) = t^k g(t) \implies F(s) = (-1)^k \frac{\dd^k}{\dd s^k} G(s) \]
		\begin{bew}
			\begin{gather*}
				G(s) = \int_0^{\infty} g(t) e^{-st} \dd t \quad \Re(s) >  \\
				\frac{\dd^k}{\dd s^k} G(s) = \int_0^{\infty} \underbrace{g(t) (-t)^{k}}_{(-1)^k f(t)} e^{-st} \dd t
			\end{gather*}
		\end{bew}
	\item Zweiter Verschiebungssatz
		\begin{gather*}
			a > 0 \\
			\L^{-1}[ e^{-sa} F(s) ] = H(t-a) f(t-a) = \begin{cases}
				f(t-a) & t \geq a \\
				0 & t < a
			\end{cases}
		\end{gather*}
		\begin{bew}
			\[ \begin{split}
				\int_0^{\infty} H(t-a) f(t-a) e^{-st} \dd t
					&= \int_a^{\infty} f(t-a) e^{-st} \dd t \\
					&\underset{u=t-a}{=} \int_0^{\infty} f(u) e^{-s(u+a)} \dd u \\
					&= e^{-sa} F(s)
			\end{split} \]
		\end{bew}
\end{enumerate}
\begin{bsp*}
	Löse
	\begin{gather*}
		\ddot{x}(t) - x(t) = e^t \\
		x(0) = 1 \\
		\dot{x}(0) = 1
		\intertext{Sei $X = \L[x]$. Ableitungsregel}
		\underbrace{s^2 X(s) - s - 1 - X(s)}_{\L[\ddot{x}]} = \frac{1}{s-1} \\
		X(s) = \frac{1}{s^2 - 1} \left( s + 1 + \frac{1}{s-1} \right) = \frac{1}{s-1} + \frac{1}{(s-1)^2 (s+1)} \\
		\frac{1}{(s-1)^2 (s+1)} \overset{\text{\scriptsize{PBZ}}}{=} \frac{A}{(s-1)^2} + \frac{B}{s-1} + \frac{C}{s+1}
	\end{gather*}
\end{bsp*}
\url{http://www.math.ethz.ch/u/felder/Teaching/PDG}

\begin{bsp*}[note = Kap. 7]
	\begin{gather*}
		\left\{ \begin{matrix*}[l]
			\frac{\dd x(t)}{\dd t} + 2x(t) = t \\
			x(0) = a
		\end{matrix*} \right. * \\
		X(s) = \L[x(0)] = \int_0^{\infty} x(t) e^{-st} \dd t
		\intertext{$*$ als Gleichung für $X(s)$}
		s X(s) - \underbrace{x(0)}_{=a} + 2 X(s) = \frac{1}{s^2} \\
		(s + 2) X(s) = \frac{1}{s^2} + a \\
		X(s) = \underbrace{\frac{1}{s+2}}_{\text{Übertragungsfunktion}} \left( \underbrace{\frac{1}{s^2} + a}_{\text{input}} \right)
		\intertext{Gesucht die inverse Laplacetransformierte (''Originalfunktion''). Partialbruchzerlegung}
		\begin{split}
			\frac{1}{s+2} \frac{1}{s^2}
				&= \frac{A}{s+2} + \frac{B}{s} + \frac{C}{s^2} \\
				&= \frac{As^2 + B(s+2)s + C(s+2)}{(s+2)s^2} \\
				&= \frac{(A+B)s^2 + (2B + C) + 2C}{(s+2)s^2}
		\end{split} \\
		A + B = 0 \\
		2B + C = 0 \\
		2C = 1 \\
		C = \frac{1}{2} , B = -\frac{1}{4} , A = \frac{1}{4} \\
		\begin{split}
			X(s)
				&= \frac{1}{4(s+2)} - \frac{1}{4s} + \frac{1}{2s^2} + \frac{a}{s+2} \\
				&= \left( a + \frac{1}{4} \right) \frac{1}{s+2} - \frac{1}{4s} + \frac{1}{2s^2}
		\end{split} \\
		x(t) = \left( a + \frac{1}{4} \right) e^{-2t} - \frac{1}{4} + \frac{t}{2}
	\end{gather*}
\end{bsp*}
\begin{bsp*}
	\begin{gather*}
		\ddot{x}(t) - x(t) = e^t \\
		x(0) = \dot{x}(0) = 1 \\
		s^2 X(s) - s \underbrace{x(0)}{1} - \underbrace{\dot{x}(0)}_{1} - X(s) = \frac{1}{s-1} \\
		(s^2 - 1) X(s) = \frac{1}{s-1} + s + 1 \\
		X(s) = \frac{1}{s^2 - 1} \left( \frac{1}{s - 1} + s + 1 \right) = \frac{1}{(s-1)^2 (s+1)} + \frac{1}{s+1} \\
		\begin{split}
			\frac{1}{(s-1)^2 (s+1)}
				&= \frac{A}{s-1} + \frac{B}{(s-1)^2} + \frac{C}{s+1} \\
				&= \frac{A(s-1)(s+1) + B(s+1) + C(s-1)^2}{(s-1)^2 (s+1)} \\
				&= \frac{(A+C)s^2 + (B-2C)s - A + B + C}{(s-1)^2 (s+1)}
		\end{split} \\
		A + C = 0 \\
		B - 2C = 0 \\
		-A + B + C = 1  \\
		A = -C \\
		B = 2C \\ 
		C + 2C + C = 1 \\
		C = \frac{1}{4} , B = \frac{1}{2} , A = -\frac{1}{4} \\
		\begin{split}
			X(s)
				&= -\frac{1}{4(s-1)} + \frac{1}{2(s-1)^2} + \frac{1}{4(s+1)} + \frac{1}{s+1} \\
				&= \frac{3}{4(s+1)} + \frac{1}{2(s-1)^2} + \frac{1}{4(s+1)}
		\end{split} \\
		x(t) = \frac{3}{4} e^t + \frac{1}{2} t e^t + \frac{1}{4} e^{-t}
	\end{gather*}
\end{bsp*}
\begin{bsp*}
	\begin{gather*}
		\ddot{x}(t) - x(t) = h(t) \\
		x(0) = a \\
		\dot{x}(0) = b \\
		H = \L[h] \\
		(s^2 - 1) X(s) = H(s) + sa + b \\
		X(s) = \frac{1}{s^2 - 1} ( H(s) + sa + b )
	\end{gather*}
\end{bsp*}
\begin{bsp*}[note = Anwendung: Gedämpfter harmonischer Oszillator mit Störkraft]
	\begin{gather*}
		\ddot{x}(t) + \omega^2 x(t) + 2k \dot{x}(t) = \underbrace{g(t)}_{\text{Störkraft}} \\
		x(0) = x_0 \\
		\dot{x}(0) = v_0 \\
		X(s) = \L[x(t)] \\
		G(s) = \L[g(\omega)] \\
		s^2 X(s) + 2ks X(s) + \omega^2 X(s) = G(s) + s x_0 + v_0 + 2k x_0 \\
		X(s) = \frac{1}{s^2 + 2ks + \omega^2} ( G(s) + s x_0 + v_0 + 2k x_0 ) \\
		x(t) = \int_0^t h(t-t') g(t') \dd t' \\
		h = \L^{-1}\left[ \frac{1}{s^2 + 2ks + \omega^2} \right]
		\intertext{$h$ heisst Einflussfunktion. $h(t-t')$ beschreibt den Einfluss auf die Lösung zur Zeit $t$ der Störkraft zur Zeit $t'$. Berechnung von $h$:}
		\frac{1}{s^2 + 2ks + \omega^2} = \frac{1}{(s+k)^2 + \omega^2 - k^2}
	\end{gather*}
	\begin{enumerate}[label = \alph*)]
		\item $k < \omega$ ''Schwache Dämpfung''
			\begin{gather*}
				\overline{\omega} \coloneqq \sqrt{\omega^2 - k^2} > 0 \\
				\sin at \laplace \frac{a}{s^2 + a^2} \\
				h(t) = \L^{-1}\left[ \frac{1}{(s+k)^2 + \overline{\omega}^2} \right] = e^{-kt} \frac{\sin \overline{\omega}t}{\overline{\omega}}
			\end{gather*}
		\item $k > \omega$ ''Starke Dämpfung''
			\begin{gather*}
				a = \sqrt{k^2 - \omega^2} > 0 \\
				\frac{1}{(s+k)^2 - a^2} = \left( \frac{1}{s + k - a} - \frac{1}{s + k + a} \right) \frac{1}{2a} \\
				h(t) = \frac{1}{2a} ( e^{-(k-a)t} - e^{-(k+a)t} ) \\
				h(t) = \frac{1}{2 \sqrt{k^2 - \omega^2}} ( e^{\overbrace{-(k - \sqrt{k^2 - \omega^2}}^{>0}) t} - e^{\overbrace{-(k + \sqrt{k^2 - \omega^2}) t}^{>0}} )
			\end{gather*}
		\item $\omega = k$ ''kritische Dämpfung''
			\[ h(t) = \L^{-1}\left[ \frac{1}{(s+k)^2} \right] = t e^{-kt} \]
	\end{enumerate}
\end{bsp*}

\subsection{Dirac \texorpdfstring{$\delta$}{delta}-Funktion}
Dirac: $\delta(t-a)$ soll eine ''Funktion'' sein, die $0$ für alle $t \neq a$, $\infty$ für $t=a$, so dass $\int_{-\infty}^{\infty} \delta(t-a) \dd t = 1$. $\delta(t-a)$ ist nicht definiert, doch aber
\[ \int_{-\infty}^{\infty} \delta(t-a) \phi(t) \dd t = \phi(a) \]
für alle Funktionen (in $\C^{\infty}$) $\phi(t)$

$\delta$-Funktion als Limes:
\begin{gather*}
	\delta_{\epsilon}(t-a) = \begin{cases} \frac{1}{\epsilon} &-\frac{\epsilon}{2} \leq t-a \leq \frac{\epsilon}{2} \\ 0 &\text{sonst} \end{cases} \quad (\epsilon > 0) \\
	\int_{-\infty}^{\infty} \delta_{\epsilon}(t-a) \dd t = 1 \\
	\delta(t-a) \coloneqq \lim_{\epsilon \rightarrow 0^{+}} \delta_{\epsilon}(t-a)
	\intertext{''im Sinne der verallgemeinerten Funktionen'' bedeutet}
	\int_{-\infty}^{\infty} \delta(t-a) \phi(t) \dd t = \lim_{\epsilon \rightarrow 0^{+}} \int_{-\infty}^{\infty} \delta_{\epsilon}(t-a) \phi(t) \dd t
\end{gather*}
Reaktion auf einem ''$\delta$-Stoss'' zur Zeit $t_0$
\begin{gather*}
	\left\{ \begin{matrix*}[l]
		\ddot{x}(t) + 2k\dot{x}(t) + \omega^2 x(t) = \delta(t-t_0) \quad (*) \quad (\text{ oder zunächst } \delta_{\epsilon}(t-a)) \\
		x(0) = \dot{x}(0) = 0
	\end{matrix*} \right. \\
	x(t) = \int_0^t h(t-t') \delta(t'-t_0) \dd t' = h(t-t_0)
	\intertext{$h(t-t_0)$ ist die Lösung von $*$}
\end{gather*}

\subsection{Laplacetransformierte von periodischen Funktionen}
$f(t+T) = f(t)$ für alle $t > 0$. \\
$T > 0$ ist die Periode.
\[ \begin{split}
	F(s)
		&= \int_0^{\infty} f(t) e^{-st} \dd t \\
		&= \underbrace{\int_0^T f(t) e^{-st} \dd t}_{\substack{\int_0^T f(t+T) e^{-st} \dd t \\\overset{t+T \rightarrow t}{=} \int_0^T f(t) e^{-st} \dd t (e^{-sT})}} + \int_T^{2T} f(t) e^{-st} \dd t + \int_{2T}^{3T} f(t) e^{-st} \dd t + \dotsb \\
		&= \int_0^T f(t) e^{-st} \dd t ( 1 + e^{-sT} + e^{-s \cdot 2T} + e^{-s \cdot 3T} + \dotsb ) \\
		&= \int_0^T f(t) e^{-st} \dd t \cdot \frac{1}{1 - e^{-sT}} 
\end{split} \]
\begin{bsp*}
	\begin{gather*}
		f(t) = \begin{cases} A &0 \leq t < h \\ 0 &h \leq t < T \end{cases} \quad \text{$T$-periodisch fortgesetzt} \\
		A > 0 , h > 0 , T > 0 \\
		\begin{split}
			F(s)
				&= \int_0^T f(t) e^{-st} \dd t \frac{1}{1 - e^{-sT}} \\
				&= A \int_0^h e^{-st} \dd t \frac{1}{1 - e^{-stT}} \\
				&= A \left[ \frac{e^{-st}}{-s} \right]_0^h \frac{1}{1 - e^{-sT}} \\
				&= \frac{A}{s} \frac{1 - e^{-sh}}{1 - e^{-sT}}
		\end{split}
	\end{gather*}
\end{bsp*}

\subsection{Laplace-Transformierte der \texorpdfstring{$\delta$}{delta}-Funktion}
\begin{gather*}
	a > 0 \\
	\L[ \delta(t-a) ] = \underbrace{\int_0^{\infty} \delta(t-a) e^{-st} \dd t}_{=\im_{\epsilon \rightarrow 0} \int_0^{\infty} \delta_{\epsilon}(t-a) e^{-st} \dd t} = e^{-sa}
	\intertext{Für $a=0$ muss das als Limes}
	\lim_{a \rightarrow 0^{+}} \L[\delta(t-a)] = 1
	\intertext{verstanden werden.}
	\left( \lim_{\epsilon \rightarrow 0} \int_0^{\infty} \delta_{\epsilon}(t) e^{-st} \dd t = \frac{1}{2} \right)
\end{gather*}

\subsection{Laplace vs. Fourier; Formel für die Inverse LT}
$f \in L^1$ d.h. $\int_{-\infty}^{\infty} \abs{f(t)} \dd t < \infty$ \\
$L^1 = L^1(\R)$ besteht aus alle Funktionen $f: \R \rightarrow \C$ die integrierbar sind (d.h. $\int_{-\infty}^{\infty} \abs{f(t)} \dd t < \infty$; Funktionen, die sich auf Mengen der Länge $0$ unterscheiben, werden als gleich betrachtet.
\begin{bsp*}
	\[ f(t) = \begin{cases} 1 &t > 0 \\ 0 &t \leq 0 \end{cases} \text{ ist ''gleich'' der Funktion } \tilde{f}(t) = \begin{cases} 1 &t \geq 0 \\ 0 &t < 0 \end{cases} \]
\end{bsp*}
$\tilde{f}(\omega) = \int_{-\infty}^{\infty} f(t) e^{-\imath\omega t} \dd t$ stetige Funktion von $\omega \tilde{f}(\omega) \rightarrow 0 , \omega \rightarrow \infty$ \\
Falls $\tilde{f} \in L^1$ dann gilt die Fourrier-Umkehrformel
\[ f(t) = \frac{1}{2\pi} \int_{-\infty}^{\infty} \tilde{f}(\omega) e^{\imath\omega t} \dd \omega \]
Falls $f(t) = 0$ für alle $t < a$ dann ist
\begin{gather*}
	\tilde{f}(z) = \int_{-\infty}^{\infty} f(t) e^{-\imath zt} \dd t \text{ für alle } z \in H^{-} = \{ z \in \C | \Im z < 0 \}
	\intertext{da}
	\tilde{f}(\omega - \imath \eta) = \int_a^{\infty} f(t) e^{-\imath\omega t} \underbrace{e^{-\eta t}}_{\leq \underbrace{e^{-\eta a}}_{<1} \text{ für } t > 0} \dd t \quad \omega \in \R , \eta \in \R_{>0}
\end{gather*}
Die komplexe Ableitung existiert. \\
$\tilde{f}(\omega)$ ist der Randwert einer auf $H^{-}$ definierten analytischen Funktion $\tilde{f}(z)$

Fourier Satz:
\begin{gather*}
	\hat{f}(\omega) = \int_{-\infty}^{\infty} f(t) e^{-\imath\omega t} \dd t \\
	f(t) = \frac{1}{2\pi} \int_{-\infty}^{\infty} \hat{f}(\omega) e^{\imath\omega t} \dd \omega \\
	( f , \hat{f} \in L^1)
	\intertext{Falls $f(t) = 0$ und in $L^2$ für $t < 0$}
	\hat{f}(z) = \hat{f}(\omega - \imath\eta) = \int_0^{\infty} f(t) e^{-(\omega - \imath\eta)t} \dd t
	\intertext{ist auch definiert und analytisch für}
	\eta > 0 \quad (z \in H_{-}) \\
	F(s) = \hat{f}(-\imath s) = \int_0^{\infty} f(t) e^{-st} \dd t = \L[f] \quad s > 0 \quad (\Re s > 0)
	\intertext{Wie kann man $f(t)$ aus $F(s)$ gewinnen?}
	\begin{split}
		f(t)
			&= \frac{1}{2\pi} \int_{-\infty}^{\infty} \hat{f}(\omega) e^{\imath\omega t} \dd \omega \\
			&= \frac{1}{2\pi} \int_{-\infty}^{\infty} F(\imath\omega) e^{\imath\omega t} \dd \omega \\
			&= \frac{1}{2\pi\imath} \int_{\gamma} F(s) e^{st} \dd s
	\end{split} \\
	\gamma : \omega \mapsto \imath\omega \qquad \omega \in \R \\
	\text{(falls $\omega \mapsto F(\imath\omega)$ integrierbar ist)}
\end{gather*}
Diese Formel gilt falls $f \in L^1$ und $\omega \mapsto F(\imath\omega)$ ebenfalls. \\
Allgemein:
\begin{satz*}
	Sei $f$ stückweise stetig mit Sprungstellen als Unstetigkeiten und es gelte $\abs{f(t)} \leq c e^{at} , t \geq 0 , a \in \R$. Dann ist F(s) analytisch für $\Re s \geq a$ und es gilt
	\[ f(t) = \frac{1}{2\pi\imath} \int_{\gamma} F(s) e^{st} \dd s \]
	wobei $\gamma$ die Gerade $t \mapsto b + \imath t$ für beliebige $b > a$ ist.
\end{satz*}
\begin{bsp*}
	\[ F(s) = \frac{1}{s^2 + a^2} \qquad a \in \C \setminus \{ 0 \} \]
	Was ist $f(t) = \L^{-1}[F(s)]$? \\
	$F(s)$ hat Pole an den Stellen $\pm \imath a$
	\begin{gather*}
		f(t) = \frac{1}{2\pi\imath} \int_{\gamma} F(s) e^{st} \dd s \quad t \geq 0 \\
		\int_{\tilde{\gamma}} \frac{1}{s^2 + a^2} e^{st} \dd s \\
		\abs{e^{st}} = e^{\frac{1}{2} \Re(s) t} \leq e^{bt} \\
		\Re(s) \leq b \text{ für } s \text{ auf } \tilde{\gamma} \\
		\abs{\frac{1}{s^2 + a^2} e^{st}} \leq \frac{c}{R^2} e^{bt} \underset{R \rightarrow \infty}{\rightarrow} 0 \\
		\begin{split}
			f(t)
				&= \lim_{R \rightarrow \infty} \frac{1}{2\pi\imath} \int \frac{1}{s^2 + a^2} e^{st} \dd s \\
				&= \res_{s = \imath a} \frac{1}{s^2 + a^2} e^{st} + \res_{s = -\imath a} \frac{1}{s^2 + a^2} e^{st} \\
				&= \frac{1}{2\imath a} e^{\imath at} - \frac{1}{2\imath a} e^{-\imath at}
		\end{split}
	\end{gather*}
\end{bsp*}

