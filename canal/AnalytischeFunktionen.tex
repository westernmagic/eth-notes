\chapter{Analytische Funktionen}
Komplexe Ableitung einer Funktion $f: \C \rightarrow \C$ (oder $\Omega \rightarrow \C$, $\Omega \subset \C$ offen) \\
\begin{def*}[note = komplexe Ableitung , index = komplexe Ableitung , indexformat = {2!~1}]
	Die \textbf{(komplexe) Ableitung} der Funktion $f$ in $z \in \Omega$ ist der Grenzwert
	\[ f'(z) = \lim_{h \rightarrow 0} \frac{f(z+h) - f(z)}{h} \]
	falls es existiert. D.h.
	\begin{gather*}
		\forall \epsilon > 0 \exists \delta > 0 : \abs{\frac{f(z+h) - f(z)}{h} - f'(z)} < \epsilon \\
		\text{ für alle $h$ mit $\abs{h} < \delta$}
	\end{gather*}
\end{def*}
\begin{bsp}
	\begin{gather*}
		f(z) = z^2 \\
		\begin{split}
			f'(z)
				&= \lim_{h \rightarrow 0} \frac{(z+h)^2 - z^2}{h} \\
				&= \lim_{h \rightarrow 0} \frac{z^2 + 2hz + h^2 - z^2}{h} \\
				&= \lim_{h \rightarrow 0} (2z + h) \\
				&= 2z
		\end{split}
	\end{gather*}
\end{bsp}
\begin{bsp}
	\begin{gather*}
		f(z) = \overline{z} \\
		\lim_{h \rightarrow 0} \frac{\overline{z+h} - \overline{z}}{h} = \lim_{h \rightarrow 0} \frac{\overline{h}}{h} \text{ existiert nicht!} \\
		\text{Wenn $h$ reell ist } \frac{\overline{h}}{h} = \frac{h}{h} = 1 \\
		\text{Wenn $h=\imath k$ imaginär ist ($k \in \R$) } \frac{\overline{h}}{h} = \frac{-\imath k}{\imath k} = -1
	\end{gather*}
	Ähnlich: $f(z) = \Re(z) , \Im(z)$ sind micht komplex differenzierbar.
\end{bsp}
\begin{bsp}
	\begin{gather*}
		f(z) = z^n \\
		\begin{split}
			f'(z)	&= \lim_{h \rightarrow 0} \frac{(z+h)^n - z^n}{h} \\
				&= \lim_{h \rightarrow 0} \frac{z^n + nhz^{n-1} + \dotsb - z^n}{h} \\
				&= \lim_{h \rightarrow 0} (n z^{n-1} + \dotsb ) \\
				&= nz^{n-1}
		\end{split}
	\end{gather*}
\end{bsp}
\begin{def*}[note = Analytische Funktion , index = analytische Funktion , indexformat = {2!~1}]
	$f: \Omega \rightarrow \C , \Omega \subset \C$ offen, heisst \textbf{analytisch} falls $f'(z)$ für alle $z \in \Omega$ existiert [und $f'$ stetig ist]\footnote{Der Satz von Goursat besagt, dass dies überflüssig ist, d.h. $f'(z)$ existiert für alle $z \in \Omega \implies f'$ stetig}
\
\end{def*}
Regeln (Beweise wie im Reellen) \\
\begin{enumerate}
	\item Linearität: $f,g$ analytisch auf $\Omega , c_1 , c_2 \in \C \implies c_1 f + c_2 g$ analytisch und es gilt $(c_1 f + c_2 g)' = c_1 f' + c_2 g'$. $\implies$ Polynome $f(z) = a_0 + a_1 z + a_2 z^2 + \dotsb + a_n z^n$ mit komplexen Koeffizienten $a_i$ sind analytisch auf ganz $\C$ und $f'(z) = a_1 + 2a_2 z + \dotsb + n a_{n-1} z^{n-1}$
	\item Produktregel: $f,g$ analytisch auf $\Omega \implies f \cdot g$ anaytisch und $(fg)' = f'g + fg'$
	\item Quotientregel: $f,g$ analytisch auf $\Omega \implies \frac{f}{g}$ analytisch auf $\Omega \setminus \{$ Nullstellen von $g \}$ und $\left( \frac{f}{g} \right)' = \frac{f'g - fg'}{g^2}$
	\item Kettenregel: $f \circ g (z) = f(g(z))$ ist analytisch wo definiert und es gilt $(f \circ g)' = f'(g(z) \cdot g'(z)$
	\item Rationale Funktionen: $f(z) = \frac{P(z)}{Q(z)} ; P , Q$ Polynome. $f$ ist analytisch auf $\C \setminus \{$ Nullstellen von $Q \}$ \\
		\begin{bsp*}
			\begin{gather*}
				f(z) = \frac{1}{z-a} \\
				f'(z) = \frac{1}{(z-a)^2}
			\end{gather*}
		\end{bsp*}
\end{enumerate}

\section{Konvergente Potenzreihen}
Potenzreihe um $a \in \C$
\[ \begin{split}
	f(z)	&= \sum_{n = 0}^\infty a_n (z-a)^n \quad a_n \in \C \\
		&= a_0 + a_1 (z-a) + a_2 (z-a)^2 + \dotsb
\end{split} \]
Aus der Analysis: \\
Es gibt einen ''Konvergenzradius'' $0 \leq R \leq \infty$ so, dass die Riehe konvergiert absolut für $\abs{z-a} \leq R$ und divergiert für $\abs{z-a} > R$ \\
\begin{bsp*}
	\[ 1 + z + z^2 + z^3 + \dotsb = \frac{1}{1 - z} , \abs{z} < 1 = R \]
\end{bsp*}
\[ \frac{1}{R} = \limsup_{n \rightarrow \infty} \sqrt[n]{\abs{a_n}} \]
\begin{satz*}
	Konvergente Potenzreihen (d.h. Potenzreihen mit Konvergenzradius $R > 0$) definieren analytische Funktionen
	\[ f(z) = \sum_{n = 0}^\infty a_n (z-a)^n \]
	auf $\Omega =$ Konvergenzkreisscheibe $= \{ z \in \C : \abs{z-a} < R \}$ und es gilt
	\[ f'(z) = \sum_{n = 0}^\infty n a_n (z-a)^{n-1} \]
\end{satz*}
\begin{bsp}
	Exponentialfunktion $\lambda \in \C$
	\begin{gather*}
		f(z) = e^{\lambda z} = \sum_{n = 0}^\infty \frac{(\lambda z)^n}{n!} = \sum_{n = 0}^\infty \frac{\lambda^n}{n!} (z-0)^n , R = \infty , \Omega = \C \\
		\begin{split}
			f'(z)	&= \sum_{n=0}^\infty \frac{\lambda^n}{n!} n z^{n-1} \\
				&= \sum_{n=1}^\infty \frac{\lambda^n}{(n-1)!} n z^{n-1} \\
				&= \sum_{m=0}^\infty \frac{\lambda^{m+1}}{m!} n z^m \\
				&= \lambda e^{\lambda z}
		\end{split}
	\end{gather*}
\end{bsp}
\begin{gather*}
	\begin{split}
		\sin(z)	&\underset{\text{Def.}}{=} \frac{e^{\imath z} - e^{-\imath z}}{2\imath} \quad (z \in \C) \\
				&= z - \frac{z^3}{3!} + \frac{z^5}{5!} - \frac{z^7}{7!} + \dotsb
	\end{split} \\
	\begin{split}
		\sin(z)	&\underset{\text{Def.}}{=} \frac{e^{\imath z} + e^{-\imath z}}{2} \quad (z \in \C) \\
				&= 1 - \frac{z^2}{2!} + \frac{z^4}{4!} - \frac{z^6}{6!} + \dotsb
	\end{split} \\
	\sin'(z) = (z)' - \left( \frac{z^3}{3!} \right)' + \left( \frac{z^5}{5!} \right)' - \dotsb = 1 - \frac{z^2}{2!} + \frac{z^4}{4!} -\dotsb = \cos(z) \\
	\cos'(z) = -\sin(z)
\end{gather*}

\section{Cauchy-Riemann Differenzialgleichungen}
Eine Funktion $\C \supset \Omega \rightarrow \C$ kann aufgefasst werden als eine Abbildung
\begin{gather*}
	\R^2 \supset \Omega \rightarrow \R^2 , (x,y) \mapsto (u(x,y) , v(x,y)) \\
	f(x+\imath y) = u(x,y) + \imath v(x,y) ; x , y \in \R
\end{gather*}
\begin{bsp*}
	\begin{gather*}
		f(z) = e^z \\
		f(x + \imath y) = e^{x + \imath y} = e^x \cos y + \imath e^z \sin y \\
		u(x,y) = e^x \cos y \\
		v(x,y) e^x \sin y
	\end{gather*}
\end{bsp*}

\subsection{Partielle Ableitung einer reellen oder komplexwertigen Funktion von zwei reellen Variablen}
\begin{gather*}
	u_x = \frac{\partial u}{\partial x} = \lim_{\R \ni \epsilon \rightarrow 0} \frac{u(x + \epsilon , y) - u( x , y )}{\epsilon} \\
	u_x = \frac{\partial u}{\partial y} = \lim_{\R \ni \epsilon \rightarrow 0} \frac{u(x , y + \epsilon) - u( x , y )}{\epsilon}
\end{gather*}
Beziehung zwischen $f'(z)$ und $u_x , u_y , v_x , v_y$. $f$ sei analytisch.
\begin{gather*}
	\begin{split}
		f'(z)	&= \lim_{\C \ni h \rightarrow 0} \frac{f(z + h) - f(z)}{h} \\
			&= \lim_{\R \ni \epsilon \rightarrow 0} \frac{f(z + \epsilon) - f(z)}{\epsilon} \\
			&= \lim_{\R \ni \epsilon \rightarrow 0} \frac{f(z + \imath\epsilon) - f(z)}{\imath\epsilon} \\
			&\underset{\text{einerseits}}{=} \begin{split}
				\lim_{\R \ni \epsilon \rightarrow 0} \frac{f(x + \epsilon + \imath y) - f(x + \imath y)}{\epsilon}	&= f_x(x + \imath y) \\
																					&= u_x(x,y) + \imath v_x(x,y)
			\end{split} \\
			&\underset{\text{anderseits}}{=} \begin{split}
				\lim_{\R \ni \epsilon \rightarrow 0} \frac{f(x + \imath(y + \epsilon)) - f(x + \imath y)}{\imath\epsilon}
					&= \frac{1}{\imath} f_y(x + \imath y) \\
					&= \frac{1}{\imath} (u_y + \imath v_y) \\
					&= v_y - \imath u_y
			\end{split}
	\end{split} \\
	f \text{ analytisch } \implies u_x = v_y , u_y = -v_x
\end{gather*}
\begin{satz*}
	$f(x + \imath y) = u(x,y) + \imath v(x,y)$ ist genau analytisch für $(x,y) \in \Omega \subset \R^2$ wenn $u_x , u_y , v_x , v_y$ definiert und stetig auf $\Omega$ sind und die Cauchy-Riemann Differentialgleichungen $u_x = v_y , u_y = -v_x$ auf $\Omega$ erfült sind.
\end{satz*}
\begin{bsp*}
	\begin{gather*}
		f(x + \imath y) = (x + \imath y)^2 = \underbrace{x^2 - y^2}_{u} + \underbrace{2\imath xy}_{v} \\
		u_x = 2x = v_y \text{ und } u_y = -2y = -v_x
	\end{gather*}
\end{bsp*}
\begin{bem}
	analytisch = holomorph
\end{bem}
\begin{bew}
	Noch zu zeigen: Aus den CR-Gleichungen von $u,v$ folgt Analytizität von $f$. \\
	Erinnerung aus der Analysis: \\
	$F: \R^2 \rightarrow \R^2$ stetig differenzierbar
	\begin{gather*}
		\begin{multlined}
			 F( x + \Delta x , y + \Delta y) = \\
			 F(x,y) + \underbrace{F'(x,y)}_{\begin{pmatrix} u_x & u_y \\ v_x & v_y \end{pmatrix}}\begin{pmatrix} \Delta x \\ \Delta y \end{pmatrix} + o\left( \sqrt{\Delta x^2 + \Delta y^2} \right) \\
			 (\Delta x , \Delta y) \rightarrow 0
		\end{multlined} \\
		 F(x,y) = \begin{pmatrix} u(x,y) \\ v(x,y) \end{pmatrix}
		\intertext{Falls $u,v$ die Cauchy-Riemann-Gleichungen erfüllen}
		F' = \begin{pmatrix} u_x & -v_x \\ v_x & u_x \end{pmatrix} \\
		F'(x,y)\begin{pmatrix} h_1 \\ h_2 \end{pmatrix} = \begin{pmatrix} v_x h_1 - v_x h_2 \\ v_x h_1 + u_x h_2 \end{pmatrix} \\
		\begin{split}
			f(z+h)	&= f(z) + (u_x h_1 - v_x h_2) + \imath (v_x h_1 + u_x h_2) + o( \abs{h} ) \\
					&= f(z) + \underbrace{(u_x + \imath v_x)}_{f_x(z)}\underbrace{(h_1 + \imath h_2)}_{h} + o(\abs{h})
		\end{split} \\
		\frac{f(z+h) - f(z)}{h} = f_x(z) + \frac{o(\abs{h})}{h}
	\end{gather*}
	Die komplexe Ableitung existiert und $f'(z) = f_x(z)$ $\blacksquare$
\end{bew}
\begin{bem}
	CR Gliechungen $\iff f_x - \imath f_y = 0$. Falls $f$ analytisch ist gilt also $f'(z) = f_x(z) = \imath f_y(z)$
\end{bem}

\subsection{Der Logarithmus}
Erinnerung: $\exp : \R \rightarrow \R_{>0} = ( 0 , \infty )$ invertierbar: es existiert eine inedeutige Funktion $\log : \R_{>0} \rightarrow \R$ (auch $\ln$) so, dass
\begin{gather*}
	e^{\log x} = x \forall x > 0 \\
	\log e^x = x \forall x \in \R
\end{gather*}
$\log z$ für $z \in \C$? \\
\[ e^{\log z} = z \]
$z$ sei gegeben. Suche $\log z = a + \imath b$, $a,b \in \R$
\begin{gather*}
	e^a e^{\imath b} = z \\
	\abs{e^a} = \abs{ z} \\
	e^{\imath b} = \frac{z}{\abs{z}}
\end{gather*}
\begin{enumerate}[label=(\alph*)]
	\item keine Lösung wenn $z=0$ ; $\log 0$ nicht definiert
	\item $z \neq 0$: $a = \log \abs{z}$. $b$ ist nicht eindeutig bestimmt
		\begin{gather*}
			z = r e^{\imath \varphi} \\
			\begin{split}
				b	&= \varphi + 2\pi k \quad k in \Z \\
					&= \arg z + 2\pi k
			\end{split} \\
			\text{Konvention: } -\pi \leq \arg z < \pi \\
			\text{Lösungen: } \log z = \log \abs{z} + \imath \arg z + 2\pi\imath k \quad k \in \Z
		\end{gather*}
\end{enumerate}
\begin{def*}[note = Hauptwert des Logarithmus , index = Hauptwert des Logarithmus , indexformat = {3!12~}]
	\begin{gather*}
		\Log : \C \setminus \R_{\leq 0} \rightarrow \C \\
		-\pi < \arg z < \pi \\
		z \mapsto \Log z = \log z + \imath \arg z
	\end{gather*}
\end{def*}
%\begin{beh}
%	$\Log$ ist eine analytische Funktion. \\
%	\begin{bew}
%		\begin{gather*}
%			\Log( re^{\imath\varphi} ) = \underbrace{\log r}_{u(x,y)} + \imath\underbrace{\varphi}_{v(x,y)} \\
%			r = \sqrt{x^2+y^2} \\
%			u(x,y) = \log \sqrt{x^2+y^2} \\
%			\tan v(x,y) = \frac{y}{x} \\
%			\cot v(x,y) = \frac{x}{y}
%			\intertext{Cauchy-RiemannGleichungen verifizieren:}
%			u_x = \frac{\partial}{\partial x} \log \sqrt{x^2+y^2} = \frac{\partial}{\partial x} \frac{1}{2} \log (x^2+y^2) = \frac{1}{2} \frac{2x}{x^2+y^2} = \frac{x}{x^2+y^2} \\
%			u_y = \frac{y}{x^2+y^2} \\
%			\tan' = 1 + \tan^2 \\
%			\tan v(x,y) = \frac{y}{x} \\
%			( 1 + \underbrace{\tan^2 v}_{\frac{y^2}{x^2}} ) v_x = -\frac{y}{x^2} \\
%			v_x = \frac{1}{1 + \frac{y^2}{x^2}} \left( -\frac{y}{x^2} \right) = -\frac{y}{x^2+y^2} \\
%			( 1 + \tan^2 v ) v_y = \frac{1}{x} \\
%			v_y = \frac{1}{1 + \frac{y^2}{x^2}} \frac{1}{x} = \frac{x}{x^2+y^2} \\
%			u_x = v_y \\
%			u_y = -v_x
%		\end{gather*}
%		die Cauchy-Riemann Gleichungen sind erfüllt , $u,v$ stetig differenzierbar für alle $(x,y) \neq (0,0)$ $\blacksquare$
%	\end{bew}
%	\begin{bem}
%		Man kann weitere ''Zweige'' des Logarithmus definieren mit Definitionsbereich ein Sektor wo $\arg z$ eindeutig und stetig ist.
%	\end{bem}
%\end{beh}\todo{Fix}

\subsubsection{Ableitung von \texorpdfstring{$\Log z$}{Log z}}
\begin{gather*}
	(e^{\Log z})' = z' , z \in \C \setminus \R_{\leq 0} \\
	\underbrace{e^{\Log z}}_{z} \Log' z = 1 \\
	\Log' z = \frac{1}{z}
\end{gather*}

\subsubsection{\texorpdfstring{$n$}{n}-te Wurzel}
Potenz: $z \mapsto z^n$ , $n = 1 , 2 , 3 , \dotsc$ \\
Hauptwert der $n$-te Wurzel:
\begin{gather*}
	\C \setminus \R_{\leq 0} \rightarrow \C \\
	z \mapsto z^{\frac{1}{n}} \coloneqq e^{\frac{1}{n} \Log z} \\
	re^{\imath\varphi} \mapsto e^{\frac{1}{n} \Log re^{\imath\varphi}} = e^{\frac{1}{n} (\log r + \imath\varphi)} = r^{\frac{1}{n}} e^{\frac{\varphi}{n}}
\end{gather*}

\subsubsection{Allgemeiner für \texorpdfstring{$a \in \C$}{a in C}}
\[ z^a \coloneqq e^{a\Log z} \]
für $z \in \C \setminus \R_{\leq 0}$. Kettenregel:
\[ (z^a)' = \frac{a}{z} e^{a \Log z} = az^{a-1} \]

Analysis:
Satz der inversen Funktionen
\[ F: \R^n \rightarrow \R^n \quad ( n = 2 \text{ für uns} ) \text{ stetig differenzierbar} \]
\begin{def*}{note = Ableitung (Tangentialabbildung) , index = Ableitung Tangentialabbildung , indexformat = {1!(2) 2!(1)}}
	\[ F'(x) , x = (x_1 , \dotsc , x_n) \]
	lineare Abbildung
	\[ F(x+h) = F(x) = F'(x) \cdot h + o(\abs{h}) \quad h \rightarrow 0 \]
\end{def*}
\begin{satz*}[note = Satz der inverse Funktion , index = Satz der inverse Funktion , indexformat = {1!~234}]
	Wenn $F'(x_0)$ invertierbar ist, dann gibt es eine Umgebung $U$ von $x_0$ so, dass $F: U \rightarrow F(U) = V$ invertierbar ist. Die inverse Abbildung $F^{-1} : V \rightarrow U$ ist stetig differenzierbar und
	\[ (F^{-1})'(F(x)) = (F'(x))^{-1} \]
\end{satz*}
$f(x+iy) = u(x,y) + \imath v(x,y)$ analytisch:
\begin{gather*}
	F' = \begin{pmatrix} u_x & u_y \\ v_x & x_y \end{pmatrix}
	\intertext{hat die Form}
	\begin{pmatrix} a & b \\ -b & a \end{pmatrix} \\
	(F')^{-1} = \frac{1}{a^2+b^2} \begin{pmatrix} a & -b \\ b & a \end{pmatrix}
\end{gather*}
hat dieselbe Form.

$\implies$ Sei $f: \Omega \rightarrow \C$ analytisch , $z_0 \in \Omega$, $f'(z_0) \neq 0$. Dann gibt es eine Umgebung $U$ von $z_0$ in $\Omega$, sodass $f: U \rightarrow f(U) \eqqcolon V$ invertierbar ist. Die inverse Funktion $f^{-1} : V \rightarrow U$ ist analytisch mit Ableitung $(f^{-1})'(f(z)) = \frac{1}{f'(z)}$. \\
\begin{bsp*}
	\begin{gather*}
		f = \exp: \C \rightarrow \C , z_0 = 0 \\
		f(z) = e^z \\
		f'(z_0) = 1 \\
		U = \{ z \in \C | -\pi < \Im z < \pi \} \\
		f(U) = \C \setminus \R_{\leq 0} \\
		f^{-1} : V \rightarrow U \text{ ist } \Log \\
		\Log' e^z = \frac{1}{e^z}
	\end{gather*}
\end{bsp*}

\subsubsection{Tangentialabbildung, Winkeltreue}
Betrachte Bilder von Kurven in $\C$. Parameterdarstellung:
\[ \gamma : t \mapsto z(t) , a \leq t \leq b \]
(z.B. Kreis: $t \mapsto re^{\imath t} , -\pi \leq t \leq \pi$ ) \\
Bild von$\gamma$ durch $f$ analytisch hat Parameter-Darstellung
\[ t \mapsto f(z(t)) \]
Kettenregel $[a,b] \overset{z}{\rightarrow} \C \overset{f}{\rightarrow} \C$
\[ \frac{\dd}{\dd t} f(z(t)) = f'(z(t)) \dot{z}(t) \quad \dot{z}(t) = \frac{\dd z(t)}{\dd t} \]
Tangentialvektor (Geschwindigkeitsvekor)
\begin{bew}
	\begin{gather*}
		z(t + \Delta t) = z(t) + \Delta z \quad \Delta z = \dot{z}(t) + o(\Delta t) \quad \Delta t \rightarrow 0 \\
		f(z+h) = f(z) + f'(z)h + o(\abs{h}) \\
		\begin{split}
			f(z(t + \Delta t)
				&= f(z(t) + \Delta z) \\
				&= f(z(t)) + f'(z(t) \Delta z + o(\abs{\Delta z}) \\
				&= f(z(t)) + f'(z(t)\dot{z}(t) \Delta t \\
				&+ \underbrace{f'(z(t)o(\Delta t) +o(\abs{\dot{z}(t) \Delta t})}_{o(\Delta t)}
		\end{split} \\
		\frac{\dd}{\dd t} f(z(t)) = f'(z(t))\dot{z}(t)
	\end{gather*}
\end{bew}
Der Tangentialvektor $\dot{w}(t)$ der Bildkurve $t \mapsto w(t) = f(z(t))$ ist
\[ \dot{w}(t) = f'(z(t))\dot{z}(t) \]
$\implies$ Haben zwei Kurven durch $z_0$ denselben Tangentialvektor in $z_0$ dann haben ihre Bilder denselben Tangentialvektor in $f(z_0)$, $w = f'(z_0) v$

Die \textbf{Tangentialabbildung} in $z_0$
\[ v \mapsto f'(z_0) v \]
ist für analytische Funktionen mit $f'(z_0) \neq 0$ eine Drehstreckung: wenn $f'(z_0) = re^{\imath\varphi}$, dann ist sie die zusammensetzung der Drehung $v \mapsto e^{\imath\varphi}v$ mit Wikel $\varphi$ und Streckung $v \mapsto rv$ mit Streckfaktor $r = \abs{f'(z_0)}$

Drehstreckungen sind \textbf{winkeltreu}.
$\angle (v_1 , v_2) = \Phi$ dann $\angle( f'(z_0) v_1 , f'(z_0)v_2 ) = \Phi$.
Man sagt, das analytische Funktionen mit $f' \neq 0$ \textbf{winkeltreu} oder \textbf{konform} sind; Winkel awischen Tangentialvektoren sind erhalten.

\begin{bsp*}
	Exponentialfunktion $z \mapsto e^z$ \\
	Bild von $\gamma : t \mapsto at$, $a = a_1 + \imath a_2$, $\varphi \frac{a_2}{a_1}$, $a_1 , a_2 \neq 0$.
	\[ t \mapsto w(t) = e^{at} = e^{a_1 t} ( \cos a_2 t + \imath \sin a_2 t ) \]
\end{bsp*}

Zweite Folgerung aus $\dot{w}(t) = f'(z(t))\dot{z}(t)$ \\
Sei $f$ analytisch auf einer zusammenhängende\footnote{je zwei Punkte in $\Omega$ können durch ein Kurve in $\Omega$ verbunden werden.} offene Menge $\Omega$ und $f'(z) = 0$ für alle $z \in \Omega$. Dann gilt: \\
$f$ ist konstant. \\
Zu zeigen: $f(z_1) = f(z_2)$ für alle $z_1 , z)2 \in \Omega$. Sei $\gamma : t \mapsto z(t) , t \in [a,b]$ sodass $z(a) = z_1 , z(b) = z_2$ \\
\begin{gather*}
	\frac{\dd}{\dd t} f(z(t)) = f'(z(t)) \dot{z}(t) = 0 \\
	f(z_1) - f(z_2) = \int_a^b \frac{\dd}{\dd t} f(z(t)) \dd t = 0
\end{gather*}
Bild von $\gamma : t \mapsto at , a = a1 + \imath a_2$
\[ t \mapsto w(t) = e^{at} = e^{a_1 t} (\cos(a_2 t) + \imath \sin(a_2 t)) \]
\begin{folge}
	Zweite Folgerung aus $\dot{w}(t) = f'(z(t)) \dot{z}(t)$:
	$f$ analytisch auf einer zusammenhängenden\footnote{je zwei Punkte in $\Omega$ können durch eine Kurve in $\Omega$ verbunden werden} offenen Menge $\Omega$. \\
	$f'(z) = 0$ für alle $z \in \Omega$ \\
	Dann gilt: $f$ ist konstant \\
	\begin{bew}
		Zu zeigen: $f(z_1) = f(z_2)$ für alle $z_1 , z_2 \in \Omega$ \\
		Sei $\gamma : t \mapsto z(t) , t \in [a,b]$ so dass
		\begin{gather*}
			z(a) = z_1 \\
			z(b) = z_2 \\
			\frac{\dd}{\dd t} f(z(t)) = f'(z(t)) \cdot \dot{z}(t) = 0 \\
			f(z_2) - f(z_1) = \int_a^b \frac{\dd}{\dd t} f(z(t)) \dd t = 0 \quad a \in D
		\end{gather*}
	\end{bew}
\end{folge}

\section{Möbiustransformationen}
\subsection{Inversion}
$I : \C^* \rightarrow \C^* , z \mapsto \frac{1}{z}$ invertierbare analytische Funktion \\
$I^{-1} = I$ \\
\begin{beh}
	$I$ bildet Kreise in $\C^*$ nach Kreise in $\C^*$ ab. \\
	\begin{bew}
		Sei $K$ ein Kreis mit Mittelpunkt $a \in \C$ Radius $R > 0$. $K = \{ z \in \C \mid \abs{z-a} = R \}$ \\
		$K$ liegt in $\C^*$ wenn $\abs{a} \neq R$
		\begin{gather*}
			I(K) = \left\{ w \in \C \middle| \abs{\frac{1}{w} - a} = R \right\} \tag{*} \\
			\implies \abs{1-wa} = R \abs{w} \\
			\abs{w-c} = \frac{R}{\abs{\abs{a}^2-R^2}} \quad \begin{matrix*}[l] \text{ Kreis mit Mittelpunkt $c$} \\ \text{und Radius } \frac{R}{\abs{\abs{a}^2-R^2}} \end{matrix*} \\
			\abs{w - \frac{1}{a}} = \frac{R}{\abs{a}} \abs{w} \quad a \neq 0
		\end{gather*}
		\begin{bew}
			Direkter Beweis, dass $*$ einen Kreis beschreibt. \\
			Verwende: $\abs{z} = \sqrt{z\overline{z}}$, Quadratisch Ergänzen. \\
			\begin{gather*}
				(1-wa)(1-\overline{w}\overline{a}) = R^2 w \overline{w} \\
				(\abs{a}^2 - R^2) w \overline{w} - aw - \overline{a}\overline{w} + 1 = 0 \\
				(\abs{a}^2 - R^2)( w \overline{w} - cw - \overline{c}\overline{w} + \frac{1}{\abs{a}^2 - R^2} = 0 \\
				\qquad c = \frac{a}{\abs{a}^2 - R^2} \\
				(w-c)(\overline{w}-\overline{c}) - \abs{c}^2 + \frac{1}{\abs{a}^2 - R^2} = 0 \\
				\abs{w-a}^2 = \frac{R^2}{(\abs{a}^2 - R^2)^2}
			\end{gather*}
		\end{bew}
	\end{bew}
\end{beh}
\todo{Fix tag}

\subsection{Allgemeine Möbius-Transformtionen}
\begin{def*}[note = Möbiustransformation , index = Möbiustransformation]
	Eine \textbf{Möbiustransformation} ist eine Funktion der Form
	\[ f(z) = \frac{az + b}{cz + d} , ad - bc \neq 0 \]
	(wenn $ad -bc = 0$ ist $f = $ const., sofern definiert)
\end{def*}
\begin{gather*}
	f_1(z) = \frac{a_1 z + b_1}{c_1 z + d_1} \\
	f_2(z) = \frac{a_2 z + b_2}{c_2 z + d_2} \\
	\begin{split}
		f_1 \circ f_2 (z)
			&= f_1(f_2(z)) \\
			&= \frac{a_1 \frac{a_2 z + b_2}{c_2 z + d_2} + b_1}{c_1 \frac{a_2 z + b_2}{c_2 z + d_2} + d_1} \cdot \frac{c_2 z + d_2}{c_2 z + d_2} \\
			&= \frac{( a_1 a_2 + b_1 b_2 ) z + a_1 b_2 + b_1 d_2}{( c_1 a_2 + d_1 c_2 ) z + c_1 b_2 + d_1 d_2} = \frac{az + b}{cz + d}
	\end{split}
	\intertext{Wobei}
	\begin{pmatrix} a & b \\ c & d \end{pmatrix} = \begin{pmatrix} a_1 & b_1 \\ c_1 & d_1 \end{pmatrix} \begin{pmatrix} a_2 & b_2 \\ c_2 & d_2 \end{pmatrix} \\
	ad - bc = \det\begin{pmatrix} a & b \\ c & d \end{pmatrix} = \det(\dots) \det(\dots) = (a_1 d_1 - b_1 c_1)(a_2 d_2 - b_2 c_2) \neq 0
\end{gather*}
Möbiustransformationen sind invertierbar:
\[ f(z) = \frac{az + b}{cz + d} \iff f^{-1}(z) \frac{dz - b}{-cz + a} \]
\begin{bem}
	Die Matrizen $\begin{pmatrix} a & b \\ c & d \end{pmatrix}$ und $\begin{pmatrix} \lambda a & \lambda b \\ \lambda c & \lambda d \end{pmatrix}$ entsprechen der gleichen Möbiustransformation
	\[ z \mapsto \frac{az + b}{cz + d} \]
	für alle $\lambda \neq 0$
\end{bem}
Analog: Kreisscheibe $\abs{z-a} \leq R$ wird nach der Kreisscheibe $\abs{w-c} \leq \frac{R}{\abs{\abs{a}^2 - R^2}}$ abgebildet, falls $R < \abs{a}$, oder nach den äusseren Gebeit $\abs{w-c} \leq \frac{R}{\abs{\abs{a}^2 - R^2}}$, falls $R > \abs{a}$ \\
Falls $R = \abs{a} , a \neq 0$
$(**): aw + \overline{a}\overline{w} = 1$ \\
$\implies$ Gleichung einer Gerade. Jede Gerade, die nicht durch $0$ geht, hat diese Form. \\
Zusammenfassend: $I$ bildet Kreise und Geraden nach Kreise und Geraden ab. (Übung: Geraden durch $0$ werden nach Geraden durch $0$ abgebildet) \\
Es ist bequem, die erweiterte Komplexe Ebene $\overline{\C} = \C \cup \{\infty\}$ einzuführen und $\frac{1}{0} = \infty$ und $\frac{1}{\infty} = 0$ zu definieren. Dann ist $I : \overline{\C} \rightarrow C , I^{-1} = I$, die verallgemeinerte Kreise nach verallgemeinerten Kreisen abgebildet werden. \\
Verallgemeinerte Kreise: Kreise oder ''Kreise durch $\infty$'' = Geraden. \\
$\overline{\C}$ = Riemannsche Zahlenkugel.

\begin{bsp*}
	\begin{enumerate}[label = \arabic*)]
		\item $id(z) = z$
			\[ \begin{pmatrix} a & b \\ c & d \end{pmatrix} = \begin{pmatrix} 1 & 0 \\ 0 & 1 \end{pmatrix} \quad \frac{1z + 0}{0z + 1} = z \]
			\begin{bem}
				Möbiustransformationen bilden eine Gruppe bezüglich $\circ$: $f \circ g(z) = f(g(z))$
			\end{bem}
		\item $t_b(z) = z + b , b \in \C$
			\[ \begin{pmatrix} a & b \\ c & d \end{pmatrix} = \begin{pmatrix} 1 & b \\ 0 & 1 \end{pmatrix} \quad \frac{1z + b}{0z + 1} = z + b \]
		\item $S_a(z) = az , a \in \C , a \neq 0$ Drehstreckung mit Faktor $a$ \\
			\[ \begin{pmatrix} a & 0 \\ 0 & 1 \end{pmatrix} \]
		\item $I(z) = \frac{1}{z}$ Inversion
			\[ \begin{pmatrix} 0 & 1 \\ 1 & 0 \end{pmatrix} \]
	\end{enumerate}
\end{bsp*}
\begin{satz*}
	Jede Möbiustransformation lässt sich schreiben als Zusammensetzung von Transformationen der Form 1) - 4)
	\begin{bew}
		\begin{gather*}
			f(z) = \frac{az + b}{cz + d} , ad - bc \neq 0
			\intertext{Sei zuerst $c \neq 0$}
			\begin{split}
				&z \overset{S_c}{\mapsto} cz \overset{t_d}{\mapsto} cz + d \overset{I}{\mapsto} \frac{1}{cz + d} \\
				&\overset{t_\lambda}{\mapsto} \frac{1}{cz + d} + \lambda = \frac{\lambda cz + \lambda + \lambda d}{cz + d} \\
				&\overset{S_\mu}{\mapsto} \frac{\mu\lambda cz + \mu(1 + \lambda d)}{cz + d}
			\end{split}
			\intertext{$\mu , \lambda$ wählen, so dass $\mu\lambda c = a$, $\mu(1 + \lambda d) = b$}
			\implies \mu + \frac{a}{c} d = b \\
			\iff \mu = b - \frac{a}{c} d = -\frac{ad - bc}{c} \\
			\lambda = \frac{a}{c\mu} \\
			\text{Wenn } c = 0 : f(z) = \frac{a}{d} z + \frac{b}{d} = t_{\frac{b}{d}} \circ S_{\frac{a}{d}}(z)
		\end{gather*}
	\end{bew}
	\begin{folge}
		Möbiustransformationen bilden verallgemeinerte Kreise nach verallgemeinerten Kreisen ab. Dasselbe gilt für verallgemeinerte Kreisscheiben.
		\begin{bew}
			Das gilt offensichtlich für $S_a$ , $t_d$ und auch für $I$.
		\end{bew}
	\end{folge}
\end{satz*}
\todo{Too long}

Wir erweitern die Definition einer MT auf $\overline{\C} = \C \cup \{ \infty \}$
\begin{gather*}
	f(z) = \frac{az + b}{cz + d} , z \neq -\frac{d}{c} \\
	f\left( -\frac{d}{c} \right) \coloneqq \infty , f(\infty) \coloneqq \lim_{z \rightarrow \infty} \frac{az + b}{cz + d} = \frac{a}{c}
	\text{Wenn } c = 0 : f(z) = \frac{az + b }{d} , z \in \C \\
	f(\infty) \coloneqq \infty \\
	\implies f: \overline{\C} \mapsto \overline{\C} \text{ bijektiv}
\end{gather*}
\begin{satz*}
	Seien $z_1 , z_2 , z_3 \in \overline{\C}$ verschiedene Punkte. Dann gibt es eine eindeutige Möbiustransformation $f$ mit $f(z_1) = 0 , f(z_2) = \infty , f(z_3) = 1$
	\begin{bew}
		für $z_1 , z_2 , z_3 \in \C$ \\
		Eindeutigkeit: \\
		\begin{gather*}
			\text{Wenn } f_1 , f_2 \text{ diese Eigenschaft haben, dann} \\
			f_1 \circ f_2^{-1}(0) = f_1(z_1) = 0 \\
			f_1 \circ f_2^{-1}(1) = 1 \\
			f_1 \circ f_2^{-1}(\infty) = \infty \\
			\text{D.h. } h \coloneqq f_1 \circ f_2^{-1} \text{ bildet } 0 , \infty , 1 \text{ nach } 0 , \infty , 1 \\
			h(0) = 0 \implies \frac{b}{d} = 0 \\
			h(\infty) = \infty \implies c = 0 \\
			h(1) = 1 \implies \frac{ad}{d} + \frac{b}{d} = \frac{a}{d} \cdot 1 = 1 \\
			\implies a = a , b = c = 0 \\
			h(z) = \frac{az}{a} = z \quad h = id \\
			f_1 \circ f_2^{-1} = id \iff f_1 = f_2
		\end{gather*}
		Existenz: Doppelverhältnis
		\begin{def*}[note = Doppelverhältnis , index = Doppelverhältnis]
			Das \textbf{Doppelverhältnis} (cross ratio) von vier verschiedenen komplexen Zahlen $z_0 , z_1 , z_2 , z_3$ ist die Zahl
			\[ ( z_0 , z_1 , z_2 , z_3 ) =  \frac{(z_0 - z_1)(z_3 - z_2)}{(z_0 - z_2)(z_3 - z_1)} \]
		\end{def*}
		\begin{beh}
			Die gesuchte Möbiustransformation ist
			\begin{gather*}
				f(z) = ( z_0 , z_1 , z_2 , z_3 ) = \frac{(z - z_1)(z_3 - z_2)}{(z - z_2)(z_3 - z_1)} = \frac{az + b}{cz + d} \\
				\begin{pmatrix} a & b \\ c & d \end{pmatrix} = \begin{pmatrix} z_3 - z_2 & -z_1 (z_3 - z_2) \\ z_3 - z_1 & -z_2 (z_3 - z_1) \end{pmatrix}
				\intertext{$\det \neq 0$, wenn $z_1 , z_2 , z_3$ verschieden sind.}
				f(z_1) = 0 \\
				f(z_2) = \infty \\
				f(z_3) = 1 \\
				\blacksquare
			\end{gather*}
		\end{beh}
	\end{bew}
	\begin{folge}
		\begin{enumerate}[label = \arabic*)]
			\item sind $z_1 , z_2 , z_3 \in \overline{\C}$ verschieden, $w_1 , w_2 , w_3 \in \overline{\C}$ verschieden, dann gibt es eine eindeutige MT $f$ welche $z_1$ nach $w_1$, $z_2$ nach $w_2$, $z_3$ nach $w_3$ abbildet.
				\begin{bew}
					Sei $f_1 : z_1 , z_2 , z_3 \mapsto 0 , \infty , 1 ; f_2 : w_1 , w_2 , w_3 \mapsto 0 , \infty , 1$. Setze $f = f_2^{-1} \circ f_1$. Eindeutigkeit wie oben.
				\end{bew}
			\item Gegeben zwei (verallgemeinerte) Kreise $K_1 , K_2$, dann gibt es eine MT $f$ die $K_1$ nach $K_2$ abbildet (nicht eindeutig)
				\begin{bew}
					Wähle je 3 Punkte auf $K_1 , K_2$, verwende 1
				\end{bew}
				Dasselbe gilt für verallgemeinerte Kreisscheiben.
			\item Sei $g$ eine MT. Dann gilt für alle $z_1 , z_2 , z_3$ verschieden:
				\[ ( z , z_1 , z_2 , z_3 ) = ( g(z) , g(z_1) , g(z_2) , g(z_3) ) \]
				\begin{bew}
					$z \mapsto ( g(z) , g(z_1) , g(z_2) , g(z_3) )$ bildet $z_1$ nach $0$ , $z_2$ nach $\infty$ , $z_3$ nach $1$ ab. $z \mapsto ( z , z_1 , z_2 , z_3 )$ auch. Also sind diese Abbildungen wegen der Eindeutigkeit gleich.
				\end{bew}
		\end{enumerate}
	\end{folge}
\end{satz*}
\todo{Too long}

\subsection{Abbildungen von Gebieten}
\begin{def*}[note = Gebiet , index = Gebiet]
	Ein \textbf{Gebiet} in $\C$ ist eine zusammenhängende offene Teilmenge von $\C$
\end{def*}
\begin{def*}[note = einfach zusammenhängend , index = einfach zusammenhängend , indexformat = {2!~1 }]
	Ein Gebiet $\Omega$ heisst \textbf{einfach zusammenhängend}, falls jede geschlossene Kurve $\gamma$ in $\Omega$ zu einem Punkt zusammenziehbar ist, d.h. Es gibt eine stetige Abbildung $(t,s) \mapsto z(t,s) \in \Omega , 0 \leq t , s \leq 1$ mit $z(1,s) = z(0,s) , z(t,0) = $const. , $t \mapsto z(t,1)$ Parameterdarstellung von $\gamma$.
\end{def*}
\begin{bsp*}
	BILD
\end{bsp*}

\subsubsection{Riemannscher Abbildungsatz}
Zu jedem einfach zusammenhängenden Gebiet $\Omega$ mit $\Omega \neq \varnothing , \Omega \neq \C$ gibt es eine analytische bijektive Abbildung $f: \Omega \rightarrow D$ mit analytischer Inversen $f^{-1} : D \rightarrow \Omega$ auf die Offene Einheitskreisscheibe $D = \{ z \in \C \mid \abs{z} < 1 \}$.

Es folgt, dass für Paare $\Omega_1 , \Omega_2$ solcher Gebiete ein solches $f: \Omega_1 \rightarrow \Omega_2$ existiert.

\begin{satz*}[note = Riemannscher Abbildungssatz , index = Riemannscher Abbildungs satz , indexformat = {12.3 3!12-~}]
	$\Omega \subset \C$ einfach zusammenhängendes Gebiet $(\Omega \neq \varnothing , \Omega \neq \C)$ dann gibt es eine konforme Abbildung $f: \Omega \rightarrow D = \{ z \in \C \mid \abs{z} < 1 \}$
\end{satz*}
\begin{def*}[note = konform , index = konform]
	$f: \Omega_1 \rightarrow \Omega_2$ heisst \textbf{konform}, falls $f$ analytisch, bijektiv mit $f'(z) \neq 0$ für alle $z \in \Omega_1$. Dann ist $f^{-1}$ auch konform.
\end{def*}
\begin{bsp}
	$\Omega$ = Obere Halbebene = $\{ z \in \C \mid\ \Im z > 0 \}$. \\
	Was ist $f: \Omega \rightarrow D$? \\
	Suche in den Möbiustransformationen. \\
	$f$ MT die
	\begin{itemize}
		\item $\infty \mapsto 1$
		\item $0 \mapsto -1$
		\item $1 \mapsto -\imath$
	\end{itemize}
	abbildet.
	\begin{gather*}
		g: \begin{cases}
			1 \mapsto \infty \\
			-1 \mapsto 0 \\
			-\imath \mapsto 1
		\end{cases} \\
		g(w) = c \cdot \frac{w+1}{w-1} \\
		g(-\imath) = c \cdot \frac{-\imath + 1}{-\imath -1} = \frac{c(-\imath+1)^2}{(-\imath-1)(-\imath+1)} = \frac{-2\imath}{-2} c = \imath c = 1 \\
		c = -\imath \\
		g(w) = -\imath \frac{w+1}{w-1} \\
		g(0) = \imath \in \Omega \\
		f = g^{-1} \\
		w = f(z) \iff z = g(w) \\
		f(z) = \frac{z-\imath}{z+\imath}
	\end{gather*}
\end{bsp}
\begin{bsp}
	Sektor: $\Omega = \{ z \mid 0 < \arg z < \alpha \}$
	\begin{gather*}
		z \mapsto z^{\frac{\pi}{\alpha}} \coloneqq e^{\frac{\pi}{\alpha} \Log z} \\
		re^{\imath\varphi} \mapsto r^{\frac{\pi}{\imath}} e^{\imath\frac{\pi}{\alpha}\varphi} \\
		\begin{array}{l}
			r > 0 \\
			0 < \varphi < \alpha
		\end{array} \iff 0 < \frac{\pi}{\alpha}\varphi < \pi
		\intertext{Die konforme Abbildung ist}
		f(z) = \frac{z^{\pi\alpha} - \imath}{z^{\pi\alpha} + \imath}
	\end{gather*}
\end{bsp}
\begin{bsp}
	Streifen $\Omega = \{ z \mid -\imath a \leq \Im z \leq \imath a \}$
	\begin{gather*}
		w = e^{\frac{\pi z}{a}} = e^{\frac{\pi x}{a} + \imath \pi \frac{y}{a}} = e^{\frac{\pi x}{a}} e^{\imath\pi \frac{y}{a}} \\
		-\pi < \frac{\pi y}{a} < \pi
		\intertext{konforme Abbildung}
		w = \frac{\imath e^{\frac{\pi z}{a}} - \imath}{\imath e^{\frac{\pi z}{a} + \imath}} = \frac{e^{\pi z}{a} - 1}{e^{\frac{\pi z}{a}} + 1}
	\end{gather*}
\end{bsp}
