\chapter{Fouriertransformationen}
''Fourrierreihen im limes $T \rightarrow \infty$''

\begin{def*}[note = Integral , index = Integral]
	Das Integral
	\[ \int{-\infty}^{\infty} f(t) \dd t \]
	einer Funktion $f: \R \rightarrow \C$ ist definert als
	\[ \lim_{M \rightarrow \infty} \int_{-M}^M f(t) \dd t \]
	( falls der Grenzwert existiert ).
\end{def*}
\begin{def*}[note = integrabel , index = integrabel]
	Eine Funktion $f$ heisst \textbf{integrabel} falls der Grenzwert
	\[ \lim_{M \rightarrow \infty} \int_{-M}^M \abs{f(t)} \dd t \]
	existiert. ( man schreibt auch $\int_{-\infty}^\infty \abs{f(t)} \dd t < \infty$ ) \\
	Dann existiert auch $\int_{-\infty}^\infty f(t) \dd t$ und es gilt
	\[ \abs{\int_{-\infty}^\infty f(t) \dd t} \leq \int_{-\infty}^\infty \abs{f(t)} \dd t \]
\end{def*}
\begin{def*}[note = Fourriertransformierte , index = Fourrier transformierte , indexformat = {1!~.2 2!1.~}]
	Die \textbf{Fourriertransformierte} einer integrablen Funktion ist die Funktion
	\[ \hat{f}(\omega) = \int_{-\infty}^\infty f(t) e^{-\imath\omega t} \dd t , \omega \in \R \]
\end{def*}
Intuition:
\[ \hat{f}(\omega) = \lim_{T \rightarrow \infty} ( T c_n |_{\omega = \frac{2\pi}{T} n} ) = \lim_{T \rightarrow \infty} \int_{-\frac{T}{2}}^{\frac{T}{2}} f(t) e^{-\imath\omega t} \dd t \]
Wohldefiniert da $\abs{f(t) e^{-\imath\omega t}} = \abs{f(t)}$ \\
Umkehrformel: Kann man $f$ aus $\hat{f}$ wiedergewinnen? \\
Intuition:
\[ \begin{split}
	f(t)	&= \lim_{T \rightarrow \infty} \sum_{\omega = \frac{2\pi}{t} n} c_n e^{\imath \omega t} \\
		&= \lim_{T \rightarrow \infty} \frac{1}{2\pi} \sum_{\omega = \frac{2\pi}{t} n} \underbrace{\frac{2\pi}{T}}_{\Delta \omega} \hat{f}(\omega) e^{\imath\omega t} \\
		&= \frac{1}{2\pi} \int_{-\infty}^\infty \hat{f}(\omega) e^{\imath\omega t} \dd \omega
\end{split} \]
(Riemann-Summe) \\
\begin{satz*}[note = Umkehrsatz von Fourier]
	Sei $f$ integrabel und $\hat{f}$ integrabel. Dann gilt
	\[ f(t) = \frac{1}{2\pi} \int_{-\infty}^\infty \hat{f}(\omega) e^{\imath\omega t} \dd \omega \]
	\begin{bew}
		später
	\end{bew}
\end{satz*}
\begin{bsp}
	\[ f(t) = \chi_{[a,b]}(t) \coloneqq \begin{cases}
		1	& a\leq t \leq b	\\
		0	& \text{sonst}	
	\end{cases} \]
	charakteristische Funktion von $[a,b]$ \\
	\[ \begin{split}
		\hat{f}(\omega)	&= \int_{-\infty}^\infty \chi_{[a,b]}(t) e^{-\imath\omega t} \dd t \\
					&= \int_a^b 1 e^{-\imath\omega t} \dd t \\
					&= \left[ \frac{1}{-\imath\omega} e^{-\imath\omega t} \right]_a^b \\
					&= \frac{1}{-\imath\omega}(e^{-\imath\omega b} - e^{-\imath\omega a}) \\
					&= e^{-\imath\omega \frac{a+b}{2}} \frac{1}{-\imath\omega} (e^{\imath\omega\frac{a-b}{2}} - e^{-\imath\omega\frac{a-b}{2}} ) \\
					&= \underbrace{e^{-\imath\omega \frac{a+b}{2}}}_{\abs{\cdot} = 1} \frac{2 \sin\left( \omega \frac{b-a}{2} \right)}{\omega}
	\end{split} \]
\end{bsp}
\begin{bem}
	Man kann zeigen, dass $\hat{f}(\omega)$ stetig ist und $\lim_{\omega \rightarrow \pm \infty} \hat{f}(\omega) = 0$ (Riemann-Lebesgue).
\end{bem}
\begin{bsp}
	\begin{gather*}
		f(t) = e^{-a \frac{t^2}{2}} , a > 0 \\
		\hat{f}(\omega) = \int_{-\infty}^\infty e^{-a\frac{t^2}{2} - \imath\omega t} \dd t \\
		\text{z.B. } \hat{f}(0) = \int_{-\infty}^\infty = e^{-a\frac{t^2}{2}} \dd t = \sqrt{\frac{2\pi}{a}} \\
		\begin{split}
			\int_{-\infty}^\infty e^{-a\frac{t^2}{2} - bt} \dd t	&= \int_{-\infty}^\infty e^{-\frac{a}{2}(t+\frac{b}{a})^2 + \frac{b^2}{2a}} \dd t \\
												&= e^{\frac{b^2}{2a}} \int_{-\infty}^\infty e^{-\frac{a}{2} (t+\frac{b}{a})^2} \dd t \\
												&= e^{\frac{b^2}{2a}} \int_{-\infty}^\infty e^{-\frac{a}{2} s^2} \dd s \\
												&= \sqrt{\frac{2\pi}{a}} e^{\frac{b^2}{2a}}
		\end{split}
		\intertext{Setze $b = \imath\omega$ (wieso das erlaubt ist, werden wir aus der Funktionentheorie lernen)}
		\hat{f}(\omega) = \sqrt{\frac{2\pi}{a}} e^{-\frac{\omega^2}{2a}}
	\end{gather*}
	Verifikation des Umkehrsatzes für diese Funktion
	\[ \begin{split}
		\frac{1}{2\pi} \int_{-\infty}^\infty \sqrt{\frac{2\pi}{a}} e^{-\frac{\omega^2}{2a}} e^{\imath\omega t} \dd \omega
			&= \frac{1}{2\pi} \sqrt{\frac{2\pi}{a}} \int_{-\infty}^\infty e^{-\frac{1}{a} \frac{\omega^2}{2} + \imath t \omega} \dd \omega \\
			&= \frac{1}{2\pi} \sqrt{\frac{2\pi}{a}} e^{-\frac{(-t)^2}{2\frac{1}{a}}} \\
			&= e^{-\frac{at^2}{2}}
	\end{split} \]
\end{bsp}
\begin{bsp}
	\begin{gather*}
		f(t) = e^{-\abs{t}} \\
		\begin{split}
			\hat{f}(\omega)	&= \int_{-\infty}^\infty e^{-\abs{t} - \imath\omega t} \dd t \\
						&= \int_0^\infty e^{-t - \imath \omega t} \dd t + \int_{-\infty}^0 e^{+t - \imath\omega t} \dd t \\
						&= \frac{-1}{1+\imath\omega} [ e^{-t-\imath\omega t} ]_0^\infty + \frac{1}{1-\imath\omega} [e^{t-\imath\omega t} ]_{-\infty}^0 \\
						&= \frac{1}{1+\imath\omega} + \frac{1}{1-\imath\omega} \\
						&= \frac{1}{1+\omega^2}
		\end{split}
	\end{gather*}
\end{bsp}

%\section{Eigenschaften von $\hat{f}$}
\begin{enumerate}[label=\arabic*)]
	\item Die Fouriertransformation ist linear $h(t) = af(t) + bg(t) \implies \hat{h}(\omega) = a\hat{f}(\omega) + b\hat{g}(\omega) \Leftarrow$ Linearität von $\int_{-\infty}^\infty$
	\item $g(t) = f(t-a) \implies \hat{g}(\omega) = e^{-\imath a \omega} \hat{f}(\omega)$
	\item $g(t) = f\left( \frac{t}{a} \right) \wedge a > 0 \implies \hat{g}(\omega) = a \hat{f}(a\omega)$ \\
		\begin{bew}
			$\hat{g}(\omega) = \int_{-\infty}^\infty g(t) e^{-\imath\omega t} \dd t = \int_{-\infty}^\infty f\left( \frac{t}{a} \right) e^{-\imath\omega t} \dd t = \int_{-\infty}^\infty f(t) e^{-\imath\omega at} a \dd t = a \hat{f}(a\omega)$
		\end{bew}
	\item $g(t) = e^{\imath at} f(t) \implies \hat{g}(\omega) = \hat{f}(\omega - a)$, analog zu 2)
	\item $g(t) = f'(t)$, $g,f$ integrabel , $f(t) \rightarrow 0$ für $t \rightarrow \pm \infty \implies \hat{g} = \imath \omega \hat{f}(\omega)$ \\
		\begin{bew}
			\[ \hat{g}(\omega) = \int_{-\infty}^\infty f'(t) e^{-\imath \omega t} \dd t = \int_{-\infty}^\infty f(t) \imath\omega e^{-\imath \omega t} \dd t \]
		\end{bew}
	\item \begin{satz*}[note = Falltungssatz]
		\begin{gather*}
			(f*g)(t) \coloneqq \int_{-\infty}^\infty f(t-s) g(s) \dd s \\
			\widehat{(f*g)}(\omega) = \hat{f}(\omega) \cdot \hat{g}(\omega)
		\end{gather*}
	\end{satz*}
\end{enumerate}

\subsubsection{Faltung(sprodukt)}
$f,g$ integrabel \\
Faltung von $f$ und $g$:
\[ (f*g)(t) = \int_{-\infty}^\infty f(t-s) g(s) \dd s \]
Man kann zeigen, dass das Integral existiert für alle $t$ und eine integable Funktion $f*g$ definiert. \\
\begin{bsp*}
	Signal $g(t)$ \\
	$\rightsquigarrow$ Signal $h(t) =$ Mittelwert von $g$ im Zeitintervall $[t-a,t]$.
	\[ h(t) = \frac{1}{a} \int_{t-a}^t g(s) \dd s \]
	ist eine Faltung von $g$ mit einer Funktion
	\[ \Phi(t) = \begin{cases}
		\frac{1}{a}	& 0 \leq t \leq a	\\
		0		&\text{sonst}	
	\end{cases} \]
	Tatsächlich gilt:
	\begin{gather*}
		(\Phi * g)(t) = \int_{-\infty}^\infty \Phi(t-s) g(s) \dd s \\
		\Phi(t-s) = \begin{cases}
			\frac{1}{a}	& 0 \leq t-s \leq a \iff t \geq s \geq t-a	\\
			0		&\text{sonst}
		\end{cases} \\
		(\Phi * g)(t) = \frac{1}{a} \int_{t-a}^t g(s) \dd s
	\end{gather*}
\end{bsp*}
\begin{satz*}
	\[ \widehat{f*g}(\omega) = \hat{f}(\omega) \hat{g}(\omega) \]
	\begin{bew}
		\[ \begin{split}
			\widehat{f*g}(\omega)	&= \int_{-\infty}^\infty (f*g)(t) e^{-\imath \omega t} \dd t \\
							&= \int_{-\infty}^\infty \left( \int_{-\infty}^\infty f(t-s) g(s) e^{-\imath\omega t} \dd s \right) \dd t \\
							&= \int_{-\infty}^\infty \left( \int_{-\infty}^\infty f(t-s) g(s) e^{-\imath\omega t} \dd t \right) \dd s \\
							&= \int_{-\infty}^\infty g(s) \underbrace{ \left(\int_{-\infty}^\infty f(t-s) e^{-\imath\omega t} \dd t \right) }_{\hat{f}(\omega) e^{-\imath\omega s}}  \dd s \\
							&= \hat{f}(\omega) \int_{-\infty}^\infty g(s) e^{-\imath\omega s} \dd s \\
							&= \hat{f}(\omega) \hat{g}(\omega)
		\end{split} \]
	\end{bew}
\end{satz*}
\begin{satz*}[note = Satz von Fubini]
	Wenn $F(t,s)$ stetig, $t \mapsto F(t,s)$ integrabel für alle $s$ , $s \mapsto F(t,s)$ integrabel für alle $t$, dann
	\[ \int_{-\infty}^\infty \left( \int_{-\infty}^\infty F(t,s) \dd t \right) \dd s = \int_{-\infty}^\infty \left( \int_{-\infty}^\infty F(t,s) \dd s \right) \dd t \]
\end{satz*}
\begin{satz*}[note = Unkehrsatz von Fourier]
	Wenn $f,\hat{f}$ integrabel, $f$ stetig, dann gilt
	\[ f(t) = \frac{1}{2\pi} \int_{-\infty}^\infty \hat{f}(\omega) e^{\imath\omega t} \dd \omega \]
	\begin{bew}
		\[ \begin{split}
				&\frac{1}{2\pi} \int_{-\infty}^\infty \hat{f}(\omega) e^{\imath\omega t} \dd \omega \\
			=	&\frac{1}{2\pi} \int_{-\infty}^\infty \left( \int_{-\infty}^\infty f(s) e^{-\imath\omega s} e^{\imath\omega t} \dd s \right) \dd \omega \\
			=	&\lim_{\epsilon \downarrow 0} \frac{1}{2\pi} \int_{-\infty}^\infty \left( \int_{-\infty}^\infty f(s) e^{-\imath\omega (s-t) - \epsilon \frac{\omega^2}{2}} \dd s \right) \dd \omega \\
			=	&\lim_{\epsilon \downarrow 0} \frac{1}{2\pi} \int_{-\infty}^\infty f(s) \left( \int_{-\infty}^\infty e^{-\imath\omega (s-t) - \epsilon \frac{\omega^2}{2}} \dd \omega \right) \dd s \\
			=	&\lim_{\epsilon \downarrow 0} \frac{1}{2\pi} \int_{-\infty}^\infty f(s) \sqrt{\frac{2\pi}{\epsilon}} e^{-\frac{(s-t)^2}{2\epsilon}} \dd s \\
			=	&\lim_{\epsilon \downarrow 0} \frac{1}{2\pi} \sqrt{\frac{2\pi}{\epsilon}} \int_{-\infty}^\infty f(t + \sqrt{\epsilon} u ) e^{-\frac{u^2}{2}} \sqrt{\epsilon} \dd u \\
			=	&f(t) \frac{1}{2\pi} \int_{-\infty}^\infty e^{-\frac{u^2}{2}} \dd u \\
			=	&f(t)
		\end{split} \]
	\end{bew}
\end{satz*}
\begin{bsp*}
	\begin{gather*}
		f(t) = e^{-a \frac{t^2}{2}} , a > 0 \\
		\hat{f}(\omega) = \sqrt{\frac{2\pi}{a}} e^{-\frac{\omega^2}{2a}}
	\end{gather*}
\end{bsp*}
\begin{gather*}
	f : \R \rightarrow \R \\
	a \cos(\omega t) + b \sin(\omega t) = Ae^{\imath\omega t} + \overline{A}e^{-\imath\omega t}
\end{gather*}
''harmonische Schwingungen mit Kreisperiode $\omega$ (Periode $T = \frac{2\pi}{\omega}$)''
\begin{gather*}
	f(t+T) = f(t) \quad f(t) = \sum_{n = -\infty}^\infty c_n e^{\frac{2\pi}{T} \imath n t}
	\intertext{Superposition von harmonischen Schwingungen mit Kreisfrequenz}
	n \cdot \omega_0 \quad \omega_0 = \frac{2\pi}{T} \\
	f \text{ allgemein aber integrabel} ( \Leftarrow f(t) \rightarrow \rightarrow 0 \text{ schnell genung für } t \rightarrow \infty ) \\
	f(t) = \int_{-\infty}^\infty \hat{f}(\omega) e^{\imath \omega t} \dd \omega \\
	c_n = \frac{1}{T} \int_{-\frac{T}{2}}^{\frac{T}{2}} f(t) e^{-\frac{2\pi\imath}{T} nt} \dd t \\
	\hat{f}(\omega) = \int_{-\infty}^\infty f(t) e^{-\imath\omega t} \dd t
\end{gather*}