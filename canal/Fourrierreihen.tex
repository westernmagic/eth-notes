\chapter{Fourierreihen}
\begin{def*}[note = Periodische Funktion , index = periodische Funktion , indexformat = {2!~1}]
	Eine Funktion $f: \R \rightarrow \R$ heisst \textbf{periodisch} mit Periode $T > 0$ (oder $T$-periodisch) falls
	\[ \forall t \in \R : f(t+T) = f(t) \]
\end{def*}
\begin{bsp*}
	\[ f(t) = \begin{cases} \sin(t) \\ \cos(t) \end{cases} \]
	ist $2\pi$-periodisch.
	
	Auch $\sin(nt)$ ist $2\pi$-periodisch für alle ganzen Zahlen $n$.
\end{bsp*}
\begin{bsp*}
	\[ f(t) = \sin\left( \frac{2\pi}{T} t \right) \]
	ist $T$-periodisch:
	\[ \sin\left( \frac{2\pi}{T} (t+T) \right) = \sin\left( \frac{2\pi}{T} t +2\pi \right) = \sin\left( \frac{2\pi}{T} t \right) \]
	Auch $\sin\left( \frac{2\pi}{T} nt \right)$, $\cos\left( \frac{2\pi}{T} nt \right)$ für $n \in \Z$
\end{bsp*}

\begin{satz*}[note = J. Fourier 1807]
	Jede\footnote{stimmt nicht; aber in den meisten Fällen} $T$-periodische Funktion lässt sich als trigonometrische Reihe
	\[ \begin{split}
		f(t) = \frac{a_0}{2}	&+ a_1 \cos\left( \frac{2\pi}{T} t \right) + a_2 \cos\left( \frac{2\pi}{T} 2t \right) + a_3 \cos\left( \frac{2\pi}{T} 3t \right) + \dots \\
						&+ b_1 \sin\left( \frac{2\pi}{T} t \right) + b_2 \sin\left( \frac{2\pi}{T} 2t \right) + b_3 \sin\left( \frac{2\pi}{T} 3t \right) + \dots \\
	\end{split} \]
	darstellen, mit reellen Koeffizienten $a_0 , a_1 , a_2 , a_3 , \dotsc , b_1 , b_2 , b_3 , \dotsc$
\end{satz*}

Fragen:
\begin{itemize}
	\item Für welche Funktionen gilt das?
	\item Wie bestimmit man die ''Fourierkoeffizienten'' $a_i, b_i$?
\end{itemize}
\begin{def*}[note = Fourierreihe , index = Fourier reihe , indexformat = {1!~.2 2!1.~}]
	Eine \textbf{Fourierreihe} ist eine Reihe der Form
	\[ \frac{a_0}{2} + \sum_{n=1}^\infty \left( a_n \sin\left( \frac{2\pi}{T} nt \right) + b_n \cos\left( \frac{2\pi}{T} nt \right) \right) \]
\end{def*}
Viel einfacher: komplexe Zahlen
\begin{gather*}
	e^{\imath \varphi} = \cos \varphi + \imath \sin \varphi \\
	e^{-\imath \varphi} = \cos \varphi - \imath \sin \varphi \\
	\cos \varphi = \frac{1}{2} \left( e^{\imath \varphi} + e^{-\imath \varphi} \right) \\
	\sin \varphi = \frac{1}{2\imath} \left( e^{\imath \varphi} + e^{-\imath \varphi} \right)
\end{gather*}
\begin{def*}[note = komplexe Fourierreihe , index = komplexe Fourier reihe , indexformat = {2!~.3!1~ 3!2.~!1~}]
	Eine \textbf{(komplexe) Fourierreihe} ist eine Reihe der Form
	\[ f(t) = \sum_{n=-N}^N c_n e^{\frac{2\pi\imath}{T} nt} , c_n \in \C \]
	Wenn sie konvergiert (d.h. wenn $\lim_{N \rightarrow \infty} \sum_{n=-N}^N c_n e^{\frac{2\pi\imath}{T} nt}$ existiert für alle $t \in \R$) definiert sie eine $T$-periodische Funktion $f: \R \rightarrow \C$ \\
	Durch Umformung lässt sich jede reelle Fourierreihe in diese Form bringen:
	\begin{gather*}
		c_0 = \frac{1}{2} a_0 \\
		c_n = \frac{1}{2} \left( a_n - \imath b_n \right) \\
		c_{-n} = \frac{1}{2} \left( a_n + \imath b_n \right) = \overline{c_n} , n > 0
	\end{gather*}
\end{def*} 
\begin{bsp*}
	Schreibe die $2\pi$-periodische Funktion $f(t) = \cos^3(t)$ als Fourierreihe.
	\[ \begin{split}
		\cos^3(t)	&= \frac{1}{8} \left( e^{\imath t} + e^{-\imath t} \right)^3 \\
				&= \frac{1}{8} \left( e^{3\imath t} + 3e^{\imath t} + 3e^{\imath t} + e^{-3\imath t} \right) \\
				&= \frac{3}{4} \cos(t) + \frac{1}{4} \cos(3t) \\
				&\text{ Fourierreihe mit } a_1 = \frac{3}{4} , a_3 = \frac{1}{4} , \text{ Rest } = 0
	\end{split}\]
\end{bsp*}

\section{Exkurs: Differential- und Integralrechnung von komplexwertigen Funktionen}
\begin{gather*}
	f: X \subset \R \rightarrow \C , f \mapsto u(t) + \imath v(t) \\
	u , v \text{ reellwertige Funktionen} \\
	\begin{split}
		f'(t)
			&= \lim_{h \rightarrow 0} \frac{f(t+h) - f(h)}{h} \\
			&= \lim_{h \rightarrow 0} \frac{u(t+h) - u(t)}{h} + \imath \frac{v(t+h) - v(t)}{h} \\
			&= u'(t) + \imath v'(t)
	\end{split} \\
	\intertext{Es gelten die gleichen Regeln wie bei reellwertigen Funktionen.}
	\int_a^b f(t) \dd t \coloneqq \int_a^b u(t) \dd t + \imath \int_a^b v(t) \dd t\\
	\intertext{Hauptsatz der Integralrechnung}
	\int_a^b f(t) = F(b) - F(a) \\
	F(t) = U(t) + \imath V(t) \\
	F'(t) = f(t) \\
	\\
	e^{at} , a \in \C \\
	e^{at} = 1 + at + \frac{(at)^2}{2!} + \dots \\
	\frac{d}{dt} e^{at} = a e^{at} \\
	\int e^{at} \dd t = \frac{1}{a} e^{at} + C , a \neq 0
\end{gather*}

\section{Fourierreihen}
Wie bestimmt man die Fourierkoeffizienten $c_n$ aus der Funktion $f(t) = \sum_{n = -\infty}^\infty c_n e^{\frac{2\pi\imath n}{T} t}$? \\
Grundlegende Identität: Orthogonalitätsrelation
\[ \frac{1}{T} \int_{-\frac{T}{2}}^{\frac{T}{2}} e^{\frac{2\pi\imath}{T} nt} e^{-\frac{2\pi\imath}{T} mt} \dd t = \begin{cases}
	1	&n = m	\\
	0	&n \neq m	
\end{cases} \]
\begin{bew}
	\begin{gather*}
		\begin{split}
			\frac{1}{T} \int_{-\frac{T}{2}}^{\frac{T}{2}} e^{\frac{2\pi\imath}{T} (n-m)t} \dd t
				&\overset{n \neq m}{=} \frac{1}{T} \left[ \frac{1}{\frac{2\pi\imath}{T} (n-m)} e^{\frac{2\pi\imath}{T} (n-m)t} \right]_{-\frac{T}{2}}^{\frac{T}{2}} \\
				&= \frac{1}{\frac{2\pi\imath}{T} (n-m)} ( e^{2\pi\imath (n-m) \frac{1}{2}} - e^{2\pi\imath (n-m) (-\frac{1}{2})} ) \\
				&= 0
		\end{split} \\
		n = m : \frac{1}{T} \int_{-\frac{T}{2}}^{\frac{T}{2}} 1 \dd t = 1 \blacksquare
	\end{gather*}
\end{bew}
Folgerung: Ein trigonometrisches Polynom ist eine Funktion der Form $f(t) = \sum_{n = -N}^{N} c_n e^{\frac{2\pi\imath}{T} nt}$ \\
Die $c_n$ bestimmt man aus $f$ durch
\[ \frac{1}{T} \int_{-\frac{T}{2}}^{\frac{T}{2}} f(t) e^{-\frac{2\pi\imath}{T} nt} \dd t = \sum_{n = -N}^N c_n \frac{1}{T} \int_{-\frac{T}{2}}^{\frac{T}{2}} e^{\frac{2\pi\imath}{T}} e^{-\frac{2\pi\imath}{T} nt} \dd t = c_m , ( -N \leq m \leq N ) \]

\section{Lineare Algebra}
Sei $V$ der Vektorraum über $\C$ aller trigonometrischen Polynomen mit Periode $T$. Basis von $V$: $e^{\frac{2\pi\imath}{T} nt}$ , $n \in \Z$ \\
Erinnerung: Ein Skalarprodukt auf einem komplexen Vektorraum $V$ ist eine Abblidung
\[ V \times V \rightarrow \C ; v , w \mapsto \langle v , w \rangle \]
\begin{enumerate}[label=\arabic*)]
	\item \[ \langle c_1 v_1 + c_2 v_2 , w \rangle = c_1 \langle v_1 , w \rangle + c_2 \langle v_2 , w \rangle \]
	\item \[ \langle v , w \rangle = \overline{\langle w , v \rangle} \]
	\item \[ \forall v \neq 0 : \langle v , v \rangle > 0 \]
\end{enumerate}
\begin{bsp*}
	\begin{gather*}
		V = \C^n \\
		\left\langle \begin{pmatrix} x_1 \\ \vdots \\ x_n \end{pmatrix} , \begin{pmatrix} y_1 \\ \vdots \\ y_n \end{pmatrix} \right\rangle =  x_1 \overline{y_1} + x_2 \overline{y_2} + \dots + x_n \overline{y_n}
	\end{gather*}
\end{bsp*}

Eine orthonormierte Basis von $V$ ist eine Basis $v_1, v_2 , \dotsc$ so, dass $\langle v_i , v_j \rangle = \delta_{ij} = \begin{cases} 1 &i = j \\ 0 &i \neq j \end{cases}$. Dann gilt für jedes $v \in V$
\[ v = \sum_n c_n v_n , c_n = \langle v , v_n \rangle \]
\[ e^{-\frac{2\pi\imath}{T} t} , 1 , e^{\frac{2\pi\imath}{T} t} , e^{-\frac{2\pi\imath}{T} 2t} \]
sind eine orthonormierte Basis des Vektorraums der trigonometrischen Polynome bezüglich des Skalarprodukts
\[ \langle f , g \rangle = \frac{1}{T} \int_{-\frac{T}{2}}^{\frac{T}{2}} f(t) \overline{g(t)} \dd t \]
(Das ist ein Skalarprodukt , 1) , 2) klar)
\begin{enumerate}[start=3,label=\arabic*)]
	\item \[ \langle f , f \rangle = \frac{1}{T} \int_{-\frac{T}{2}}^{\frac{T}{2}} f(t) \overline{f(t)} \dd t = \frac{1}{T} \int_{-\frac{T}{2}}^{\frac{T}{2}} \abs{f(t)}^2 \dd t > 0 \] wenn $f$ nicht identisch $0$ ist.
\end{enumerate}
Orthogonalitätsrelationen $\iff \langle e^{\frac{2\pi\imath}{T} t} , e^{\frac{2\pi\imath}{T} t} \rangle = \delta_{n,m}$ \\
Saubere Notation $e_n(t) \coloneqq e^{\frac{2\pi\imath}{T} nt}$
\begin{gather*}
	\langle e_n , e_n \rangle = \delta_{n,m} \\
	f = \sum_{n = -N}^N c_n e_n , c_n = \langle f , e_n \rangle
\end{gather*}
\begin{def*}[note = Fourierkoeffizient , index = Fourier koeffizient , indexformat = {1!~.2}]
	Sei $f$ eine $T$-periodische Funktion (d.h. $f(t+T) = f(t)$ für alle $t \in \R$) ; die \textbf{Fourierkoeffiziente} von $f$ sind
	\[ c_n = \frac{1}{T} \int_{-\frac{T}{2}}^{\frac{T}{2}} f(t) e^{-\frac{2\pi}{T} \imath nt} \dd t \]
\end{def*}
\begin{def*}[note = Lipschitz Stetigkeit , index = Lipschitz stegig , indexformat = {2!1~ }]
	$f$ heisst Lipschitz-stetig falls es ein $C > 0$ gibt so, dass $\abs{f(t) - f(t')} \leq c \abs{t-t'}$ für alle $t$ nahe genug zu $t'$.
\end{def*}
\begin{bsp*}
	Stetig differenzierbare Funktionen auf $[a,b]$ sind Lipschitz-stetig.
\end{bsp*}
\begin{satz*}
	Ist $f$ $T$-periodisch und Lipschitz-stetig, dann gilt
	\[ f(t) = \sum_{n = -\infty}^\infty c_n e^{\frac{2\pi\imath}{T} nt} \]
	für alle $t \in \R$, wobei
	\[ c_n = \frac{1}{T} \int_{-\frac{T}{2}}^{\frac{T}{2}} f(t) e^{-\frac{2\pi\imath}{T} nt} \dd t \]
%	\begin{bew}
%		Siehe Skript (Blatter)
%	\end{bew}
\end{satz*}\todo{Fix}
Daraus folgt dasselbe für reelle Fourierreihen: \\
$f$ reelle $T$-periodische Lipschitzstegige Funktion. Dann gilt:
\[ f(t) = \frac{a_0}{2} + \sum_{n=1}^\infty \left( a_n \cos\left( \frac{2\pi}{T} nt \right) + b_n \sin\left( \frac{2\pi}{T} nt \right) \right) \]
wobei $a_n = c_n + c_{-n} = \frac{2}{T} \int_{-\frac{T}{2}}^{\frac{T}{2}} f(t) \cos\left( \frac{2\pi}{T} nt \right) \dd t$, $b_n = -\frac{c_n - c_{-n}}{\imath} = \frac{2}{T} \int_{-\frac{T}{2}}^{\frac{T}{2}} f(t) \sin\left( \frac{2\pi}{T} nt \right) \dd t$ \\
\begin{bem}
	Wenn $f$ eine gerade Funktion ist, d.h. $f(t) = f(-t)$, dann $b_n \frac{2}{T} \int_{-\frac{T}{2}}^{\frac{T}{2}} \underbrace{f(t) \sin\left( \frac{2\pi}{T} nt \right)}_{\text{ungerade Funktion}} \dd t = 0$ \\
	Man erhält eine ''Kosinusreihe''
	\[ \frac{a_0}{2} + \sum_{n=1}^\infty a_n \cos\left( \frac{2\pi}{T} nt \right) \]
	Analog: Falls $f$ ungerade ist (d.h $f(-t) = -f(t)$), dann hat $f$ eine Sinusreihe
	\[ f(t) = \sum_{n=1}^\infty b_n \sin\left( \frac{2\pi}{T} nt \right) \]
\end{bem}
\begin{bsp*}
	$f(t) = 1 - \frac{2}{\pi} \abs{t}$ für $-\pi \leq t \leq \pi$ ; $2\pi$-periodisch fortgesetzt ; $f(t) = f(-t)$
	\begin{gather*}
		f(t) = \frac{a_0}{2} + \sum_{n=1}^\infty a_n \cos( nt ) \\
		a_n = \frac{2}{2\pi} \int_{-\pi}^{\pi} ( 1 - \frac{2}{\pi} \abs{t} ) \cos( nt ) \dd t
	\end{gather*}
\end{bsp*}

Falls $f$ z.B. Lipschitz-stetig ist, dann konvergiert die Fourier-Reihe in jedem Punkt $t$ gegen $f$, d.h. die Folge der Partialsummen
\[ s_N(t) = \frac{a_0}{2} + \sum_{n=1}^N ( \dots ) \]
erfüllt $\lim_{N \rightarrow \infty} s_N(t) = f(t)$ für alle $t \in \R$

\begin{bsp*}
	\begin{gather*}
		f(t) = 1 - \frac{2}{\pi} \abs{t} , -\pi \leq t \leq \pi , 2\pi\text{-periodisch fortgesetzt} \\
		f(t) = f(-t) \implies b_n = 0 \\
		\begin{split}
			a_n	&= \frac{2}{2\pi} \int_{-\pi}^\pi f(t) \cos(nt) \dd t \\
				&= \frac{2}{2\pi} \cdot 2 \int_0^\pi f(t) \cos(nt) \dd t \\
				&= \frac{2}{\pi} \int_0^\pi (1-\frac{2}{\pi} t) \cos(nt) \dd t \\
				&= \frac{2}{\pi} \left( \int_0^\pi \cos(nt) \dd t + \frac{2}{\pi} \int_0^\pi t \cos(nt) \dd t \right) \\
				&= \frac{2}{\pi} \left( \left[ \frac{\sin(nt)}{n} \right]_0^\pi - \frac{2}{n} \left[ \frac{1}{n} t \sin(nt) + \frac{1}{n^2} \cos(nt) \right]_0^\pi \right) \\
				&= \frac{4}{\pi^2} \frac{1}{n^2} ( -\underbrace{\cos(n\pi)}_{(-1)^n} + 1 ) \\
				&= \begin{cases}
					0					&n \text{ gerade} 	\\
					\frac{8}{\pi^2} \frac{1}{n^2}	&n \text{ ungerade}	
				\end{cases}
		\end{split} \\
		\begin{split}
			1 - \frac{2}{\pi} \abs{t}	&= \frac{8}{\pi^2} \sum_{\substack{n=0\\n \text{ ungerade}}}^\infty \frac{1}{n^2} \cos(nt) , -\pi \leq t \leq \pi \\
								&= \frac{8}{\pi^2} \sum_{k=0}^\infty \frac{1}{(2k+1)^2} \cos((2k+1) t)
		\end{split}
		\intertext{Es folgt}
		1 = \frac{8}{\pi^2} \sum_{\text{ungerade Zahlen } n \geq 1} \frac{1}{n^2} \quad (t=0)
		\intertext{Also:}
		1 + \frac{1}{3^2} + \frac{1}{5^2} + \frac{1}{7^2} + \dots = \frac{\pi^2}{8} \quad \text{(Euler)}
	\end{gather*}
\end{bsp*}
\begin{bsp*}
	Fourier-Reihe einer nicht stetigen Funktion, ''Rechtecksignal''
	\begin{gather*}
		f(t) = \begin{cases}
			1	& 0 \leq t < \frac{T}{2}		\\
			-1	& -\frac{T}{2} \leq t < 0	
		\end{cases}, T\text{-periodisch fortgesetzt} \\
		\sum_{n=1}^\infty b_n \sin\left( \frac{2\pi}{T} nt \right) \\
		\begin{split}
			b_n	&= \frac{2}{T} \int_{-\frac{T}{2}}^{\frac{T}{2}} f(t) \sin\left( \frac{2\pi}{T} nt \right) \dd t \\
				&= \frac{2}{T} \cdot 2 \int_0^{\frac{T}{2}} 1 \sin\left( \frac{2\pi}{T} nt \right) \dd t \\
				&= \frac{4}{T} \left[ -\frac{T}{2\pi n} \cos\left( \frac{2\pi}{T} nt \right) \right]_0^{\frac{T}{2}} \\
				&= \frac{2}{\pi n} ( - (-1)^n + 1 ) \\
				&= \begin{cases}
					0			& n \text{ gerade}	\\
					\frac{4}{\pi n}	& n \text{ ungerade}
				\end{cases}
		\end{split}
		\intertext{Partialsumme ($n$ ungerade)}
		s_N(t) = \frac{4}{\pi} \left( \sin\left( \frac{2\pi}{T} 3t \right) \right)+ \dots + \frac{1}{N} \sin\left( \frac{2\pi}{T} Nt \right)
	\end{gather*}
\end{bsp*}
\begin{gather*}
	f(t+T) = f(t) \forall t \in \R \\
	c_n = \frac{1}{T} \int{-\frac{T}{2}}^{\frac{T}{2}} f(t) e^{-\frac{2\pi\imath}{T} nt} \dd t = \langle f , e_n \rangle , e_n(t) = e^{\frac{2\pi\imath}{T} nt}
	\intertext{Partialsummen der Fourier-Reihe von $f$}
	s_N(t) = \sum_{n=-N}^N c_n e^{\frac{2\pi\imath}{T} nt}
	\intertext{Falls $f$ z.B. stetig differenzierbar}
	f(t) = \lim_{N \rightarrow \infty} s_N(t) = \underbrace{\sum_{n=-\infty}^\infty c_n e^{\frac{2\pi\imath}{T} nt}}_{\text{Fourierreihenentwickelung von } f}
\end{gather*}
\begin{satz*}
	Falls $f$ stetig differenzierbar ausser an endlich vielen Sprungstellen ist, wobei der Linke und rechte Grenzwert von $f$ und $f'$ an den Sprungstellen existiert, dann gilt immer noch
	\[ \lim_{N \rightarrow \infty} s_N(t) = f(t) \]
	ausser für $t$ Sprungstelle. \\
	Für $t_0$ eine Sprungstelle
	\[ \lim_{N \rightarrow \infty} s_N(t_0) = \frac{1}{2} ( \lim_{t \rightarrow t_0^-} f(t) + \lim_{t \rightarrow t_0^+} f(t) ) \]
\end{satz*}
\begin{bem}
	Wenn $g(t)$ $T$-periodisch ist dann
	\[ \forall a , b : b - a = T \implies \int_{-\frac{T}{2}}^{\frac{T}{2}} g(t) \dd t \int_a^b g(t) \dd t \]
	Es folgt insbesondere
	\[ c_n = \frac{1}{T} \int_0^T f(t) e^{-\frac{2\pi\imath}{T} nt} \dd t \]
\end{bem}

\section{Überschwingungen (Gibbsphänomen)}
\begin{bsp*}
	Siehe oben.
	\begin{gather*}
		f(t) = \begin{cases}
			1	& 0 \leq t < \frac{T}{2}		\\
			-1	& -\frac{T}{2} \leq t < 0	
		\end{cases}, T\text{-periodisch fortgesetzt} \\
		s_N(t) = \frac{4}{\pi} \left( \sin\left( \frac{2\pi}{T} 3t \right) \right)+ \dots + \frac{1}{N} \sin\left( \frac{2\pi}{T} Nt \right)
		\intertext{Wir betrachten $s_N(\frac{a}{N}$ für grosse $N$ und $0 \leq a \leq 10$. Für $T=2\pi$}
		s_N\left( \frac{a}{N} \right) = \frac{4}{\pi} \sum_{\substack{n=1\\n \text{ ungerade}}}^N \frac{1}{n} \sin\left( n \frac{a}{N} \right) = \frac{2}{\pi} \sum_{\substack{n=1\\n \text{ ungerade}}}^N \frac{\sin\left( \frac{na}{N} \right)}{\frac{na}{N}} \cdot 2\frac{2}{N}
		\intertext{Riemannsumme für das Integral $\frac{2}{\pi} \int_0^a \frac{\sin x}{x} \dd x$}
		\lim_{N \rightarrow \infty} s_N\left( \frac{a}{N} \right) = \frac{2}{\pi} \int_0^a \frac{\sin x}{x} \dd x = \frac{2}{\pi} Si(a) \\
		Si(a) = \int_0^a \frac{\sin x}{x} \dd x \qquad \text{''Integralsinus''} \\
		Si'(a) = \frac{\sin a}{a} = 0 \text{ für } a = n\pi , n \in \Z^{>0}
		\intertext{Für grosse $N$ sieht also die Partialsumme $s_N(t)$ für $t \approx 0$ folgendermassen}
		\frac{2}{\pi} Si(\pi) \approx 1.8
		\intertext{Bei $s_N(t)$ ist der Sprung um ca. $20\%$ grösser als bei $f(t)$}
	\end{gather*}
\end{bsp*}

\section{Differenzierbarkeit}
Prinzip: je glatter die Funktion desto schneller konvergiert die Foureierreihe. \\
Sei z.B. $f$ stetig differenzierbar
\begin{gather*}
	\begin{split}
		c_n	&= \frac{1}{T} \int_{-\frac{T}{2}}^{\frac{T}{2}} f(t) e^{-\frac{2\pi\imath n}{T} t} \dd t \\
			&\overset{n \neq 0}{=} \underbrace{0}_{\substack{[T\text{-periodisch}]_{-\frac{T}{2}}^{\frac{T}{2}}}} -\frac{1}{T} \int_{-\frac{T}{2}}^{\frac{T}{2}} f'(t) \left( -\frac{T}{2\pi\imath n} \right) \underbrace{e^{-\frac{2\pi\imath n}{T} t}}_{\abs{\cdot}=1} \dd t \\
			&= \frac{1}{2\pi\imath n} \int_{-\frac{T}{2}}^{\frac{T}{2}} f'(t) \underbrace{e^{-\frac{2\pi\imath n}{T} t}}_{\abs{\cdot}=1}
	\end{split} \\
	\abs{c_n} \leq \frac{1}{\abs{n}} \frac{T}{2\pi} \underbrace{\max_t \abs{f'(t)}}_{\cos t}
	\intertext{Der $n$-te Fourierkoeffizient von $f'(t)$ ist $d_n = \frac{2\pi\imath}{T} n c_n$. Dies erhält man auch durch Differenzieren der Fouierreihe.}
	f(t) = \sum_{n = -\infty}^\infty c_n e^{\frac{2\pi\imath n}{T} t} \\
	f'(t) = \sum_{n = -\infty}^\infty \frac{2\pi\imath n}{T} c_n e^{\frac{2\pi\imath n}{T} t}
	\intertext{Wenn $f$ $k$-mal stetig differenzierbar ist}
	c_n = \frac{1}{2\pi\imath n} \int_{-\frac{T}{2}}^{T}{2} f'(t) e^{-\frac{2\pi\imath n}{T}} \dd t = \frac{T}{(2\pi\imath n)^2} \int_{-\frac{T}{2}}^{T}{2} f''(t) e^{-\frac{2\pi\imath n}{T}} \dd t \\
	\abs{c_n} \leq \frac{T^k}{(2\pi\abs{n})^k} \max_{-\frac{T}{2} \leq t \leq \frac{T}{2}} \abs{f^{(k)}(t)} = \frac{\text{const.}}{\abs{n}^k}
\end{gather*}
\begin{bem}
	\[ \abs{\int_a^b f(t) \dd t} \leq \max_{a \leq t\leq b} \abs{f(t)} \cdot (b-a) \]
\end{bem}

\section{Parsevalidentität}
Lineare Algebra: $V$ komplexer Vektorraum mit Skalarprodukt, $e_1 , e_2 , \dotsc$ orthonormierte Basis
\begin{gather*}
	V \ni v = \sum_n \langle v , e_n \rangle e_n \\
	\| v \| = \sqrt{\langle v , v \rangle} \qquad\text{Norm von } v \\
	\| v \| = \sqrt{\sum_n \abs{ \langle v , e_n \rangle }^2}
\end{gather*}
\begin{bew}
	\[ \begin{split}
		\langle v , v \rangle	&= \langle \sum_n \langle v , e_n \rangle e_n , \sum_m \langle v , e_m \rangle e_m \rangle \\
						&= \sum_{n,m} \langle v , e_n \rangle \overline{\langle v , e_m \rangle} \underbrace{\langle e_n , e_m \rangle}_{\delta_{n,m}} \\
						&= \sum_n \abs{ \langle v , e_n \rangle }^2
	\end{split} \]
\end{bew}
\begin{gather*}
	V = \{ \text{trigonometrische Polynome} \} \\
	f = \sum c_n e_n \text{ mit } e_n(t) = e^{\frac{2\pi\imath}{T} nt} \text{ und } c_n = \langle f , e_n \rangle \\
	\| v \|^2 = \sum_n \abs{ \langle v , e_n \rangle}^2 \\
	\underbrace{\frac{1}{T} \int_{-\frac{T}{2}}^{\frac{T}{2}} \abs{f(t)}^2 \dd t}_{\langle f , f \rangle} = \sum_{n = -\infty}^\infty \abs{c_n}^2
\end{gather*}
\textbf{Parsevalidentität}: gilt allgemein für stetige Funktionen $f$

\section{Eine Anwendung von Fourier-Reihen}
\begin{bsp*}
	Wärmeleitung auf einem Ring \\
	$f( \varphi )$ = Anfangstemeratur zur Zeit $t=0$ im Punkt $(r \cos \varphi , r \sin \varphi)$ \\
	Zeitevolution der Temperatur wird durch die Wärmeleitungsgleichung
	\[ \frac{\partial u}{\partial t}(t, \varphi) = D \frac{\partial^2 u}{\partial \varphi^2}( t , \varphi ) \quad D > 0 \]
	gegeben $u(t,\varphi)$ Temperatur zur Zeit $t$ \\
	Anfangsbedingung: $u(0,\varphi) = f(\varphi)$ sei gegeben
	\begin{gather*}
		f( \varphi + 2\pi ) = f( \varphi ) , u( t , \varphi ) = u( t ,\varphi + 2\pi ) \\
		f( \varphi ) = \sum_{n = -\infty}^\infty c_n e^{\imath n \varphi} \quad c_n \text{ gegeben} \\
		u( t , \varphi ) = \sum_{n = -\infty}^\infty c_n(t) e^{\imath n \varphi} \\
		\intertext{$c_n(t)$ aus $c_n = c_n(0)$ bestimmen? Wärmegleichung}
		\sum_{n = -\infty}^\infty \frac{\dd c_n(t)}{\dd t} e^{\imath n \varphi} = D \sum_{n = -\infty}^\infty (\imath n)^2 c_n(t) e^{\imath n \varphi} \\
		\sum_{n = -\infty}^\infty \left( \frac{\dd c_n(t)}{\dd t} + D n^2 c_n(t) \right) e^{\imath n \varphi} = 0 \text{ für alle } t > 0 , \varphi \\
		\frac{\dd c_n(t)}{\dd t} = -D n^2 c_n(t) \text{ für alle } n \\
		c_n(0) = c_n \\
		c_n(t) = e^{-D n^2 t} c_n \\
		\text{Lösung: } u( t , \varphi ) = \sum_{n = -\infty}^\infty c_n e^{-D n^2 t} e^{\imath n \varphi} \\
		\text{Sei } f(\varphi) = \begin{cases}
			1	& 0 < \varphi \leq \pi		\\
			-1	& -\pi < \varphi \leq 0	
		\end{cases} \\
		\begin{split}
			\implies f(\varphi)	&= \frac{4}{\pi} \sum_{\substack{n=1 \\ n \text{ ungerade}}}^\infty \frac{1}{n} \sin(n\varphi) \\
							&= \frac{4}{\pi} \sum_{\substack{n=1 \\ n \text{ ungerade}}}^\infty \frac{1}{n} \frac{1}{2\imath} ( e^{\imath n \varphi} - e^{-\imath n \varphi} ) \\
							&= \frac{4}{\pi} \sum_{\substack{n=-\infty \\ n \text{ ungerade}}}^\infty \frac{1}{n} \frac{1}{2\imath} e^{\imath n \varphi}
		\end{split} \\
		\implies c_n(0) = \frac{4}{\pi} \frac{1}{n} \frac{1}{2\imath} \\
		c_n(t) = \frac{4}{\pi} \frac{1}{n} \frac{1}{2\imath} e^{-D n^2 t} \\
		\begin{split}
			\implies u(t,\varphi)	&= \frac{4}{\pi} \sum_{n=-\infty}^{\infty} \frac{1}{n} \frac{1}{2\imath} e^{-D n^2 t} e^{\imath n \varphi} \\
							&= \frac{4}{\pi} \sum_{\substack{n=1 \\ n \text{ ungerade}}}^\infty \frac{1}{n} e^{-D n^2 t} \sin(n\varphi)
		\end{split}
	\end{gather*}
	Bemerkung:
	\begin{itemize}
		\item $c_n(t)$ ist exponentiell klein für $n$ gross falls $t > 0$
		\item $u(t,\varphi)$ glatt als Funktion von $\varphi$
		\item Für $t < 0$ konvergiert die Reihe im Allgemeinen nicht, d.h. $c_n(t) \rightarrow \infty$ für $n \rightarrow \infty$
	\end{itemize}
\end{bsp*}

\section{Partialsummen als beste Näherung im quadratischen Mittel}
$f$ $T$-periodisch \\
\begin{gather*}
	e_n(t) = e^{\frac{2\pi\imath}{T} nt} \\
	c_n = \scal{ f , e_n } = \frac{1}{T} \int_{-\frac{T}{2}}^{\frac{T}{2}} f(t) \overline{e_n(t)} \dd t \\
	s_N(t) = \sum_{n = -N}^N c_n e_n(t) \\
	\begin{split}
		V_N	&= \{ \text{ trigonometrische Polynome von Grad } \leq N \} \\
			&\coloneqq \left\{ \text{ Linearkombinationen } \sum_{n = -N}^N d_n e_n \text{ von } e_{-N} , e_{-N+1} , \dotsc , e_N \right\}
	\end{split}
\end{gather*}
$V_N$ Unterraum des Raumes der $T$-periodischen Funktionen. \\
Die ortogonale Projektion eines Vektors $f$ auf dem Unterraum $V_N$ ist $\sum_{n = -N}^N \scal{f,e_n} e_n$ und es ist der Punkt in $V_N$ der am nächsten bezüglich des Abstands $\norm{f-p} = \sqrt{\scal{f-p,f-p}}$ zu $f$ liegt. Es gilt nämlich für $p = \sum_{n = -N}^N d_n e_n \in V_N$
\begin{gather*}
	\begin{split}
		\norm{f-p}	&= \scal{f - \sum_{n = -N}^N d_n e_n , f - \sum_{m = -N}^N d_m e_m} \\
				&= \scal{f,f} - \sum_{m = -N}^N \overline{d_m} \scal{f,e_m} - \sum_{n = -N}^N d_n \scal{e_n,f} + \sum_{n = -N}^N d_n \overline{d_n} \\
				&= \scal{f,f} + \sum_{n = -N}^N ( - \overline{d_n} c_n - d_n \overline{c_n} + d_n \overline{d_n} ) \\
				&= \scal{f,f} + \sum_{n = -N}^N ( -c_n \overline{c_n} + (d_n - c_n)(\overline{d_n} - \overline{c_n})) \\
				&= \scal{f,f} - \sum_{n = -N}^N \abs{c_n}^2 + \sum_{n = -N}^N \abs{d_n - c_n}^2 = 0 \iff c_n \\
				&= d_n
	\end{split}
\end{gather*}
$\norm{f-p}$ ist am kleinsten, wenn $p = s_N$

Allgemein: $W$ unitärer Vektorraum d.h. ein komplexer Vektorraum mit einem Skalarprodukt [ z.B. $W = \{$ stetige, $T$-periodische Funktionen $\}$ , $\scal{f,g} = \frac{1}{T} \int_{-\frac{T}{2}}^{\frac{T}{2}} f(t) \overline{g(t)} \dd t$ ]. $V \subset W$ endlichdimensionaler Unterraum [ z.B $V = \{$ trigonometrische Polynome von Grad $\leq N \}$ ]. $e_{-N} , \dotsc , e_N$ orthonormierte Basis von $V$ [ im Beispiel $e_n(t) = e^{\frac{2\pi\imath}{T} nt}$ ]. Orthogonalprojektion von $f \in W$ auf $V$:
\[ \pi(f) = \sum_{n = -N}^N \scal{f,e_n} e_n \]
$\pi(f)$ ist die beste Näherung von $f$ in $V$, d.h. $\pi(f)$ ist der Vektor $p$ in $V$, der den kleinsten Abstand $\norm{f-p} = \sqrt{\scal{f-p,f-p}}$ zu $f$ hat.

Für Fourierreihen: Der Abstand zwischen stitigen $T$-periodische Funktionen $f$ und einem beliebigen trigonometrischen Polynom $p$ von  Grad $\leq N$ ist am kleinsten wenn $p = S_N = \sum_{n = -N}^N c_n e^n , c_n = \scal{f,e_n} = \frac{1}{T} \int_{-\frac{T}{2}}^{\frac{T}{2}} f(t) e^{-\frac{2\pi\imath}{T} nt} \dd t$. $\sqrt{\frac{1}{T} \int_{-\frac{T}{2}}^{\frac{T}{2}} \abs{f(t) - p(t)}^2 \dd t}$ ist am kleinsten, wenn $p(t) = S_N(t)$
