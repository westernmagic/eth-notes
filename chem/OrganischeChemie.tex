\chapter{Organische Chemie}
\begin{longtabu}spread 0cm [l]{ | *{11}{ X[-1] | } }
	\hline
	\multicolumn{2}{ | c | }{} &\multicolumn{2}{ c | }{Nomenklatur} &\multicolumn{4}{ c | }{Identifiktation} & & & \\ \tabucline{3-8}
	\multicolumn{2}{ | c | }{Stoffklasse} &Präfix &Suffix &Fuktionelle Gruppe &Atom &Hybridisierung &Oxidationszahl &Eigenschaften &Herstellung &Reaktionen \hline \endhead
		&Alkane
			&-
			&-an
			&Einfachbindung
			&\ce{C}
			&sp\textsuperscript{3}
			&
			&
			&
			&\begin{itemizet}
				\item Verbrennung / Oxidation
				\item Radikale Halogenierung
			\end{itemizet}
		\\ \tabucline{2-}
		&Cycloalkane
			&Cyclo-
			&-an
			&Ring
			&\ce{C}
			&
			&
			&
			&
			&
		\\ \tabucline{2-}
		&Alkene
			&
			&-en
			&Doppelbindung
			&\ce{C}
			&sp\textsuperscript{2}
			&
			&
			&
			&\begin{itemizet}
				\item Katalytische Hydrierung (cis-/syn-Addition)
				\item Brom-Addtion (Elektrophile-/trans-/anti-Addtiton)
				\item Bromwasserstoff-Addition (\ce{HX})
				\item Radikalische Polymerisation
			\end{itemizet}
		\\ \tabucline{2-}
		&Alkine
			&
			&-in
			&Dreifachbindung
			&\ce{C}
			&sp
			&
			&
			&
			&\begin{itemizet}
				\item Katalytische Hydrierung
				\item Halogenwasserstoff-Addition (\ce{HX})
				\item Salzbildung (endständige Alkine mit starken Basen)
			\end{itemizet}
		\\ \tabucline{2-}
		&Aromate
			&
			&
			&\begin{itemizet}
				\item planar
				\item cyclisch
				\item konjugierter $\pi$-Bindungssystem mit $4n+2 , n \in \Z_{\geq 1}$ $\pi$-Elektronen
			\end{itemizet}
			&\ce{C}
			&sp\textsuperscript{2}
			&
			&
			&
			&Elektrophile aromatische Addition
		\\ \tabucline{2-}
		&Alkohole
			&hydroxy-
			&-ol
			&\chemfig{-OH}
			&\ce{C}
			&sp\textsuperscript{3}
			&+I
			&\begin{itemizet}
				\item \ce{H}-Brücken (Donor \& Acceptor)
					\begin{itemizet}[ label = $\implies$ ]
						\item hohe Sdp.
						\item Viskozität von Polyolen
					\end{itemizet}
				\item pKs = 16-18
				\item farblos
			\end{itemizet}
			&\begin{itemizet}
				\item Biologische Prozesse
				\item Hydratisierung von Alkanen
				\item Reduktion von Aldehyden/Ketonen
				\item \textbf{nicht durch direkte Substitution von \ce{H} in Alkanen}
			\end{itemizet}
			&\begin{itemizet}
				\item Elimination mit \ce{H+} an benachbartem Atom $\implies$ Alken
				\item Bildung von Estern (mit Carbonsäuren)
				\item Acetatbildung (mit Aldehyden/Ketonen)
				\item Oxidation zu Aldehyden/Ketonen (aus prim./sek. Alk.)
				\item Substitution mit einem Nucleophil nach Aktivierung
			\end{itemizet}
		\\ \tabucline{2-}
		&Ether
			&
			&-ether
			&\chemfig{C-O-C}
			&\ce{C}
			&sp\textsuperscript{3}/sp\textsuperscript{2}/sp
			&+I
			&\begin{itemizet}
				\item flüchtig
				\item leicht brennbar / hochexplosiv mit \ce{O2}
				\item Lipid-LSM
				\item inerter LSM für viele Reaktionen (ausser Oxiran)
			\end{itemize}
			&
			&
		\\ \tabucline{2-}
		&Peroxide
			&
			&
			&
			&
			&
			&
			&\begin{itemizet}
				\item explosiv
				\item schwache \chemfig{O-O} Bindung (homolytische Spaltung)
			\end{itemizet}
			&\begin{itemizet}
				\item aus Ethern mit $\ce{O2} + hv$ (Radikalreaktion)
					\begin{itemizet}[ label = $\implies$ ]
						\item Ether in brauen Flaschen aufbewahren
					\end{itemizet}
				\item durch Umsetzung von Hydroperoxid-Ionen (\ce{HOO-}) mit Halogenalkanen
			\end{itemizet}
			&\begin{itemizet}
				\item katalytisch zu Alk. + \ce{O2} abgebaut
			\end{itemizet}
		\\ \tabucline{2-}
		&Phenole
			&
			&
			&\chemfig{Ar-OH}
			&\ce{C}
			&sp\textsuperscript{2}
			&+I
			&\begin{itemizet}
				\item farblos
				\item hohe Schmp. \& Sdp.
				\item pKs $\approx$ 10
			\end{itemizet}
			&
			&\begin{itemizet}
				\item elektrophile aromatische Substitution
			\end{itemizet}
		\\ \tabucline{2-}
		&Enole
			&
			&
			&\chemfig{C(=[:120])(-[:-120])-OH}
			&\ce{C}
			&sp\textsuperscript{2}
			&+I
			&\begin{itemizet}
				\item pKs = 10-12
				\item unstabil $\implies$ Carbonyl-Tautomerie
				\item Enolate (deprotonierte Form): delok. Lad.
			\end{itemizet}
			&
			&
		\\ \hline
	\multirow{2}*{Carbonylverbindungen}
		&Aldehyde
			&oxo-
			&-al
			&\chemfig{R-C(=[:60]O)(-[:-60]H)}
			&\multirow{2}*{\ce{C}}
			&\multirow{2}*{sp\textsuperscript{2}}
			&\multirow{2}*{+II}
			&\begin{itemizet}
				\item bathochrone Verschiebung (grösseres $\lambda$)
			\end{itemizet}
			&\begin{itemizet}
				\item Ox. von prim. Alk,
				\item Red. von Carbonsäuren
			\end{itemizet}
			&\begin{itemizet}
				\item Nachweis:
					\begin{itemizet}
						\item Tollens-Reaktion (Silberspiegel): Ox. mit \ce{[Ag(NH3)2]+}
						\item Fehling-Reaktion: Ox. mit \ce{Cu{2+}}
					\end{itemizet}
				\item Ox. zu Carbonsäuren
				\item Addition mit Nucleophilen
			\end{itemizet}
		\\ \tabucline{2-}
		&Ketone
			&oxo-
			&-on
			&\chemfig{C(-[:120]R)(-[:-120]R')=O}
			&
			&
			&
			&\begin{itemizet}
				\item bathochrome Verschiebung (grösseres $\lambda$)
				\item stabiler als Ald.
				\item gute LSM für org. Stoffe
				\item \ce{Nu-} langsamer als bei Ald.
			\end{itemizet}
			&\begin{itemizet}
				\item Ox. von sek. Alk.
			\end{itemizet}
			&
		\\ \hline
		&Carbonsäuren (CS)
			&
			&
			&\schemestart
				\chemfig{C(-[:180])(=[:60]\lewis{02,O})-[:-60]\lewis{46,O}-H}
				\arrow{<->}
				\chemfig{C(-[:180])(-[:60]\lewis{137,O}^{-})=[:-60]\lewis{5,O}^{+}-H}
			\schemestop
			&\ce{C}
			&sp\textsuperscript{2}
			&+III
			&\begin{itemizet}
				\item Ox. von Ald.
			\end{itemizet}
			&\begin{itemizet}
				\item hydrophil
				\item niedere CS \ce{H2O} lös.
				\item \ce{H}-Brücken $\implies$ Dimere
				\item pKs $\leq$ 5
					\begin{itemizet}
						\item Elektronenakeptoren erhöhen
						\item Elektronendonatoren erniedrigen
					\end{itemizet}
				\item[!] 3-Oxocarbonsäuren instabil (Zerfallen in Raumtemperatur zu Ketonen + \ce{C)2})
			\end{itemizet}
			&\begin{itemizet}
				\item als Säure mit \ce{H2O}
				\item Salzbildung mit Basen (vollständige Dissoziation in \ce{H2O} Lsg.
					\begin{itemizet}[ label = $\drsh$ ]
						\item Micellen / Emulgation
					\end{itemizet}
			\end{itemizet}
		\\ \hline
	\multirow{6}*{Carbonsäurenderivate}
		&\multirow{2}*{Carbonsäurechloride}
			&\multirow{2}*{}
			&\multirow{2}*{}
			&\multirow{2}*{}
			&\multirow{2}*{}
			&\multirow{2}*{}
			&\multirow{2}*{}
			&\multirow{2}*{}
			&\multirow{2}*{mit Thionylchlorid , Phosphorpentachlorid, etc.}
			&\begin{itemizet}
				\item Acylierungsmittel (aktivierte Carbonsäurederivate)
			\end{itemizet}
		\\ \tabucline{11-}
		&
			&
			&
			&
			&
			&
			&
			&
			&
%			&\multirow{5}*{\begin{itemizet}
%				\item Reaktion mit Nuclephilen (sortiert nach absteigender Reaktivität)
%				\item Protonierung mit starken Säuren
%				\item Reaktion mit Starken Basen
%			\end{itemizet}}
		\\ \tabucline{2-}
		&Carbonsäureanhydride
			&
			&
			&
			&
			&
			&
			&\begin{itemizet}
				\item 2 Carbonsäuren unter Abspaltung \ce{H2O}
				\item Carbonsäurechlorid + Carbonsäure
				\item aus Dicarbonsäuren $\rightarrow$ cyclische Anhydride
			\end{itemizet}
			&\begin{itemizet}
				\item Acylierungsmittel (\enquote{Acylgruppen-Überträger})(auf \ce{OH-} / \ce{NH2} übertragen)
			\end{itemizet}
			&
		\\ \tabucline{2-}
		&Thioester
			&
			&
			&
			&
			&
			&
			&\begin{itemizet}
				\item aus Carbonsäurechloride/-anhydride mit Thiolen/Thiophenolen
			\end{itemizet}
			&\begin{itemizet}
				\item \enquote{aktivierte Carbonsäure}
			\end{itemizet}
			&
		\\ \tabucline{2-}
		&Carbonsäureester
			&
			&
			&
			&
			&
			&
			&\begin{itemizet}
				\item Carbonsäurechloride/-anhydride mit Alk./Phenole
				\item Fischer-Verseterung: Carbonsäure + Alk. (Säure Kat.)
			\end{itemizet}
			&\begin{itemizet}
				\item relativ flüchtig
				\item riechen fruchtig
			\end{itemizet}
			&%todo
		\\ \tabucline{2-}
		&Carbonsäureamide
			&
			&
			&
			&
			&
			&
			&\begin{itemizet}
				\item Carbonsäurechloride/-anhydride/Thioester mit Ammoniak/prim./sek. Aminen
				\item \textbf{nicht aus Carbonsäuren mit Aminen} $\rightarrow$ Salzen (Ammoniumcaboxylate)
				\item Lactane = cyclische Amide aus Amincarbonsäuren
			\end{itemizet}
			&\begin{itemizet}
				\item relativ schwerflüchtig (insbesondere un-/monosubst. $\Leftarrow$ \ce{H}-Brücken
				\item neutrale Verbindungen $\leftarrow$ delok. \ce{e-}-Paar von \ce{N}
			\end{itemizet}
			&%todo
		\\ \hline
		&Kohlensäurederivate
			&
			&
			&
			&
			&
			&
			&
			&
			&
		\\ \tabucline{2-}
		&Amine
			&amino-
			&-amin
			&
			&
			&
			&
			&
			&
			&
		\\ \tabucline{2-}
		&Imine
			&
			&
			&
			&
			&
			&
			&
			&
			&
		\\ \tabucline{2-}
		&Nitrile
			&
			&
			&
			&
			&
			&
			&
			&
			&
		\\ \tabucline{2-}
		&Amide
			&
			&
			&
			&
			&
			&
			&
			&
			&
		\\ \hline
	\multirow{3}*{\ce{S}-Verbindungen}
		&Thiole
			&
			&
			&
			&
			&
			&
			&
			&
			&
		\\ \tabucline{2-}
		&Sulfone
			&
			&
			&
			&
			&
			&
			&
			&
			&
		\\ \tabucline{2-}
		&Sulfonsäuren
			&
			&
			&
			&
			&
			&
			&
			&
			&
		\\ \hline
		&Halogenverbindungen
			&
			&
			&
			&
			&
			&
			&
			&
			&
		\\ \hline
\end{longtabu}
