\chapter{Grenzwerte}
Ziel: Verhalten einer Funktion am Rand ihren Definitionsbereichs.\\
\begin{def*}[note = offener Ball , index = offener Ball]
	Zu $x_0 \in \R$ und $r > 0$ sei
	\[ B_r(x_0) \coloneqq \{ x \in \R^n | \abs{x-x_0} < r \} \]
	den \textbf{offenen Ball} mit Radius $r$ um $x_0$ (ohne Rand).
\end{def*}

Betrachte eine Telmenge $X \subset \R^n$.\\
\begin{def*}[note = Innere , index = Innere]
	$X^\circ \coloneqq \{ x_0 \in X | \exists r > 0 : B_r(x_0) \subset X\}$ heisst das \textbf{Innere von $\mathbf{X}$} oder die \textbf{Menge der inneren Punkte von $\mathbf{X}$}.
\end{def*}
\begin{def*}[note = offen , index = offen]
	$X$ heisst \textbf{offen}, falls $X = X^\circ$ ist.
\end{def*}
\begin{def*}[note = Abschluss , index = Abschluss]
	$\overline{X} \coloneqq \{ y \in \R^n | \forall r > 0 : B_r(y) \cap X \neq \varnothing \}$ heisst der \textbf{Abschluss von $\mathbf{X}$}.
\end{def*}
\begin{def*}[note = abgeschlossen , index = abgeschlossen]
	$X$ ist \textbf{abgeschlossen}, wenn $X = \overline{X}$ ist.
\end{def*}
\begin{def*}[note = Rand , index = Rand]
	$\partial X \coloneqq \overline{X} \setminus X^\circ$ heisst der \textbf{Rand von $\mathbf{X}$}.
\end{def*}
\begin{def*}[note = dicht , index = dicht]
	Eine Teilmenge $Y \subset X$ mit $X \subset \overline{Y}$ heisst \textbf{dicht im $\mathbf{X}$}.
\end{def*}
\begin{bem}
	\begin{itemize}
		\item $X^\circ \subset X \subset \overline{X}$
		
		\item Jeder ''offener Intervall'' $]a,b[ \subset \R$ ist offen.
		\item Der offene Ball $B_r(x_0)$ ist offen.
		\item $X^\circ$ ist offen.
		\item Jede durch endlich viele strikte Ungleichunen in stetigen Funktionen definierte Menge ist offen.
		
		\item Jeder ''abgeschlosser Intervall'' ist abgeschlossen.
		\item Jede ''abgeschlossene Kugel'' $\{ x \in \R^n : \abs{x-x_0} \leq r \}$ ist abgeschlossen.
		\item $\overline{X}$ ist abgeschlossen.
		\item $f_i, g_i: \R^n \rightarrow -r$ stetig für $i = 1, \dotsc r \implies \{ x \in \R^n | f_1(x) \leq g_1(x), \dotsc , f_r(x) \leq g_r(x) \}$ ist abgeschlossen.
	\end{itemize}
\end{bem}
\begin{bsp*}
	Graph einer stetigen Funktion $f: \R \rightarrow \R = \{ (x,y) \in \R^2 | y = f(x) \}$ ist abgeschlossen.
\end{bsp*}
\begin{bem}
	$\varnothing, \R^n$ sind sowohl offen als auch abgeschlossen.\\
	$]a,b]$ ist abgeschlossen falls $a = -\infty$, sonst nicht abgeschlossen (nie offen).
\end{bem}
\begin{bsp*}
	\begin{gather*}
		\Q \subset \R \\
		\Q^\circ = \varnothing \\
		\overline{\Q} = \R
	\end{gather*}
\end{bsp*}
\begin{bem}
	$\partial B_r(x_0) = \{ x \in \R^n | \abs{x-x_0} = r \}$ heisst \textbf{Sphäre}\index{Sphäre}
\end{bem}
\begin{def*}[note = Grenzwert , index = Grenzwert]
	Sei $f: X \rightarrow \R^n$ eine Funktion, $X \subset \R^m$ und $x_0 \in \R^m$ mit $x \in \overline{X \setminus \{ x_0 \}}$\\
	$f$ hat bei $x_0$ den \textbf{Grenzwert} $y_0 \in \R^n$ falls gilt:
	\[ \forall \epsilon > 0 \exists \delta > 0 : \forall x \in X : 0 < \abs{x-x_0} < \delta \implies \abs{f(x) - y_0} < \epsilon \]
	Dann schreibt man:
	\begin{gather*}
		f(x) \rightarrow y_0 \text{ für } x \rightarrow x_0 \\
		\lim_{x \rightarrow x_0} f(x) = y_0 \qquad \text{limes}
	\end{gather*}
	\begin{bem}
		Der Grenzwert ist eindeutig bestimmt, falls er existiert.
	\end{bem}
\end{def*}
\begin{fakt}
	Für $x_0 \in X$ ist $f$ stetig in $x_0$ \gdw $\lim_{x \rightarrow x_0} f(x) = f(x)$ ist.
\end{fakt}
\begin{bsp*}
	\[ f: \R \rightarrow \R , x \mapsto \begin{cases}
		\sin \frac{1}{x}	&x \neq 0	\\
		0			&x = 0	
	\end{cases} \]
	$\lim_{x \rightarrow 0} f(x)$ existiert nicht!
\end{bsp*}
\begin{bsp*}
	\begin{gather*}
		\delta : \R \rightarrow \R , x \mapsto \begin{cases}
			1	&x = 0	\\
			0	&x \neq 0	
		\end{cases} \\
		\lim_{x \rightarrow 0} \delta(x) = 0 \neq 1 = f(x)
	\end{gather*}
\end{bsp*}
\begin{bsp*}
	\[ \begin{split}
		f(t)	&\coloneqq \frac{t^3 - 3t^2 -3t + 10}{t^2 - 5t + 6} \qquad \begin{matrix*}[l] \text{rationale Funktion } ; \\ \R \setminus \{ 2,3 \} \rightarrow \R \end{matrix*} \\
			&= \frac{(t-2) \cdot (t^2 - t -5)}{(t-2) \cdot (t-3)} \\
			&= \frac{t^2 - t - 5}{t-3} \\
			&\rightsquigarrow f \text{ hat eine stetige Fortsetzung } \R \setminus \{ 3 \} \rightarrow \R
	\end{split} \]
\end{bsp*}
\begin{bsp*}
	\[ \lim_{x \rightarrow 0} \frac{\sin x}{x} = 1 \]
	\begin{bew}[head = Geometrische Beweisidee:]
		BILD1
	\end{bew}
\end{bsp*}

Varianten:\\
\begin{bsp*}
	\begin{gather*}
		F: \R \rightarrow \R , x \mapsto \begin{cases}
			\frac{\abs{x-2}}{x^2-4}	&x \neq \pm 2	\\
			0					&x = -2		\\
			\frac{1}{4}				&x = 2		
		\end{cases}\\
		F \text{ stetig in } x_0 \neq \pm 2 \\
		F \text{ rechtsseitig stetig in } x_0 < 2 , \lim_{x \rightarrow 2 -} F(x) = \frac{1}{4} \neq F(x_0) \\
		\lim_{x \rightarrow -2 +} F(x) = -\infty \\
		\lim_{x \rightarrow -2 -} F(x) = +\infty \\
		\lim_{x \rightarrow -2} F(x) \text{ existiert auch nicht als uneigentlicher Grenzwert} \\
		\lim_{x \rightarrow \infty} F(x) = 0 \\
		\lim_{x \rightarrow -\infty} F(x) = 0
	\end{gather*}
\end{bsp*}
\begin{def*}[note = Einseitige Grenzwerte , index = Grenzwert!einseitiger]
	\begin{gather*}
		\lim_{x \searrow x_0} f(x) = \lim_{x \rightarrow x_0 +} f(x) = y_0 \iff \\
		\forall \epsilon > 0 \exists \delta > 0 \forall x \in X : 0 < x - x_0 < \delta \implies \abs{f(x) - y_0} < \epsilon \\
		\lim_{x \nearrow x_0} f(x) = \lim_{x \rightarrow x_0 +} f(x) = y_0 \iff \\
		\forall \epsilon > 0 \exists \delta > 0 \forall x \in X : 0 < x_0 - x < \delta \implies \abs{f(x) - y_0} < \epsilon
	\end{gather*}
\end{def*}
\begin{bem}
	$f$ rechtsstetig in $x_0 \iff \lim_{x \rightarrow x_0 +} f(x) = f(x_0)$ \\
	$f$ linksstetig in $x_0 \iff \lim_{x \rightarrow x_0 -} f(x) = f(x_0)$
\end{bem}
\begin{def*}[note = uneigentlicher Grenzwert , index = Grenzwert!uneigentlicher]
	\begin{tabular}{ll}
		$f(x) > N$					&''nahe'' $\infty$	\\
		$\abs{f(x) - y_0} < \epsilon$	&''nahe'' $y_0$	
	\end{tabular}\\
	\begin{gather*}
		\begin{split}
			&\lim_{x \rightarrow x_0} f(x) = \infty, \text{ falls } \\
			&\forall N > 0 \exists \delta > 0 \forall x \in X : 0 < \abs{x-x_0} < \delta \implies f(x) > N
		\end{split} \\
		\begin{split}
			&\lim_{x \rightarrow x_0} f(x) = -\infty, \text{ falls } \\
			&\forall N > 0 \exists \delta > 0 \forall x \in X : 0 < \abs{x-x_0} < \delta \implies f(x) < -N
		\end{split}
	\end{gather*}
	Analog: einseitge Grenzwerte
\end{def*}
\begin{def*}
	\begin{tabular}{ll}
		$\abs{x-x_0} < \delta$	&''nahe'' $x_0$	\\
		$x > M$				&''nahe'' $\infty$	
	\end{tabular}\\
	Sei $X \subset \R$ nach oben unbeschränkt. \\
	Dann gilt:
	\begin{gather*}
		\lim_{x \rightarrow \infty} = y_0 \in \R^m , \text{ falls} \\
		\forall \epsilon > 0 \exists M : \forall x \in X : x > M \implies \abs{f(x) - y_0} < \epsilon
	\end{gather*}
	Analog: \\
	Sei $X$ nach unten unbeschränkt. \\
	Dann gilt:
	\begin{gather*}
		\lim_{x \rightarrow -\infty} = y_0 \in \R^m , \text{ falls} \\
		\forall \epsilon > 0 \exists M : \forall x \in X : x < -M \implies \abs{f(x) - y_0} < \epsilon
	\end{gather*}
	
	Kombination:\\
	\begin{align*}
		\lim_{x \rightarrow \infty}	&=	&	&\infty \text{ falls } \forall N \exists M \forall x \in X : x >	&	&M \implies f(x) >	&	&N \\
		\lim_{x \rightarrow \infty}	&=	&- 	&\infty \text{ falls } \forall N \exists M \forall x \in X : x >	&	&M \implies f(x) <	&-	&N \\
		\lim_{x \rightarrow -\infty}	&=	&	&\infty \text{ falls } \forall N \exists M \forall x \in X : x <	&-	&M \implies f(x) >	&	&N \\
		\lim_{x \rightarrow -\infty}	&=	&-	&\infty \text{ falls } \forall N \exists M \forall x \in X : x <	&-	&M \implies f(x) <	&-	&N 
	\end{align*}
\end{def*}
\begin{bsp*}
	\begin{gather*}
		\lim_{x \rightarrow \infty} \frac{1}{x^2} = 0 \\
		\lim_{x \rightarrow 0} \frac{1}{x^2} = +\infty \\
		\lim_{x \rightarrow 0} \frac{1}{x} \text{ nicht definiert} \\
		\begin{split}
			n > 0 \rightsquigarrow	&\lim_{x \rightarrow \infty} x^n = \infty \\
								&\lim_{x \rightarrow -\infty} x^n = \begin{cases}
									\infty	&\text{falls } n \text{ gerade}	\\
									-\infty	&\text{falls } n \text{ ungerade}	
								\end{cases}
		\end{split}
	\end{gather*}
\end{bsp*}

\subsubsection{Rechnen mit Grenzwerten}
Seien $X \xrightarrow{f} Y \xrightarrow{g} Z$ und  $\lim_{x \rightarrow a} f(x) = b \in Y$.\\
\begin{itemize}
	\item Ist $b \in Y$ und $g$ stetig in $b$, dann existiert $\lim_{x \rightarrow a} g(f(x)) = g(\lim_{x \rightarrow a} f(x)) = g(b)$. \\
		Zu $\epsilon > 0$ existiert $\delta > 0$ mit: $\forall y \in Y : \abs{y-b} < \delta \implies \abs{g(y)-g(b)} < \epsilon$ \\
		Zu diesem $\delta$ existiert $\gamma > 0$ mit: $\forall x \in X : 0 < \abs{x-a} < \gamma \implies \abs{f(x)-b} < \delta$ \\
		$\implies \abs{g(f(x)-g(b)} < \epsilon$ \\
	\item Ist $b \notin Y$ und $\lim_{y \rightarrow b} g(y) = c$, so gilt $\lim_{x \rightarrow a} g(f(x)) = c$.
\end{itemize}
\begin{bem}
	Dabei dürfen $a, b, c$ auch $\pm \infty$ sein.
\end{bem}
\begin{bsp*}
	\begin{enumerate}[label=\alph*)]
		\item \[ \begin{split}
			\lim_{x \rightarrow \infty} \frac{x^2 + 5x + 7}{2x^2 - 3x - 5} &= \lim_{x \rightarrow \infty} \frac{ 1 + \frac{5}{x} + \frac{7}{x^2}}{2 - \frac{3}{x} - \frac{5}{x^2}} \\
				&= \lim_{t \rightarrow 0+} \frac{1 + 5t + 7t^2}{2 - 3t - 5t^2} \\
				&= \frac{1 + 5 \cdot 0 + 7 \cdot 0^2}{2 - 3 \cdot 0 - 5 \cdot 0^2} \\
				&= \frac{1}{2} \quad \text{Da die Funktion stetig ist.}
		\end{split} \]
		\item \[ \lim_{x \rightarrow \infty} e^{\frac{1}{x}} = \lim_{t \rightarrow 0+} e^t = e^0 = 1 \quad \text{Da } \lim_{x \rightarrow \infty} \frac{1}{x} = 0 \text{ und } \frac{1}{x} > 0 \]
		\item \[ \lim_{x \rightarrow 0+} e^{\frac{1}{x}} = \lim_{t \rightarrow \infty} e^t = \infty \quad \text{Da } \lim_{x \rightarrow 0+} \frac{1}{x} = \infty \]
		\item \begin{gather*}
			\lim_{x \rightarrow 0-} e^{\frac{1}{x}} = \lim_{t \rightarrow -\infty} e^t = \lim_{s \rightarrow \infty} \frac{1}{e^s} = \lim_{u \rightarrow \infty} \frac{1}{u} = 0 \\
			\begin{split}
				\text{Da }
					&\lim_{x \rightarrow 0-} \frac{1}{x} = -\infty \\
					&t=-s \\
					&e^t = e^{-s} = \frac{1}{e^s} \\
					&\lim_{s \rightarrow e^s} = \infty \\
					&u = e^s
			\end{split}
		\end{gather*}
	\end{enumerate}
\end{bsp*}
\begin{fakt}
	Für jede vektorwertige Funktion $f = (f_1, f_2, \dotsc , f_n)$ gilt:
	\[ \lim_{x \rightarrow a} f(x) = b = (b_1, b_2, \dotsc , b_n) \iff \forall i : \lim_{x \rightarrow a} f_i(x) = b_i \]
\end{fakt}
\begin{bsp*}
	\[ \begin{split}
		\lim_{x \rightarrow 0+} \frac{\sin x}{\sqrt{5x^2 + x^4}}
			&= \lim_{x \rightarrow 0+} \frac{\sin x}{\abs{x}} \cdot \frac{1}{\sqrt{5 + x^2}} \\
			&= \lim_{x \rightarrow 0+} \frac{\sin x}{x} \cdot \lim_{x \rightarrow 0+} \frac{1}{\sqrt{5 + x^2}} \\
			&= 1 \cdot \frac{1}{\sqrt{5}} = \frac{1}{\sqrt{5}}
	\end{split} \]
\end{bsp*}
\begin{def*}[note = Majorantenkriterium , index = Majorantenkriterium]
	Ist $\lim_{x \rightarrow a} g(x) = 0$ und $\abs{f(x)} \leq \abs{g(x)}$ für alle $x$ nahe $a$, so gilt $\lim_{x \rightarrow a} f(x) = 0$.
\end{def*}
\begin{def*}[note = Minorantenkriterium , index = Majorantenkriterium]
	Ist $\lim_{x \rightarrow a} g(x) = \infty$ und $f(x) > g(x)$ für alle $x$ nahe $a$, so gilt $\lim_{x \rightarrow a} f(x) = \infty$. \\
	Analog $-\infty$
\end{def*}
\begin{bsp*}
	\[ e^x \geq x \text{ für alle } x \geq 0 \text{ und } \lim_{x \rightarrow \infty} = \infty \implies \lim_{x \rightarrow \infty} e^x = \infty \]
\end{bsp*}
\begin{bsp*}
	\[ \lim_{x \rightarrow \infty} \frac{\sin x}{x} = 0 \quad \text{Da } \abs{\sin x} \leq 1 \quad \abs{\frac{\sin x}{x}} \leq \abs{\frac{1}{x}} \quad \lim_{x \rightarrow \infty} \frac{1}{x} = 0 \]
\end{bsp*}
\begin{bsp*}
	\[ \lim_{x \rightarrow \infty} \frac{x}{2 + \sin \frac{1}{x}} = \infty \quad \frac{x}{2 + \sin \frac{1}{x}} \quad \lim_{x \rightarrow \infty} \frac{x}{3} = \infty \]
\end{bsp*}
\begin{bsp*}
	\begin{gather*}
		\lim_{(x,y) \rightarrow (0,0)} \frac{x^2 y^2}{x^2 + y^2} \\
		\abs{\frac{x^2 y^2}{x^2 + y^2}} \leq \abs{y^2} \\
		\lim_{(x,y) \rightarrow (0,0)} y^2 = \lim_{y \rightarrow 0} y^2 = 0
	\end{gather*}
\end{bsp*}

\section{Asymptoten}
\begin{def*}[note = Asymptote , index = Asymptote]
	\begin{enumerate}[label = \alph*)]
		\item Sei $X \subset \R$ nach oben unbeschränkt und seien $f, g : X \rightarrow \R$. Wir nennen $f, g$ \textbf{zueinander asymptotisch} für $x \rightarrow \infty$, falls gilt:
			\[ \lim_{x \rightarrow \infty} f(x) - g(x) = 0 \]
		\item Ist $g(x) = px + q$ eine lineare Funktion, asymptotisch zu $f$, dann heisst die Gerade $\graph(g)$ \textbf{Asymptote von $\mathbf{f}$ für $\mathbf{x \rightarrow \infty}$}
	\end{enumerate}
	Analog $x \rightarrow -\infty$
	
	Bestimmung \\
	Die Asymptote ist eindeutig, falls sie existiert.\\
	\begin{gather*}
		(\text{Wären } g(x) = px + q \\
		g'(x) = p'x + q' \text{ beide Asymptoten für } x \rightarrow \infty \\
		\implies \lim_{x \rightarrow \infty} g(x) - g'(x) = 0 - 0 = 0 \\
		g(x) - g'(x) = [f(x) - g'(x)] - [f(x) - g(x)] \\
		\lim_{x \rightarrow \infty} (p - p') x + (q - q') = 0
	\end{gather*}
\end{def*}
\begin{bsp*}
	\begin{gather*}
		f(t) = \frac{t^2 - t - 5}{t-3} = t + 2 + \frac{1}{t-3} \quad \text{Polynomdivision} \\
		\implies \text{Asymptote } g(t) = t + 2 . \quad [ \lim_{x \rightarrow \pm \infty} \frac{1}{t-3} = 0
	\end{gather*}
\end{bsp*}
\begin{bsp*}
	\begin{gather*}
		\begin{split}
			f(x)	&= \sqrt{x(x+a)} \text{ für } x \rightarrow +\infty \\
				&= x \sqrt{1 + \frac{a}{x}}
		\end{split}\\
		\text{Ansatz: } g(x) = x + q \\
		\begin{aligned}
			\text{Ziel: }	&\lim_{x \rightarrow \infty} \sqrt{x(x+a)} - (x+q) \\
			=		&\lim_{x \rightarrow \infty} \frac{(\sqrt{x(x+a)} - (x+q))(\sqrt{x(x+a)} - (x+q))}{(\sqrt{x(x+a)} - (x+q))} \\
			=		&\lim_{x \rightarrow \infty} \frac{x(x+a) - (x+q)^2}{x(\sqrt{1 + \frac{a}{x}} + 1 + \frac{q}{x})} \\
			=		&\lim_{x \rightarrow \infty} \frac{(a-2q)x - q^2}{x(\sqrt{1 + \frac{a}{x}} + 1 + \frac{q}{x})} \\
			=		&\lim_{x \rightarrow \infty} \frac{(a-2q) - \frac{q^2}{x}}{(\sqrt{1 + \frac{a}{x}} + 1 + \frac{q}{x})} \\
			=		&\frac{a - 2q}{2}
		\end{aligned}
	\end{gather*}
	Asymptote $x+\frac{a}{2}$
\end{bsp*}
Anwendung:\\
\begin{def*}[note = beschränkt , index = beschränkt]
	Eine Teilmenge $X \subset \R^n$ heisst \textbf{beschränkt},wenn die Menge $\{ \abs{x} : x \in X \}$ nach oben beschränkt ist. Äquivalent: Es existiert $r > 0$ mit $X \subset B_r(0)$.
	
	Eine Funktion $f: X \rightarrow \R^n$ heisst beschränkt wenn
	\[ \image(f) \iff  \exists r > 0 : \forall x \in X : \abs{f(x)} \leq r \]
	\begin{bem}
		$X \subset \R^n , f: X \rightarrow \R^n , f$ stetig in $x_0 \in X$.\\
		Dann $\exists r > 0$ sodass die Einschränkung von $f$ auf $X \cap \overline{B_r(x_0)}$ beschränkt ist. \\
		\begin{bew}
			\begin{gather*}
				\text{Für } \epsilon \coloneqq 1 \text{ existiert } \delta > 0 : \forall x \in X : \abs{x-x_0} < \delta \\
				\implies \abs{f(x)-f(x_0)} < 1 \\
				r \coloneqq \frac{\delta}{2} \text{ tut's für alle } x \in X \cap B_r(x_0) \text{ ist } \\
				\abs{f(x)} \leq \abs{f(x)-f(x_0)} + \abs{f(x_0)} \leq 1 + \abs{f(x_0)} \quad \blacksquare
			\end{gather*}
		\end{bew}
	\end{bem}
\end{def*}
\begin{def*}[note = kompakt , index = kompakt]
	$X$ heisst \textbf{kompakt} falls es abgeschlossen und beschränkt ist.
\end{def*}
\begin{bsp*}
	$\overline{B_r(x_0)}$ ist kompakt. \\
	$[a,b] \subset \R$
\end{bsp*}
\begin{satz*}
	$f: X \rightarrow \R^n$ stetig, $X \subset \R^n$ kompakt $\implies f$ beschränkt.
	
	Folge: Sei $f: \R \rightarrow \R$ stetig, sodass $\lim_{x \rightarrow +\infty} f(x) \in \R \ni \lim_{x \rightarrow -\infty} f(x)$ existieren. Dann ist $f$ beschränkt.
\end{satz*}
\begin{bsp*}
	\[ f(x) \coloneqq \begin{cases}
		\frac{\sin x}{x}	&x \neq 0	\\
		1			&x = 0	
	\end{cases} \]
\end{bsp*}
\begin{bsp*}
	\[ \frac{x^2 + 5x}{x^2 + 1} \]
\end{bsp*}
