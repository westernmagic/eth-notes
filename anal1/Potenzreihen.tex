\chapter{Potenzreihen}
\begin{def*}[note = Potenzreihe , index = Potenzreihe]
	Ein Ausdruck der Form
	\[ f(z) = \sum_{k=0}^\infty a_k \cdot z^k \]
	heisst Potenzreihe.
	
	(reele P.R.: $a_k, z \in \R$; komplexe P.R.: $a_k, z \in \C$)
\end{def*}
\begin{fakt}
	Auf der Menge $U$ aller $z$, wo die Reihe konvergiert, ist dadurch eine Funktion definiert.\\
	Ist $z \in U$ und $\abs{z'} < \abs{z}$, so ist $z' \in U$ und die Reihe konvergiert absolut in $z'$.\\
	\begin{bew}
		\begin{gather*}
			z \in U \implies \lim_{k \rightarrow \infty} a_k z^k = 0 \\
			\text{Insbesondere } \exists c > 0 \forall k \geq 0 : \abs{a_k z^k} \leq c \\
			\sum_{k=0}^\infty \abs{a_k z^k} = \sum_{k=0}^\infty \abs{a_k z^k \left( \frac{z'}{z} \right)^k} = \leq \sum_{k=0}^\infty c \cdot \abs{\frac{z'}{z}}^k < \infty \\
			\text{Majorantenkriterium}
		\end{gather*}
		$\implies$ Entweder absolute konvergenz auf $\R$ oder $U = [-a,a]$ für $a < \infty$ und wir haben konvergenz auf $]-a,a[$
	\end{bew}
\end{fakt}

\[ f(z) = \sum_{k=0}^\infty a_k \cdot z^k, a_k \in \C \]

\begin{fakt}
	Falls $f(\zeta)$ konvergiert für $\zeta \in \C$ dann konvergiert $f(\zeta')$ absolut für jedes $\zeta' \in \C$ mit $\abs{\zeta'} < \abs{\zeta}$.
	
	Folge: Konvergenzbereich von $f \coloneqq \{ \zeta \in \C | f(\zeta)$ konvergiert $\}$
	
	Mit $\rho \coloneqq \sup\{ \abs{\zeta} : f(\zeta)$ konv. $\}$ gilt: \\
	$f(\zeta)$ divergiert für $\abs{\zeta} > \rho$\\
	$f(\zeta)$ irgendetwas für $\abs{\zeta} = \rho$\\
	$f(\zeta)$ konvergiert absolut für $\abs{\zeta} < \rho$
	
	$\implies$ Konvergenzbereich ist eine Kreisscheibe mit oder ohne oder mit einem Teil des Randes.\\
	Spezialfall $\rho = \infty$ absolute Konvergenz auf $\C$
	
	\begin{bem}
		$\rho \geq 0$ immer
	\end{bem}
	Spezialfall: $\rho = 0$: Konvergenz nur in $z=0$
\end{fakt}
\begin{def*}[note = Konvergenzradius , index = Konvergenzradius]
	$\rho$ heisst \textbf{Konvergenzrtadius} von $f$.
\end{def*}

\subsection{Bestimmung des Konvergenzradiuses}
\subsubsection{Quotientenkriterium}
Falls der Grenzwert $\lim_{k \rightarrow \infty} \abs{\frac{a_k}{a_{k+1}}}$ existiert oder $= \infty$, so ist er gleich $\rho$

Idee:\\
Sei $\alpha \coloneqq \lim_{k \rightarrow \infty} \abs{\frac{a_k}{a_{k+1}}}$ und $\zeta \in \C$ mit $\abs{\zeta} < \alpha$.\\
Wähle $\alpha' \in ] \abs{\zeta} , \alpha [$ und $k_0$ mit $\forall k \leq k_0 : \abs{frac{a_k}{a_{k+1}}} > \alpha'$
\begin{gather*}
	\begin{split}
		\sum_{k=0}^\infty \abs{a_k \cdot \zeta^k}	&= \sum_{k=0}^\infty \abs{a_k} \cdot \abs{\zeta^k} \\
										&= \sum_{k=0}^\infty \abs{a_k} \cdot \alpha'^k \cdot \left( \frac{\abs{\zeta}}{a'} \right)^k \\
										&\leq \sum_{k=0}^{k_0 - 1} (\text{etwas}) + \sum_{k=k_0}^\infty c \cdot q^k \quad \text{konvergent.} 
	\end{split}\\
	\alpha' \cdot \abs{a_{k+1}} \leq \abs{a_k} \\
	\alpha' \cdot \abs{a_{k+2}} \leq \abs{a_{k+1}} \\
	\implies \forall k \geq k_0 : \alpha'^{k-k_0} \abs{a_k} \leq \abs{a_{k_0}} \\
	\alpha'^k \cdot \abs{a_k} \leq  \alpha'^{k_0} \cdot \abs{a_{k_0}} \eqqcolon c
\end{gather*}
\todo{Double superscript}

\begin{bsp*}
	\begin{gather*}
		a \in \C \setminus \{ 0 \} , \sum_{k=0}^\infty \frac{z^k}{a^k} \quad \text{hat Konv. Radius:} \\
		a_k = \frac{1}{a^k} \implies \lim_{k \rightarrow \infty} \abs{\frac{a_k}{a_{k+1}}} = \lim_{k \rightarrow \infty} \abs{\frac{\frac{1}{a^k}}{\frac{1}{a^{k+1}}}} = \lim_{k \rightarrow \infty} \abs{a} = \abs{a}
	\end{gather*}
\end{bsp*}
\begin{bsp*}
	\begin{gather*}
		\begin{split}
			\alpha \in \R , \sum_{k=1}^\infty \frac{z^k}{k^\alpha} : a_k = \frac{1}{k^\alpha} \rightarrow \lim_{k \rightarrow \infty} \abs{\frac{\frac{1}{k^\alpha}}{\frac{1}{(k+1)^\alpha}}}
				&= \lim_{k \rightarrow \infty} \left( \frac{k+1}{k} \right)^\alpha\\
				&= \lim_{k \rightarrow \infty} \left(1 + \frac{1}{k} \right)^\alpha\\
				&= 1 \\
		\end{split} \\
		\lim_{k \rightarrow \infty} \frac{1}{k} = 0 \\
		\lim_{y \rightarrow 0} (1+y)^\alpha = (1+0)^\alpha = 1 \\
		\implies \text{ Konvergenzradius } 1 \\
		\alpha = 0 \implies \text{ Divergent für alle } \zeta \in \C \text{ mit } \abs{\zeta} = 1 \\
		\alpha > 1 \implies \text{ absolut konvergent für alle } \zeta \in \C \text{ mit } \abs{\zeta} = 1 \\
		\begin{split}
			0 \leq \alpha \leq 1 \implies	&\text{ Divergenz für alle } \zeta = 1	\\
									&\text{ Konvergenz für } \zeta = -1	
		\end{split}
	\end{gather*}
\end{bsp*}
\begin{bsp*}
	\begin{gather*}
		\sum_{k=0}^\infty \frac{z^k}{k!} \text{ hat Konvergenzradius } \infty . \\
		a_k = \frac{1}{k!} , \lim_{k \rightarrow \infty} \abs{\frac{\frac{1}{k!}}{\frac{1}{(k+1)!}}} = \lim_{k \rightarrow \infty} (k+1) = \infty
	\end{gather*}
\end{bsp*}
\begin{bsp*}
	\begin{gather*}
		\sum_{k=0}^\infty k^k \cdot z^k \\
		\begin{split}
			a_k = k^k , \lim_{k \rightarrow \infty} \abs{\frac{a_k}{a_{k+1}}}
				&= \lim_{k \rightarrow \infty} \frac{k^k}{(k+1)^{k+1}} \\
				&= \lim_{k \rightarrow \infty} \underbrace{\left( \frac{k}{k+1} \right)^k}_{\leq 1} \cdot \underbrace{\frac{1}{k+1}}_{\rightarrow 0} \\
				&= 0 \quad \text{Majorantenkriterium}
		\end{split}
	\end{gather*}
\end{bsp*}
\begin{bsp*}
	\[ \sum_{k=0}^\infty (\sin k) \cdot z^k \]
	$a_k = \sin k ; \lim_{k \rightarrow \infty} \abs{\frac{\sin k}{\sin (k+1)}}$ existiert nicht \\
	$\abs{\zeta} < 1 \implies$ konvergenz bei $\zeta : \abs{(\sin k) \cdot z^k} \leq \abs{z}^k$\\
	Da $\sin k \rightarrow 0$ bei $k \rightarrow \infty$, ist $\rho = 1$
\end{bsp*}
\begin{gather*}
	\sum_{k=0}^\infty \sin \frac{k \pi}{2} \cdot z^k = \sum_{k=0}^\infty \begin{Bmatrix*}[l]
		0	&\text{falls } k \text{ gerade}	\\
		1	&\text{falls } k = 1 + 4l | l \in \Z	\\
		-1	&\text{falls } k = 3 + 4l		
	\end{Bmatrix*} \cdot z^k \\
	\frac{\sin \frac{k\pi}{2}}{\sin \frac{(k+1)\pi}{2}} \\
	k = 2m + 1 \\
	\sum_{m=0}^\infty (-1) \cdot z^{2m + 1} = \left( \sum_{m=0}^\infty (-1) \cdot (z^2)^m \right) \cdot z \quad \text{Reihe in } z^2 = y \\
	= z \cdot \left( \sum_{m=0}^\infty (-1) \cdot y^m \right) \begin{cases}
		\text{konvergiert für}	&\abs{y} < 1	\\
		\text{divergiert}		&\abs{y} > 1	
	\end{cases} \\
	\implies \text{ hat Konv. Radius } 1
\end{gather*}

\subsubsection{Würzelkriterium}
$\alpha \coloneqq  \lim_{k \rightarrow \infty} \frac{1}{\sqrt[k]{\abs{a_k}}}$ existiert, \\
dann ist $\alpha = \rho$

\begin{bsp*}
	Für $k \in \Z^{\geq 0}$ und $\alpha \in \C$ sei 
	\[
		\binom{\alpha}{k} \coloneqq \begin{cases}
			\frac{\alpha ( \alpha - 1 ) \dots ( \alpha - k + 1 )}{k!}	&k \geq 1	\\
			1										&k = 0	
		\end{cases}
	\]
\end{bsp*}

\subsection{Binomialkoeffizient}
Für $\alpha \in \Z^{\geq 0}$ ist $\binom{\alpha}{k} =$ Anzahl der $k$-elementigen Teilmengen einer Menge mit $\alpha$ Elementen. \\
\begin{def*}[note = Binomische Reihe , index = Binomische Reihe]
	\[ \sum_{k=0}^\infty \binom{\alpha}{k} \cdot z^k \]
	Konvergenzradius falls $\alpha \notin \Z^{\geq 0}$
	\[ \lim_{k \rightarrow \infty} \abs{\frac{a_k}{a_{k+1}}} = \lim_{k \rightarrow \infty} \frac{\abs{\frac{\alpha (\alpha - 1) \dots (\alpha - k + 1)}{k!}}}{\abs{\frac{\alpha (\alpha - 1) \dots (\alpha - k)}{(k - 1)!}}} = \lim_{k \rightarrow \infty} \abs{\frac{k+1}{\alpha - k}} = 1
	\]
\end{def*}
\begin{satz*}
	Für $\alpha , x \in \R , \abs{x} < 1$, gilt
	\[ \sum_{k=0}^\infty \binom{\alpha}{k} \cdot x^k = (1 + x)^\alpha \]
\end{satz*}

Spezialfall: $\alpha = n \in \Z^{\geq 0} , z \in \C , \abs{z} < 1$ \\
Reihe bricht ab, \\
$(1 + z)^n = \sum_{k=0}^n \binom{n}{k} \cdot z^k$ und Konvergenzradius $\infty$

Spezialfall $\alpha = -n , n \in \Z^{> 0}$
\begin{gather*}
	\frac{1}{(1+z)^n} = \sum_{k=0}^\infty \binom{-n}{k} \cdot z^k \\
	\begin{split}
		\binom{-n}{k}	&= \frac{(-n)(-n-1) \dots (-n-k+1)}{k!} \\
					&= (-1)^k \cdot \frac{(n+k-1)(n+k-2) \dots (n+1) n}{k!}
	\end{split} \\
	\frac{1}{(1-z)^n} = \sum_{k=0}^\infty \binom{n+k-1}{k} \cdot z^k \\
	n=1 \\
	\frac{1}{1-z} = \sum_{k=0}^\infty z^k \\
	n=2 \\
	\frac{1}{(1-z)^2} = \sum_{k=0}^\infty (k+1) \cdot z^k \\
	\intertext{Spezialfall: $\alpha = \frac{1}{2}$}
	\rightsquigarrow \sqrt{1+x} = \sum_{k=0}^\infty \binom{\frac{1}{2}}{k} \cdot z^k = 1 + \frac{1}{2} x - \frac{1}{8} x^2 + \dots
\end{gather*}

\begin{fakt}
	Jede Potenzreihe definiert in Inneren ihres Konvergenzradiuses eine stetige Funktion.
\end{fakt}

\subsection{Rechnen mit Potenzreihen}
\begin{align*}
	\frac{1}{1-z} \cdot \frac{1}{1-z}	&= \frac{1}{(1-z)^2} \\
							&= \sum_{k=0}^\infty (k+1) \cdot z^k \text{ für } \abs{z} < 1 \\
							&= \left( \sum_{k=0}^\infty z^k \right) \left( \sum_{l=0}^\infty z^l \right) \\
							&= \sum_{k=0}^\infty \sum_{l=0}^\infty z^{k+l} \\
							&= \sum_{n=0}^\infty \abs{\{ (k,l) , k, l \in \Z^{\geq 0}, k + l = n \}} \cdot z^n \\
							&= \sum_{n=0}^\infty (n+1) \cdot z^k
\end{align*}

\subsubsection{Produkt}
\[ \left( \sum_{k=0}^\infty a_k \cdot z^k \right) \left( \sum_{kl0}^\infty b_k \cdot z^l \right) = \sum_{n=0}^\infty \left( \sum_{l=0}^n a_k b_{n-k} \right) z^k \quad l = n - k \]
falls beide Reihen konvergieren.

Analog: Summe, Differenz, Quotient, Komposition, Umkehrfunktion sind wieder Potenzreihen.

\subsection{Exponentialfunktion}
\[ \exp(z) \coloneqq \sum_{k=0}^\infty \frac{z^k}{k!} \]
Konvergenzradius $= \infty \rightsquigarrow$ stetige Funktion $\C \rightarrow \C$

\subsubsection{Additionstheorem}
Für $z, w \in \C$ gilt \\
\[ \exp(z+w) = \exp(z) \cdot \exp(w) \]
\begin{bew}
	\begin{align*}
		\exp(z+w)	&= \sum_{n=0}^\infty \frac{(z+w)^n}{n!} \\
				&= \sum_{n=0}^\infty \sum_{k=0}^\infty \binom{n}{k} z^k \cdot w^{n-k} \\
				&= \sum_{n=0}^\infty \sum_{k=0}^\infty \frac{z^k \cdot w^{n-k}}{k!(n-k)!} \\
				&= \sum_{k,l \geq 0} \frac{z^k w^l }{k!l!} \\
				&= \left( \sum_{k=0}^\infty \frac{z^k}{k!} \right) \left( \sum_{l=0}^\infty \frac{w^l}{l!} \right) \\
				&= \exp(z) \cdot \exp(w) \quad \blacksquare
	\end{align*}
\end{bew}
Insbesondere gilt für alle $z \in \C$ und $n \in \Z, \exp(nz) = \exp(z)^n$
\begin{align*}
	n \geq 1 :	&\exp(z+ \dots +z)		\\
	n = 0	 :	&\exp(0z) = 1 = \exp(z)^0	\\
	n < 0 :	&1 = \exp(0) = \exp(nz - nz) = \exp(nz) \cdot \exp(-nz)
\end{align*}

\begin{def*}[note = Eulersche Zahl , index = Eulersche Zahl]
	\[ e = \exp(1) = \sum_{k=0}^\infty \frac{1}{k!} \]
\end{def*}
\begin{satz*}
	Für $x \in \R$ gilt
	\[ \exp(x) = e^x \]
	\begin{bew}
		\begin{gather*}
			\xi = \frac{m}{n} ; m, n \in \Z^{> 0} \text{ ist} \\
			\exp(\xi) > 0 \\
			\exp(\xi)^n = \exp(n\xi) = \exp(m) = \exp(1)^m = e^m \\
			\implies \exp(\xi) = \underset{+}{\sqrt[n]{e^m}} = e^{\frac{n}{m}} = e^\xi \\
			\implies \exp(-\xi) = \frac{1}{\exp(\xi)} = \frac{1}{e^\xi} = e^{-\xi} \\
			\overset{?}{=} \exp{\xi} = e^\xi \text{ für alle } \xi \text{ aus } \Q . \\
			\text{Da beide Seiten stetig } \\
			\implies \exp(x) = e^x \text{ für alle } x \in \R \quad \blacksquare
		\end{gather*}
	\end{bew}
\end{satz*}
\begin{bem}[note = Abkürzung]
	Für $z \in \C$
	\[ e^z \coloneqq \exp(z) \]
\end{bem}

\subsubsection{Eigenschaften}
\begin{enumerate}[label = \alph*)]
	\item $\exp : \R \rightarrow \R^{>0}$ ist bijektiv, stetig und streng monoton wachsend. \\
	\item Für jedes $q \in \Z$ gilt
	\begin{gather*}
		\lim_{x \rightarrow \infty} \frac{e^x}{x^q} = \infty \\
		\lim_{x \rightarrow \infty} x^q \cdot e^{-x} = 0
		\intertext{Denn:}
		\frac{e^x}{x^q} = \frac{1}{x^q} \cdot \sum_{k=0}^\infty \frac{x^k}{k!} \geq \frac{1}{x^q} \frac{x^{q+1}}{(q+1)!} \frac{1}{(q+1)!} \rightarrow \text{ für } x \rightarrow \infty \implies \frac{e^x}{x^q} \rightarrow \infty
		\intertext{und}
		\lim_{x \rightarrow \infty} x^q \cdot e^{-x} = \lim_{x \rightarrow \infty} \frac{1}{x^q \cdot e^{-x}} = 0
	\end{gather*}
\end{enumerate}
Erinnerung: $\exp : \R \rightarrow \R^{>0}$ bijektiv, streng monoton wachsend.

\subsection{Logarithmus}
\begin{def*}[note = natürliche Logarithmus , index = Logarithmus!natürliche]
	Dir Umkehrfunktion der obigen ist der \textbf{natürliche Logarithmus} $\log: \R^{>0} \rightarrow \R$.\\
	bijektiv, streng monoton wachsend
\end{def*}
Historisch: \\
\begin{tabular}{ l c l l }
	$\log_{10}$			&$=$				&lg	&						\\
	$\log_2$				&$=$				&lb	&binärlog.					\\
	\multirow{2}{*}{$\log_e$}	&\multirow{2}{*}{$=$}	&ln	&\multirow{2}{*}{natürlicherlog.}	\\
						&				&log	&
\end{tabular}\\
\subsubsection{Rechenregeln}
\begin{gather*}
	e^{\log y} = y \text{ für } y > 0 , y \in \R \\
	\log e^x = x \text{ für } x \in \R \\
	\log (xy) = \log x + \log y \\
	\log \frac{1}{x} = - \log x \\
	\log x^y = y \log x \\
	a^x = e^{x \log a} = \exp(\log a \cdot x) \\
	\begin{split}
			&y = a^x \\
		\iff	&x = \log_a y \\
		\iff	&y = e^{\log_a x} \\
		\iff	&(\log a) \cdot x = \log y \rightsquigarrow \log_a y = \frac{\log y}{\log a}
	\end{split}
\end{gather*}

Komplexität:
\begin{tabular}{ l c l l }
	$+$		&:	&$O(n)$			&			\\
	$\cdot$	&:	&$O(n^2)$			&Schulemethode	\\
	$\cdot$	&:	&$O(n \cdot \log n)$	&Optimiert		
\end{tabular}

\subsubsection{Grenzwerte}
\begin{gather*}
	\lim_{t \rightarrow \infty} \log t = \infty \\
	\lim_{t \rightarrow 0+} \log t = -\infty
	\intertext{Für jedes $\alpha > 0$ gilt:}
	\lim_{t \rightarrow \infty} \frac{\log t}{t^\alpha} = 0 \\
	\lim_{t \rightarrow 0+} t^\alpha \log t = 0
	\intertext{Denn:}
	\lim_{t \rightarrow \infty} \frac{\log t}{e^{\alpha \log t}} = \lim_{s \rightarrow \infty} \frac{s}{e^{\alpha s}} = \lim_{u \rightarrow \infty} \frac{\frac{u}{\alpha}}{e^u} = 0 \\
	\lim_{t \rightarrow 0+} t^\alpha \log t = \lim_{s \rightarrow \infty} \frac{-\log s}{s^\alpha} = 0
\end{gather*}
\begin{fakt}
	\[ \lim_{z \rightarrow 0} \frac{e^z - 1}{z} = 1 \quad z \in \C \]
	\begin{bew}
		\[ \frac{e^z - 1}{z} = \frac{ \left( \sum_{k=0}^\infty \frac{z^k}{k!} \right) - 1}{z} = \sum_{k=1}^\infty \frac{z^{k-1}}{k!} \rightarrow \sum_{k=0}^\infty \frac{0^{k-1}}{k!} = 1 \]
	\end{bew}
\end{fakt}
\begin{fakt}
	\[ \lim_{x \rightarrow 0} \frac{\log (1+x)}{x} \quad x \in \R \]
	\begin{bew}
		\begin{gather*}
			y = \log (1+x) \\
			\rightsquigarrow e^y = 1+x \\
			e^y - 1 = x \\
			\lim_{x \rightarrow 0} \frac{\log (1+x)}{x} = \lim_{y \rightarrow 0} \frac{y}{e^y - 1} = 1
		\end{gather*}
	\end{bew}
\end{fakt}
\begin{fakt}
	\[ \lim_{n \rightarrow \infty} \left( 1 + \frac{x}{n} \right)^n = \exp(n) \quad x \in \R \]
	\begin{bew}
		\begin{gather*}
			x \neq 0 \implies  \lim_{n \rightarrow \infty} \frac{x}{n} = 0 \\
			\text{aber } \frac{x}{n} \neq 0 \\
			\implies \lim_{n \rightarrow \infty} \frac{n \log \left( 1 + \frac{x}{n} \right)^n}{n \frac{x}{n}} = 1 \\
			\implies \lim_{n \rightarrow \infty} \frac{\log \left( 1 + \frac{x}{n} \right)^n}{x} = 1 \\
			\implies \lim_{n \rightarrow \infty} \log \left( 1 + \frac{x}{n} \right)^n = x \\
			\text{stetigkeit von } \exp \implies \lim_{n \rightarrow \infty} \left( 1 + \frac{x}{n} \right)^n = e^x \quad \blacksquare
		\end{gather*}
	\end{bew}
	Insbesondere
	\[ e = \lim_{n \rightarrow \infty} \left( 1 + \frac{1}{n} \right)^n \]
\end{fakt}

\subsection{Potenzreihenentwickelung}
\begin{gather*}
	1 + y = e^x = 1 + x + \frac{x^2}{2} + \frac{x^3}{6} \\
	y = x + \frac{x^2}{2} + \frac{x^3}{6} + \dots \\
	\text{Ansatz: } x = y + ay^2 + by^3 + \dots \\
	\begin{split}
		y	&= (y + ay^2 + by^3) + \frac{1}{2} (y + ay^2 + by^3)^2 + \frac{1}{6} (y + ay^2 + by^3)^3 + \dots \\
			&= y + y^2 \underbrace{\left( a + \frac{1}{2} \right)}_{a = \frac{1}{2}} + y^3 \underbrace{\left( b + \frac{1}{2} 2a + \frac{1}{6} \right)}_{\substack{b + a + \frac{1}{6} = 0 \\ b = \frac{1}{2} - \frac{1}{6} = \frac{1}{3}}} + O(x^4)
	\end{split} \\
	\log(1+y) = y - \frac{y^2}{2} + \frac{y^3}{3} + \dots
\end{gather*}
\begin{fakt}
	\[ \log(1+y) = y - \frac{y^2}{2} + \frac{y^3}{3} - \frac{y^4}{4} + \dots = \sum_{k=1}^\infty \frac{ (-1)^{k-1} y^k}{k} \quad \text{falls } \abs{y} < 1 \]
\end{fakt}

Erinnerung:\\
\begin{gather*}
	\exp(\imath t) = \cos t + \imath \sin t = \sum_{k=0}^\infty \frac{(\imath t)^k}{k!} = \sum_{\substack{k=0\\\text{gerade}}}^\infty \frac{(-1)^{\frac{k}{2}} t^k}{k!} + \sum_{\substack{k=1\\\text{ungerade}}}^\infty \frac{(-1)^{\frac{k-1}{2}} t^k}{k!} \\
	\implies \cos t = \sum_{l=0}^\infty \frac{(-1)^l t^{2l}}{(2l)!} = 1 - \frac{t^2}{2} + \frac{t^4}{24} \mp \dots \\
	\implies \sin t = \sum_{l=0}^\infty \frac{(-1)^l t^{2l+1}}{(2l+1)!} = t - \frac{t^3}{6} + \frac{t^5}{120} \mp \dots
\end{gather*}
\begin{bem}
	Durch
	\[ \sin t = \frac{e^{\imath t} - e^{-\imath t}}{2\imath} \]
	und
	\[ \cos t = \frac{e^{\imath t} + e^{-\imath t}}{2} \]
	oder der entsprechenden Potenzreihenentwickelung definieren $\sin$ und $\cos$ auch Funktionen $\C \rightarrow \C$.
\end{bem}
\begin{bem}
	$\pi$ ist die kleinste reele Zahl $> 0$ mit $\sin \pi = 0$
\end{bem}
\begin{bem}
	\begin{gather*}
		e^{x + \imath y} = e^x e^{\imath y} = e^x ( \cos y + \imath \sin y) \quad x, y \in \R \\
		e^{z + 2\pi \imath k} = e^z \quad \text{für } z \in \C , k \in \Z \\
		\C \rightarrow \C , z \mapsto e^z : \\
		\text{Bild } = \C \setminus \{ 0 \} \\
		e^z = e^w \iff w = z + 2\pi \imath k \quad \text{für ein }k \in \Z
	\end{gather*}
\end{bem}
\begin{satz*}[note = Pythagoras , index = Pythagoras!Satz von]
	\[ \cos^2 t + \sin^2 t =1 \]
	\begin{bew}
		\[ \begin{split}
				&\left( \frac{e^{\imath t} + e^{-\imath t}}{2} \right)^2 + \left( \frac{e^{\imath t} - e^{-\imath t}}{2\imath} \right)^2 \\
			=	&\frac{e^{2\imath t} + 2 + e^{-2\imath t}}{4} + \frac{e^{2\imath t} - 2 + e^{-2\imath t}}{-4} = 1
		\end{split} \]
	\end{bew}
	Analog:
	\begin{gather*}
		\cos 2t = \cos^2 t - \sin^2 t \\
		\sin 2t = 2\cos t \sin t
	\end{gather*}
\end{satz*}

\section{Hyperbolische Funktionen}
\begin{gather*}
	\cosh t = \frac{e^t + e^{-t}}{2} = 1 + \frac{t^2}{2} + \frac{t^4}{24} + \dots \\
	\sinh t = \frac{e^t - e^{-t}}{2} = t + \frac{t^3}{6} + \frac{t^5}{120} + \dots \\
	\cosh(\imath t) = \cos(t) \iff \cosh(t) = \cos(\imath t) \\
	\sinh(\imath t) = \imath \sin(t) \iff \sinh(t) = \frac{\sin(\imath t)}{\imath} = -\imath \sin(\imath t)
\end{gather*}

\subsection{Umkehrfunktionen}
\begin{gather*}
	x = \cosh t = \frac{e^t + e^{-i}}{2} \quad t \geq 0 \\
	e^t 2x = (e^t + e^{-t}) e^t \\
	0 = e^{2t} - 2x e^t + 1 \\
	e^t = \frac{2x \pm \sqrt{4x^2 - 4}}{2} = x \pm \sqrt{x^2 - 1} \quad \text{nur +, da } x \geq 1 \\
	t = \log( x + \sqrt{x^2 - 1} ) = \arcosh(t)
\end{gather*}
\begin{bem}
	\begin{gather*}
		\begin{split}
			\tan x = \frac{\sin x}{\cos x}	&= \frac{x - \frac{x^3}{6} + \frac{x^5}{120} \mp \dots}{1 - \frac{x^2}{2} + \frac{x^4}{24} \mp \dots} \\
									&= x \frac{1 - \frac{x^2}{6} + \frac{x^4}{120} \mp \dots}{1 - \frac{x^2}{2} + \frac{x^4}{24} \mp \dots} \\
									&= x ( a + b x^2 + c x^4 + \dots ) \quad \text{Ansatz}
		\end{split} \\
		\begin{split}
			1 - \frac{x^2}{6} + \frac{x^4}{120}	&= \left( 1 - \frac{x^2}{2} + \frac{x^4}{24} \right) (a + b x^2 + c x^4 +  \dots ) \\
										&= a + \left( b - \frac{a}{2} \right) x^2 + \left( c - \frac{b}{2} + \frac{a}{24} \right) x^4 + \dots
		\end{split} \\
		\begin{matrix*}[l]
			\implies	&1 = a							&a = 1	\\
					&-\frac{1}{6} = b - \frac{a}{2}			&b  =  \frac{1}{2} - \frac{1}{6} = \frac{1}{3}	\\
					&\frac{1}{120} = c - \frac{b}{2} + \frac{a}{24}	&c = \frac{2}{15}				
		\end{matrix*}\\
		\implies \tan x = x + \frac{x^3}{3} + \frac{2 x^5}{15} + \dots
	\end{gather*}
\end{bem}

\section{''Klein-o'' und ''Gross-O'' Notation}
Betrachte $g(x)$ für $x \rightarrow x_0$\\
\begin{def*}[note = Gross O , index = gross O]
	$O(g(x))$ bezeichnet irgendeine Funktion $f$ mit $\abs{\frac{f(x)}{g(x)}}$ beschränkt für $x \rightarrow x_0$
\end{def*}
\begin{def*}[note = Klein o , index = klein o]
	$o(g(x)$ bezeichnet irgendeine Funktion $f$ mit $\abs{frac{f(x)}{g(x)}} \rightarrow 0$ für $x \rightarrow x_0$
\end{def*}
\begin{bsp*}
	$f(x) = O(1)$ für $x \rightarrow x_0$ bedeutet $f(x)$ beschränkt für $x \rightarrow x_0$ \\
	$\implies \frac{1}{x^n} = O(1) = o(1)$ für $x \rightarrow x_0, n > 0$
\end{bsp*}
\[ \left. \begin{matrix}
	f(x)	&=	&O(1)	\\
	g(x)	&=	&O(1)	
\end{matrix} \right\} f(x) = h(x) \quad \textbf{BLÖDSINN} \]
\begin{bsp*}
	\begin{gather*}
		f(x) = \sum_{k=0}^\infty a_k x^k \quad \text{Konvergenzradius } \rho > 0 \\
		\begin{split}
			\implies \left( f(x) - \sum_{k=0}^n a_k x^k \right)	&= \sum_{k=n+1}^\infty a_k x^k \\
												&= \underbrace{\left(\sum_{k=0}^\infty a_{k+n+1} x^k \right)}_{\substack{\implies\text{hat Konvergenzradius }\rho\\\text{stetige Funktion nahe } x=0\\\text{geht gegen } a_{n+1} \text{ für } x \rightarrow 0\\\implies\text{beschränkt nahe } x=0}} x^{n+1}
		\end{split} \\
		\implies f(x) = \sum_{k=0}^n a_k x^k + O(x^{n+1}) \quad \text{für } x \rightarrow 0
	\end{gather*}
\end{bsp*}
\begin{bsp*}
	$x^n = o(e^x)$ für jedes $n$ für $x \rightarrow \infty$
\end{bsp*}
\begin{bsp*}
	$e^x$ ist nicht $O(x^n)$ für $x \rightarrow \infty$
\end{bsp*}
\begin{bem}
	$f$ ist stetig in $x_0 \iff f(x) = f(x_0) + o(1)$ für $x \rightarrow x_0$
\end{bem}

Erinnerung:\\
$f(x) = O(g(x)$ falls $\abs{\frac{f(x)}{g(x)}}$ beschränkt.\\
$f(x) = o(g(x))$ falls $\abs{\frac{f(x)}{g(x)}} \rightarrow 0$\\

\subsubsection{Rechenregeln}
\begin{gather*}
	\left. \begin{matrix*}[l]
		f(x) = O(g(x))	\\
		g(x) = O(h(x))	
	\end{matrix*} \right\} f(x) = O(h(x)) \\
	\log(x) = O(x), x \rightarrow x_0 \\
	\left. \begin{matrix*}[l]
		f_1(x) = O(g(x))	\\
		g_2(x) = O(g(x))	
	\end{matrix*} \right\} f_1(x) + f_2(x) = O(g(x))
\end{gather*}
Analog $o( \cdot )$
