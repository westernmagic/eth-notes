\chapter{Integral}
\subsection{Inhalt einer Teilmenge von $\R^n$}
\begin{tabular}{ l l l }
	$n=1$:	&Länge		&$b-a$			\\
	$n=2$:	&Flächeninhalt	&$a \cdot b$		\\
	$n=3$:	&Volumen		&$a \cdot b \cdot c$	
\end{tabular}\\
\begin{bem}
	Nicht jede Teilmenge hat einen Inhalt
\end{bem}
Sei $f$ eine Funktion auf $[a,b]$. \\
Eine \textbf{Zerlegung} $\mathcal{Z}$ von $[a,b]$ besteht aus endlich vielen Zwischenpunkten
\[ a = b_0 < b_1 < \dots < b_r = b \]
sowie ''Stützpunkten''
\[ x_i \in [b_{i-1},b_i] \quad \text{für alle } 1 \leq i \leq r \]
Die \textbf{Feinheit} von $\mathcal{Z}$ ist $\delta(\mathcal{Z}) \coloneqq \max\{ b_i - b_{i-1} \mid 1 \leq i \leq r \}$
Die zugehörige Riemann-Summe:
\[ S_f(\mathcal{Z} \coloneqq \sum_{i=1}^r f(x_i) \cdot (b_i - b_i-1) \]
\begin{bem}
	ist $f \geq 0$, so ist dies der Flächeninhalt der Treppenfläche
\end{bem}
\begin{def*}[note = Riemann-Integral , index = Riemann-Integral]
	Wenn $\lim_{\delta(\mathcal{Z}) \rightarrow 0} S_f(\mathcal{Z})$ existiert, so heisst $f$ \textbf{Riemann-integrierbar}, und der Grenzwert heisst das \textbf{Riemann-Intergal}
	\[ \int_a^b f(x) dx \]
	D.h.
	\[ \forall \epsilon > 0 \exists \delta \forall \mathcal{Z} \text{ Zerlegung }: \delta(\mathcal{Z}) < \delta \implies \abs{S_f(\mathcal{Z}) - \int_a^b f(x) dx} < \epsilon \]
\end{def*}
\begin{satz*}
	Jede stetige Funktion ist Riemann-integrierbar
\end{satz*}
\begin{fakt}
	Ist $f: [a,b] \rightarrow \R^{\geq 0}$ stetig, so ist $\int_a^b f(x) dx$ der Flächeninhalt der von $\graph(f) ; y = 0 ; x = a ; x = b$ begrenzten Fläche.
\end{fakt}

\subsection{Grundeigenschaften}
\begin{itemize}
	\item Falls $f = \begin{pmatrix}f_1 \\ \vdots \\ f_n\end{pmatrix}$ vektorwertig, so ist $f$ integrierbar \gdw jedes $f_i$ integrierbar  ist, und dann gilt
		\[ \int_a^b f(x) dx = \left( \int_a^b f_i(x) dx \right)_{i=1, \dotsc , n} \]
	\item Speziell: $f: [a,b] \rightarrow \C$ ist integrierbar \gdw $\Re(f)$ und $\Im(f)$ integrierbar sind, und dann ist
		\begin{gather*}
			\Re\left( \int_a^b f(x) dx \right) = \int_a^b \Re(f(x)) dx \\
			\Im\left( \int_a^b f(x) dx \right) = \int_a^b \Im(f(x)) dx
		\end{gather*}
	\item $f, g$ integrierbar $\implies f + g$ integrierbar, und
		\[ \int_a^b (f(x) + g(x)) dx = \int_a^b f(x) dx + \int_a^b g(x) dx \]
		\begin{bew}
			\[ \begin{split}
				S_{f+g}(\mathcal{Z})	&\overset{\text{def.}}{=} \sum_{i=1}^r [f(x_i) + g(x_i)] \cdot (b_i - b_i-1) \\
								&\sum_{i=1}^r f(x_i) \cdot (b_i - b_i-1) + \sum_{i=1}^r g(x_i) \cdot (b_i - b_i-1) \\
								&\overset{\text{def.}}{=} S_{f}(\mathcal{Z}) + S_{g}(\mathcal{Z})
			\end{split} \]
			Für $\epsilon > 0$ sei $\delta > 0$ sodass
			\begin{gather*}
				\begin{split}
					\forall \mathcal{Z} : \delta(\mathcal{Z}) < \delta	&\implies \abs{S_f(\mathcal{Z}) - \int_a^b f(x) dx} < \epsilon \\
														&\text{und } \abs{S_g(\mathcal{Z}) - \int_a^b g(x) dx} < \epsilon
				\end{split} \\
				\begin{split}
					\implies	&\abs{S_{f+g}(\mathcal{Z}) - \left[ \int_a^b f(x) dx + \int_a^b g(x) dx \right] } \leq \\
							&\abs{S_f(\mathcal{Z}) - \int_a^b f(x) dx} + \abs{S_g(\mathcal{Z}) - \int_a^b g(x) dx} < \epsilon + \epsilon < 2 \epsilon
				\end{split} \\
				\implies \lim_{\delta(\mathcal{Z}) \rightarrow 0} S_{f+g} = \int_a^b f(x) dx + \int_a^b g(x) dx \quad \blacksquare
			\end{gather*}
		\end{bew}
\end{itemize}

\subsubsection{Grundeigenschaften des Integrals}
\begin{itemize}
	\item falls die rechte Seite existiert, existiert auch die linke Seite
		\begin{enumerate}[label=(\alph*)]
			\item \[ \int_a^b [f(x) + g(x)] dx = \int_a^b f(x) dx + \int_a^b g(x) dx \]
			\item \[ \int_a^b \lambda \cdot f(x) dx = \lambda \cdot \int_a^b f(x) dx \]
		\end{enumerate}
	\item falls beide Seiten existieren
		\begin{enumerate}[label=(\alph*) , resume]
			\item \[ \abs{\int_a^b f(x) dx} \leq \int_a^b \abs{f(x)} dx \]
			\item \[ \int_a^b f(x) dx \leq \int_a^b g(x) dx \text{ falls } f \leq g \text{ auf } [a,b] \]
		\end{enumerate}
\end{itemize}
\begin{bem}[note = zu (c)]
	Integral zählt Flächteile oberhalb der x-Achse positive, unterhalb der x-Achse negativ.
\end{bem}
\begin{bem}[note = zu (d)]
	\[ \int_a^b [g(x) - f(x)] dx \]
	Flächeninhalt des von $\graph(f) , \graph(g) , x=a , x=b$ umgrenzten Bereichs.
\end{bem}
\begin{enumerate}[label=(\alph*) , start=5]
	\item \[ \int_a^b f(x) dx = \int_a^c f(x) dx + \int_c^b f(x) dx \]
	falls $a \leq c \leq b$
\end{enumerate}
\begin{bem}
	\[ \int_a^b f(x) dx = 0 \]
	ist definiert!
\end{bem}
\begin{bsp}
	$f(x) = c$ konstant \\
	$\mathcal{Z}$ Zerlegung: $a = b_0 < b_1 < \dots < b_r = b ; x_i \in [b_{i-1} , b_i]$
	\begin{gather*}
		\begin{split}
			\rightsquigarrow S_{\mathcal{Z}}(f)	&\overset{\text{def.}}{=} \sum_{i=1}^r \underbrace{f(x_i)}_{=c} (b_i - b_{i-1}) \\
										&= c \cdot \sum_{i=1}^r (b_i - b_{i-1}) \\
										&= c \cdot (b_r - b_0) = c ( b-a )
		\end{split} \\
		\implies \int_a^b c dx = c (b-a)
	\end{gather*}
\end{bsp}
\begin{bsp}
	$f(x) = \frac{c}{b} \cdot x$ \\
	Für $r \geq 1$ wähle $b_i \coloneqq \frac{ib}{r} , x_i = b_i$ für die Zerlegung $\mathcal{Z}_r$
	\[ \begin{split}
		S_{\mathcal{Z}_r}(f)	&= \sum_{i=1}^r \left( \frac{c}{b} \cdot \frac{ib}{r} \right) \cdot \underbrace{\left( \frac{ib}{r} - \frac{(i-1)b}{r} \right)}_{=\frac{b}{r}} \\
						&= \frac{c}{b} \cdot \frac{b}{r} \cdot \frac{b}{r} \cdot \sum_{i=1}^r i \\
						&= \frac{cb}{r^2} \cdot \frac{r(r+1)}{2} \\
						&= \frac{cb}{2} \cdot \left( 1 + \frac{1}{r} \right) \rightarrow \frac{cb}{2} \text{ für } r \rightarrow \infty
	\end{split} \]
	Folge:
	\[ \int_0^b \left( \frac{c}{b} x \right) dx = \frac{cb}{2} \]
	= Fläche eines rechtwinkligen Dreiecks mit Katheten $b,c$
\end{bsp}
\begin{bsp}
	$f(x) = \frac{1}{x}$ auf $[1,b]$ \\
	Für $r \geq 1$ sei $\mathcal{Z}_r$ die Zerlegung mit $b_i = b^{\frac{i}{r}}$ und $x_i = b_{i-1}$
	\begin{gather*}
		\begin{split}
			\delta(\mathcal{Z}_r)	&= \max\{ b^{\frac{i}{r}} - b^{\frac{i-1}{r}} \mid 1 \leq i \leq r \} \\
							&=  \max\{ b^{\frac{i}{r}} \cdot ( 1 - b^{\frac{-1}{r}} ) \mid 1 \leq i \leq r \} \\
							&= b \cdot ( 1 - b^{\frac{-1}{r}} ) \rightarrow 0 \text{ für } r \rightarrow \infty
		\end{split} \\
		\begin{split}
			S_{\mathcal{Z}_r}(f)	&= \sum_{i=1}^r \frac{1}{b^{\frac{i-1}{r}}} \cdot (  b^{\frac{i}{r}} - b^{\frac{i-1}{r}} ) \\
							&= \sum_{i=1}^r (b^{\frac{1}{r}} - 1) \\
							&= r ( b^{\frac{1}{r}} - 1 )
		\end{split} \\
		\begin{split}
			\lim_{r \rightarrow \infty} r ( b^{\frac{1}{r}} - 1)	&= \lim_{x \rightarrow 0+} \frac{b^x - 1}{x} \\
												&\overset{\text{BH}}{=} \lim_{x \rightarrow 0+} \frac{(\log b) \cdot b^x}{1} \\
												&=\log b
		\end{split}\\
		\intertext{Antwort:}
		\int_1^b \frac{1}{x} dx = \log b \text{ für alle } b \geq 1
	\end{gather*}
\end{bsp}
\begin{bem}
	Für $a > b$ sei
	\[ \int_a^b f(x) dx \coloneqq -\int_b^a f(x) dx \]
	und $f$ auf $[b,a]$
	
	Dann gilt (e) für alle $a, b, c \in \R$, sofern alle Integrale definiert sind.
	\[ \int_a^b f(x) dx = \int_a^c f(x) dx + \int_c^b f(x) dx \]
\end{bem}
\begin{bem}
	Jede stückweise stetige Funktion ist Riemann-integrierbar, dh.
	\[ \exists \text{ Zerlegung } [a,b] = [a_0,a_1] \cup [a_2 , a_i] \cup \dots \cup [a_{n-1} , a_n] \text{ sodass } f|_{]a_{i-1} , a_i[} \]
	die Einschränkung einen stetigen Funktion auf $[a_{i-1} , a_i]$ ist, für alle $i$. Dann ist
	\[ \int_a^b f(x) dx = \sum_{i=1}^n \int_{a_{i-1}}^{a_i} f(x) dx \]
\end{bem}
\begin{fakt}
	Falls für alle bis auf endlich viele $x \in [a,b]$ gibt $f(x) = g(x)$, und $\int_a^b g(x) dx$ existiert, so existiert $\int_a^b f(x) dx$ und sie sind gleich.
\end{fakt}

\subsubsection{Hauptsatz der Infinitesimalrechnung}
\begin{satz*}[note = Hauptsatz der Infinitesimalrechnung (Version A) , index = Hauptsatz der Infinitesimalrechnung]
	Sei $f$ auf $[a,b]$ stetig. Dann ist
	\[ [a,b] \ni t \mapsto F(x) \coloneqq \int_a^t f(x) dx \]
	differenzierbar mit Ableitung $F' = f$.
	
	Begründung: \\
	Für $t$ fest sei $t' > t ; t , t' \in [a,b]$
	\[ \implies F(t') - F(t) = \int_t^{t'} f(x) dx \]
	Erinnerrung: \\
	$f$ stetig in $t$ heisst
	\[ \forall \epsilon > 0 \exists \delta > 0 \forall x \in [a,b] : \abs{x-t} < \delta \implies \abs{f(x) - f(t)} < \epsilon \]
	\begin{gather*}
		\begin{split}
			\underbrace{\abs{F(t') - [F(t) + (t'-t) \cdot f(t)]}}_{\overset{\text{soll}}{=} o(t'-t)}	&= \abs{\int_t^{t'} f(x) dx - \int_t^{t'} f(t) dx} \\
																	&= \abs{\int_t^{t'} [f(x) - f(t)] dx} \\
																	&\leq \int_t^{t'} \underbrace{\abs{f(x) - f(t)}}_{\substack{< \epsilon\\\text{falls } \abs{t'-t} < \epsilon}} dx \\
																	&\leq \int_t^{t'} \epsilon dx \\
																	&= (t'-t) \cdot \epsilon
		\end{split} \\
		\frac{\text{linke Seite}}{t'-t} \leq \epsilon \text{ falls } \abs{t'-t} < \delta \\
		\text{dh. } \lim_{t' \rightarrow t+} \frac{\text{linke Seite}}{t'-t} = 0 \quad \forall \epsilon > 0\exists \delta
	\end{gather*}
\end{satz*}
\begin{def*}[note = Stammfunktion , index = Stammfunktion]
	Sei $f$ eine auf einem Intervall $I$ definierte Funktion. Dann heisst jede auf $I$ definierte differenzierbare Funktion $F$ mit $F' = f$ eine \textbf{Stammfunktion} von $f$.
\end{def*}
\begin{bem}
	Mit $F$ ist auch $F + c$ eine Stammfunktion von $f$ für jede Konstante $c$; \\
	und jede weitere Stammfunktion von $f$ hat diese Gestalt. \\
\end{bem}
\begin{satz*}[note = Hauptsatz der Infinitesimalrechnung (Version B) , index = Hauptsatz der Infinitesimalrechnung]
	Sei $f$ auf $[a,b]$ stetig und $F$ eine Stammfunktion von $f$. Dann gilt:
	\[ \int_a^b f(x) dx = F(b) - F(a) \]
	\begin{bew}
		Mit $F_0(t) \coloneqq \int_a^t f(x) dx$ gilt \\
		$F = F_0 + c$ für $c$ konstant. Dann ist $F_0(a) = \int_a^a f(x) dx = 0$ \\
		$\implies \int_a^b f(x) dx = F_0(b) - F_0(a) = F(b) - F(a) \quad \blacksquare$
	\end{bew}
\end{satz*}

\subsubsection{Prinzip zur Berechnung von Integralen:}
\begin{enumerate}[label=(\alph*)]
	\item Errate eine Stammfunktion $F$
	\item Werte aus $F(b) - F(a) \eqqcolon F(x)|_{x=a}^{x=b}$
\end{enumerate}
\begin{bsp}
	\begin{gather*}
		\int_a^b x^s dx = \begin{cases}
			&s \in \R , 0 < a < b		\\
			&s \in \Z^{<0} , a < b < 0	\\
			&s \in \Z^{\geq 0} , a < b 	
		\end{cases} \\
		\intertext{Für $s \neq -1$ ist $\frac{x^{s+1}}{s+1}$ eine Stammfunktion von $x^s$}
		\implies \int_a^b s^s dx = \left. \frac{x^{s+1}}{s+1} \right|_{x=a}^{x=b} = \frac{b^{s+1} - a^{s+1}}{s+1}
		\intertext{Für $s = -1$ ist $\log\abs{x}$ eine Stammfunktion}
		\implies \int_a^b \frac{1}{x} dx = \log\abs{x} |_a^b = \log\abs{b} - \log\abs{a}
	\end{gather*}
\end{bsp}
\begin{bsp*}
	\[ \int_a^b \frac{1}{x^2+1} dx = \arctan x |_a^b = \arctan b - \arctan a \]
\end{bsp*}

\section{Integrationstechniken}
Bezeichnung: $\int f(x) dx$ steht für irgendeine Stammfunktion von $f$. ''unbestimmtes Integral'' \\
\begin{bem}
	\[ \underbrace{\int_a^b f(x) dx}_{\substack{\text{bestimmtes}\\\text{Integral}}} = \left. \left[ \int f(x) dx \right] \right|_a^b \]
\end{bem}
\begin{bem}[note = Konvention]
	$\int f(x) dx =$ Formel + Konstante
\end{bem}

\subsubsection{Partielle Integration}
Produktregel: $(fg)' = fg' + f'g$ \\
dh. $fg$ ist Stammfunktion von $f'g + f'g$
\begin{gather*}
	\int (fg' + f'g) dx = fg + c \\
	\int f(x) g'(x) dx = f(x) g(x) - \int f'(x) g(x) dx
\end{gather*}
Das funktioniert gut, wenn $f'$ einfacher ist als $f$ und $g$ nicht komplizierter als $g'$ \\
\begin{bsp*}
	\begin{gather*}
		\int \underbrace{x}_{\downarrow} \underbrace{e^x}_{\uparrow} dx = x e^x - \int 1 e^x dx = x e^x - e^x + c \\
		\begin{matrix*}[l]
			f(x) = x	&f'(x) = 1		\\
			g'(x) = e^x	&g(x) = e^x	
		\end{matrix*}
	\end{gather*}
\end{bsp*}
\begin{bsp*}
	\begin{gather*}
		n \in \Z^{\geq 0} , \lambda \in \C \setminus \{ 0 \} \\
		I_n \coloneqq \int \underbrace{t^n}_{\downarrow} \underbrace{e^{\lambda t}}_{\uparrow} dt = t^n \frac{e^{\lambda t}}{\lambda} - \int n t^{n-1} \frac{e^{\lambda t}}{\lambda} = \frac{t^n e^{\lambda t}}{\lambda} - \frac{n}{\lambda} I_{n-1} \\
		\intertext{Induktion $\rightsquigarrow$}
		I_n = (-1)^n \frac{n!}{\lambda^{n+1}} \left( \sum_{k=0}^n \frac{(-\lambda t)^k}{k!} \right) e^{\lambda t}
	\end{gather*}
\end{bsp*}
\begin{bsp*}
	\[ \begin{split}
		\int \log t dt	&= \int (\underbrace{\log t}_{\downarrow}) \cdot \underbrace{1}_{\uparrow} \\
					&= (\log t) t - \int \underbrace{\frac{1}{t} t}_{=1} dt \\
					&= (\log t) t - t + c
	\end{split} \]
\end{bsp*}
\begin{bsp*}
	\[ \begin{split}
		\int (\log t)^2 dt	&= \int (\log t)^2 \cdot 1 dt \\
					&= (\log t) t - \int 2 (\log t) \frac{1}{t} t dt \\
					&= (\log t)^2 t - 2 \int \log t dt \\
					&= (\log t)^2 t - 2 (\log t) t + 2t + c
	\end{split} \]
\end{bsp*}
\begin{bsp*}
	\begin{gather*}
		\begin{split}
			I	&= \int \underbrace{e^{\alpha t}}_{\uparrow} \underbrace{\cos \beta t}_{\downarrow} dt \\
				&= \frac{e^{\alpha t}}{\alpha} \cos \beta t - \int \frac{e^{\alpha t}}{\alpha} (-\beta \sin \beta t) dt \\
				&= \frac{e^{\alpha t}}{\alpha} \cos \beta t + \frac{\beta}{\alpha} \int \underbrace{e^{\alpha t}}_{\uparrow} \underbrace{\sin \beta t}_{\downarrow} dt \\
				&= \frac{e^{\alpha t}}{\alpha} \cos \beta t + \frac{\beta}{\alpha} \left( \frac{e^{\alpha t}}{\alpha} \sin \beta t - \int \frac{e^{\alpha t}}{\alpha} \beta \cos \beta t dt \right) \\
				&= \frac{e^{\alpha t}}{\alpha} \cos \beta t + \frac{\beta e^{\alpha t}}{\alpha^2} \sin \beta t - \frac{\beta^2}{\alpha^2} \underbrace{\int e^{\alpha t} \cos \beta t dt}_{=I} \\
		\end{split}\\
	\left(1 + \frac{\alpha^2}{\beta^2} \right) I = \frac{e^{\alpha t}}{\alpha} \cos \beta t + \frac{\beta e^{\alpha t}}{\alpha^2} \sin \beta t + c \\
		I = \int e^{\alpha t} \cos \beta t dt = \frac{\alpha^2}{\alpha^2 + \beta^2} \left( \frac{e^{\alpha t}}{\alpha} \cos \beta t + \frac{\beta e^{\alpha t}}{\alpha^2} \sin \beta t + c \right) \\
		\begin{split}
			\int e^{\alpha t} \cos \beta t dt	&= \int e^{\alpha t} \cdot \frac{e^{\imath \beta t} + e^{-\imath \beta t}}{2} dt \\
									&= \int \frac{e^{(\alpha + \imath \beta) t} + e^{(\alpha -\imath \beta) t}}{2} dt \\
									&= \frac{\frac{e^{(\alpha + \imath \beta) t}}{\alpha + \imath \beta} + \frac{e^{(\alpha - \imath \beta) t}}{\alpha - \imath \beta}}{2} + c \\
									&= \frac{e^{\alpha t}}{\alpha^2 + \beta^2} (\alpha \cos \beta t + \beta \sin \beta t) + c
		\end{split}
	\end{gather*}
\end{bsp*}

\subsubsection{Substitution}
Sei $f' = f:$ Kettenregel: $(F(\varphi(y)))' = F'(\varphi(y)) \cdot \varphi'(y) = f(\varphi(y)) \cdot \varphi'(y)$ \\
$y \mapsto F(\varphi(y))$ ist Stammfunktion von $y \mapsto f(\varphi(y)) \cdot \varphi'(y)$
\[ \int f(\varphi(y)) \cdot \varphi'(y) dy = \left. \left( \int f(x) dx \right) \right|_{x=\varphi(y)} \]

Anwendung in wei Richtungen: \\
1. Fall; Integrand hat die Form $f(\varphi(y)) \cdot \varphi'(y)$ \\
\begin{bsp*}
	\begin{gather*}
		\int \sin^3 y \cdot cos y dy = \begin{vmatrix}
			x = \sin y \\
			\frac{dx}{dy} = cos y
		\end{vmatrix} = \int x^3 dx = \frac{x^4}{4} + c = \frac{\sin^4 y}{4} + c
	\end{gather*}
\end{bsp*}
\begin{bem}
	\begin{gather*}
		x = \varphi(y) \\
		\frac{dx}{dy} = \varphi'(y) \\
		\implies dx = \varphi'(y) dy \\
		\implies dx = \frac{dx}{dy} \cdot dy
	\end{gather*}
\end{bem}
\begin{bsp*}
	\begin{gather*}
		\begin{split}
			\int \tan t dt	&= \int \frac{\sin t}{\cos t} dt \\
						&= -\int \frac{1}{\cos t} ( -\sin t) dt \\
			u = \cos t \\
			\frac{du}{dt} = -\sin t \\
			du = -\sin t dt \\
						&= -\int \frac{1}{u} du \\
						&= -\log \abs{u} + c \\
						&= -\log \abs{\cos t} + c
		\end{split}
	\end{gather*}
\end{bsp*}

2. Fall: \\
\begin{bsp*}
	\begin{gather*}
		\begin{split}
			\int \frac{x}{\sqrt{2x-3}} dx	&= \begin{vmatrix}
										y = \sqrt{2x-3} \\
										y^2 = 2x-3 \\
										x = \frac{y^2+3}{2} \\
										\frac{dx}{dy} = y \\
										dx = y \cdot dy
									\end{vmatrix} \\
								&= \int \frac{\frac{y^2+3}{2}}{y} \cdot y dy \\
								&= \frac{1}{2} \left( \frac{1}{3} y^3 + 3y \right) + c \\
								&= \frac{y}{6} (y^2+9) + c \\
								&= \frac{\sqrt{2x-3}}{6} (2x+6) + c
		\end{split}
	\end{gather*}
\end{bsp*}
\begin{bsp*}
	\begin{gather*}
		\begin{split}
			\int \frac{x^2 dx}{\sqrt{1-x^2}}	&= \begin{vmatrix}
											x = \sin t \\
											dx = \cos t dt \\
											1 - x^2 = \cos^2 t \\
											t \in ] -\frac{\pi}{2} , \frac{\pi}{2} [
										\end{vmatrix} \\
									&= \int \frac{\sin^2 t}{\cos t} \cdot \cos t dt \\
									&= \sin^2 t dt \\
			\cos 2t = 1 - 2\sin^2 t \\
			\sin^2 t = \frac{1 - \cos 2t}{2} \\
									&= \int \frac{1 - \cos 2t} dt \\
									&= \frac{t}{2} - \frac{1}{2} \frac{\sin 2t}{2} + c \\
									&= \frac{t}{2} - \frac{1}{2} \sin t \cos t + c \\
									&= \frac{\arcsin x}{2} - \frac{x}{2} \sqrt{1-x^2} + c
		\end{split}
	\end{gather*}
\end{bsp*}
\begin{bsp*}
	\begin{gather*}
		\begin{split}
			\int \frac{dx}{\sqrt{1+e^x}}	&= \begin{vmatrix}
										y = e^x \\
										dz = e^x dx = y dx \\
										dx = \frac{1}{y} dy
									\end{vmatrix} \\
								&= \int \frac{dy}{y \cdot \sqrt{1+y}} \\
								&= \begin{vmatrix}
										u = \sqrt{1+y} \\
										y = u^2 - 1 \\
										dy = 2u du
									\end{vmatrix} \\
								&= \int \frac{2u du}{(u^2-1)u} \\
								&= \int \frac{2}{u^2-1} du \\
								&= \int \left( \frac{-1}{u+1} + \frac{1}{u-1} \right) du \\
								&= -\log \abs{u+1} + \log\abs{u-1} + c \\
								&= \log\abs{\frac{u-1}{u+1}} + c \\
								&= \log\abs{\frac{\sqrt{1+e^x}-1}{\sqrt{1+e^x}+1}} + c
		\end{split}
	\end{gather*}
\end{bsp*}
\begin{bsp*}
	\begin{gather*}
		\int \frac{dx}{\sqrt{(b-x)(x-a)}} \quad , x \in ]a,b[ \\
		\begin{split}
			(b-x)(x-a)	&= -x^2 + (a+b) x - ab \\
					&= - \left( x - \frac{a+b}{2} \right)^2 + \left( \frac{a+b}{2} \right)^2 - ab \\
					&= \left( \frac{b-a}{2} \right)^2 - \left( x - \frac{a+b}{2} \right)^2 \\
					&= \left( \frac{b-a}{2} \right)^2 \left( 1 - \left( \frac{2x-a-b}{b-a} \right)^2 \right)
		\end{split} \\
		y = \frac{2x-a-b}{b-a} \\
		dy = \frac{2 dx}{b-a} \\
		dx = \frac{b-a}{2} dy \\
		\\
		\begin{split}
			\int \frac{dx}{\sqrt{(b-x)(x-a)}}	&= \int \frac{\frac{b-a}{2} dy}{\frac{b-a}{2} \sqrt{1-y^2}} \\
									&= \int \frac{dy}{\sqrt{1-y^2}} \\
									&= \arcsin y + c \\
									&= \arcsin \frac{2x-a-b}{b-a} + c
		\end{split}
	\end{gather*}
\end{bsp*}

Vereinfachung von $\sqrt{ax^2+bx+c}$ kann führen zu:
\begin{itemize}
	\item \[ \sqrt{1-y^2} : \begin{cases}
			y = \sin x \\
			y = \cos x
		\end{cases}\]
	\item \[ \sqrt{1+y^2} : y = \sinh x \]
	\item \[ \sqrt{y^2-1} : y = \cosh x \]
\end{itemize}

\subsubsection{Integration von rationalen Funktionen}
\begin{bsp*}
	\begin{gather*}
		\int \frac{1-x^6}{x(x^2+1)^2} dx \\
		\text{Ansatz: } \frac{1-x^6}{x(x^2+1)^2} = \frac{a}{x} + \frac{b+cx+dx^2+e^3}{(x^2+1)^2} + f + gx \\
		\begin{matrix*}[l]
			1-x^6	&= a(x^2+1)^2		&+ (b+cx+dx^2+e^3)x \\
					&				&+ (f+gx)x(x^2+1)^2 \\
					&= a(x^4+2x^2+1)	&+ (bx+cx^2+dx^3+e^4)x \\
					&				&+ f(x^5+2x^3+x) \\
					&				&+ g(x^6+2x^4+x^2)
		\end{matrix*} \\
		\begin{matrix*}[l]
			\implies	&g = -1				\\
					&f= 0	&b + f = 0		\\
					&a = 1	&a + e + 2g = 0	\\
					&b = 0				\\
					&e = 1				\\
					&c = -1				\\
					&d = 0
		\end{matrix*} \\
		\begin{split}
			\dots						&= \int \left( \frac{1}{x} + \frac{x^3-x}{(x^2+1)^2} - x \right) dx \\
									&= \log\abs{x} - \frac{x^2}{2} + \int \frac{x^3-x}{(x^2+1)^2} dx \\
			\int \frac{x^3-x}{(x^2+1)^2} dx	&= \begin{vmatrix}
											y = x^2+1 \\
											dy = 2xdx \\
											x^3-x = x(x^2-1) = \frac{2x}{2} (y-2)
										\end{vmatrix} \\
									&= \int \frac{\frac{y-2}{2}}{y^2} dy \\
									&= \int \left( \frac{1}{2y} - \frac{1}{y^2} \right) dy \\
									&= \frac{1}{2} \log\abs{y} + \frac{1}{y} + c \\
									&= \frac{1}{2} \log(x^2+1) + \frac{1}{x^2+1} + c
		\end{split}
		\intertext{Gesamtresultat}
		\log\abs{x} - \frac{x^2}{2} + \frac{1}{2} \log(x^2+1) + \frac{1}{x^2+1} + c
	\end{gather*}
\end{bsp*}
\begin{bsp*}
	\begin{gather*}
		\int \frac{x^2-1}{(x^2+1)^2} dx \\
		\frac{x^2-1}{(x^2+1)^2} = \frac{ax^2}{(x^2+1)^2} + \frac{b}{(x^2+1)} \\
		x^2-1 = ax^2 + b(x^2+1) \\
		b = -1 \\
		1 = a + b \implies a = 2 \\
		\\
		\int \frac{x^2-1}{(x^2+1)^2} dx = \frac{2x^2 dx}{(x^2+1)^2} - \underbrace{\int \frac{dx}{x^2+1}}_{=\arctan x + c} \\
		\begin{split}
			\int \underbrace{\frac{2x}{(x^2+1)^2}}_{\uparrow} \underbrace{x}_{\downarrow} dx &\qquad \left( \frac{1}{x^2+1} \right)' = \frac{-2x}{(x^2+1)^2} \\
				&= \frac{-1}{x^2+1} x - \int \frac{-1}{x^2+1} \cdot 1 dx \\
				&= \frac{-x}{x^2+1} + \arctan x + c
		\end{split} \\
		\int \frac{x^2-1}{(x^2+1)^2} dx = \frac{-x}{x^2+1} + c
	\end{gather*}
\end{bsp*}

Methode:
\begin{enumerate}[label=(\arabic*)]
	\item Faktorisiere Nenner
	\item Partialbruchzerlegung
	\item $\int \frac{\text{Polynom}}{(x-a)^n} dx \rightsquigarrow$ Substituiere $y = x-a \rightsquigarrow \int \frac{\text{Polynom}(y)}{y^n} dy$ bekannt.
	\item $\int \frac{\text{Polynom}}{quad(x)^n} dx \rightsquigarrow$ Substituiere $y = ax+b$ sodass Nenner $(y^2+1)^n$ wird
	\item $\int \frac{(\text{Polynom in } y^2) \cdot y}{(y^2+1)^n} dy \rightsquigarrow$ Substituiere $y^2+1 = z$
	\item $\int \frac{\text{Polynom} in y^2}{(y^2+1)^n} dy$ \\
		$n=1$: $\arctan y$ \\
		$n>1$: $\int \frac{dy}{(y^2+1)^n} = \frac{2n-3}{2n-2} \int \frac{dy}{(y^2+1)^{n-1}} + \frac{1}{2n-2} \cdot \frac{y}{(y^2+1)^{n-1}}$
\end{enumerate}

\subsection{Uneigentliche Integrale}
\begin{tabular}{ l l l }
	Bisher:	&Integral einer Funktion auf	&$[a,b]$	\\
	Jetzt:		&'' - ''				&$[a,b[$	\\
			&'' - ''				&$]a,b]$	\\
			&'' - ''				&$]a,b[$	
\end{tabular}
\[ \int_a^b = \int_a^c + \int_c^b \text{ für } a < c < b \]
\begin{def*}[note = uneigentlischer Grenzwert , index = uneigentlischer Grenzwert]
	Sei $f$ auf $[a,b[$ definiert, so dass für jedes $c \in [a,b[$ $\int_a^c f(x) dx$ existiert. Dann heisst
	\[ \int_a^b f(x) dx \coloneqq \lim_{c \rightarrow b-} \int_a^c f(x) dx \]
	das \textbf{uneigentliche Integral} von $f$ über $[a,b[$, falls der limes existiert. Falls nicht, sagt man, dass das uneigentliche Interval divergiert. \\
	\begin{bem}
		Dies gilt wenn $f$ stetig ist. Denn dann ist $f|_{[a,c]}$ stetig.
	\end{bem}
	
	Analog: $f$ auf $]a,b]$ definiert, $f|_{[c,b]}$ integrierbar für alle $a < c < b $
	\[ \rightsquigarrow \int_a^b f(x) dx \coloneqq \lim_{c \rightarrow a+} \int_c^b f(x) dx \]
	
	Analog: $f$ auf $]a,b[$ definiert, $f|_{[c,d]}$ integrierbar für alle $a < c < d < b$
	\[ \rightsquigarrow \int_a^b f(x) dx \coloneqq \lim_{c \rightarrow a+} \lim_{d \rightarrow b-} \int_c^d f(x) dx \]
\end{def*}
\begin{bsp}
	\begin{gather*}
		\int_a^\infty \frac{dx}{x^s} \quad a > 0 \quad x \mapsto \frac{1}{x^s} \text{ stetig auf } [a,\infty[ \\
		\int_a^\infty \frac{dx}{x^s} \overset{\text{def.}}{=} \lim_{b \rightarrow \infty} \int_a^b \frac{dx}{x^s} \\
		= \begin{cases}
			\lim_{b \rightarrow \infty} \left( \left. \frac{x^{1-s}}{1-s} \right|_a^b \right) = \lim_{b \rightarrow \infty} \left( \frac{b^{1-s} - a^{1-s}}{1-s} \right) = \begin{cases}
				\frac{a^{1-s}}{s-1}	&s > 1	\\
				\infty			&s < 1 	
			\end{cases}	&s \neq 1	\\
			\lim_{b \rightarrow \infty} ( \log x |_a^b ) = \lim_{b \rightarrow \infty} (\log b - \log a) = \infty	&s = 1
		\end{cases}
	\end{gather*}
\end{bsp}

Anwendung 1: Ist $f$ auf $[a,\infty[$ stetig, und $\abs{f(x)} \leq \frac{c}{x^s}$ mit konstanten $s>1$ und $c$, dann konvergiert $\int_a^\infty f(x) dx$

Anwendung 2: Ist $f$ auf $[a,\infty[$ stetig und $f(x) \geq \frac{c}{x^s}$ mit konstanten $s \leq 1$ und $c > 0$, dann divergiert $\int_a^\infty f(x) dx$ gegen $\infty$

\subsubsection{Majorantenkriterium}
$f$ auf $[,a\infty[$ definiert auf jedem $[a,b]$ integrierbar, und $\abs{f(x)} \leq \frac{c}{x^s}$ mit Konstanten $c$ und $s > 1$ $\implies \int_a^infty f(x) dx$ existiert.

\subsubsection{Minorantenkriterium}
$f$ auf $[,a\infty[$ definiert auf jedem $[a,b]$ integrierbar, und $f(x) \geq \frac{c}{x^s}$ mit Konstanten $c > 0$ und $s \leq 1$ $\implies \int_a^infty f(x) dx$ divergiert gegen $\infty$.

\begin{bsp*}
	\begin{gather*}
		\int_0^\infty \frac{t^2+4}{(1+4t^2)^{\frac{3}{2}}} \dd t \\
		\intertext{Nur Problem bei $\infty$}
		\frac{t^2+4}{(1+4t^2)^{\frac{3}{2}}} = \frac{t^2(1+\frac{4}{t^2})}{t^3(\frac{1}{t^2}+4)^{\frac{3}{2}}} = \frac{1}{t} \frac{1+\frac{4}{t^2}}{(\frac{1}{t^2}+4)^{\frac{3}{2}}} \rightarrow \frac{1}{8} \text{ für } t \rightarrow \infty \implies = \infty
	\end{gather*}
\end{bsp*}
\begin{bsp*}
	\begin{gather*}
		\int_2^\infty \frac{dx}{x(\log x)^s} \quad s > 0 \\
		= \int_{\log 2}^\infty \frac{dy}{y^s} \begin{cases}
			\text{konvergent}	&s > 1	\\
			\text{divergent}		&s \leq 1	
		\end{cases}
	\end{gather*}
\end{bsp*}
\begin{bsp*}
	\[ \int_{10}^\infty \frac{dx}{x \cdot \log x \cdot \log \log x} \text{ divergent} \]
\end{bsp*}
\begin{bsp*}
	\begin{gather*}
		\begin{split}
			\int_{-\infty}^\infty \frac{dt}{1+t^2}	&= \lim_{a \rightarrow -\infty} \lim_{b \rightarrow \infty} \arctan t |_a^b \\
										&= \lim_{a \rightarrow -\infty} \lim_{b \rightarrow \infty} (\arctan b - \arctan a) \\
										&= \frac{\pi}{2} - \left( -\frac{\pi}{2} \right) \\
										&= \pi \\
										&= \lim_{b \rightarrow \infty} \int_{-b}^b \frac{dt}{1+t^2}
		\end{split}
	\end{gather*}
\end{bsp*}
\begin{bsp*}
	Vorsicht! \\
	\begin{gather*}
		\int_0^\infty \frac{\sin x}{x} dx \\
		\intertext{Kein Problem bei 0, da stetige Fortsetzung =1} %FIXME
		\int_1^t \underbrace{\sin x}_{\uparrow} \underbrace{\frac{1}{x}}_{\downarrow} dx = \underbrace{\left. -\cos x \cdot \frac{1}{x} \right|_1^t}_{\frac{\cos 1}{1} - \frac{\cos t}{t}} - \underbrace{-\int_1^t \cos x \cdot \frac{-1}{x^2} dx}_{\substack{\text{konvergiert für}\\t \rightarrow \infty\\\text{nach Majorantenkriterium!}}} = \text{ konvergiert für } t \rightarrow \infty \\
		\int_0^t = \int_0^1 + \int_1^t \implies \int_0^\infty \frac{\sin x}{x} dx \text{ konvergiert}
	\end{gather*}
\end{bsp*}

Analog für $\int_{-\infty}^\cdot \dots$

\begin{bsp*}
	\begin{gather*}
		a < b \\
		\int_a^b \frac{dx}{(x-a)^s} = \lim_{c \rightarrow a+} \int_c^b \frac{dx}{(x-a)^s} \\
		s < 1 \implies \lim_{c \rightarrow a+} \left. \frac{(x-a)^{1-s}}{1-s} \right|_c^b =  \lim_{c \rightarrow a+} \frac{(b-a)^{1-s} - (c-a)^{1-s}}{1-s} \text{ konvergiert} \\
		s > 1 \text{ divergiert} \\
		s = 1 :  \lim_{c \rightarrow a+} \log(x-a)|_c^b =  \lim_{c \rightarrow a+} [\log(b-a) - \log(c-a)] \text{ divergiert}
	\end{gather*}
\end{bsp*}

\subsubsection{Majorantenkriterium}
$f$ auf $]a,b]$ definiert, auf jedem $[c,b]$ integrierbar, und $\abs{f(x)} \leq \frac{c}{(x-a)^s}$ für Konstanten $c$ und $s < 1$. Dann konvergiert $\int_a^b f(x) dx$

\subsubsection{Minorantenkriterium}
$f$ auf $]a,b]$ definiert, auf jedem $[c,b]$ integrierbar, und $\abs{f(x)} \geq \frac{c}{(x-a)^s}$ für Konstanten $c>0$ und $s \geq 1$. Dann divergiert $\int_a^b f(x) dx$

\begin{bsp*}
	\begin{gather*}
		\int_0^\infty \frac{dx}{\sqrt{\tan x}} \\
		\intertext{Kein Problem bei $\frac{\pi}{2}$, stetig fortsezbar = 0}
		\text{bei } 0: \frac{1}{\sqrt{\tan x}} = \frac{1}{\sqrt{x \frac{\sin x}{x} \frac{1}{\cos x}}} = \frac{1}{x^{\frac{1}{2}}} \sqrt{ \frac{\cos x}{\frac{\sin x}{x}}} \rightarrow 1 \text{ für } x \rightarrow 0 \\
		\implies \exists \delta > 0 : \forall x > 0 : x < \delta \implies \sqrt{ \frac{\cos x}{\frac{\sin x}{x}}} \leq 2 \\
		\text{Majorantenkriterium: } \frac{1}{\sqrt{\tan x}} \leq \frac{2}{x^{\frac{1}{2}}} \\
		\implies \int_0^\delta \frac{dx}{\sqrt{\tan x}} \text{ konvergiert} \\
		\implies \int_0^{\frac{\pi}{2}} \frac{\dd x}{\sqrt{\tan x}} \text{ konvergiert}
	\end{gather*}
\end{bsp*}
\begin{bsp*}
	\begin{gather*}
		\int_0^{\frac{\pi}{2}} \frac{dx}{\sin x} \text{ divergent} \\
		\frac{1}{\sin x} = \underbrace{\frac{x}{\sin x}}_{\geq \frac{1}{2}} \frac{1}{x} \implies \int_0^\delta \frac{dx}{\sin x} = \infty
	\end{gather*}
\end{bsp*}
\begin{bsp*}
	\begin{gather*}
		\int_{-1}^1 \frac{dx}{x^3} = \int_{-1}^1 x^{-3} dx = \left. \frac{x^{-2}}{-2} \right|_{-1}^1 = \frac{1^{-2}}{-2} - \frac{(-1)^{-2}}{-2} = 0 \textbf{ FALSCH} \\
		\int_{-1}^1 \frac{dx}{x^3} = \underbrace{\int_{-1}^0 \frac{dx}{x^3}}_{\infty} - \underbrace{\int_0^1 \frac{dx}{x^3}}_{-\infty}
	\end{gather*}
\end{bsp*}
