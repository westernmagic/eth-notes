\chapter{Funktionen}
Funktionsterm = Formel \\
\begin{bsp*}
	\begin{gather*}
		f(x) = \frac{1}{x+1} \\
		g(x) = 0 \\
		f(x) - f(x) \overset{?}{=} g(x)
	\end{gather*}
	Definitionsbereich?
\end{bsp*}\marginpar{$[a,b) = [a,b[$ halb-offenes Intervall\\\\ $\mapsto$ wird abgebildet auf\\\\ $\{\} = \emptyset$}
\begin{def*}[note = Funktion , index = Funktion]
	Eine Funktion $f: A \rightarrow B$ besteht aus: \\
	\qquad Definitionsbereich $A$, \\
	\qquad Zielbereich $B$, und \\
	\qquad einer Zuordnung eines $f(x) \in B$ für jedes $x \in A$
\end{def*}
\begin{bem}
	Zielbereich $B$ nicht mit Bildmenge verwechseln! \\
	\begin{def*}[note = Bildmenge , index = Bildmenge]
		\[
			\text{Bildmenge } = \begin{cases}
				\{ f(a)  | a \in A \} \subset B \\
				\{ b \in B | \exists a \in A : f(a) = b \}
			\end{cases}
		\]
	\end{def*}
\end{bem}
\begin{bsp*}
	\[ f: [0,\infty [ \rightarrow \R , x \mapsto 0 \]
\end{bsp*}
\begin{bsp*}
	\[ f: \emptyset \rightarrow \R , x \mapsto 0 \qquad B = \varnothing \]
\end{bsp*}

\section{Beschreibung von Funktionen}
\subsubsection{durch ihren Graphen}
\begin{def*}[note = Graph , index = Graph]
	\[ \graph(f) \coloneqq \{ ( a , f(a) ) | a \in A \} \supset A \times B \]
\end{def*}
\begin{def*}[note = Kreuzmenge , index = Kreuzmenge]
	\[ A \times B \coloneqq \{ ( a , b ) | a \in A , b \in B \} \]
\end{def*}

\subsubsection{Fallunterscheidung}
\begin{bsp*}[note = $\sgn : \R \rightarrow \R$]
	\[
		\sgn x =
		\begin{cases}
			1	&x > 0	\\
			0	&x = 0	\\
			-1	&x < 0	
		\end{cases}
	\]
\end{bsp*}
\begin{bsp*}
	\[ f: \R \rightarrow \R, x \mapsto \frac{1}{x+1} \]
\end{bsp*}
\begin{fakt}
	$f$ ist durch $A , B , \graph(f)$ eindeutig bestimmt, denn \\
	$f(a)$ = das einzige $b \in B$ mit $(a,b) \in \graph(f)$ \\
	$\implies (a,b) = (a',f(a')) | a \in A$ \\
	$\implies a = a', \: b = f(a') = f(a)$
\end{fakt}
\begin{bem}
	Eine Teilmenge $\Gamma \subset A \times B$ ist der Graph einer Funktion $A \rightarrow B$ \gdw für jeden $a \in A$ ein eindeutiges $b \in B$ existiert mit $(a,b) \in \Gamma$ \\
	\[ \forall a \in A \:\exists! b \in B : (a,b) \in \Gamma \]
\end{bem}

\subsection{Arten, eine Funktion anzugeben}
\begin{itemize}
	\item Formel
	\item Graph
	\item Wertetabelle
	\item Differentialgleichung
	\item Fallunterscheidung
	\item implizit
	\item Funktionalgleichung (Eigenschaften)
\end{itemize}
\begin{def*}[note = Polynomfunktion , index = Funktion!Polynom-]
	\[ f(x) = \sum_{i=0}^n a_i x^i \]
\end{def*}
\begin{def*}[note = Gebrochenrationalefunktion , index = Funktion!Gebrochenrationale-]
	\[ f(x) = \frac{\sum_{i=0}^n a_i x^i}{\sum_{k=0}^m b_k x^k} \]
\end{def*}
\begin{def*}[note = Potenzfunktion , index = Funktion!Potenz-]
	\[\begin{matrix*}[l]
		n \in \Z^{\geq 0}:	&f(x) = x^n	\\
		n=0 :				&f(x) = x^0 \coloneqq 1	&|x \in \R
	\end{matrix*}\]
\end{def*}\marginpar{$0^0 = 1$}
\begin{def*}[note = Wurzelfunktion , index = Funktion!Wurzel-]
	\[ f(x) = x^{\frac{1}{m}} \]
\end{def*}
\begin{def*}[note = rationale Potenzfunktion , index = Funktion!Potenz-!rationale]
	\[ f(x) = x^{\frac{n}{m}} \]
\end{def*}
\begin{def*}[note = allgemeine Potenzfunktion , index = Funktion!Potenz-!allgemeine]
	\[ f(x) = x^a \qquad | x > 0, a \in \R \]
\end{def*}
\begin{def*}[note = algebraische Funktion , index = Funktion!algebraische]
	zusammengesetzt aus rationalen Funktionen und deren Umkehrfunktionen.
\end{def*}
\begin{def*}[note = elementare Funktion , index = Funktion!elementare]
	zusammengesetzt aus algebraischen Funktionen, exponential Funktion, trigonometrischen Funktion und deren Umkehrfunktionen.
\end{def*}
weitere Funktionen: Bezel, Gamma

\subsubsection{Implizite Funktionen}
\begin{bsp*}
	\begin{gather*}
		x^2 + y^2 = 1 \\
		f: [-1,1] \rightarrow \R , x \mapsto \sqrt{1-x^2} \\
		g: [-1,1] \rightarrow \R , x \mapsto -\sqrt{1-x^2}
	\end{gather*}
\end{bsp*}
Prinzip:\\
Gegeben eine Teilmenge $C \subset \R^2$.\\
Wähle eine Teilmenge $C' \subset C$, sodass $C' = \graph(f)$ für $f: I \rightarrow \R , I \subset \R , I$ Intervall\\
\begin{bsp*}
	\[ C: x^3 + y^3 = 3xy \]
	GRAPH\\
	Funktionen:
	\begin{gather*}
		f_1: [0,\infty[ \rightarrow ]-\infty,0] \\
		f_2: [0,a] \rightarrow [0,a] \\
		f_3: ]-\infty,a] \rightarrow [0,\infty[ \\
		\intertext{\textbf{nicht verwechseln mit}}
		g: \R^2 \rightarrow \R, (x,y) \mapsto x^3 + y^3 - 3xy \\
		\drsh g( x , f(x) ) = 0 \\
		\drsh g( h(y) , y ) = 0
	\end{gather*}
\end{bsp*}

\subsubsection{Funktionalgleichung}
\begin{bsp*}
	\begin{gather*}
		e^{x+y} = e^x \cdot e^y \\
		\exp(x+y) = \exp(x) \cdot \exp(y)
	\end{gather*}
\end{bsp*}
\begin{fakt}
	Die Exponentialfunktion $\exp: \R \rightarrow \R$ ist die einzige stetige Funktion mit $\exp(x+y) = \exp(x) \cdot \exp(y)$ und $\exp(1) = e$
\end{fakt}
\begin{bsp*}
	\[ (n+1)! = n! \cdot (n+1) \]
\end{bsp*}

\subsubsection{Differentialgleichungen}
Gleichung zwischen $x, f(x), f'(x), \dotsc$\\
Anfangswerte, um die Funktion zu bestimmen

\section{Eigenschaften von Funktionen}
\begin{def*}[note = Injektivität , index = Injektivität]
	Eine Funktion $f: X \rightarrow Y$ heisst \textbf{injektiv}, wenn\\
	\[ \forall x, x' \in X : f(x) = f(x') \implies x = x' \]
\end{def*}
\begin{def*}[note = Surjektivität , index = Surjektivität]
	Eine Funktion $f: X \rightarrow Y$ heisst \textbf{surjektiv}, wenn\\
	\[ \forall y \in Y \exists s \in X : f(x) = y \]
\end{def*}
\begin{def*}[note = Bijektivität , index = Bijektivität]
	Eine Funktion $f: X \rightarrow Y$ heisst bijektiv, wenn sie sowohl injektiv als auch surjektiv ist, dh. \\
	\[ \forall y \in Y \exists! x \in X : f(x) = y \]
\end{def*}
\begin{tabular}{l l l l}
	Definitionsbereich	&von $f$:	&$\dom(f)$	&(domain)					\\
	Zielbereich			&von $f$:	&$\range(f)$							\\
	Bildmenge			&von $f$:	&$\image(f)$	&$= \{ f(x) | x \in X \} \subset Y$	
\end{tabular}\\
\begin{bem}
	$f$ sujektiv $\iff \image(f) = \range(f)$
\end{bem}
\begin{def*}[note = Umkehrfunktion , index = Umkehrfunktion]
	Ist $f$ bijektiv, so heisst die Funktion $f^{-1}: Y \rightarrow X, y \mapsto ($das einzige $x \in X$ mit $f(x) = y)$ die \textbf{Umkehrfunktion} von $f$.
\end{def*}
\begin{bem}
	\begin{gather*}
		\forall x \in X : f^{-1}(f(x)) = x \\
		\forall y \in Y : f(f^{-1}(y)) = y
	\end{gather*}
\end{bem}
\begin{bem}
	\begin{gather*}
		\graph(f^{-1}) = \{ (y , f^{-1}(y)) | y \in Y \} = \{ (f(x) , x) | x \in X \} \\
		\implies \text{Spiegelung an der $x=y$ Diagonale}
	\end{gather*}
\end{bem}
\begin{bsp*}
	\begin{tabular}{l l l r r}
		a.)	&$\R \rightarrow \R$						&,$x \mapsto \sin x$	&nicht injektiv,	&nicht surjektiv		\\
		b.)	&$\R \rightarrow [ -1 , 1 ]$					&,$x \mapsto \sin x$	&nicht injektiv,	&surjektiv			\\
		c.)	&$[ -\frac{\pi}{2} , \frac{\pi}{2} [ \rightarrow \R$		&,$x \mapsto \sin x$	&injektiv,		&nicht surjektiv		\\
		d.)	&$[ -\frac{\pi}{2} , \frac{\pi}{2} [ \rightarrow [-1,1]$	&,$x \mapsto \sin x$	&injektiv,		&surjektiv			\\
			&									&				&			&$\implies$ bijektiv	
	\end{tabular}
\end{bsp*}
\begin{def*}[note = arcsin , index = arcsin arccos , indexformat = {1 2}]
	Umkehrfunktion von $[ -\frac{\pi}{2} , \frac{\pi}{2} [ \rightarrow [-1,1], x \mapsto \sin x$ ist $\arcsin: [-1,1] \rightarrow [ -\frac{\pi}{2} , \frac{\pi}{2} [$ \\
	analog ist $\arccos: [-1,1] \rightarrow [0,\pi]$ die Umkehrfunktion von $[0,\pi] \rightarrow [-1,1], x \mapsto \cos x$
\end{def*}
\begin{bsp*}
	\begin{gather*}
		[0,\infty[ \rightarrow [0,\infty[, x \mapsto x^n \qquad |n \in \R^{>0} \qquad \text{bijektiv} \\
		\intertext{Umkehrfunktion:}
		[0,\infty[ \rightarrow [0,\infty[, y \mapsto y^{\frac{1}{n}} \qquad \text{Wurzelfunktion}
	\end{gather*}
\end{bsp*}
\begin{def*}[note = gerade , index = gerade]
	$f: \R \rightarrow \R$ heisst \textbf{gerade}, falls\\
	\[ \forall x \in \R : f(-x) = f(x) \]
	$y$-Achse Symmetrie
\end{def*}
\begin{def*}[note = ungerade , index = ungerade]
	$f: \R \rightarrow \R$ heisst \textbf{ungerade}, falls\\
	\[ \forall x \in \R : f(-x) = -f(x) \]
	Ursprung Symmetrie
\end{def*}
\begin{bsp*}
	\begin{gather*}
		f(x) = 1 , x^2 , x^4 , \dotsc , \cos x \text{ sind gerade} \\
		f(x) = x , x^3 , x^5 , \dotsc , \sin x , \tan x \text{ sind ungerade}
	\end{gather*}
\end{bsp*}
\begin{bem}
	Jede Funktion $f: \R \rightarrow \R$ ist die Summe einer geraden und einer ungeraden Funktion, nämlich:\\
	\[ \underbrace{\frac{f(x) + f(-x)}{2}}_{\text{gerade}} + \underbrace{\frac{f(x) - f(-x)}{2}}_{\text{ungerade}} \]
\end{bem}
\begin{def*}[note = monoton , index = monoton]
	Sei $I \in \R$ ein Intervall.
	Eine Funktion $f: I \rightarrow \R$ heisst
	\begin{description}
		\item[monoton wachsend] ,wenn \\
			$\forall x , x' \in I : x < x' \implies f(x) \leq f(x')$
		\item[streng monoton wachsend] ,wenn \\
			$\forall x , x' \in I : x < x' \implies f(x) < f(x')$
		\item[monoton fallend] ,wenn \\
			$\forall x , x' \in I : x < x' \implies f(x) \geq f(x')$
		\item[streng monoton fallend] ,wenn \\
			$\forall x , x' \in I : x < x' \implies f(x) > f(x')$
	\end{description}
\end{def*}
\begin{bsp*}
	\begin{gather*}
		[0,\infty[ \rightarrow \R , x \mapsto x^n \quad | n > 0 \text{ ist streng monoton wachsend}\\
		\R \rightarrow \R , x \mapsto x^2 \text{ ist nicht monoton}\\
		\R \rightarrow \R , x \mapsto \begin{cases}
			x^3	&, x \geq 0	\\
			0	&, x < 0	
		\end{cases} \text{ ist monoton wachsend}
	\end{gather*}
\end{bsp*}
\begin{bem}
	$f$ ist streng monoton $\implies$ injektiv
\end{bem}
\begin{bsp*}
	\begin{gather*}
		\text{Seien } x, x' \in [0,\infty[ , x < x' \implies x' > 0 \wedge x \geq 0 \\
		{x'}^n - x^n = ( \underbrace{x' - x}_{>0} )( \underbrace{\underbrace{{x'}^{n-1}}_{>0} + \underbrace{{x'}^{n-2} \cdot x}_{>0} + \dots + \underbrace{x^{n-1}}_{>0}}_{>0} ) \\
		{x'}^n - x^n > 0 \\
		{x'}^n > x^n
	\end{gather*}
\end{bsp*}

\section{Spezielle Definitions- und Zielbereiche, Bedeutung von Funktionen}
\begin{def*}[note = Folge , index = Folge]
Eine Funktion
	\begin{gather*}
		\Z^{\geq 0} \rightarrow Y \\
		\Z^{\geq 1} \rightarrow Y
	\end{gather*}
	heisst \textbf{Folge in Y}.\\
\end{def*}
\begin{bsp*}
	\[ \left( \frac{1}{n} \right)_{n \geq 1} = \left( 1 , \frac{1}{2} , \frac{1}{3} , \dotsc \right) \]
\end{bsp*}
Varianten:\\
\begin{def*}[note = Aufzählung , index = Aufzählung]
	Eine bijektive Funktion:\\
	$\Z^{\geq 1} \rightarrow Y$: ist abzählbar unendlich \\
	oder \\
	$\{ 1 , 2 , 3 , \dotsc , n \} \rightarrow Y$: $Y$ hat Kardinalität $n$\\
	heisst \textbf{Aufzählung} $Y$.
\end{def*}
\begin{bem}
	Vorsicht: \\
	$\R$ ist unendlich, aber nicht abzählbar.\\
	$\Z , \Q$ sind abzählbar unendlich
\end{bem}
reelle Funktionen: $X \rightarrow Y$ für $X, Y \subset \R$.\\
mögliche Bedeutung:\\
Raum: Linienkoordinaten, Zeit, physikalische Grössen

Funktionen mehrerer Variablen: \\
$X \subset \R^n , Y \subset \R ; X \rightarrow Y$\\
\begin{bsp*}
	\[ \R^2 \rightarrow \R , (x,y) \mapsto 0 \]
\end{bsp*}
Bedeutung:
\begin{itemize}
	\item Höhe über Meeresspiegel
	\item Noten in Analysis als Funktion von Arbeitsaufwand und Talent
	\item Volumen eines von $a,b$ abhängigen Körpers
	\item Beschreibung einer Fläche als $\graph(f)$
\end{itemize}
$(n=2)$ Visualisierung durch Höhenlinien\\
$\{ (x,y) | f(x,y) = h \}$ für festes $h$.\\
\begin{bsp*}
	\[ f: \R^2 \rightarrow \R , (x,y) \mapsto xy \]
\end{bsp*}
\todo{Graph}

$\R^n = \{ (x_1 , \dotsc , x_n ) | x_i \in \R \}$\\
mögliche Bedeutungen:
\begin{itemize}
	\item Ortsvektor
	\item Richtungsvektor
\end{itemize}

$\R \supset I \rightarrow \R^n , t \mapsto ( x_1(t) , \dotsc , x_n(t) ) \quad I$ Intervall\\
parametrisierte Kurve in $\R^n$ gibt eine Bildmenge (kein Graph!)\\
\begin{bsp*}[note = Schraubenlinie]
	\begin{gather*}
		\R \rightarrow \R^3 , \varphi \mapsto \begin{pmatrix}
			r \cos \varphi \\
			r \sin \varphi \\
			\frac{h}{2\pi} \varphi
		\end{pmatrix}\\
		h, r > 0 \\
		\text{Bildmenge } = \left\{ \begin{pmatrix}
			x \\
			y \\
			z
		\end{pmatrix} \in \R^3 \middle| x = r \cos\left( \frac{2 \pi z}{h} \right) , y = r \sin\left( \frac{2 \pi z}{h} \right) \right\}
	\end{gather*}
\end{bsp*}

$f: \R^2 \supset X \rightarrow \R^3$\\
Parametriesierung einer Fläche im $\R^3$.\\
\begin{bsp*}
	\begin{gather*}
		\R \times [0,r] \rightarrow \R^3 , (\varphi , \rho) \mapsto \begin{pmatrix}
			\rho \cos \varphi \\
			\rho \sin \varphi \\
			\frac{h}{2 \pi} \varphi
		\end{pmatrix}\\
		h > 0 \text{ fest}
	\end{gather*}
\end{bsp*}
\begin{bsp*}[note = Kugeloberfläche mit Radius $r>0$]
	Kugelkoordinaten
	\[
		[0,2 \pi] \times \left[ -\frac{\pi}{2} , \frac{\pi}{2} \right] \rightarrow \R^3 , (\varphi , \theta ) \mapsto \begin{pmatrix}
			r \cos \theta \sin \varphi \\
			r \cos \theta \sin \varphi \\
			r \sin \theta
		\end{pmatrix}
	\]
\end{bsp*}

Bedeutung einer Funktion $\R^n \supset X \rightarrow \R^n$\\
Umparametriesierung eines Bereichs\\
\begin{bsp*}[note = {Kreisscheibe $\subset \R^2 , x^2 + y^2 \leq r^2$}]
	\begin{gather*}
		[-1,1] \times [-1,1] \ni (x,y) \mapsto ( r x \sqrt{1-y^2} , r y )\\
		[-1,1] \times [-1,1] \ni (x,y) \mapsto r \cdot (x,y) \cdot \begin{cases}
			0									&(x,y) = (0,0)	\\
			\frac{\max( \abs{x} , \abs{y}}{\sqrt{x^2 + y^2}}	&\text{sonst}			
		\end{cases}
	\end{gather*}
\end{bsp*}
\begin{bsp*}[note = linearer Koordinatenwechsel]
\end{bsp*}

Richtungsvektoren:
$\R^2 \supset X \rightarrow \R^2$\\
\textbf{Vektorfeld}: jedem Punkt in $X$ wird ein Richtungsvektor zugeordnet.\\
\begin{bsp*}
	Geschwindigkeitsvektor einer fliessender Flüssigkeit $\rightarrow$ Differentialgleichung
\end{bsp*}

\section{Stetigkeit}
\begin{gather*}
	f: X \rightarrow Y \\
	X \subset \R^m , Y \subset \R^n , x \mapsto f(x) , x' \mapsto f(x')\\
	\text{Abstand von } x, x' \in \R ^m \text{ ist } \abs{x-x'} \coloneqq \sqrt{(x_1 - x'_1)^2 + \dots + (x_n - x'_n)^2} \\
	x, x' \text{ \enquote{nahe} } \iff \abs{x-x'} \text{ \enquote{klein} } \iff \abs{x-x'} < \delta \\
	f(x), f(x') \text{ \enquote{nahe} } \iff \abs{f(x) - f(x')} < \epsilon
\end{gather*}
\begin{def*}[note = Stetigkeit , index = Stetigkeit]
	\begin{enumerate}[label=(\alph*)]
		\item $f$ ist stetig in $x_0 \in X$ falls gilt:
			\[ \forall \epsilon > 0 \exists \delta > 0 : \forall x \in X : \abs{x-x_0} < \delta \implies \abs{f(x) - f(x')} < \epsilon \]
		\item $f$ ist stetig, falls $f$ stetig in jedem $x_0 \in X$ ist.
	\end{enumerate}
\end{def*}
\begin{fakt}
	Jede Polynomfunktion ist stetig. \\
	$\R \supset X \rightarrow Y$ ist stetig in $x_0$ \gdw $f|x \wedge [x_0,\infty[$ und $f|x \wedge ]-\infty,x_0]$ stetig in $x_0$ sind.
\end{fakt}

\subsubsection{Grundeigenschaften}
\begin{enumerate}[label=(\alph*)]
	\item
		$f: X \rightarrow Y$ stetig \\
		$g: Y \rightarrow Y$ stetig \\
		$\implies$ die zusammengesetzte Funktion $g \circ f: X \rightarrow Z , x \mapsto g(f(x))$ ist stetig.\\
		\begin{bew}
			\begin{gather*}
				\text{Sei } x_0 \in X, \text{ sei } \epsilon > 0 \\
				g \text{ stetig in } f(x_0) \rightsquigarrow \exists \delta > 0 : \forall z \in Y: \\
				\abs{y - f(x_0)} < \delta \implies \abs{g(y) - g(f(x_0))} < \epsilon \\
				f \text{ stetig in } x_0 \implies \gamma > 0 : \forall x \in X \\
				\abs{x - x_0} < \gamma \implies \abs{f(x) - f(x_0)} < \delta \\
				\text{Zusammen:} \\
				\abs{x - x_0} < \gamma \implies \abs{g(f(x)) - g(f(x_0))} < \epsilon \\
				\text{d.h. } g \circ f \text{ stetig in } x_0 \quad \blacksquare
			\end{gather*}
		\end{bew}
	\item
		$f = (f_1 , \dotsc , f_n)$\\
		$f_1 , \dotsc , f_n : X \rightarrow \R$\\
		$\drsh f$ stetig $\iff$ jedes $f_i$ stetig
	
	\item
		Die Grundrechenarten sind stetig.\\
		\begin{bew}
			$+ : \R^2 \rightarrow \R , (x,y) \mapsto x + y$\\
			Sei $(x_0 , y_0 ) \in \R^2$. Sei $\epsilon > 0$. Setze $\delta \coloneqq \frac{\epsilon}{2}$\\
			Dann gilt für alle $(x,y) \in \R^2$:
			\begin{gather*}
				\abs{ (x,y) - (x_0,y_0) } < \delta \\
				\implies \abs{x-x_0} < \delta \wedge \abs{y-y_0} < \delta \\
				\implies \abs{ (x-x_0) + (y-y_0) } \leq \abs{x-x_0} + \abs{y-y_0} < 2 \delta = \epsilon \\
				\implies \abs{ (x+y) - (x_0+y_0)} < \epsilon
			\end{gather*}
			analog $-$
		\end{bew}
		\begin{bew}
			$\cdot : \R^2 \rightarrow \R , (x,y) \mapsto x \cdot y$\\
			Sei $(x_0,y_0) \in \R^2 , \epsilon > 0$\\
			\begin{gather*}
				\abs{x} = \abs{ (x-x_0) + x_0 } \leq \abs{x-x_0} + \abs{x_0} < \delta + \abs{x} \\%% ORLY?
				\delta = \frac{\min\{\epsilon,1\}}{1 + \abs{x_0} + \abs{y_0}} \implies \delta \leq 1 \\
				\begin{split}
				\abs{x \cdot y - x_0 \cdot y_0}	&= \abs{x \cdot (y-y_0) + (x-x_0) \cdot y_0}					\\
										&\leq \abs{x \cdot (y-y_0)} + \abs{(x-x_0) \cdot y_0}			\\
										&= \abs{x} \cdot \abs{(y-y_0)} + \abs{(x-x_0)} \cdot \abs{y_0}	\\
										&\leq \abs{x} \cdot \delta + \abs{y_0} \cdot \delta				\\
										&< ( ( \delta + \abs{x_0} ) + \abs{y_0} ) \cdot \delta			\\
										&= (\delta + \abs{x_0} + \abs{y_0}) \cdot \delta				\\
										&\leq (1 + \abs{x_0} + \abs{y_0}) \cdot \delta				\\
										&\leq \epsilon										
				\end{split}
			\end{gather*}
			Dann gilt $\forall (x,y) \ in \R^2 : \abs{ (x,y) - (x_0,y_0) } < \delta \implies \abs{ x \cdot y - x_0 \cdot y_0 } < \epsilon$\\
			analog $:$
		\end{bew}
	\item
		\begin{folge}
			jede Rationale Funktion ist stetig, wo definiert.
		\end{folge}
	\item
		$f$: Intervall $\rightarrow$ Intervall bijektiv, stetig $\implies f^{-1}$ stetig
\end{enumerate}
\todo{Fix vertical spacing}
\begin{bsp*}
	Für $n \in \Z^{\geq 0} : [0,\infty[ \rightarrow [0,\infty[, x \mapsto x^n$\\
	ist $[0,\infty[ \rightarrow [0,\infty[, y \mapsto \sqrt[n]{y}$
\end{bsp*}
\begin{bsp*}
	$\R^n \rightarrow \R , x = ( x_1 , \dotsc , x_n ) \mapsto \abs{x} \coloneqq \sqrt{x_1^2 + \dotsc + x_n^2}$ ist stetig
\end{bsp*}
\begin{bsp*}
	\begin{gather*}
		\R^n \rightarrow \R , (x_1 , \dotsc , x_n) \mapsto \max\{x_1 , \dotsc , x_n\} \text{ ist stetig}\\
		\underline{n=2:} \max\{ x_1 , x_2 \} = \begin{cases}
			x_1	&\text{falls } x_1 \geq x_2	\\
			x_2	&\text{falls } x_1 \leq x_2	
		\end{cases}
	\end{gather*}
\end{bsp*}

Unstetige Beispiele\\
\begin{bsp*}
	\[ \sgn: \R \rightarrow \R , x \mapsto \begin{cases}
		1	&\text{falls } x > 0	\\
		0	&\text{falls } x = 0	\\
		-1	&\text{falls } x < 0	
	\end{cases} \]
	ist stetig ausserhalb von $0$ aber unstetig in $0$.
\end{bsp*}
\begin{bsp*}
	$\R \rightarrow \R , x \mapsto \lfloor x \rfloor \coloneqq$ die grösste Zahl $\leq x$\\
	unstetig in jedem $x_0 \in \Z$\\
	Die Funktion ist in jedem Punkt rechtseitig stetig.
\end{bsp*}
\begin{def*}[note = rechts-/linksseitige Stetigkeit , index = Stetigkeit]
	$f: X \rightarrow Y$ mit $X \subset \R$ heisst im $x_0 \in X$ \textbf{rechtsseitig stetig}\index{Stetigkeit!rechtsseitig}, falls\\
	\[ \forall \epsilon > 0 \exists \delta > 0 : \forall x \in X : x > x_0 \wedge \abs{x-x_0} < \delta \implies \abs{f(x)-f(x_0)} < \epsilon \]
	
	$f: X \rightarrow Y$ mit $X \subset \R$ heisst im $x_0 \in X$ \textbf{linksseitig stetig}\index{Stetigkeit!linksseitig}, falls\\
	\[ \forall \epsilon > 0 \exists \delta > 0 : \forall x \in X : x < x_0 \wedge \abs{x-x_0} < \delta \implies \abs{f(x)-f(x_0)} < \epsilon \]	
	$f$ ist in $x_0$ stetig $\iff$ $f$ ist in $x_0$ rechts- und linksseitigstetig
\end{def*}
\todo{Overfull}

\section{Grundeigenschaften von \texorpdfstring{$\R$}{R}}
Jede nichtleere endliche Teilmenge $S \subset \R$ hat ein eindeutiges Maximum $\max(S)$.\\
\begin{tabular}{lll}
	falls nichtleer nicht endlich:	&entweder	&$\forall a \in \R \exists x \in S : x > a$ \\
						&oder:	&$\exists a \in \R \forall x \in S : x \leq a$
\end{tabular}\\
($a$ ist obere Schranke) \\
Dann gibt es eine eindeutige kleinste obere Schranke.\\
\begin{def*}[note = Supremum , index = Supremum]
	\[
		\text{Supremum } S \subset \R : \sup(S) \coloneqq  \begin{cases}
			\infty				&\begin{matrix*}[l]
									\text{falls $S$ keine}					\\
									\text{obere Schranke hat}
								\end{matrix*}							\\
			\begin{matrix*}[l]
				\text{kleinste}	\\
				\text{obere}	\\
				\text{Schranke}
			\end{matrix*}			&\begin{matrix*}[l]
									\text{falls $S \neq \varnothing$ und}		\\
									\text{es existiert}					\\
									\text{eine obere Schranke}
								\end{matrix*}							\\
			-\infty					&\text{falls $S = \varnothing$}			
		\end{cases}
	\]
\end{def*}

\subsection{Supremum und Infimum}
\begin{itemize}
	\item Betrachte $X \subset \R$
	\item Eine Zahl $a \in \R$ mit $\forall x \in X : x \leq a$ heisst \textbf{eine obere Schranke} von $X$.
	\item Falls so ein $a$ existiert, heisst $X$ \text{nach oben beschränkt}.
	\item Falls so ein $a$ in $X$ selbst existiert, so ist sie eindeutig, nämlich den Maximum $\max(X)$.
	
	\item Eine Zahl $a \in \R$ mit $\forall x \in X : x \geq a$ heisst \textbf{eine untere Schranke} von $X$.
	\item Falls so ein $a$ existiert, heisst $X$ \textbf{nach unten beschränkt}.
	\item Falls so ein $a$ in $X$ selbst existiert, so ist sie eindeutig, nämlich den Minimum $\min(X)$.
	
	\item Ist $X$ nichtleer und nach oben beschränkt, so besitzt es eine eindeutige \text{kleinste obere Schranke}, gennant \textbf{Supremum} $\mathbf{\sup(X)}$. (d.h. $\sup(X) = \min\{ a \in \R | a$ ist obere Schrank von $X \}$
	\item $\sup( \varnothing ) \coloneqq -\infty$
	\item $\sup(X) \coloneqq +\infty$ falls $X$ nicht nach oben beschränkt ist.
	\item Wenn $\max(X)$ existiert, so ist $\max(X) = \sup(X)$.
	
	\item Ist $X$ nichtleer und nach unten beschränkt, so besitzt es eine eindeutige \text{grösste untere Schranke}, gennant \textbf{Infimum} $\mathbf{\inf(X)}$. (d.h. $\inf(X) = \max\{ a \in \R | a$ ist untere Schrank von $X \}$
	\item $\inf( \varnothing ) \coloneqq +\infty$
	\item $\inf(X) \coloneqq -\infty$ falls $X$ nicht nach oben beschränkt ist.
	\item Wenn $\min(X)$ existiert, so ist $\min(X) = \inf(X)$.
\end{itemize}
\begin{bsp*}
	\begin{gather*}
		\max [0,1] = 1 = \sup [0,1] = \sup ]0,1[ \\
		\max ]0,1[ \text{ existiert nicht!}
	\end{gather*}
\end{bsp*}

\subsubsection{Charakteririerung von \texorpdfstring{$\sup X$}{sup X}}
Eine $a \in \R$ mit:\\
\begin{gather*}
	\forall x \in X : x \leq a, \\
	\forall \epsilon > 0 : \exists x \in X : x > a - \epsilon
\end{gather*}

\subsubsection{Eigenschaften}
\begin{itemize}
	\item $X \subset X' \subset \R \rightsquigarrow \sup X \leq \sup X' \qquad (\text{dabei } -\infty < x < +\infty \text{ für jedes } x \in \R )$
	\item $c, b \in \R ; X, Y \subset \R$
	\begin{itemize}
		\item $X + b \coloneqq \{ x + b | x \in X \}$
		\begin{itemize}
			\item $\sup(X + b) = \sup(X) + b$
		\end{itemize}
		\item $c \dot X \coloneqq \{ c \cdot x | x \in X \}$
		\begin{itemize}
			\item $\sup(c \cdot X) = c \cdot \sup(X)$ falls $c > 0$
			\item $\sup(c \cdot X) = c \cdot \inf(X)$ falls $c < 0$
		\end{itemize}
		\item $X + Y \coloneqq \{ x + y | x \in X , y \in Y \}$
		\begin{itemize}
			\item $\sup(X + Y) = \sup(X) + \sup(Y)$ falls $X, Y \neq \varnothing$
			\begin{itemize}
				\item $a \coloneqq \sup(X) , b \coloneqq \sup(Y)$. Dann\\
					$\forall x \in X \forall y \in Y : ( x \leq a \wedge y \leq b ) \implies x + y \leq a + b$ und \\
					$\forall \epsilon > 0 : (( \exists x \in X : x > a - \epsilon ) \wedge ( \exists y \in Y : y > b - \epsilon )) \implies x + y > a + b - 2 \epsilon$
			\end{itemize}
		\end{itemize}
	\end{itemize}
\end{itemize}
\begin{bsp*}
	\begin{gather*}
		x = \xi_r \dots \xi_2 \xi_1 \xi_0 . \eta_1 \eta_2 \eta_3 \dots >0 \\
		x_n = \xi_r \dots \xi_0 . \eta_1 \dots \eta_n | x = \sup\{ x_0 , x_1 , \dotsc \}
	\end{gather*}
\end{bsp*}
\begin{bsp*}
	$\pi = 3.14159\dots$\\
	Umfang eines regelmässiges $n$-Ecks $U_n$ eingeschrieben in ein Kreis mit Radius $1$ \\
	$2\pi = \sup\{ U_n | n \geq 2 \}$
\end{bsp*}
\begin{satz*}
	Für $a > 0$ existiert genau eine stetige Funktion $f: \R \rightarrow \R^{> 0}$, mit $f\left( \frac{m}{n} \right) = \sqrt[n]{a^m}$ für alle $m, n \in \Z, n > 0$.\\
	Bezeichnung: $a^x \coloneqq f(x)$\\
	Denn:\\
	Für $0 < a < 1$ setze $a^x \coloneqq \left( \frac{1}{a} )^{-x} \right)$. \\
	Für $a =1$ setze $a^x$. \\
	Sei also $a > 1$. \\
	Dann ist die Abbildung $\Q \rightarrow \R, \xi \mapsto a^\xi$ streng monoton wachsend.\\
	Setze $f(x) \coloneqq \sup\{ a^\xi | \xi \in \Q , \xi \leq x \}$ nichtleer, nach oben beschränkt durch $a^\eta$ für $\eta \in \Q , \eta \geq x$.\\
	Falls $x \in \Q$, ist $\sup = \max = a^x$\\
	$f$ stetig in $x_0 \in \R$? \\
	Sei $\delta \in \Q , \delta > 0$, wähle \\
	\begin{gather*}
		\xi \in \Q : x_0 - 2\delta < \xi < x_0 - \delta \\
		\rightsquigarrow x_0 + \delta < \xi + 3\delta \\
		\rightsquigarrow \forall x \in \R : \abs{x-x_0} < \delta \\
		\implies x \in ] x - \delta , x_0 + \delta [ \\
		\implies \xi < x < \xi + 3\delta \\
		\implies a^\xi < a^x < a^{\xi + 3\delta} \\
		\implies \abs{ a^x - a^{x_0} } < a^{\xi - 3\delta} = a^\xi \cdot ( a^{3\delta } )
	\end{gather*}
	Zu $\epsilon > 0$ nimm $\delta \in \Q, \delta > 0$ und $\xi$ so, dass $a^\xi \cdot ( a^{3\delta} - 1 ) < \epsilon$.\\
	Eindeutigkeit: \dotfill nei machemer nöd :P
	
	Eigenschaften:
	\begin{gather*}
		a^{x+y} = a^x \cdot a^y \\
		(a^x)^y = a^{xy} \\
		(ab)^x = a^x \cdot b^x
	\end{gather*}
	Die Funktion $\R^{> 0} \times \R \rightarrow \R^{>0} , (a,x) \mapsto a^x$ ist stetig.
\end{satz*}
\begin{satz*}[note = Zwischenwertsatz]
	Sei $f: [a,b] \rightarrow \R$ stetig. Dann minnt $f$ jeden Wert zwischen $f(a)$ und $f(b)$ an.\\
	\[
		\left( \text{d.h.} \qquad \begin{matrix*}[l]
			f(a) \leq f(b) \implies	&[ f(a) , f(b) ] \subset \image(f)	\\
			\text{sonst }		&[ f(b) , f(a) ] \subset \image(f)
		\end{matrix*} \right)
	\]
	\begin{bew}
		Sei $f(a) \leq f(b)$; sonst ersetze $f$ durch $-f$.\\
		Sei $f(a) \leq y \leq f(b)$.\\
		Setze $g: [a,b] \rightarrow \R , x \mapsto f(x) - y$\\
		$\implies g$ stetig, $g(a) \leq 0 \leq g(b)$.\\
		Gesucht: Nullstelle von $g$.\\
		Halbierungsprinzip:
		\begin{gather*}
			a_0 \coloneqq a \\
			b_0 \coloneqq b \\
			\text{falls } \sgn\left( g\left( \frac{a_0 + b_0}{2} \right)\right) = \sgn(g(a_0)) \text{ setze} \\
			a_1 \coloneqq \frac{a_0 + b_0}{2} \\
			b_1 \coloneqq b_0 \\
			\text{sonst} \\
			a_1 \coloneqq a_0 \\
			b_1 \coloneqq \frac{a_0 + b_0}{2} \\
			\text{usw.} \\
			b_n - a_n = \frac{b_0 - a_0}{2^n} \\
			a_0 \leq a_1 \leq \dots \\
			b_0 \geq b_1 \geq \dots \\
			x \coloneqq \sup\{ a_0 , a_1, \dotsc \} = \inf\{ b_0, b_1, \dotsc \} \text{ tut's!}
		\end{gather*}
		S.122\\
		Folge: Ist $I \subset \R$ ein Intervall und $f: I \rightarrow \R$ stetig und streng monoton os induziert $f$ eine bijektive Abbildung $I \rightarrow f(I)$.\\
		\begin{bsp*}
			\[ x \mapsto x^n , \sin x , \dotsc \]
		\end{bsp*}
	\end{bew}
\end{satz*}
\todo{Too long}
\begin{bsp*}
	$a > 1 \rightsquigarrow \R \rightarrow \R^{>0} ,x \mapsto a^x$ ist stetig und streng monoton wachsend. Denn:\\
	Für $x < x'$ ist $a^x - a^{x'} = \underbrace{a^x}_{>0} \cdot (\underbrace{a^{x-x'}}_{>0} - 1 )$ \\
	Für $y = \frac{m}{n} > 0$ ist $a^y = \sqrt[n]{a^m} > 1$
\end{bsp*}
Umkehrfunktion:
\[ \R^{>0} \rightarrow \R, y \mapsto \log_a y \]

$\R$ ist ein angeordneter Körper \\
\begin{gather*}
	a < b \iff b - a = c^2 \text{ für ein } c \in \R \setminus \{0\} \\
	a \leq b \iff b - a = c^2 \text{ für ein } c \in \R
\end{gather*}

\begin{tabular}{ll}
	$a > 0$	&positiv		\\
	$a \geq 0$	&nichtnegativ	\\
	$a < 0$	&negativ		\\
	$a \leq 0$	&nichtpositiv	
\end{tabular}

\subsubsection{Dezimalentwickelung}
Jede reelle Zahl ist\\
$x = \pm \xi_n \xi_{n-1} \dots \xi_0 . \eta_1 \eta_2 \eta_3 \dots$\\
mit $n \in \Z^{\geq 0}, \xi_i, \eta_i \in \{ 0, 1, 2, 3, 4, 5, 6, 7, 8, 9 \}$\\
$x = \pm \left( \sum_{i=0}^n \xi_i \cdot 10^i + \sum_{j=1}^\infty \eta_i \cdot10^-i \right)$\\
Diese Darstellung ist eindeutig bis auf\\
$ \dots . \dots \eta_i 9999\dots = \dots . \dots (\eta_i +1)$\\
$x$ ist rational $\iff$ Nachkommastellen werden schliesslich periodisch.

\subsubsection{Vektoren}
$\R^n = \{ (x_1 , \dotsc , x_n) | \text{alle } x_i \in \R \}$\\
$\abs{x} = \sqrt{x_1^2 + \dots + x_n^2}$ Betrag / euklidische Norm von $x$\\
\begin{satz*}[note = Dreiecksungleichung]
	\[ \abs{x+y} \leq \abs{x} + \abs{y} \]
\end{satz*}
\begin{def*}[note = Norm , index = Norm]
	Eine \textbf{Norm} auf $\R^n$ ist eine Funktion $\R^n \rightarrow \R , x \mapsto ||x||$ sodass:
	\begin{enumerate}
		\item $\forall x \in \R^n : \norm{x} \geq 0$ und $\norm{x} = 0$ \gdw $x=0$
		\item $\forall x \in \R^n \forall \lambda \in \R: \norm{\lambda \cdot x} = \abs{\lambda} \cdot \norm{x}$
		\item $\forall x , y \in \R^n : \norm{x + y} \leq \norm{x} + \norm{y}$
	\end{enumerate}
\end{def*}
\begin{bsp*}
	\begin{tabular}{ll}
		$\norm{\cdot} = \abs{\cdot}$								&Standard-euklidische Norm			\\
		$\norm{x}_1 \coloneqq \abs{x_1} + \dots + \abs{x_n}$				&\enquote{Taxifahrernorm} (Weg entlang Quadrate)	\\
		$\norm{x}_\infty \coloneqq \max\{ \abs{x_1}, \dotsc , \abs{x_n} \}$	&Maximumumsnorm					
	\end{tabular}
\end{bsp*}
\todo{Overfull}
\begin{fakt}
	\[ \norm{x}_{\infty} \leq \abs{x} \leq \norm{x}_1 \leq n \cdot \norm{x}_{\infty} \]
	Folge: In der Definition von Stetigkeit und $\lim$ und $\sum$ kann man eine beliebige Norm auf $\R^n$ nehmen anstatt $\abs{\cdot}$
\end{fakt}

\subsubsection{Standardskalarprodukt auf \texorpdfstring{$\R^n$}{R^n}}
\begin{gather*}
	\scal{x,y} \coloneqq x_1 y_1 + \dots + x_n y_n \\
	\scal{x,x} = \abs{x}^2
\end{gather*}
