\subsection{Semantik der Aussagenlogik}
Wahrheitswerte der Atomformeln $\rightarrow$ Wahrheitswerte der zusammengesetzten Formeln
\subsubsection{Vollständige Wahrheitstabelle}
$((( A \wedge B ) \vee ( \neg C )) \wedge ( A \vee B ))$
\begin{gather*}
	\begin{array}{ c c c c c c c c c c >{\columncolor{green}}c c c c c c }
		(((	& A	& \wedge	& B	& )	& \vee	& (	& \neg	& C	& ))	& \wedge	& (	& A	& \vee	& B	& ))	\\
			& 0	& 0		& 0	&	& 1		&	& 1		& 0	&	& 0		&	& 0	& 0		& 0		\\
			& 0	& 0		& 0	&	& 0		&	& 0		& 1	&	& 0		&	& 0	& 0		& 0		\\
			& 0	& 0		& 1	&	& 1		&	& 1		& 0	&	& 1		&	& 0	& 1		& 1		\\
			& 0	& 0		& 1	&	& 0		&	& 0		& 1	&	& 0		&	& 0	& 1		& 1		\\
			& 1	& 0		& 0	&	& 1		&	& 1		& 0	&	& 1		&	& 1	& 1		& 0		\\
			& 1	& 0		& 0	&	& 0		&	& 0		& 1	&	& 0		&	& 1	& 1		& 0		\\
			& 1	& 1		& 1	&	& 1		&	& 1		& 0	&	& 1		&	& 1	& 1		& 1		\\
			& 1	& 1		& 1	&	& 1		&	& 0		& 1	&	& 1		&	& 1	& 1		& 1		
	\end{array}
\end{gather*}
\begin{def*}[note = Belegung , index = Belegung]
	D , $\epsilon_D$ \hyperref[Formel!korrekte]{wie oben}. \\
	\begin{gather*}
		\begin{array}{ l l }
			\mathcal{A} : D \rightarrow \{ 0 , 1 \}				& \text{Belegung} \\
			\mathcal{\hat{A}} : \varepsilon_D \rightarrow \{ 0 , 1 \}	& \text{Fortsetzung von } \mathcal{A}
		\end{array}
	\end{gather*}
	\begin{itemize}
		\item $\mathcal{\hat{A}}( D ) \coloneqq \mathcal{A}( D ) \qquad \text{für Atomformel D}$
		\item $\mathcal{\hat{A}}(( F \wedge G )) \coloneqq \mathcal{\hat{A}}( F ) \wedge \mathcal{\hat{A}}( G )$
		\item $\mathcal{\hat{A}}(( F \vee G )) \coloneqq \mathcal{\hat{A}}( F ) \vee \mathcal{\hat{A}}( G )$
		\item $\mathcal{\hat{A}}(( \neg F )) \coloneqq 1 - \mathcal{\hat{A}}( F )$
	\end{itemize}
\end{def*}

\subsubsection{Vereinfachungen:}
\begin{itemize}
	\item Weglassen von Klammern
	\item Andere Junktoren ( als Abkürzungen )
\end{itemize}
\subsubsection{Semantische Äquivalenz}
\begin{gather*}
	\begin{array}{ c c c c c c c >{\columncolor{green}}c c c c c c }
		((	& \neg	& (	& A	& \wedge	& B	& ))	& \wedge	& (	& A	& \vee	& B	& ))	\\
			& 1		&	& 0	& 0		& 0	&	& 0		&	& 0	& 0		& 0		\\
			& 1		&	& 0	& 0		& 1	&	& 1		&	& 0	& 1		& 1		\\
			& 1		&	& 1	& 0		& 0	&	& 1		&	& 1	& 1		& 0		\\
			& 0		&	& 1	& 0		& 1	&	& 0		&	& 1	& 1		& 1					
	\end{array}\\
	\intertext{Diese beiden Formel sind syntaktisch verschieden, aber semantisch äquivalent (logisch gleichwertig).}
	\begin{array}{ c >{\columncolor{green}}c c }
		A	& \oplus	& B	\\
		0	& 0		& 0	\\
		0	& 1		& 1	\\
		1	& 0		& 0	\\
		1	& 0		& 1	
	\end{array}\\
	A \oplus B \equiv (( \neg ( A \wedge B ) \wedge ( B \vee A )) \equiv ((( \neg A ) \wedge B ) \vee ( A \wedge ( \neg B )))
\end{gather*}
\begin{def*}[note = {semantische Äquivalenz, Tautologie, Unerfüllbarkeit} , index = Semantik!Äquivalenz]
	Formeln F und G sind semantisch äquivalent, falls sie für jede Belegung den gleichen Wahrheitswert haben. \\
 	Wir schreiben $F \equiv G$ oder $F \iff G$. \\
	\begin{gather*}
		0 \coloneqq ( A \wedge ( \neg A ) ) \\
		1 \coloneqq ( A \vee ( \neg A ) )
	\end{gather*}
	Eine Formel $F \equiv 1$ heisst \textbf{Tautologie}\index{Formel!Tautologie}.\\
	Eine Formel $F \equiv 0$ heisst \textbf{unerfüllbar}\index{Formel!unerfüllbar}.
\end{def*}
\begin{satz*}
	$F \equiv G$ gilt \gdw $F \leftrightarrow G$ eine Tautologie ist. \\
	$( F \leftrightarrow G \coloneqq ( ( F \wedge G ) \vee ( ( \neg F ) \wedge ( \neg G ) )$ \\
	\begin{bew}
		$F \equiv G$ : Für jede Belegung haben $F$ und $G$ den gleichen Wahrheitswert.\\
		Gleichbedeutend mit: Für jede Belegung ist $F \leftrightarrow G$ wahr.
	\end{bew}
\end{satz*}
\begin{bsp*}[note = Äquivalenzen, index = logische Gesetze]
	\begin{gather*}
		\begin{array}{ l l l }
			\text{Idempotenz:}		& ( F \wedge F )				&\equiv F							\\
								& ( F \vee F )				&\equiv F							\\
			\text{Kommutativität:}		& ( F \wedge G )				&\equiv ( G \wedge F )					\\
								& ( F \vee G )				&\equiv ( G \vee F )					\\
			\text{Assoziativität:}		& ( ( F \wedge G  ) \wedge H )	&\equiv ( F \wedge ( G  \wedge H ) )		\\
								& ( ( F \vee G  ) \vee H )		&\equiv ( F \vee ( G  \vee H ) )			\\
			\text{Distibutivität:}		& ( F \wedge ( G  \vee H ) )		&\equiv ( ( F \wedge G ) \vee ( F \wedge H ) )	\\
								& ( F \vee ( G  \wedge H ) )		&\equiv ( ( F \vee G ) \wedge ( F \vee H ) )	\\
			\text{de Morgan:}		& ( \neg ( F \wedge G ) )		&\equiv( ( \neg F ) \vee ( \neg G ) )		\\
								& ( \neg ( F \vee G ) )			&\equiv( ( \neg F ) \wedge ( \neg G ) )		\\
			\text{Doppelte Negation:}	& ( \neg ( \neg F ) )			&\equiv F							
		\end{array}
	\end{gather*}
\end{bsp*}

\subsubsection{Vereinfachungen:}
\begin{itemize}
	\item $A \rightarrow B \text{ für } ( ( \neg A ) \vee B )$
	\item $A \oplus B , A \leftrightarrow B$
	\item $A \wedge B \wedge C$
	\item Priorität (absteigend):
	\begin{itemize}
		\item $\neg$
		\item $\wedge , \vee$
		\item $\leftarrow , \leftrightarrow$
	\end{itemize}
\end{itemize}
Formel $F$ in $A_1 , \dotsc , A_n$ entspricht einer Funktion:\\
Belegung $\mapsto$ Auswertung\\
$\{ 0 , 1 \} \rightarrow \{ 0 , 1 \}$

\subsubsection{Berechnung und Auswertung in Wahrheitstabellen}
\begin{gather*}
	\begin{array}{ c c c c c c >{\columncolor{green}}c c }
		(	& \neg	& A_1	& \oplus	& A_2	& )	& \rightarrow	& A_3	\\
			& 1		& 0		& 1		& 0		&	& 0			& 0		\\
			& 1		& 0		& 1		& 0		&	& 1			& 1		\\
			& 1		& 0		& 0		& 1		&	& 1			& 0		\\
			& 1		& 0		& 0		& 1		&	& 1			& 1		\\
			& 0		& 1		& 0		& 0		&	& 1			& 0		\\
			& 0		& 1		& 0		& 0		&	& 1			& 1		\\
			& 0		& 1		& 1		& 1		&	& 0			& 0		\\
			& 0		& 1		& 1		& 1		&	& 1			& 1		
	\end{array}
\end{gather*}\\
Formeln mit gleicher Funktionen sind \enquote{semantisch Äquivalent}.

\subsubsection{Normalformen}
Formel $\rightarrow$ Wahrheitsfunktion \\
Formel $\leftarrow$ Wahrheitsfunktion ? \\
\begin{gather*}
	\begin{array}{ c c c >{\columncolor{green}}c }
		A 	& B	& C	& F	\\
		0	& 0	& 0	& 0	\\
		0	& 0	& 1	& 0	\\
		0	& 1	& 0	& 1	\\
		0	& 1	& 1	& 1	\\
		1	& 0	& 0	& 1	\\
		1	& 0	& 1	& 1	\\
		1	& 1	& 0	& 0	\\
		1	& 1	& 1	& 1	\\
	\end{array}\\
	\begin{array}{ l l }
	F 	& \equiv ( \neg A \wedge B \wedge \neg C ) \vee ( \neg A \wedge B \wedge C ) \vee ( A \wedge \neg B \wedge \neg C ) \vee ( A \wedge  \neg B \wedge C ) \vee ( A \wedge B \wedge C ) \\
		& \equiv ( ( \neg A \wedge B ) \wedge ( C \vee \neg C ) ) \vee ( ( A \vee \neg B ) \wedge ( C \vee \neg C ) ) \vee ( A \wedge B \wedge C ) \\
		& \equiv ( \neg A \wedge B ) \vee ( A \wedge \neg B ) \vee ( A \wedge B \wedge C )
	\end{array}\\
	\textbf{ Disjunktive Normalform DNF}\\
	\begin{array}{ l l }
		F	& \equiv ( A \vee B \vee C ) \wedge ( A \vee B \vee \neg C ) \wedge ( \neg A \vee \neg B \vee C )	\\
			& \equiv ( ( A \vee B ) \vee ( C \wedge \neg C ) ) \wedge ( \neg A \vee \neg B \vee C )			\\
			& \equiv ( A \vee B ) \wedge ( \neg A \vee \neg B \vee C )							
	\end{array}\\
	\textbf{Konjunktive Normalform KNF}
\end{gather*}\\
\begin{def*}[note = {DNF, KNF} , index = Normalform]
	\begin{itemize}
		\item \textbf{Literal}\index{Literal}: $F$ Atomformel, dann sind $F, \neg F$ Literale.
		\item $F$ in \textbf{KNF}\index{KNF}\index{Normalform!konjunktive}, falls es Literale $L_{ij}$ gibt mit $\bigwedge_{i=1}^k ( \bigvee_{j=1}^{m_i} L_{ij} )$
		\item $F$ in \textbf{DNF}\index{DNF}\index{Normalform!disjunktive}, falls es Literale $L_{ij}$ gibt mit $\bigvee_{i=1}^k ( \bigwedge_{j=1}^{m_i} L_{ij} )$
	\end{itemize}
\end{def*}

\subsubsection{Modell und semantische Folgerung}
\begin{def*}[note = Modell , index = Modell]
	Sei $F$ eine Formel und $\mathcal{A}$ eine Belegung mit $\mathcal{A}( F ) \equiv 1$ . Dann heisst $\mathcal{A}$ ein \textbf{Modell} in $F$, man schreibt $\mathcal{A} \models F$
\end{def*}
\begin{bsp*}
	\begin{gather*}
		\begin{array}{ c c c c c >{\columncolor{green}}c c }
			(	& A	& \wedge	& B	& )	& \rightarrow	& C	\\
				& 0	& 0		& 0	&	& 1			& 0	\\
				& 0	& 0		& 0	&	& 1			& 1	\\
				& 0	& 0		& 1	&	& 1			& 0	\\
				& 0	& 0		& 1	&	& 1			& 1	\\
				& 1	& 0		& 0	&	& 1			& 0	\\
				& 1	& 0		& 0	&	& 1			& 1	\\
				& 1	& 1		& 1	&	& 0			& 0	\\
				& 1	& 1		& 1	&	& 1			& 1	
		\end{array}
	\end{gather*}
	Tatsachen:
	\begin{itemize}
		\item $F \equiv G$ \gdw $( \mathcal{A} \models G \text{ \gdw} \mathcal{A} \models G )$ .
		\item $F$ Tautologie, falls $\mathcal{A} \models F$ für alle $\mathcal{A}$ .
		\item $F$ unerfüllbar, falls $\mathcal{A} \models F$ für kein $\mathcal{A}$ .
	\end{itemize}
\end{bsp*}
\marginpar{\color{red}$\models$ gleich wie $\implies$}
\begin{def*}[note = semantische Folgerung , index = semantische Folgerung]
	\begin{gather*}
		G \text{ ist eine semantische Folgerung von } F_1 , \dotsc , F_n \\
		F_1 , \dotsc , F_n \models G \\
		F_1 , \dotsc , F_n \implies G \\
		\text{falls für alle } \mathcal{A} \text{ mit } \mathcal{A} \models F_1 , \dotsc , \mathcal{A} \models F_n \text{ auch } \mathcal{A} \models G \text{ gilt.}
	\end{gather*}
\end{def*}
\begin{bem}
		$F \equiv G$ gleichbedeutend mit $F \models G$ und $G \models F$ ( Manchmal sieht man $ F \vDash\joinrel\Dashv G$ oder $F \iff G$ )
\end{bem}
\begin{bsp*}
		\begin{gather*}
		A , A \rightarrow B , B \rightarrow C \models C\\
		\begin{split}
			& \text{Die einzige Belegung, die } A , A \rightarrow B \text{ und } B \rightarrow C \text{ erfüllt  } \\
			& \text{ ist die Belegung } A \equiv B \equiv C \equiv 1 \\
		\end{split}\\
		\text{Für jede } \mathcal{A} \text{ die das erfüllt, ist auch } C \text{ erfüllt.} \\
		F \equiv G \text{ \gdw } F \leftrightarrow G \text{ Tautologie} \\
		F \models G \text{ \gdw } F \rightarrow G \text{ Tautologie} \\
		A \wedge ( A \rightarrow B ) \wedge ( B \rightarrow C ) \rightarrow C \text{ Tautologie}
	\end{gather*}
\end{bsp*}
\begin{satz*}
	$F_1 , \dotsc , F_n \models G \text{ \gdw } \bigwedge_{i=0}^n F_i \rightarrow G \text{ Tautologie}$\\
	\begin{bew}
		\[\bigwedge_{i=1}^n F_i \rightarrow G \equiv \neg ( \bigwedge_{i=1}^n F_i  ) \vee G \equiv ( \bigvee_{i=1}^n F_i  ) \vee G\]
	\end{bew}
	Beides gilt \gdw für jede Belegung $\mathbb{A}$ gilt:\\
	\qquad falls $\mathcal{A}( F_i ) = 1$ für alle $i$ dann auch $\mathcal{A}( G ) = 1$
\end{satz*}
\begin{bem}
	\begin{itemize}
		\item $F$ Tautologie \gdw $1 \models F$
		\item $F$ unerfüllbar \gdw $F \models 0$
	\end{itemize}
\end{bem}
