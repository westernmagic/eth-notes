\section{Planare Graphen}
[Abbot: Flatland]\\
BILD\\
\begin{def*}[note = planar , index = planar]
	$G=(V,E)$ \textbf{planar}, wenn er so gezeichnet werden kann (in die Ebene eingebettet), dass sich keine Kanten überkreuzen (Die Kanten müssen keine Geradestücke sein.
\end{def*}
\begin{satz*}[note = Eulersche Polyederformel , index = Eulersche Polyederformel]
	Sei $G=(V,E)$ zusammenhängender Graph, planar, die Ebene in $f$ Gebiete unterteilt. Dann $\abs{V} + f - \abs{E} = 2$.
	\begin{bew}
		Solange $G$ Kreise enthält, entferne Kanten: $\abs{V}$ unverändert, $\abs{E}: -1$, $f: -1$. Für Bäume stimmt die Formel: $\abs{V} = \abs{E} + 1, f = 1$
	\end{bew}
	\begin{korr*}
		$G=(V,E)$ planar, $\abs{V} \geq 3$. Dann $\abs{E} \leq 3 \abs{V} - 6$\\
		\begin{bew}
			\begin{gather*}
				\abs{V} + f - \abs{E} = 2 \\
				3f \leq 2 \abs{E} \\
				f \leq \frac{2}{3} \abs{E} \\
				\abs{V} + f - \abs{E} = 2 \leq \abs{V} + \frac{2}{3} \abs{E} - \abs{E} = \abs{V} - \frac{\abs{E}}{3} \\
				\abs{E} \leq 3 \abs{V} - 2
			\end{gather*}
		\end{bew}
	\end{korr*}
	\begin{gather*}
		\abs{E} < 3 \abs{V} \\
		\sum_{v \in V} \deg v = 2 \abs{E} < 6 \abs{V} \\
		\frac{\sum_{v \in V} \deg v}{\abs{V}} < 6
	\end{gather*}
\end{satz*}
\todo{Too long}

\begin{gather*}
	K_5: \abs{E} = 10 , \abs{V} = 5 \\
	10 \leq 3 \cdot 5 - 6 = 9 \quad \lightning \\
	\implies K_5 \text{ nicht planar}\\
	\\
	K_{3,3}: \abs{V} = 6 , \abs{E} = 9 \\
	9 \leq 3 \cdot 6 - 6 = 12 ?
\end{gather*}
\begin{korr*}
	$G=(V,E)$ planar und bipartit(Dreiecksfrei), $\abs{V} \geq 3$. Dann $\abs{E} \leq 2 \abs{V} - 4$\\
	\begin{bew}
		\begin{gather*}
				4f \leq 2 \abs{E} \\
				f \leq \frac{\abs{E}}{2} \\
				\abs{V} + f - \abs{E} = 2 \leq \abs{V} + \frac{\abs{E}}{2} - \abs{E} \\
				\abs{E} \leq 2 \abs{V} - 4
		\end{gather*}
	\end{bew}
	\[ \implies \overline{\deg(v)} < 4 \]
\end{korr*}

Nicht planar sind:
\begin{itemize}
	\item $K_5 , K_{3,3}$
	\item $K_6 , K_{3,4} , K_{4,4} , \dotsc$
	\item Graphen, die $K_5$ oder $K_{3,3}$ als Teilgraphen enthalten.
	\item Unterteilungen des $K_5$ oder des $K_{3,3}$ $(*)$
	\item Graphen, die $(*)$ als Teilgraphen enthalten.
\end{itemize}
\begin{satz*}[note = (Kuratowski)]
	Die Liste ist vollständig.
\end{satz*}

\subsection{Färbung von Graphen}
\begin{def*}[note = Knotenfärbung , index = Knotenfärbung]
	\textbf{(Knoten-)Färbung} von $G=(V,E)$ mit $k$ Farben: $c: V \rightarrow \{ 1, 2, \dotsc , k \}$ so, dass $(u,v) \in E \implies c(u) \neq c(v)$.
\end{def*}
\begin{def*}[note = chromatische Zahl , index = chromatische Zahl]
	\textbf{chronatische Zahl} $X(G)$: min. Anzahl Farben $k$, sodass eine $k$-Färbung von $G$ existiert.
\end{def*}
\begin{bsp*}
	\begin{gather*}
		X(K_n) = n \\
		X(K_{m,n}) = 2 \\
		X(M_{m,n}) = 2 \quad ( \neq( m = 1 \wedge n = 1 )) \\
		X(C_n) = \begin{cases}
			2	&n \text{ gerade}	\\
			3	&n \text{ ungerade}
		\end{cases} \\
		X(\text{Baum}) = 2 \quad ( \abs{V} \geq 2 )
	\end{gather*}
\end{bsp*}
\begin{satz*}
	$G=(V,E)$ zusammenhängend, $\abs{V} \geq 2$ bipartit ($X(G) = 2$) \gdw es keinen Kreis ungerader Länge enthält.\\
	\begin{bew}
		\enquote{$\Rightarrow$} klar, da $X($ungerader Kreis$) > 2$ und $X($Teilgraph$) \leq X($Graph$)$ \\
		\enquote{$\Leftarrow$} Spannbaum von $G$. Querverbindungen:
		\begin{itemize}
			\item zwischen Etagen gleicher Parität $\implies$ ungerader Kreis $\implies$ existieren nicht.
			\item ungleicher Parität: kein Problem
		\end{itemize}
	\end{bew}
\end{satz*}

Enscheidung, ob \dots
\begin{itemize}
	\item $X(G) = 2$: lineare Zeit
	\item $X(G) = 3$: NP-vollständig
\end{itemize}

\begin{satz*}
	$G=(V,E)$ planar $\implies X(G) \leq 4$
\end{satz*}
Analoge, \enquote{historisches} Problem: Wie viele Farben sind nötig, um eine Landkarte so einzufärben, dass angrenzende Länder verschiedene Farben haben? (Länder zusammenhängend) $\rightarrow$ Färben planarer Graphen

\begin{description}
	\item[3?] $K_4$ planar
	\item[4?] $K_5$ nicht planar $\implies$ 4-Färbbarkeit möglich.
\end{description}

\begin{satz*}
	$G$ planar $\implies X(G) \leq 6$\\
	\begin{bew}
		\begin{gather*}
			G \text{ planar } \implies \abs{E} \leq 3 \abs{V} - 6 \\
			\sum_{v \in V} \deg(v) = 2 \abs{E} \leq 6\abs{V} - 6 < 6\abs{V} \\
			\frac{1}{\abs{V}} \sum_{v \in V} \deg(v) < 6 \\
			\exists v \in V : \deg(v) < 6
		\end{gather*}
		Induktion über $\abs{V} = n$ \\
		IV: $n \leq 6$ \\
		$n \rightarrow n+1$
	\end{bew}
\end{satz*}
