\chapter{Algebra}
\section{Gruppen}
\clearpage
\begin{def*}[note = Gruppe , index = Gruppe]
	$(G,*) , G$ Menge (endlich oder unendlich) \\
	$* : G \times G \rightarrow G$ Operation \\
	$(G,*)$ \textbf{Gruppe}, falls
	\begin{itemize}
		\item $\forall a , b , c \in G : (a*b)*c = a*(b*c)$ (Assoziativität)
		\item $\exists e : a*e = e*a = a \forall a$ (Neutralelement)
		\item $\forall a \exists b : a*b = b*a = e$ (Inverse)
	\end{itemize}
	Folgerungen:
	\begin{itemize}
		\item Klammer weglassen
		\item \[ a^n = \begin{cases}
			\underbrace{a * a * \dots * a}_{n \text{ mal}}				&n \geq 0	\\
			e											&n = 0	\\
			\underbrace{a^{-1} * a^{-1} * \dots * a^{-1}}_{-n \text{ mal}}	&n \geq 0	
			\end{cases} \] $\rightarrow$ Potenzgesetze
		\item Neutralelement eindeutig: $e, e'$ Neutrlelemente: \[ e \overset{e'}{=} e * e' \overset{e}{=} e' \]
		\item Inverse eindeutig: $b, b'$ Inverse von $a$ \[ b = b*e = b*(a*b') = (b*a)*b' = b' \implies b = b' = a^{-1} \]
	\end{itemize}
\end{def*}
\begin{bem}
	Kommutativität ($\forall a, b \in G : a * b = b * a$) \textbf{nicht} gefordert. \\
	Falls sie gilt: \textbf{Abelsche} Gruppe
\end{bem}
Folgerung:
$\forall a, b \in G : a * x = b$ hat \text{genau eine} Lösung. \\
\begin{bew}[note = Exsitenz]
	\begin{gather*}
		x = a^{-1} * b \\
		a * ( a^{-1} * b ) = ( a * a^{-1} ) * b = b
	\end{gather*}
\end{bew}
\begin{bew}[note = Eindeutigkeit]
	$x, x'$ Lösungen \\
	\begin{gather*}
		a * x = b = a * x' \quad \mid a^{-1} * \\
		a^{-1} * a * x = a^{-1} * a * x' \\
		x = x'
	\end{gather*}
\end{bew}

\begin{bsp*}[head = Beispiele von Gruppen]
	\begin{itemize}
		\item $( \Z , + )$ , $( \Z_m^* , \cdot )$
		\item $( \Q , + )$ , $( \Q \setminus \{ 0 \} , \cdot )$
		\item $( \R , + )$ , $( \R \setminus \{ 0 \} , \cdot )$
		\item $( \Z_p , + )$ , $( \Z_p \setminus \{ 0 \} , \cdot )$
		\item ( $\R^n$ , Vektoraddition )
		\item ( $M^{n \times n}(\R)$ mit $\det(M) \neq 0$, Matrixmultiplikation )
		\item ( Permutationen auf einer Menge $M$ , Komposition )
	\end{itemize}
\end{bsp*}
\begin{def*}{Geometrische Gruppen}
	Abbildung der Ebene auf sich, die gewisse Grössen invariant lassen.
\end{def*}
\begin{bsp*}
	\begin{itemize}
		\item Rotation um 0.
		\item Ähnlichkeitstransformationen
		\item Kongruenztransformationen: (nicht kommutativ)
			\begin{itemize}
				\item Translationen
				\item Rotationen
				\item Spiegelungen
			\end{itemize}
		\item Symmetriegruppen:
			\begin{itemize}
				\item Symmertiegruppe $S_3$ des gleichseitegen Dreiecks: Menge aller Kongruenztransformationen, die den Dreieck invariant lassen. \\
					Anzahl Ecken: $6 = 3!$
				\item Symmetriegruppe $S_4$ des Quadrats \\
					Anzahl Elemente: 8
				\item Symmetriegruppe $S_n$ des $n$-Ecks \\
					Anzahl Elemente: $2n$ \\
					Die Gruppe wird erzeugt von:
						\begin{itemize}
							\item Drehung um $\frac{2\pi}{n}$
							\item eine Spiegelung
						\end{itemize}
						oder
						\begin{itemize}
							\item alle Spiegelungen
						\end{itemize}
						$\implies$ 1 oder 2 Spiegelungen
			\end{itemize}
	\end{itemize}
\end{bsp*}
\begin{bsp*}[head = Beispiele von endlichen Gruppen: systematisch]
	\begin{description}
		\item[1 Element?] Triviale Gruppe $G = \{ e \} , e * e = e$
		\item[2 Elemente?] \{ Identität, eine feste Spiegelung \} , $( \Z_2 , + )$ , $( \Z_3^* , \cdot )$ , XOR
		\item[3 Elemente?] Es gibt nur eine Gruppe mit 3 Elementen. \\
			\begin{tabular}{ c | c c c }
				$\cdot$	&e	&a	&b	\\ \hline
				e		&e	&a	&b	\\
				a		&a	&b	&e	\\
				b		&b	&e	&a	
			\end{tabular} \quad
			\begin{tabular}{ c | c c c }
				$( \Z_3 , + )$	&0	&1	&2	\\ \hline
				0			&0	&1	&2	\\
				1			&1	&2	&0	\\
				2			&2	&0	&1	
			\end{tabular}
		\item[4 Elemente?]
			\begin{tabular}{ c | c c c c }
				$*$		&e	&a		&b		&c	\\ \hline
				e		&e	&a		&b		&c	\\
				a		&a	&\fbox{e}	&c		&b	\\
				b		&b	&c		&\fbox{e}	&a	\\
				c		&a	&b		&a		&e	
			\end{tabular} $\not\equiv$
			\begin{tabular}{ c | c c c c }
				$*$		&e	&a		&b	&c	\\ \hline
				e		&e	&a		&b	&c	\\
				a		&a	&\fbox{e}	&c	&b	\\
				b		&b	&c		&\fbox{a}	&e	\\
				c		&a	&b		&e	&a	
			\end{tabular} $\equiv$
			\begin{tabular}{ c | c c c c }
				$*$		&e	&a		&b	&c	\\ \hline
				e		&e	&a		&b	&c	\\
				a		&a	&\fbox{b}	&c	&e	\\
				b		&b	&c		&e	&a	\\
				c		&a	&e		&a	&b	
			\end{tabular} $\equiv ( \Z_4 , + ) \equiv ( \Z_5^* , \cdot )$
	\end{description}
\end{bsp*}
\begin{def*}[note = Isomorphismus , index = Isomorphismus]
	\textbf{Isomorphismus} zwischen $( G , * ) , ( G' , \circ )$:
	$\varphi : G \rightarrow G'$ bijektiv, sodass
	\[ \forall a, b \in G : \varphi( a * b ) = \varphi( a ) \circ \varphi( b ) \]
	$G$ und $G'$ sind \textbf{isomorph}, $G \equiv G'$
\end{def*}
Es gilt dann auch:
\[ \varphi( e ) = e' \in G' \]
\begin{bew}
	\begin{gather*}
		\varphi(a) = \varphi(a * e) = \varphi(a) \circ \varphi(e) \\
		\implies \varphi(e) = e' \in G'
	\end{gather*}
\end{bew}
\[ \varphi(a^{-1}) = \varphi(a)^{-1} \]
\begin{bew}
	\[ e' = \varphi(e) = \varphi(a * a^{-1}) = \varphi(a) \circ \underbrace{\varphi(a^{-1})}_{=\varphi(a)^{-1}} \quad \blacksquare \]
\end{bew}
\begin{gather*}
	\text{CRS: } \Z_2 \times \Z_3 = \Z_6 \\
	\Z_2 \times \Z_2 \\
	\begin{pmatrix}a\\b\end{pmatrix} * \begin{pmatrix}c\\d\end{pmatrix} = \begin{pmatrix}a \oplus c\\b \oplus d\end{pmatrix}
\end{gather*}
\begin{tabular}{ c | c c c c }
	$*$		&$\begin{pmatrix}0\\0\end{pmatrix}$	&$\begin{pmatrix}1\\0\end{pmatrix}$		&$\begin{pmatrix}0\\1\end{pmatrix}$	&$\begin{pmatrix}1\\1\end{pmatrix}$	\\ \hline
	$\begin{pmatrix}0\\0\end{pmatrix}$		&$\begin{pmatrix}0\\0\end{pmatrix}$	&$\begin{pmatrix}1\\0\end{pmatrix}$		&$\begin{pmatrix}0\\1\end{pmatrix}$	&$\begin{pmatrix}1\\1\end{pmatrix}$	\\
	$\begin{pmatrix}1\\0\end{pmatrix}$		&$\begin{pmatrix}1\\0\end{pmatrix}$	&$\begin{pmatrix}0\\0\end{pmatrix}$	&$\begin{pmatrix}1\\1\end{pmatrix}$	&$\begin{pmatrix}0\\1\end{pmatrix}$	\\
	$\begin{pmatrix}0\\1\end{pmatrix}$		&$\begin{pmatrix}0\\1\end{pmatrix}$	&$\begin{pmatrix}1\\1\end{pmatrix}$		&$\begin{pmatrix}0\\0\end{pmatrix}$	&$\begin{pmatrix}1\\0\end{pmatrix}$	\\
	$\begin{pmatrix}1\\1\end{pmatrix}$		&$\begin{pmatrix}1\\1\end{pmatrix}$	&$\begin{pmatrix}0\\1\end{pmatrix}$		&$\begin{pmatrix}1\\0\end{pmatrix}$	&$\begin{pmatrix}0\\0\end{pmatrix}$	
\end{tabular}\\
\begin{def*}[note = Produkt , index = Produkt]
	$(G , \circ_G) , (G' , \circ_{G'})$ Gruppen \\
	\textbf{Produkt} $G \times G' : (G \times G' , \circ ):$ \\
	\[ (a,a') \circ (b,b') \coloneqq (a \circ_G b , a' \circ_{G'} b') \quad a , b \in G ; a' , b' \in G' \]
	$(G \times G' , \circ)$ ist Gruppe
\end{def*}
\begin{def*}[note = Untergruppe , index = Untergruppe]
	$(G , \circ)$ Gruppe \\
	$\varnothing \neq H \subseteq G$ \textbf{Untergruppe} von $G$, falls $H$ selbst eine Gruppe bezüglich $\circ$ ist:
	\begin{itemize}
		\item $e \in H$
		\item $a , b \in H \implies a \circ b \in H$
		\item $a \in H \implies a^{-1} \in H$
	\end{itemize}
\end{def*}
\begin{bsp*}
	\begin{itemize}
		\item Untergruppen von $\Z_{10} : \{ 0 , 1 , \dotsc , 9 \}$ \\
			$\Z_{10} , \{ 0, 2, 4, 6, 8 \} , \{ 0, 5 \} , \{ 0 \}$
		\item $\{ -1 , 1 \}$ ist UG von $\Z_{10}^* \quad (n > 2)$ \\
			$2 \mid \abs{Z_n^*} = \varphi(n) = \prod (p_i -1) p_i^{e_i - 1}$
	\end{itemize}
\end{bsp*}
\begin{satz*}[note = (Lagrange)]
	$G$ endliche Gruppe, $H$ Untergruppe von $G$. \\
	Dann $\abs{H} \mid \abs{G}$\\
	\begin{bew}
		\begin{gather*}
			a \in G \\
			a H \coloneqq \{ a \cdot h \mid h \in H \} \\
		\end{gather*}
		Nebenklasse \\
		Beobachtungen:
		\begin{itemize}
			\item Falls $e \notin H$, dann ist $a H$ keine Untergruppe von $G$ $(e \notin a \cdot H) \quad [e = a \cdot h \implies a = \underbrace{e}_{\in H} \cdot \underbrace{h^{-1}}_{\in H} \implies a \in H \: \lightning]$
			\item Falls $a \cdot H \cap b \cdot H \neq \varnothing \implies aH = bH$. \\
				Sei $c = a \cdot h = b \cdot h' \implies a = b \cdot \underbrace{h' \cdot h^{-1}}_{\in H} \quad h , h' \in H$\\
				Sei $x \in aH$ \\
				$x = a \cdot \underbrace{h''}_{\in H} = b \cdot \underbrace{h' \cdot h^{-1} \cdot h''}_{\in H} \in b H$\\
				Analog: $bH \subseteq aH$. Also $aH = bH$
			\item $\underbrace{\abs{aH}}_{=\abs{\{ a \cdot h \mid h \in H \}}} = \abs{H}$ \\
				$ah = ah' \overset{a^{-1} \cdot}{\iff} h = h'$
		\end{itemize}
		disjunkte Vereinigung
		\[ G = H \cup aH \cup a'H \cup \dots \implies \abs{H} \mid \abs{G} \]
	\end{bew}
	Konsequenzen? \\
	Betrachte $ e = a^0 , a^1 , a^2 , a^3 , \dotsc , a^i , \dotsc a^j$
	$a^i = a^j \implies \underbrace{a^j (a^i)^{-1}}_{a^{j-1}} = e$
\end{satz*}
\begin{def*}[note = Ordnung , index = Ordnung]
	\begin{gather*}
		G , a \in G \\
		\ord(a) \coloneqq \min\{ i > 0 \mid a^i = e \} \\
		\textbf{Ordnung} \text{ von } a \\
		H_a \coloneqq \{ e , a , a^2 , \dotsc , a^{\ord(a) - 1} \}
	\end{gather*}
	\begin{itemize}
		\item $H_a$ hat genau $\ord(a)$ Elemente.
		\item $H_a$ ist Abelsche Untergruppe von $G$ \\
			\begin{itemize}
				\item $a^i \circ a^j = a^{i+j} = a^{k \cdot \ord(a) + R_{\ord(a)}(i+j)} = \underbrace{(\underbrace{a^{\ord(a)}}_{e})^k}_{e} \circ a^{\overbrace{R_{\ord(a)}(i+j)}^{< \ord(a)}} \in H_a$
				\item $(\underbrace{a^i}_{\in H_a})^{-1} = a^{\ord(a) - i} \in H_a \quad [ a^i \circ a^{\ord(a) - i} = a^{\ord(a)} = e ]$
				\item $a^i \circ a^j = a^{i+j} = a^{j+i} = a^j \circ a^i$
			\end{itemize}
	\end{itemize}
\end{def*}
\begin{satz*}[note = (Lagrange)]
	$G$ endliche Gruppe, $a \in G$. \\
	Dann $\ord(a) \mid \abs{G}$ und
	\[ a^{\abs{G}} = e \]
	\begin{bew}
		\begin{gather*}
			\ord(a) = \abs{H_a} , H_a \text{ UG von } G \\
			\implies \abs{H_a} \mid \abs{G} \implies \ord(a) \mid \abs{G} \\
			\implies \exists k \in \Z : k \cdot \ord(a) = \abs{G} \\
			a^{\abs{G}} = a^{k \cdot \ord(a)} = ( a^{\ord(a)} )^k = e^k = e \quad \blacksquare
		\end{gather*}
	\end{bew}
\end{satz*}
Spezialfall: \\
\begin{satz*}[note = (Fermat)]
	$p$ prim , $p \nmid a$:
	\[ a^{p-1} \equiv 1 \pmod p \]
	\begin{bew}
		Lagrange für $G = \Z_p^*$
	\end{bew}
\end{satz*}
\begin{satz*}[note = (Euler)]
	$n \in \N , \ggt(a,n) = 1$ \\
	Dann
	\[ a^{\varphi(n)} \equiv 1 \pmod n \]
	\begin{bew}
		$\varphi(n) = \abs{\Z_n^*}$\\
		$a \in \Z_n^*$\\
		Lagrange
	\end{bew}
\end{satz*}
\begin{bsp*}
	\begin{gather*}
		p , q , p \neq q , p \nmid a , q \nmid a , n = p \cdot q \\
		a = (p-1)(q-1) \equiv 1 \pmod {pq}
	\end{gather*}
\end{bsp*}

\subsection{RSA 1978 (Rivest, Shamir, Adleman)}
Public-Key-Kryptolosystem (1976 Diffie, Hellman)

Ingredienzen von RSA:
\begin{description}
	\item[Effizienz] Modulare Arithmetik, erweiterter Euklid
	\item[Korrektheit] Euler (Fermat + CRS)
	\item[Sicherheit] $p \cdot q \rightarrow p , q$ schwierig?
\end{description}
\begin{tabularx}{\textwidth}{ X c X }
	Alice								&		&Bob								\\
&&	Wählt grosse Primzahlen $p , q$.													\\
&&	Berechne $n \coloneqq p \cdot q$												\\
&&	$e$ zufällig mit $\ggt(e,(p-1)(q-1)) = 1$											\\
&&	Berechne $d = e^{-1} \pmod{(p-1)(q-1)}$											\\
&&	$pk = (n,e) , sk = (n,d)$														\\
&	$\overset{pk}{\leftarrow}$														&\\
	Meldung $M (\leq n)$															&&\\
	ENC: $C = R_n(M^e)$															&&\\
	Effiziente Berechnung:															&&\\
	$M , m^2 , M^4 , M^6 , \dotsc , M^{2i}$												&&\\
	$e = (e_r e_{r-1} \dots e_1 e_0)_2$ Binärdarstellung									&&\\
	$C = ((((M^{e_r})^2 \cdot M^{e_{r-1}})^2 \cdot M^{e_{r-2}})^2 \cdot M^{e_{r-3}})^2 \dotsm M^{e_0}$ 	&&\\
	in $\Z_n^* \implies R_n(\cdot)$ bei jedem Schritt										&&\\
&	$\overset{C}{\rightarrow}$														&\\
&&	DEC: $M = R_n(C^d)$															
\end{tabularx}

\subsubsection{Wieso ist RSA korrekt?}
\begin{gather*}
	M \rightarrow C = R_n(M^e) \rightarrow R_n(C^d) \overset{??}{=} M \\
	R_n(C^d) = R_n((R_n(M^e))^d) = R_n(M^{e \cdot d}) = R_n(M^{1 + k(p-1)(q-1)}) \\
	d \equiv e^{-1} \pmod{(p-1)(q-1)} \iff e \cdot d \equiv 1 \pmod{(p-1)(q-1)}
\end{gather*}

\subsubsection{Sicherheit}
RSA gebrochen, falls $\underbrace{p \cdot q}_{n} \rightarrow p , q$ \quad Bester Algorithmus?
\begin{description}
	\item[naiv:] $\sqrt{n} = e^{\frac{1}{2} \log n}$
	\item[NFS:] $e^{(\log n)^{\frac{1}{3}}}$
\end{description}

\subsubsection{Digitale Signaturen}
\begin{tabularx}{\textwidth}{ X c X }
	Alice				&	&Bob		\\
	$SK = (n,d)$					\\
	$PK = (n,e)$					\\
&	$\overset{PK}{\rightarrow}$			\\
	Ich schulde Bob 1000 CHF = $v$		\\
	$S = R_n(v^d)$					\\
&	$\overset{v,S}{\rightarrow}$			\\
&&	Verifizieren: $R_n(S^e) \overset{?}{=} v$	\\
&&	Bob geht zum Richter:				\\
&&	$v, S, (n,e) = PK$ Alice				
\end{tabularx}

\subsubsection{Anzahl Gruppen mit 5 Elementen}
\begin{gather*}
	G = \{ e , a , b , c , d \} \\
	\ord(a) \mid 5 , \quad a \neq e \implies \ord(a) \neq 1 \\
	\implies \ord(a) = 5
	\intertext{Also:}
	G = \{ e , a , a^2 , a^3 , a^4 \} \\
	a^i \circ a^j = a^{R_5(i+j)} \\
	\implies G \equiv \Z_5 \\
	\intertext{Isomorphismus: $a^i \leftrightarrow i$}
\end{gather*}

\subsubsection{Gruppen mit 6 Elementen}
$\Z_6 \equiv \Z_7^* \equiv \Z_2 \times \Z_3 \not\equiv S_3$ (Symmetriegruppe des Dreiecks) $\equiv$ Permutation on $\{ 1, 2, 3 \}$\\
\[ \begin{array}{ c | c | c }
	\Z_7^*	&\Z_6	&\Z_2 \times \Z^3				\\ \hline
	1		&0		&\begin{pmatrix}0\\0\end{pmatrix}	\\
	3		&1		&\begin{pmatrix}1\\1\end{pmatrix}	\\
	2		&2		&\begin{pmatrix}0\\2\end{pmatrix}	\\
	6		&3		&\begin{pmatrix}1\\0\end{pmatrix}	\\
	4		&4		&\begin{pmatrix}0\\1\end{pmatrix}	\\
	5		&5		&\begin{pmatrix}1\\2\end{pmatrix}	
\end{array} \]

\subsubsection{Gruppen mit 7 Elementen}
$\Z_7$

\subsubsection{Gruppen mit \texorpdfstring{$p$}{p} Elementen (\texorpdfstring{$p$}{p} prim)}
$\Z_p$

\begin{def*}[note = zyklisch , index = zyklisch]
	Eine (endliche) Gruppe $G$ heisst \textbf{zyklisch}, falls es $g \in G$ gibt mit $\ord(g) = \abs{G}$:
	\[ G = \{ g^0 , g^1 , \dotsc , g^{\abs{G}-1} \} \]
	$g$ heisst \textbf{Generator} von $g$, $G \equiv \langle g \rangle$
\end{def*}
\begin{bem}
	$G$ zyklisch $\implies G$ abelsch \\
	\[ a \circ b = g^i \circ g^j = g^{i+j} = g^{j+i} = b \circ a \]
	Umkehrung gilt nicht! (Gegenbeispiel $\Z_2 \times \Z_2$)
\end{bem}
\begin{satz*}
	$\abs{G}$ prim $\implies G$ zyklisch \\
	Jedes $g \neq e$ ist Generator. \\
	\begin{bew}
		\begin{gather*}
			g \neq e , \ord(g) \mid \abs{G} , \ord(g) \neq 1 \\
			\implies \ord(G) = \abs{G} \implies G = \langle g \rangle \quad \blacksquare
		\end{gather*}
	\end{bew}
\end{satz*}
$G$ zyklisch : $G \equiv \Z_{\abs{G}}$ \\
\begin{bsp*}
	\begin{gather*}
		\Z_p^* \equiv \Z_{p-1} = \langle a \rangle \\
		\varphi : R_p(a^x) \leftrightarrow x
	\end{gather*}
	BILD $\Z_{10}$ \\
	BILD $\Z_{\abs{G}}$ \\
	$i \mapsto g^i$: square \& multiply \\
	\# ops $\leq 2 \cdot \log \abs{G}$
	$\rightarrow$ effizient \\
	$g^i \mapsto i$ schwierig \\
	Diskreter Logarithmus:
	\begin{enumerate}[label=\arabic*)]
		\item naiv: $g^0 , g^1 , g^2 , \dotsc , g^i \rightarrow \abs{G}$ Schritte
		\item \enquote{Baby-step giant-step} falls Speicher für $M$ Gruppenelemente \\
			Tabelle: $(0,g^i) , (1,g^{i+1}) , \dotsc , (M,g^{i+M})$ sortiert nach zweiten Eintrag \\
			BILD \\
			$g^{jM} = g^{i+k} \implies i = jM - k$ \\
			\# Schritte: $M + \frac{\abs{G}}{M}$, optimal: $M \approx \sqrt{\abs{G}} \rightsquigarrow \sim \sqrt{\abs{G}}$
	\end{enumerate}
\end{bsp*}
\begin{bsp*}
	\begin{tabular}{ l l l }
		$\abs{G} =$	&$2^{1000}$				\\
		$i \mapsto g^i$:	&$2000$					\\
		$g^i \mapsto i$:	&naiv:	&$\sim 2^{1000}$	\\
					&BS-GS:	&$\sim 2^{500}$		
	\end{tabular}
\end{bsp*}

\subsubsection{Diffie-Hellman-Protokoll}
\begin{tabular}{ c c c }
	Alice				&	&Bob												\\
					&$\overset{G,g,G=\langle g \rangle}{\rightarrow}$	&				\\
	$x$				&$\overset{g^x}{\rightarrow}$				&				\\
					&$\overset{g^y}{\leftarrow}$				$y$				\\
	$(g^y)^x = g^{xy}$	&									&$(g^x)^y = g^{xy}$	\\
		&$g^x, g^y \rightsquigarrow g^{xy}$						&				\\
		&vermutlich schwierig									&				
\end{tabular}\\
Ziel: Verständnis von fehlerkorrigierenden Codes\\
Problem: Verrauschter Kanal\\
BILD

\subsection{Körper und Polynome}
\begin{def*}[note = Körper , index = Körper]
	$(K,+,\cdot)$ Körper, falls
	\begin{itemize}
		\item $(K,+)$ Abelsche Gruppe (Neutralelement $0$)
		\item $(K^* = K \setminus \{ 0 \},\cdot)$ Abelsche Gruppe (Neutralement $1 \neq 0$)
		\item Distibutivität: $\forall a, b, c \in K : a \cdot (b+c) = a \cdot b + a \cdot c$
	\end{itemize}
\end{def*}

\subsubsection{Beispiele von Körper}
\[ \C , \R , \Q , ( \Z_p , + , \cdot ) (p \text{ prim}) \]
$p=2$
\begin{gather*}
	\begin{array}{ c | c c }
		+	&0	&1	\\ \hline
		0	&0	&1	\\
		1	&1	&0	
	\end{array} \quad \begin{array}{ c | c c }
		\cdot	&0	&1	\\ \hline
		0	&0	&0	\\
		1	&0	&1	
	\end{array}
\end{gather*}

\subsubsection{Eigenschaften}
\begin{itemize}
	\item $0 \cdot a = 0$ \\
		\begin{bew}
			\begin{gather*}
				0 \cdot a = (0+0) \cdot a = 0 \cdot a + 0 \cdot a \quad + (-0 \cdot a) \\
				0 = 0 \cdot a
			\end{gather*}
		\end{bew}
	\item Nullteilerfreiheit: $a \cdot b = 0 \implies a = 0 \vee b = 0$ \\
		\begin{bew}
			\begin{gather*}
				a \neq 0 \implies \exists a^{-1} \\
				b = a^{-1} \cdot \underbrace{a \cdot b}_{=0} = 0
			\end{gather*}
		\end{bew}
	\item $(a+b)^2 = a^2 + \underbrace{2}_{=(1+1)}ab + b^2$
	\item Gleichungssystem Lösen: Gauss-Elimination \\
		$Ax = b$ hat eindeutige Lösung, falls $\det(A) \neq 0 \quad (A \in K^{n \times n} , b \in K^n)$
\end{itemize}

\subsubsection{Endliche Körper}
$GF(2)$ \enquote{Galois-field} (1811 - 1832) \\
\begin{gather*}
	GF(2) \\
	\begin{array}{ c | c c }
		+	&0	&1	\\ \hline
		0	&0	&1	\\
		1	&1	&0	
	\end{array} \quad \begin{array}{ c | c c }
		\cdot	&0	&1	\\ \hline
		0	&0	&0	\\
		1	&0	&1	
	\end{array} \\
	GF(3) \\
	\begin{array}{ c | c c c }
		+	&0	&1	&2	\\ \hline
		0	&0	&1	&2	\\
		1	&1	&2	&0	\\
		2	&2	&0	&1
	\end{array} \quad \begin{array}{ c | c c c }
		\cdot	&0	&1	&2	\\ \hline
		0	&0	&0	&0	\\
		1	&0	&1	&2	\\
		2	&0	&2	&1	
	\end{array} \\
	\text{Körper mit 4 Elementen? } (\Z_4,+,\cdot) \text{ kein Körper } 2 \cdot 2 = 0 \\
	GF(4) \\
	\begin{array}{ c | c c c c }
		+	&0	&1	&a	&b	\\ \hline
		0	&0	&1	&a	&b	\\
		1	&1	&0	&b	&a	\\
		a	&2	&b	&0	&1	\\
		b	&b	&a	&1	&0	
	\end{array} \quad \begin{array}{ c | c c c c }
		\cdot	&0	&1	&a	&b	\\ \hline
		0	&0	&0	&0	&0	\\
		1	&0	&1	&a	&b	\\
		a	&0	&a	&b	&1	\\
		b	&0	&b	&1	&a	
	\end{array}
\end{gather*}
$\Z$ $\rightarrow$ \begin{tabular}{c}Euklid\\Restsatz\end{tabular} $\rightarrow$ Teilbarkeit $\rightarrow$ \begin{tabular}{c}ggT\\Erweiterter\\Eulkid\\$\ggt(a,b) = u \cdot a + v \cdot b$\end{tabular} $\rightarrow$ \begin{tabular}{c}Primzahlen\\eindeutige Zerlegung\\$p | a \cdot b \implies p | a \vee p | b$\end{tabular} $\rightarrow$ \begin{tabular}{c}Arithmetik modulo $p$\\Existenz von Inversen\\$\rightarrow$ Primkörper $\Z_p = GF(p)$\end{tabular} \\
\begin{tabular}{c}Polynome\\über Körper\\(z.B. über $GF(4)$\end{tabular} $\rightarrow$ '' $\rightarrow$ '' $\rightarrow$ '' $\rightarrow$ \begin{tabular}{c}Irreduzible Polynome\\eindeutige Zerlegung\end{tabular} $\rightarrow$ \begin{tabular}{c}Arithmetik modulo\\irreduzible Polynome\\Existenz der Inverse\\$\rightarrow$ Erweiterungskörper\end{tabular}\\
\begin{bsp*}
	\begin{gather*}
		GF(4) = GF(2^2) \\
		[ GF(2): 1+1=0 , + = - , (a+b)^2 = a^2 + b^2 ] \\
		p(x) = x^2 + x + 1 \text{ Primpolynom} \\
		K = \{ 0 , 1 , x , x+1 \} \\
		\begin{array}{ c | c c c c }
		+	&0	&1	&x	&x+1	\\ \hline
		0	&0	&1	&x	&x+1	\\
		1	&1	&0	&x+1	&x	\\
		x	&x	&x+1	&0	&1	\\
		x+1	&x+1	&x	&1	&0	
	\end{array} \\
	\begin{array}{ c | c c c c }
		\cdot	&0	&1	&x	&x+1	\\ \hline
		0	&0	&0	&0	&0	\\
		1	&0	&1	&x	&x+1	\\
		x	&0	&x	&x+1	&1	\\
		x+1	&0	&x+1	&1	&x	
	\end{array}
	\end{gather*}
\end{bsp*}

\subsubsection{Polynome über einem Körper K}
\begin{def*}[note = Polynomring , index = Polynomring]
	$K$ Körper. $K[x]$ Polynomring,
	\begin{gather*}
		K[x] \coloneqq \left\{ \sum_{i=0}^n a_i x^i \mid a_i \in K , a_n \neq 0 \right\} \cup \{ 0 \} \\
		a_n \neq 0 : \deg\left( \sum_{i=0}^n a_i x^i \right) = n \\
		\enquote{\deg(0) = -\infty} \\
		\deg(a \cdot b) = \deg(a) + \deg(b)
	\end{gather*}
\end{def*}
\begin{satz*}[note = (Euklid)]
	$K$ Körper, $a(x) , b(x) \in K[x] , b(x) \not\equiv 0$.\\
	Dann gibt es eindeutige $q(x) , r(x) \in K[x]$ mit
	\[ a(x) = q(x) \cdot b(x) + r(x) , \quad \deg(r(x)) < \deg(b(x)) \]
	\begin{bew}
		Existenz:
		\begin{description}
			\item[1. Fall] \[ \deg(a) < \deg(b) : q(x) \equiv 0 , r(x) \equiv a(x) \]
			\item[2. Fall]
				\begin{gather*}
					\deg(a) \geq \deg(b) :	\\
					a(x) = a_n x^n + a_{n-1} x^{n-1} + \dots + a_1 x + a_0 \\
					b(x) = b_m x^m + b_{m-1} x^{m-1} + \dots + b_1 x + b_0 \\
					q_1(x) = \frac{a_n}{b_m} x^{n-m} \\
					\deg(\underbrace{a(x) - q_1(x) b(x)}_{=a'(x)}) \leq \deg(a) - 1
				\end{gather*}
				Falls $\deg(a') \geq \deg(b)$: Analog: $\deg(a'(x) - q_2(x) b(x)) \leq \deg(a') - 1 \leq \deg(a) - 2$ \\
				Wiederholen, bis $\deg(\underbrace{a(x) - ( q_1(x) + q_2(x) + \dots ) b(x)}_{=r(x)}) < \deg(b)$
		\end{description}
		Eindeutigkeit:
		\begin{gather*}
			\deg(r) < \deg(b) , \deg(r') < \deg(b) \\
			\begin{split}
				a(x)	&\equiv q(x) b(x) + r(x) \\
					&\equiv q'(x) b(x) + r'(x)
			\end{split} \\
			(q(x) - q'(x)) b(x) \equiv \underbrace{r'(x) - r(x)}_{<\deg(b)} \\
			\begin{split}
				\implies	&q(x) \equiv q'(x) \\
						&r(x) \equiv r'(x)
			\end{split} \quad \blacksquare
		\end{gather*}
	\end{bew}
\end{satz*}

\subsubsection{Teilbarkeit und irreduzible Polynome}
\begin{def*}[note = irreduzibel , index = irreduzibel]
	$a(x) , b(x) \in K[x] , b \not\equiv 0$. Dann
	\begin{gather*}
		b(x) \mid a(x) :\iff \exists c(x) \in K[x] : b(x) \cdot c(x) = a(x) \\
		(\deg(p) \geq 1) \\
		p(x) \text{ irreduzibel } \\
		:\iff ( p(x) = s(x) \cdot t(x) \implies \deg(s) = 0 \vee \deg(t) = 0 )
	\end{gather*}
\end{def*}
\begin{bsp*}
	\[ \R , x+1 = \frac{1}{5} \cdot (5x + 5) \]
\end{bsp*}

Wie erkennt man, ob ein Polynom irreduzibel ist?\\
Nullstellen und lineare Faktoren\\
\begin{satz*}
	$a(x) \in K[x] , c \in K$ mit $a(c) = 0$. ($a(x)$ als Funktion $K \rightarrow K$, ausgewertet an $x = c$). Dann und nur dann gilt
	\[(x-c) | a(x) \]
	\begin{bew}
		Euklid: 
		\begin{gather*}
			a(x) = q(x) (x-c) + r(x) \\
			\deg(r) < 1 \\
			r \in K \\
			a(x) = q(x) (x-c) + r \\
			c \text{ ist Nullstelle} \\
			0 = a(c) = q(c) \underbrace{(c-c)}_{0} + r \\
			\implies r = 0 \\
			\implies a(x) = q(x) (x-c) \implies (x-c) | a(x) \quad \blacksquare
		\end{gather*}
	\end{bew}
	\[ x-c \equiv 1 \cdot x^1 + (-c) \cdot x^0 \]
\end{satz*}

\subsubsection{Irreduzible Polynome über GF(2)}
\begin{description}
	\item[Grad 1] $x, x+1$
	\item[Grad 2] 
		\[ \begin{array}{ c c c c c c }
				&	&	&	&	&\text{NS}	\\
			x^2	&	&	&	&	&0		\\
			x^2	&	&	&+	&1	&1		\\
			x^2	&+	&x	&	&	&0,1		\\
			x^2	&+	&x	&+	&1	&\leftarrow
		\end{array} \]
	\item[Grad 3] 
		\[ \begin{array}{ c c c c c c c c }
				&	&	&	&	&	&	&\text{NS}	\\
			x^3	&	&	&	&	&	&	&0		\\
			x^3	&	&	&	&	&+	&1	&1		\\
			x^3	&	&	&+	&x	&	&	&0,1		\\
			x^3	&	&	&+	&x	&+	&1	&\leftarrow\\
			x^3	&+	&x^2	&	&	&	&	&0,1		\\
			x^3	&+	&x^2	&	&	&+	&1	&\leftarrow\\
			x^3	&+	&x^2	&+	&x	&	&	&0		\\
			x^3	&+	&x^2	&+	&x	&+	&1	&1	
		\end{array} \]
	\item[Grad 4]
		Keine NS, aber nicht irreduzibel: $(x^2+x+1)(x^2+x+1) = x^4+x^2+1$ \\
		$x^4 + \dots + 1$ ungerade \# Summanden: \\
		$x^4+x^3+1$ \\
		$x^4+x+1$ \\
		$x^4+x^3+x^2+x+1$
\end{description}
Es gibt irreduzible Polynome beliebigen Grades über $GF(p)$

\subsubsection{ggT und Erweiterter Euklid}
\[ a(x) | b(x) \wedge b(x) | a(x) \implies a(x) = k \cdot b(x) , k \in K^* \]
\begin{bew}
	\begin{gather*}
		a(x) = c(x) \cdot b(x) \equiv c(x) \cdot c'(x) \cdot a(x) \\
		\implies c(x) \cdot c'(x) \equiv 1 \implies c(x) = k \in K^* , c' = \frac{1}{k} \text{ konstant}
	\end{gather*}
\end{bew}
\begin{def*}[note = ggT , index = ggT]
	$c(x)$ ist ggT von $a(x)$ und $b(x)$, falls 
	\begin{itemize}
		\item $c(x) | a(x)$
		\item $c(x) | b(x)$
		\item $d(x) | a(x) \wedge d(x) | b(x) \implies d(x) | c(x)$
	\end{itemize}
	ggT ist bis auf Konstante eindeutig.
\end{def*}

Euklid:
\begin{gather*}
	K = GF(3) \quad \ggt(x^3+2x+1 , 2x^2+x+2) = u \cdot (x^3+2x+1) + v \cdot (2x^2+x+2)
\end{gather*}
\begin{gather*}
	\begin{array}{ | c | c | c | }
									\hline
					&u		&v		\\ \hline
		x^3 + 2x + 1	&1		&0		\\ \hline
		2x^2 + x + 2	&0		&1		\\ \hline
		2x			&1		&x+1		\\ \hline
		2			&2x+1	&2x^2 + 2	\\ \hline
		1			&x + 2	&x^2 + 1	\\ \hline
	\end{array} \\
	1 = (x+2)(x^3+2x+1) + (x^2+1)(2x^2+x+2)
\end{gather*}
Euklid Restsatz, EEA
\[ \ggt(a(x),b(x)) = u(x)a(x) + v(x)b(x) \]
(Lemma von Bézont)
\begin{gather*}
	p \in K[x] \text{ irreduzibel} \\
	p \nmid a : \ggt(a,p) = 1(x) \\
	u \cdot a + v \cdot p \equiv 1(x) \\
	u \equiv a^{-1}(x) \pmod{p(x)} \text{ Inverses}
\end{gather*}
$\implies$ Die Polynome modulo $p(x)$ sind ein \textbf{Körper} \\
Grundmenge: Polynome vom Grad $< n$ (falls $n \deg(p)$)
\[ \abs{ \{ a_0 + a_1 x + a_2 x^2 + \dots + a_{n-1} x^{n-1} \mid a_i \in K \} } = \abs{K}^n \]
\begin{gather*}
	p \text{ irreduzibel} \\
	p(x) \mid a(x) \cdot b(x) \implies p(x) \mid a(x) \vee p(x) \mid b(x) \\
	p \nmid a \implies \ggt(p,a) = 1 = ua + vp \implies b = \underbrace{uab}_{p \mid} + vpb \implies p \mid b
\end{gather*}
$\rightarrow$ Eindeutigkeit der Zerlegung in irreduzible Polynome \\
$\rightarrow$ Polynom vom Grad $n$ hat höchstens $n$ Nullstellen.

\subsubsection{Klassifikation endlicher Körper}
\begin{itemize}
	\item $p$ prim, $u \in \N \implies$ Dann existiert genau ein Körper mit $p^n$ Elementen: $GF(p^n)$ ($GF(p)$, irreduzibles Polynom vom Grad $n$, Grösse: $\abs{GF(p)}^n = p^n$)
	\item Das sind alle!
\end{itemize}
Charakteristik von $K$: $X(K) \coloneqq \ord_{(K,+)}(1)$ \\
\begin{beh}
	$X(K)$ ist prim. \\
	\begin{bew}
		\begin{gather*}
			\ord(1) = a \cdot b \\
			\underbrace{1 + 1 + \dots + 1}_{a \cdot b} = 0 \\
			\underbrace{(1 + 1 + \dots + 1)}_{a} \cdot \underbrace{(1 + 1 + \dots + 1)}_{b} \implies a = 1 \vee b = 1 \implies \ord(1) \text{ prim}
		\end{gather*}
	\end{bew}
\end{beh}
\[ K' = \{ 0 , 1 , 1 + 1 , \dotsc , \underbrace{1 + 1 + \dotsc}_{p-1} \} \]
$K'$ ist ein Körper, $GF(p)$ \\
$K$ ist Vektorraum über $K'$ der $\dim d$ \\
$\abs{K} = \abs{GF(p)}^d = p^d$

Polynom ($\not\equiv 0$) vom Grad $n$ hat höchstens $n$ Nullstellen. \\
Heisst auch: Falls Zwei Polynome $p_1(x) , p_2(x)$ von Grad $\leq n$ an $n+1$ Stellen den Gleichen Wert annehmen, dann sind die identisch. \\
($p_1(x) - p_2(x)$ Polynome von Grad $\leq n$, $n+1$ Nullstellen $\implies p_1(x) - p_2(x) \equiv 0$)

\subsubsection{Lagrange-Interpolation}
$\exists ! p(x) , \deg(p) \leq n$ durch die $n+1$ Stützstellen. \\
Eindeutigkeit: siehe oben. \\
Existenz:
\begin{gather*}
	p(x) = \sum \beta_i u_i(x) \\
	u_i(x) = \frac{(x-\alpha_0) (x-\alpha_1) \dotsm (x-\alpha_{i-1}) (x-\alpha_{i+1}) \dotsm (x-\alpha_n)}{(\alpha_i-\alpha_0)(\alpha_i-\alpha_1) \dotsm (\alpha_i-\alpha_{i-1}) (\alpha_i-\alpha_{i+1}) \dotsm (\alpha_i-\alpha_n)} \\
	u_i(d_j) = \begin{cases}
		0	&i \neq j	\\
		1	&i = j		
	\end{cases}
\end{gather*}

\subsubsection{Secret Sharing}
Geheimnis $s$ \\
Bedingung: $k$ aus $n$ sollen das Geheimnis kennen, $k-1$ nicht.
 \[ p(x) = \underbrace{s}_{=b_0} + b_1 x + b_2 x^2 + \dots + b_{k-1} x^{k-1} \]
 $k-1$ Shares $\rightarrow$ keine Information \\
 
 Codierungstheorie: \\
 $a_i \in GF(p)$ \\
 BILD \\
 \begin{gather*}
 	p(x) = \sum_{i=0}^k a_i x^i \\
 	\beta_i = p(\alpha_i)
\end{gather*}
Es reicht, wenn $k+1$ Werte korrekt ankommen (Falls man weiss welche ...).

Minimaldistanz eines Codes: \\
min. Hammingdistanz $d_{min}$ zwischen zwei Codewörtern. \\
Polynomcode: $d_{min} = n + 1 - k$ für lineare Codes \textbf{optimal} \\
\# Fehler korrigierbar: $\lfloor \frac{d_{min} - 1}{2} \rfloor$ \\
\# Fehler detektierbar: $d_{min} - 1$ \\
Reed-Solomon-Codes
	