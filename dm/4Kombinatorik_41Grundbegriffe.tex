\chapter{Kombinatorik}
\section{Grundbegriffe}
\begin{bsp*}
	$u \cdot r$ Kästchen\\
	Auf wie viele Arten kann man auf kürzestem Weg von $A$ nach $B$?\\
	\begin{itemize}
		\item Ansatz 1: \enquote{Divide et impera}
		\begin{gather*}
			\begin{array}{ l l }
				\text{Pascal-Dreieck:}		& \binom{n}{k} = \binom{n-1}{k-1} + \binom{n-1}{k}	\\
				\text{Verankerung:}		& \binom{n}{0} = \binom{n}{n} = 1					\\
				\text{Lösung:}			& \binom{u+r}{r}								
			\end{array}
		\end{gather*}
		\item Ansatz 2: Schnittfolge
		\begin{gather*}
			R_1 U_1 R_2 R_3 U_2 \dots U_u R_r \\
			\begin{array}{ l l }
				\text{\# Permutationen:}	& (u+r)!			\\
				\text{\# Wege:}			& \frac{(u+r)!}{u!r!}	
			\end{array}\\
			\intertext{Wir haben gesehen:}
			\binom{n}{k} = \frac{n!}{k!(n-k)!}
		\end{gather*}
	\end{itemize}
\end{bsp*}

\subsubsection{Interpretation von \texorpdfstring{$\binom{n}{k}$}{n tief k}}
\begin{itemize}
	\item Anzahl $k$-elementige Teilmengen einer $n$-Menge. \\
		\begin{bew}[note = {Direkter Beweis: Induktion über $n$}]
			\begin{gather*}
				n \rightarrow n+1 \\
				\# k\text{-Teilmengen einer } n+1\text{-Menge} \\
				\binom{n}{k} + \binom{n}{k-1} = \binom{n+1}{k}
			\end{gather*}
		\end{bew}
	\item Binomialkoeffizienten\\
		\[ (x+y)^n = \sum_{i=0}^n \binom{n}{i} x^i y^{n-i} \]
\end{itemize}

\subsubsection{\enquote{Urnenmodelle}}
\begin{bsp*}
	$n=3$ Kugeln\\
	Zeihe $k=3$ \\
	\begin{tabular}{ c | c c c | c c c }
								&\multicolumn{3}{c|}{geordnet}	&\multicolumn{3}{c}{ungeordnet}		\\	\hline
		\multirow{3}{*}{mit Zurücklegen}	& (1,1)	& (1,2)	& (1,3)	& (1,1)	& (1,2)	& (1,3)		\\
								& (2,1)	& (2,2)	& (2,3)	&		& (2,2)	& (2,3)		\\
								& (3,1)	& (3,2)	& (3,3)	&		&		& (3,3)		\\	\hline
		\multirow{3}{*}{ohne Zurücklegen}	&		& (1,2)	& (1,3)	&		& (1,2)	& (1,3)		\\
								& (2,1)	&		& (2,3)	&		&		& (2,3)		\\
								& (3,1)	& (3,2)	&		&		&		&				
	\end{tabular}\\ \phantom{.} \\
	\begin{tabular}{ l | l | l }
						& geordnet															& ungeordnet							\\	\hline
		mit Zurücklegen		& $n^k$ \footnote{Anzahle Wörter der Länge $k$ über einen Alphabet mit $n$ Zeichen}	& $\binom{n+k-1}{n} = \binom{n+k-1}{n-1}$ \footnote{\begin{gather*}
		\underbrace{1 2 3 1 1 2 1 3 2 3 \dots 3 2}_k \\
		\text{sortieren (kommt nicht drauf an, da ungeordnet)}\\
		\begin{array}{ c c c c c c c c c c c c c c c }
			1	&1	&1	&1	&\dots	&1	&|	&2	&2	&\dots	&2	&|	&3	&\dots	&3	\\
			*	&*	&*	&*	&\dots	&*	&|	&*	&*	&\dots	&*	&|	&*	&\dots	&*	
		\end{array}\\
		\left.\begin{array}{ l l l }
			\# \text{Anzahl}	& x	&: k	\\
			\# \text{Anzahl}	& |	&: n-1
		\end{array} \right\} \implies \binom{n+k-1}{n} = \binom{n+k-1}{n-1}
		\end{gather*}}	\\	\hline
		ohne Zurücklegen	& $\frac{n!}{(n-k)!} \eqqcolon n^{\underline{k}}$								& $\binom{n}{k}$						
	\end{tabular}
\end{bsp*}
\begin{bsp*}{\# Abstimmungsergebnisse}
	\begin{gather*}
		\underbrace{20 \text{ Stimmen}}_{k=20}, \underbrace{3 \text{ Varianten}}_{n=3} \\
		\binom{22}{20} = \binom{22}{2} = \frac{22 \cdot 21}{2} = 231
	\end{gather*}
\end{bsp*}
