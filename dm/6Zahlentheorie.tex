\chapter{Zahlentheorie}
\[ \N , \Z \]
\begin{tabular}{ l c l }
	bisher				&				&ab jetzt					\\ \hline
	Allgemeinheit			&$\leftrightarrow$	&Struktur					\\
	(Mengen, Graphen)		&				&(Zahlen, Graphen, Körper)		\\ \hline
	neu (Euler)				&				&alt (Euklid)				\\ \hline
	(Mathematische			&$\leftrightarrow$	&(Informatikrelevante)			\\
	Informatik				&				&Mathematik				\\ \hline
	sehr viele Anwendungen	&$\leftrightarrow$	&Sehr spezielle, überrraschende	\\
	(Richtige Sprache)		&				&Anwendungen (Kommunikation:	\\
						&				&Kryptographie, Codierung )		
\end{tabular} \\
BILD \\
historisch: populär, einfach(formulierbare) ungelöste Probleme
\begin{itemize}
	\item Goldbach-Vermutung
	\item Gibt es $\infty$ viele Primzahlzwillinge?
	\item Fermat: $x^n + y^n = z^n \quad (n \geq 3 , x, y, z \in \N_{> 0} )$
	\item $x^n - y^n = 1 \quad (m, n \geq 2 , x, y \in \N_{> 0} )$\\
		$3^2 - 2^3 = 1$
\end{itemize}
heute: algorithmische Aspekte: Quelle schwieriger Probleme. \\
Vorgehen: nicht axiomatisch: Grundtatsachen als wahr angenommen\\
\begin{bsp*}
	\begin{itemize}
		\item $a \cdot b = 0 \iff a = 0 \vee b = 0$
		\item $a \geq b \iff -a \leq -b$
		\item $(-a) \cdot b \iff -(a \cdot b)$
		\item $a^2 \geq 0 \forall a$
	\end{itemize}
\end{bsp*}
Erstes Ziel: Existenz und Eindeutigkeit der Primzahlfaktoriesierung. \\
Schlüsseltatsache:
\[ p \text{ prim: } p | a \cdot b \implies p | a \vee p | b \]
