\part{Übungsnotizen}
\setcounter{chapter}{0}
\chapter{Logik}
\section{Beweis für unendlich viele Primzahlen}

Fermat-Zahlen: $F_n = 2^{2n} +1 , n \geq 0$ \\
\begin{enumerate}
	\renewcommand{\labelenumi}{\alph{enumi})}
	\item $F_0 \cdot F_1 \cdot \dotsm \cdot F_{n-1} + 2 = F_n$
	\item $d \mid F_n \wedge d \mid F_k \wedge n \neq k \rightarrow d = 1$
	\item $\drsh$ es gibt enendlich viele Primzahlen
\end{enumerate}
\begin{gather*}
	\prod_{i=0}^0 ( F_i ) + 2 = 3 + 2 = 5 = F_1 \\
	\prod_{i=0}^n ( F_i ) + 2 = ( \prod_{i=0}^{n-1} F_i ) \cdot F_n + 2 \\
	( F_n - 2 ) \cdot F_n + 2 = ( 2^{2n} + 1 - 2)(2^{2n} + 1) = ( 2^{2n} -1 )( 2^{2n} +1 ) +1
\end{gather*}
