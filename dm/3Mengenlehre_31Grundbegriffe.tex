\chapter{Mengenlehre}
\section{Grundbegriffe}
Elementbeziehung:\\
\begin{gather*}
	x \in A : \text{Objekt } x \text{ ist Element der Menge } A \\
	x \notin A : \iff \neg ( x \in A )
\end{gather*}

\subsection{Cantor: ''naiv'': Jedes Objekt (insbes. jede Menge ) kann in jeder Menge sein oder nicht.}
Eine Menge kann sich selbst enthalten.\\
Problem: \textbf{Russel Antinomie}\index{Russel Antinomie}:\\
''Ein Barbier rasiert jeden, der sich selbst nicht rasiert. Rasiert er sich selbst?'' \\
\begin{gather*}
	\begin{array}{ l l }
		M \coloneqq \{ A \mid A \notin A \}			& M \in M ?				\\
		M \in M \implies M \notin M \implies M \in M	&\text{Nicht entscheidbar.}	
	\end{array}
\end{gather*}\\
\begin{tabular}{ l p{7cm} }
Ausweg: \textbf{Klassen}\index{Klasse} (Eigenschaften),	&\textbf{Mengen}\index{Menge} \\
										& $\drsh$ Axiomensystem ZFC (Zermelo Fränkel ''Choice'' (Auswahlaxiom))
\end{tabular}

\subsubsection{Auswahlaxiom}
Konsequenz: \text{Banach-Tarski-Paradox}\index{Banach-Tarski-Paradox}: \\
Eine Kugel in 5 teilen und daraus 2 gleich grosse Kugeln bilden.

\subsection{Wie darf man Mengen bilden?}
Mengen dürfen weiter Mengen enthalten, eg. $\{ \N , \R \}$

\subsubsection{Gleichheit von Mengen (Extensionalitätsaxiom)}
\begin{gather*}
	A , B \text{ Mengen} \\
	\begin{aligned}
		\text{Dann } A = B	&\iff \forall x~( x \in A \leftrightarrow x \in B ) \\
						&''\Rightarrow'': \text{ Logik der Gleichheit} \\
						&''\Leftarrow'': \text{ Axiom}
	\end{aligned}
\end{gather*}\\
\begin{bsp*}
	\[ \{ a , b , c \} = \{ b , c , a \} = \{ a , a , b , c \} \neq \{ \{ a \} , \{ b \} , \{ c \} \} \]
\end{bsp*}

\subsubsection{Mengenbildung mit Prädikaten}
\begin{gather*}
	A \text{ Menge}, P( \cdot ) \text{Prädikat} \\
	\text{Dann ist } B = \{ x \in A \mid P( x ) \}
\end{gather*} \\
\begin{bsp*}
	\begin{gather*}
		A = \{ x \in \mathbb{N} \mid 0 \leq x \leq 10 \} = \{ 0 , 1 , 2 , 3 , 4 , 5 , 6 , 7 , 8 , 9 , 10 \} \\
		B = \{ x \in A \mid \Prim( x ) \} = \{ 2 , 3 , 5 , 7 \}
	\end{gather*}
\end{bsp*}

\subsubsection{Teilmengenrelation}
\[ A \subseteq B : \iff \forall x~( x \in A \rightarrow x \in B ) \]
\begin{satz*}
	$A \subseteq B \wedge B \subseteq A \implies A = B$ \\
	Beweis: Extensionalität
\end{satz*}

\subsubsection{Existenz}
Es gibt mindestens eine Menge.
\begin{gather*}
	A \text{ Menge.} \\
	\varnothing \coloneqq \{ x \in A \mid x \neq x \} \\
	\text{Dann gilt für jede Menge } B: \\
	\varnothing \subseteq B .
\end{gather*}
\begin{gather*}
	\text{Die ganze Mathematik wird aus } \varnothing \text{ konstruiert.} \\
	\begin{aligned}
		\varnothing											& \eqqcolon 0	\\
		\{ \varnothing \}											& \eqqcolon 1	\\
		\{ \varnothing , \{ \varnothing \} \}							& \eqqcolon 2	\\
		\{ \varnothing , \{ \varnothing \} , \{ \varnothing , \{ \varnothing \} \} \}	& \eqqcolon 3	\\
		\mathbb{N} \coloneqq \{ 0 , 1 , 2 , 3 , \dotsc \}					& = \omega	\\
		\{ 0 , 1 , 2 , 3 , \dotsc , \omega \}								& = \omega + 1
	\end{aligned}
\end{gather*}

\subsubsection{Mengenbildungen}
Falls $A$ und $B$ Mengen sind, dann auch:
\begin{align*}
	& A \cap B : x \in A \cap B :\iff x \in A \wedge x\in B					& \text{Durchschnitt}			\\
	& A \cup B : x \in A \cup B :\iff x \in A \vee x\in B						& \text{Vereinigung}			\\
	& A \setminus B : x \in A \setminus B :\iff x \in A \wedge x\notin B			& \text{Durchschnitt}			\\
	& A \bigtriangleup B : x \in A \bigtriangleup B :\iff x \in A \oplus x\in B		& \text{symmetrische Differenz}	\\
	& \mathbb{P}( A ) = 2^A : x \in \mathbb{P}( A ) :\iff x \subseteq A			& \text{Potenzmenge}			\\
	\multicolumn{2}{l}{\text{Falls Universum } U \text{ Menge:}}										\\
	&\overline{A} \coloneqq U \setminus A								& \text{Komplementärmenge}	
\end{align*}
\[ \overline{A \setminus B} \]

\subsubsection{Mengenfamilien}
\begin{gather*}
	A_i \text{ Menge für } i \in I , I \text{ Menge} \\
	\begin{aligned}
	& x \in \bigcup_{i \in I} A_i :\iff \exists i \in I : x \in A_i 		& \bigcup_{i \in \varnothing} A_i = \varnothing \\
	& x \in \bigcap_{i \in I} A_i :\iff \forall i \in I : x \in A_i 		& \bigcup_{i \in \varnothing} A_i \text{ darf nicht existieren } \rightarrow \text{Russel Antinomie}
	\end{aligned}
\end{gather*}

\subsubsection{Das kartesische Produkt (Descartes, 17 Jh.)}
Das geordnete Paar
\begin{gather*}
	(a , b) = (c , d) :\iff a = c \wedge b = d \\
	\intertext{Definition als Menge:}
	(a , b) \coloneqq \{ \{ a \} , \{ a , b \} \} \qquad (a , a) = \{ \{ a \} , \{ a , a \} \} = \{ \{ a \} , \{ a \} \} = \{ \{ a \} \}\\
	\\
	A , B \text{ Mengen.} \\
	A \times B \coloneqq \{ (a , b) \mid a \in A \wedge b \in B \} \\
	A \times B \neq B \times A \text{ falls } A \neq B \\
	I = \{ 1 , 2 , \dotsc , k \} \\
	\bigtimes_{i=0}^k A_i \coloneqq \{ (a_1 , \dotsc , a_k ) \mid \forall i : a_i \in A_i \}
\end{gather*}
\begin{bsp*}
	\begin{gather*}
		\mathbb{R} \times \mathbb{R} \\
		R \subseteq \mathbb{R} \times \mathbb{R} \\
		R = \{ (x , y) \in \mathbb{R}^2 \mid x \leq y \}
	\end{gather*}
\end{bsp*}
