\subsection{Äquivalenzrelationen}
\begin{def*}[note = Äquivalenzrelation , index = Äquivalenzrelation]
	Eine Relation $\sim$ auf $A$ heisst Äquivalenzrelation auf $A$, falls sie
	\begin{itemize}
		\item reflexiv,
		\item symmetrisch und
		\item transitiv
	\end{itemize}
	ist.
\end{def*}
\begin{bsp*}
	$A = \{$ Menschen $\}$ \\
	$a \sim b : \iff a $ und $b$ im gleichen Jahr geboren. \\
	$\rightsquigarrow$ Partition der Grundmenge
\end{bsp*}
\begin{bsp*}
	\begin{gather*}
		\sim \text{ Relation auf } \mathbb{R}^2 \qquad [ \sim \subseteq (\mathbb{R}^2)^2 ] \\
		(x,y) \sim (x',y') :\iff x + y = x' + y' \\
		\intertext{(Definition via Gleichheit einer Eigenschaft $\rightarrow$ Äquivalenzrelation)}\\
		[(x,y)] \coloneqq \{ (x',y') \mid (x',y') \sim (x,y) \} \quad \text{Äquivalenzklasse von } (x,y) \\
		\intertext{Die Äquivalenzklassen partitionieren die Ebene.}
	\end{gather*}
\end{bsp*}
\begin{def*}[note = Partition , index = Partition]
	Sei $A$ eine Menge. Eine Partition von A ist eine Mengenfamilie $(A_i)_{i \in I}$ mit\\
	\begin{gather*}
		\bigcup_{i \in I} A_i = A \\
		\forall i \neq j : A_i \cap A_j = \varnothing \qquad (A_i , A_j \textbf{ disjunkt} )
	\end{gather*}
\end{def*}
\begin{satz*}
	$A$ Menge\\
	\begin{itemize}
		\item $\sim$ Äquivalenzrelation auf $A \implies [a] \coloneqq \{ b \in A \mid b \sim a \}$ Partition.
		\item $(A_i)_{i \in I}$ Partition von $A$. $x \sim y :\iff \exists i : x \in A_i \ni y$. Dann ist $\sim$ Äquivalenzrelation.
	\end{itemize}
\end{satz*}
\begin{bew}
	Zu zeigen ist: Äquivalenzklassen entweder disjunkt oder identisch.\\
	\begin{gather*}
		z \in [x] \cap [y] \\
		\text{Zu zeigen: } [x] = [y] \\
		\left.\begin{array}{l}
			z \in [x] \implies z \sim x \implies x \sim z \\
			z \in [y] \implies z \sim y
		\end{array} \right\} \implies x \sim y \\
		\left.\begin{array}{l}
			\forall x' \in [x] : x \sim x \\
			x \sim y
		\end{array} \right\} \implies x' \sim y \implies x \in [y] \\
		[x] \subseteq [y], \text{ analog } [y] \subseteq [x] \implies [x] = [y] 
	\end{gather*}
\end{bew}
\begin{bsp*}
	\begin{itemize}
		\item Semantische Äquivalenz
		\item $\equiv \pmod m$ Äquivalenzrelation auf $\mathbb{Z}$
		\begin{itemize}
			\item Äquivalenzklassen\\
			\begin{gather*}
				[0] = \{ \dotsc, -2m, -m, 0, 2m, 3m, \dotsc \} \\
				[1] = \{ \dotsc, -2m+1, -m+1, 1, 2m+1, 3m+1, \dotsc \} \\
				\vdots \\
				[m-1] = \dots \\
				\text{Anzahl Klassen: } | \{[0], [1], \dotsc , [m-1] \} | = m
			\end{gather*}
			\item neue Struktur $\mathbb{Z}_m = \mathbb{Z}/\equiv_m$ \\
				Auf $\mathbb{Z}_m$ werden Operatoren definiert:\\
				\begin{gather*}
					[a] + [b] \coloneqq [a+b] \\
					[a] = [a'] \\
					[b] = [b'] \\
					[a+b] \stackrel{?}{=} [a'+b'] \\
					a' \in [a] \\
					a' \equiv a \pmod m \\
					\begin{aligned}
						\left.\begin{array}{l}
							a' \equiv a + k \cdot m \\
							b' \equiv b + l \cdot m
						\end{array} \right\} &\implies a' + b' \equiv a + b + (k+l ) \cdot m \\
						&\implies a' + b' \in [a+b] \\
						&\iff [a'+b'] = [a+b]
					\end{aligned}
				\end{gather*}
		\end{itemize}
	\end{itemize}
\end{bsp*}
