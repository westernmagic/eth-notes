\chapter{Mechanik}

\section{Kinematik}
Beschreibung von Bewegungen \\
einfachster Fall:
\begin{itemize}
	\item geradlinige Bahn (1D)
	\item gleichförmige Bewegung
\end{itemize}
\todo{Diagram}

\subsection{Weg-Zeit-Diagramm}
\todo{Graph}
\[ \tan \alpha = \frac{\Delta s}{\Delta t} \overset{!}{=} v \]
konst. Steigung von $s(t) \implies$ konst. $v$

\subsection{Geschwindigkeit}
\begin{def*}[ note = Geschwindigkeit , index = Geschwindigkeit ]
	\begin{gather*}
		v = \frac{\Delta s}{\Delta t} \\
		[v] = \frac{m}{s}
	\end{gather*}
\end{def*}

\subsection{Geschwindigkeits-Zeit Diagramm}
\todo{Graph}
\begin{gather*}
	v(t) = \frac{\Delta s}{\Delta t} \\
	\text{Fläche} = v \cdot \Delta t = \Delta s \text{ ! } = \text{ zurückgelegter Weg}
\end{gather*}

\subsection{Nicht-gleichförmige Bewegungen}
\todo{Graph}
Geschwindigkeit $v(t)$
\begin{gather*}
	\overline{v} = \frac{\Delta s}{\Delta t} = \text{ mittlere Geschwindigkeit zw. $t_1$ und $t_2$} \\
	v(t_1) = \lim_{t_1 \rightarrow t_2} \frac{\Delta s}{\Delta t} \underset{\text{\scriptsize{Math.}}}{=} s'(t) \\
	v(t) = s'(t) \\
\end{gather*}

\subsubsection{Schreibweise}
\begin{gather*}
	\Delta t \rightsquigarrow \dd t \\
	v(t) = s'(t) \eqqcolon \frac{\dd s}{\dd t}
\end{gather*}
1. Ableitung \\
$v(t)$: Momentangeschwindigkeit
\begin{bsp*}[ note = $v$ nimmt gleichmässig zu ]
	Graph
	\[ \text{Fläche } \underset{\text{\scriptsize{Math.}}}{\overset{!}{=}} \int_{t_1}^{t_2} v(t) \dd t \]
\end{bsp*}
\todo{Graph}
Änderung der Geschwindigkeit mit der Zeit:
\begin{def*}[ note = Beschleunigung , index = Beschleunigung ]
	\begin{gather*}
		a = \frac{\Delta v}{\Delta t} \\
		[a] = \frac{m}{s^2}
	\end{gather*}
\end{def*}
Fall:
\begin{itemize}
	\item gleichförmige Beschleunigung: $a =$ konst.
	\item beliebige Funktion $a(t)$
\end{itemize}
\todo{Graph}
analog:
\begin{gather*}
	a(t) = \frac{\dd v}{\dd t} \\
	a(t) = \frac{\dd}{\dd t} \left( \frac{\dd s}{\dd t} \right) = s''(t) \eqqcolon \frac{\dd^2 s}{\dd t^2}
\end{gather*}
