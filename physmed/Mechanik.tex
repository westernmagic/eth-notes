\chapter{Mechanik}

\section{Kinematik}
Beschreibung von Bewegungen \\
einfachster Fall:
\begin{itemize}
	\item geradlinige Bahn (1D)
	\item gleichförmige Bewegung
\end{itemize}
\includegraphics{Bild1}

\subsection{Weg-Zeit-Diagramm}
\includegraphics{Bild2}
\[ \tan \alpha = \frac{\Delta s}{\Delta t} \overset{!}{=} v \]
konst. Steigung von $s(t) \implies$ konst. $v$

\subsection{Geschwindigkeit}
\begin{def*}[ note = Geschwindigkeit , index = Geschwindigkeit ]
	\begin{gather*}
		v = \frac{\Delta s}{\Delta t} \\
		[v] = \frac{m}{s}
	\end{gather*}
\end{def*}

\subsection{Geschwindigkeits-Zeit Diagramm}
\includegraphics{Bild3}
\begin{gather*}
	v(t) = \frac{\Delta s}{\Delta t} \\
	\text{Fläche} = v \cdot \Delta t = \Delta s \text{ ! } = \text{ zurückgelegter Weg}
\end{gather*}

\subsection{Nicht-gleichförmige Bewegungen}
\includegraphics{Bild4}
Geschwindigkeit $v(t)$
\begin{gather*}
	\overline{v} = \frac{\Delta s}{\Delta t} = \text{ mittlere Geschwindigkeit zw. $t_1$ und $t_2$} \\
	v(t_1) = \lim_{t_1 \rightarrow t_2} \frac{\Delta s}{\Delta t} \underset{\text{\scriptsize{Math.}}}{=} s'(t) \\
	v(t) = s'(t) \\
\end{gather*}

\subsubsection{Schreibweise}
\begin{gather*}
	\Delta t \rightsquigarrow \dd t \\
	v(t) = s'(t) \eqqcolon \frac{\dd s}{\dd t}
\end{gather*}
1. Ableitung \\
$v(t)$: Momentangeschwindigkeit
\begin{bsp*}[ note = $v$ nimmt gleichmässig zu ]
	\includegraphics{Bild5}
	\[ \text{Fläche } \underset{\text{\scriptsize{Math.}}}{\overset{!}{=}} \int_{t_1}^{t_2} v(t) \dd t \]
\end{bsp*}
\todo{Undefined seq.}
Änderung der Geschwindigkeit mit der Zeit:
\begin{def*}[ note = Beschleunigung , index = Beschleunigung ]
	\begin{gather*}
		a = \frac{\Delta v}{\Delta t} \\
		[a] = \frac{m}{s^2}
	\end{gather*}
\end{def*}
Fall:
\begin{itemize}
	\item gleichförmige Beschleunigung: $a =$ konst.
	\item beliebige Funktion $a(t)$
\end{itemize}
\includegraphics{Bild6}
analog:
\begin{gather*}
	a(t) = \frac{\dd v}{\dd t} \\
	a(t) = \frac{\dd}{\dd t} \left( \frac{\dd s}{\dd t} \right) = s''(t) \eqqcolon \frac{\dd^2 s}{\dd t^2}
\end{gather*}

\begin{rep*}[note = Kinematik (1D-Fall)]
	\includegraphics{Bild7}
\end{rep*}
\begin{bsp*}[note = Der freie Fall]
	\includegraphics{Bild8} \\
	auf der Erdoberfläche
	\[ a(t) = a_{\text{Fall}} = g = 9.81 \frac{\text{m}}{{\text{s}}^2} \]
	\includegraphics{Bild9}
	\[ \text{Fläche } = g( t - t_0 ) \]
	\includegraphics{Bild10}
	\begin{gather*}
		v(t) = v_0 + g( t - t_0 ) \\
		\Delta v = g( t - t_0 ) \
		\text{Fläche } = v_0 ( t - t_0 ) + \frac{1}{2} g( t - t_0 )^2 = \Delta S
	\end{gather*}
	\includegraphics{Bild11}
	\[ s(t) = s_0 + v_0 ( t - t_0 ) + \frac{1}{2} g( t - t_0 ) \]
	
	einfachster Fall:
	\includegraphics{Bild12}
	\begin{gather*}
		s(0) = 0 \\
		t_0 = 0 \\
		s_0 = 0 \\
		v_0 = 0 \\
		\implies s(t) = \frac{1}{2} gt^2
	\end{gather*}
\end{bsp*}

\subsection{Bewegungen in der Ebene}
\subsubsection{Ortsvektor \texorpdfstring{$\vec{r}(t)$}{r(t)}}
\includegraphics{Bild13}
\begin{itemize}[label = $\rightarrow$]
	\item Länge (Betrag)
	\item Richtung
\end{itemize}

Geschwindigkeit: $\frac{\text{Weg}}{\text{Zeit}}$
Weg: $\Delta \vec{r} = \vec{r}(t) - \vec{r}(t_0)$ \\
$\vec{v} = \frac{\Delta \vec{r}}{\Delta t}$ mittlere Geschwindigkeit \\
Momentangeschwindigkeit: $\vec{v}(t) = \lim_{\Delta t \rightarrow 0} \frac{\Delta \vec{r}}{\Delta t} \overset{\text{\scriptsize{Math.}}}{=} \frac{\dd \vec{r}}{\dd t}$ \\
\includegraphics{Bild14}
\begin{gather*}
	\vec{r}(t) = \vec{x}(t) + \vec{y}(t) \\
	\implies \text{ Komponentenschreibweise} \\
	\vec{r}(t) = ( x(t) , y(t) ) \\
	\frac{\dd \vec{r}}{\dd t} = \left( \frac{\dd x}{\dd t} , \frac{\dd y}{\dd t} \right)
\end{gather*}
$v(t)$:
\begin{itemize}
	\item Betrag (Schnelligkeit)
	\item Richtung! (tangential zu Bahn)
\end{itemize}
\subsubsection{Schnelligkeit}
\[ \abs{\vec{v}(t)} = v(t) = \sqrt{{v_1}^2(t) + {v_2}^2(t)} \]
\includegraphics{Bild15}

\subsubsection{Momentanbeschleunigung}
\[ \vec{a}(t) = \frac{\dd \vec{v}}{\dd t} = \frac{\dd^2 \vec{s}}{\dd t^2} \]

\subsection{Wann ist eine Bewegung beschleunigt?}
Wenn $\vec{v}$ sich ändert!
\begin{itemize}[label = $\rightarrow$]
	\item Betrag
	\item Richtung!
\end{itemize}
\begin{bsp*}[note = Kreisbewegung mit konstante Umlaufgeschwindigkeit]
	\includegraphics{Bild16}\\
	$\implies v =$ konst., $\vec{v}$ dreht $\implies$ Zentripetalbeschleunigung
	\[ a = \frac{v^2}{r} \]
\end{bsp*}
\begin{bsp*}[note = beliebige Kreisbewegung ($v \neq$ konst.)]
	\includegraphics{Bild17} \\
	\[ \begin{matrix*}[l]
		a_Z(t) = \frac{v^2(t)}{r}		&\text{Zentripetalbeschleunigung} \\
		a_T(t) = \frac{\dd v}{\dd t}	&\text{Tangentialbeschleunigung}
	\end{matrix*} \]
\end{bsp*}

\subsection{Bewegungen im 3D-Raum}
\includegraphics{Bild18} \\
nicht Neues!
\[ \vec{r}(t) = ( x(t) , y(t) , z(t) ) \]

\section{Dynamik}
$\implies$ Ursache der Bewegung
\subsection{Kraft/Masse}
\begin{def*}[ note = Kraft , index = Kraft ]
	\dots Wirkung!
\end{def*}
\begin{bsp*}
	\begin{itemize}
		\item Gewicht heben
		\item Deformation (Messung!)
		\item Bewegung
	\end{itemize}
\end{bsp*}
\begin{def*}[ note = Masse , index = Masse ]
	''Trägheit'' (''\dots schwieriger in Bewegung zu setzen'')
\end{def*}
Kraft: Vektor! \\
\includegraphics{Bild19} \\
Länge, Richtung, Angriffspunkt

\subsection{Die Newtonschen Prinzipien (1686)}
\begin{gather*}
	\underbrace{\vec{F}}_{\text{Ursache}} = \underbrace{m \cdot \vec{a}}_{\text{Wirkung}} \\
	[F] = \text{kg} \frac{\text{m}}{\text{s}^2} = \text{N} \quad ( \text{Newton} )
\end{gather*}
$\implies$ 2. Newtonsche Prinzip (Axiom) (Aktionsprinzip) \\
Anwendung: \\
\begin{itemize}
	\item Mann kennt Kraft $\implies$ Beschleunigung + Bahn berechnen
	\item Ich sehe Beschleunigung $\implies$ Was für Kräfte wirken
\end{itemize}
\subsubsection{1. Newtonsches Prinzip (Trägheitsprinzip)}
kräftefreie Körper ( $\vec{F} = \vec{0}$ , $\sum_i \vec{F_i} = \vec{0}$ )
\begin{itemize}[ label = $\rightarrow$ ]
	\item Körper in Ruhe (ist + bleibt)
	\item bewegt sich mit konst. Geschwindigkeit $\vec{v} =$ konst.
\end{itemize}
$\implies$ Bewegungszustände

\subsubsection{Newtonsches Prinzip (Reaktionsprinzip)}
(action = reactio) \\
Kräfte rühren immer von Wechselwirkungen (WW)
\begin{bsp*}[note = Feder]
	\includegraphics{Bild20}
	\[ \vec{F_{21}} = -\vec{F_{12}} \]
	Reaktionspartner $\rightarrow$ greifen immer an verschiedenen Körper an
\end{bsp*}
\begin{rep*}[ note = Dynamik ]
	Kraft: erzeugt Bewegung \\
	Masse: Trägheit \\
	Newtonsche Prinzipien \\
	\begin{enumerate}
		\item kräftefreier Körper: $\vec{v} =$ konst. (z.B. $\vec{v} = \vec{0}$)
		\item $\vec{F} = m \vec{a}$ Ursache \& Wirkung
		\item Kräfte rühren \textbf{immer} von WW her \\
			\includegraphics{Bild21} \\
			$\vec{F_{21}} = -\vec{F_12}$
	\end{enumerate}
\end{rep*}
\begin{bsp*}[ note = Hammer \& Nagel ]
	\includegraphics{Bild22}
	\begin{enumerate}[start = 2]
		\item $\vec{F_{NH}} = M \vec{a_H}$
		\item $\vec{F_HN} = -\vec{NH}$
	\end{enumerate}
\end{bsp*}

\subsection{Arten von Kräften}
\subsubsection{Gravitationskraft}
(Anziehung von Massen) \\
auf Erdoberfläche Gewichtskraft $\vec{G}$ \\
\begin{tabular}{ll}
	Betrag:			&$mg$ \\
	Richtung:			&zum Erdmittelpunkt \\
	Angriffspunkt:	&Schwerpunkt \\
	Reaktionspartner:	&\includegraphics{Bild23}
\end{tabular}

\subsubsection{Elektromagnetische Kräfte}
(Anziehung / Abstossung von Ladungen)
\begin{itemize}[ label = $\rightarrow$ ]
	\item Coulombkraft (elektrische Kraft; verschiedene Erscheingungsformen)
	\item magnetische Kraft
	\item Lorentzkraft
\end{itemize}

\subsubsection{starke Kraft}
\begin{itemize}[ label = $\rightarrow$ ]
	\item Stabilität der Atomkeime
\end{itemize}

\subsubsection{schwache Kraft}
\begin{itemize}[ label = $\rightarrow$ ]
	\item Radioaktivität
\end{itemize}

\subsection{Coulombkraft und ihre Erscheinungsformen}
\begin{itemize}
	\item elastische Kräfte im festen Körpern (Kohäsion)
	\item Berührungskräfte - Die Normalkraft
		\begin{bsp*}[ note = Quader auf Tisch in Ruhe ]
			\includegraphics{Bild24}
			\begin{gather*}
				\sum \vec{F_a} = \vec{0} \\
				\implies \vec{G} + \sum \vec{\dd F} = \vec{0}
			\end{gather*}
		\end{bsp*}
\end{itemize}

\subsubsection{Coulombgesetz}
\includegraphics{Bild25}
\[ F_{21} = F_{12} = \frac{1}{4 \pi \epsilon_0} \frac{Q_1 Q_2}{r^2} \]
elektirsche Feldkonstante $\epsilon_0 = 8.85 \cdot 10^{-12} \frac{As}{Vm}$

\subsubsection{Kraftgesetz zwischen zwei Atomen}
\includegraphics{Bild26}

\subsubsection{Kraftkurve}
\includegraphics{Bild27} \\
GGW-Abstand (chemische Bindung)

\subsection{Reibungskräfte}
(parallel zur Berührungsfläche)

\subsubsection{Haftreibung}
\includegraphics{Bild28} \\
Quader unbewegt \\
\[ \vec{F_H} = -\vec{F_F} \]

maximale Haftreibung:
\[ F_H \leq \underbrace{\mu_H}_{\text{Haftreibungszahl}} F_N \]

\subsection{Gleitreibung \texorpdfstring{$\vec{F_R}$}{F_R}}
\includegraphics{Bild29}
\begin{itemize}
	\item Richtung: versucht immer Relativbewegung zu bremsen.
	\item unabhängig von $v$
\end{itemize}
\[ F_R = \mu_G \cdot F_N \]

\subsection{Kraftstösse}
\includegraphics{Bild30}












