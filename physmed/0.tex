\setcounter{chapter}{-1}
\chapter{Wozu Physik für Mediziner?}
Physik = Lehre der Naturgesetze
\begin{enumerate}
	\item Mensch \& Tier: Teil der Natur $\rightarrow$ Verständnis des Organismus
		\begin{bsp*}
			\begin{itemize}
				\item Hüftgelenk $\rightarrow$ Mechanik, Festigkeitslehre
				\item Auge $\rightarrow$ Optik
				\item Reizübertragung (Nerven) $\rightarrow$ Elekrizität
				\item Blutzirkulation $\rightarrow$ Strömungslehre
			\end{itemize}
		\end{bsp*}
	\item Diagnostik-/Theraoiewerkzeuge $\rightarrow$ physikalische Apparate
		\begin{bsp*}
			\begin{itemize}
				\item Röntgenapparatur, CT, MRI $\rightarrow$ Verstehen der Resultate $\rightarrow$ Schutz von Patient + Personal
			\end{itemize}
		\end{bsp*}
	\item Besondere Berufsbilder
		\begin{itemize}
			\item Gerichtsmediziner
			\item Sicherheit, Unfallverhütung
			\item Strahlenschutz
		\end{itemize}
	\item Analytisches Denken! Probleme lösen: (Diagnose, Entscheidungen treffen)
\end{enumerate}